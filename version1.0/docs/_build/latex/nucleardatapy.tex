%% Generated by Sphinx.
\def\sphinxdocclass{report}
\documentclass[letterpaper,10pt,english]{sphinxmanual}
\ifdefined\pdfpxdimen
   \let\sphinxpxdimen\pdfpxdimen\else\newdimen\sphinxpxdimen
\fi \sphinxpxdimen=.75bp\relax
\ifdefined\pdfimageresolution
    \pdfimageresolution= \numexpr \dimexpr1in\relax/\sphinxpxdimen\relax
\fi
%% let collapsible pdf bookmarks panel have high depth per default
\PassOptionsToPackage{bookmarksdepth=5}{hyperref}

\PassOptionsToPackage{warn}{textcomp}
\usepackage[utf8]{inputenc}
\ifdefined\DeclareUnicodeCharacter
% support both utf8 and utf8x syntaxes
  \ifdefined\DeclareUnicodeCharacterAsOptional
    \def\sphinxDUC#1{\DeclareUnicodeCharacter{"#1}}
  \else
    \let\sphinxDUC\DeclareUnicodeCharacter
  \fi
  \sphinxDUC{00A0}{\nobreakspace}
  \sphinxDUC{2500}{\sphinxunichar{2500}}
  \sphinxDUC{2502}{\sphinxunichar{2502}}
  \sphinxDUC{2514}{\sphinxunichar{2514}}
  \sphinxDUC{251C}{\sphinxunichar{251C}}
  \sphinxDUC{2572}{\textbackslash}
\fi
\usepackage{cmap}
\usepackage[T1]{fontenc}
\usepackage{amsmath,amssymb,amstext}
\usepackage{babel}



\usepackage{tgtermes}
\usepackage{tgheros}
\renewcommand{\ttdefault}{txtt}



\usepackage[Bjarne]{fncychap}
\usepackage{sphinx}

\fvset{fontsize=auto}
\usepackage{geometry}


% Include hyperref last.
\usepackage{hyperref}
% Fix anchor placement for figures with captions.
\usepackage{hypcap}% it must be loaded after hyperref.
% Set up styles of URL: it should be placed after hyperref.
\urlstyle{same}


\usepackage{sphinxmessages}




\title{nucleardatapy}
\date{Nov 25, 2024}
\release{0.1}
\author{Jérôme Margueron, IRL NPA, USA}
\newcommand{\sphinxlogo}{\vbox{}}
\renewcommand{\releasename}{Release}
\makeindex
\begin{document}

\ifdefined\shorthandoff
  \ifnum\catcode`\=\string=\active\shorthandoff{=}\fi
  \ifnum\catcode`\"=\active\shorthandoff{"}\fi
\fi

\pagestyle{empty}
\sphinxmaketitle
\pagestyle{plain}
\sphinxtableofcontents
\pagestyle{normal}
\phantomsection\label{\detokenize{index::doc}}


\sphinxAtStartPar
\sphinxstylestrong{nucleardatapy} (/in short nuda/) is a Python library for nuclear physicists facilitating the access to theoretical or experimental nuclear data. It is specificaly designed for equation of state practitionners interested in the modeling of neutron stars, and it offers \sphinxstyleemphasis{simple} and \sphinxstyleemphasis{intuitive} APIs.

\sphinxAtStartPar
All data are provided with their reference, so when using these data in a scientific paper, reference to data should be provided explicitely. The reference to this toolkit could be given, but it should not mask the reference to data.

\sphinxAtStartPar
This python toolkit is designed to provide:
1) microscopic calculations in nuclear matter,
2) phenomenological predictions in nuclear matter,
3) experimental data for finite nuclei.

\sphinxAtStartPar
Check out the {\hyperref[\detokenize{source/usage::doc}]{\sphinxcrossref{\DUrole{doc}{Usage}}}} section for further information, including how to
{\hyperref[\detokenize{source/usage:installation}]{\sphinxcrossref{\DUrole{std,std-ref}{install}}}} the project.

\begin{sphinxadmonition}{note}{Note:}
\sphinxAtStartPar
This project is under active development.
\end{sphinxadmonition}


\chapter{Contents}
\label{\detokenize{index:contents}}
\sphinxstepscope


\section{Usage}
\label{\detokenize{source/usage:usage}}\label{\detokenize{source/usage::doc}}

\subsection{Installation}
\label{\detokenize{source/usage:installation}}\label{\detokenize{source/usage:id1}}
\sphinxAtStartPar
To use nucleardatapy, first download the .zip file from the git repository, or clone it in your local computer:

\begin{sphinxVerbatim}[commandchars=\\\{\}]
\PYG{g+gp}{\PYGZdl{} }git\PYG{+w}{ }clone\PYG{+w}{ }https://github.com/jeromemargueron/nucleardatapy
\end{sphinxVerbatim}

\sphinxAtStartPar
If you have downloaded the .zip file, you can unzip it anywhere in your local computer:

\begin{sphinxVerbatim}[commandchars=\\\{\}]
\PYG{g+gp}{\PYGZdl{} }unzip\PYG{+w}{ }nucleardatapy.zip
\end{sphinxVerbatim}

\sphinxAtStartPar
Then, in all cases, you shall enter into the new folder \sphinxtitleref{/nucleardatapy}:

\begin{sphinxVerbatim}[commandchars=\\\{\}]
\PYG{g+gp}{\PYGZdl{} }\PYG{n+nb}{cd}\PYG{+w}{ }nucleardatapy
\end{sphinxVerbatim}

\sphinxAtStartPar
and launch the install script:

\begin{sphinxVerbatim}[commandchars=\\\{\}]
\PYG{g+gp}{\PYGZdl{} }bash\PYG{+w}{ }install.sh
\end{sphinxVerbatim}

\sphinxAtStartPar
This will copy the Python toolkit into \$HOME/mylib/ as well as a few samples. It will also give you the content of the global variable NUCLEARDATAPY\_TK. If you edit install.sh, you can change the version (by default it is set to the latest one) as well as the destination folder (by default it is \$HOME/mylib).

\sphinxAtStartPar
Finally, you will have to create the global variable NUCLEARDATAPY\_TK with its right content. If you do not want to create it each time you open a new terminal, then you can define it in your .profile or .zprofil or .bash file as:

\begin{sphinxVerbatim}[commandchars=\\\{\}]
\PYG{g+go}{export NUCLEARDATAPY\PYGZus{}TK=\PYGZdl{}HOME/mylib/nucleardatapy}
\end{sphinxVerbatim}

\begin{sphinxadmonition}{note}{Note:}
\sphinxAtStartPar
The exact path to write above is given at the end of the installation.
\end{sphinxadmonition}


\subsection{Use nucleardatapy}
\label{\detokenize{source/usage:use-nucleardatapy}}\label{\detokenize{source/usage:use}}
\sphinxAtStartPar
Go to the folder \sphinxtitleref{mylib/nucleardatapy/samples/nucleardatapy\_samples/} and try that:

\begin{sphinxVerbatim}[commandchars=\\\{\}]
\PYG{g+gp}{\PYGZdl{} }python3\PYG{+w}{ }sample\PYGZus{}SetupMicroMatter.py
\end{sphinxVerbatim}


\subsection{Test nucleardatapy}
\label{\detokenize{source/usage:test-nucleardatapy}}\label{\detokenize{source/usage:test}}
\sphinxAtStartPar
A set of tests can be easily performed. They are stored in tests/ folder.

\begin{sphinxVerbatim}[commandchars=\\\{\}]
\PYG{g+gp}{\PYGZdl{} }bash\PYG{+w}{ }run\PYGZus{}tests.sh
\end{sphinxVerbatim}


\subsection{Get started}
\label{\detokenize{source/usage:get-started}}\label{\detokenize{source/usage:id2}}
\sphinxAtStartPar
How to obtain microscopic results for APR equation of state:

\begin{sphinxVerbatim}[commandchars=\\\{\}]
\PYG{k+kn}{import} \PYG{n+nn}{os}
\PYG{n}{nucleardatapy\PYGZus{}tk} \PYG{o}{=} \PYG{n}{os}\PYG{o}{.}\PYG{n}{getenv}\PYG{p}{(}\PYG{l+s+s1}{\PYGZsq{}}\PYG{l+s+s1}{NUCLEARDATAPY\PYGZus{}TK}\PYG{l+s+s1}{\PYGZsq{}}\PYG{p}{)}
\PYG{n}{sys}\PYG{o}{.}\PYG{n}{path}\PYG{o}{.}\PYG{n}{insert}\PYG{p}{(}\PYG{l+m+mi}{0}\PYG{p}{,} \PYG{n}{nucleardatapy\PYGZus{}tk}\PYG{p}{)}

\PYG{k+kn}{import} \PYG{n+nn}{nucleardatapy} \PYG{k}{as} \PYG{n+nn}{nuda}

\PYG{n}{mic} \PYG{o}{=} \PYG{n}{nuda}\PYG{o}{.}\PYG{n}{SetupMicroMatter}\PYG{p}{(} \PYG{n}{model} \PYG{o}{=} \PYG{l+s+s1}{\PYGZsq{}}\PYG{l+s+s1}{1998\PYGZhy{}VAR\PYGZhy{}AM\PYGZhy{}APR}\PYG{l+s+s1}{\PYGZsq{}} \PYG{p}{)}

\PYG{n}{mic}\PYG{o}{.}\PYG{n}{print\PYGZus{}outputs}\PYG{p}{(} \PYG{p}{)}
\end{sphinxVerbatim}

\sphinxstepscope


\section{API}
\label{\detokenize{source/api:api}}\label{\detokenize{source/api::doc}}

\begin{savenotes}\sphinxattablestart
\centering
\begin{tabulary}{\linewidth}[t]{\X{1}{2}\X{1}{2}}
\hline

\sphinxAtStartPar
{\hyperref[\detokenize{source/generated/nucleardatapy:module-nucleardatapy}]{\sphinxcrossref{\sphinxcode{\sphinxupquote{nucleardatapy}}}}}
&
\sphinxAtStartPar
This module provides microscopic, phenomenological and experimental data constraints.
\\
\hline
\end{tabulary}
\par
\sphinxattableend\end{savenotes}

\sphinxstepscope


\subsection{nucleardatapy}
\label{\detokenize{source/generated/nucleardatapy:module-nucleardatapy}}\label{\detokenize{source/generated/nucleardatapy:nucleardatapy}}\label{\detokenize{source/generated/nucleardatapy::doc}}\index{module@\spxentry{module}!nucleardatapy@\spxentry{nucleardatapy}}\index{nucleardatapy@\spxentry{nucleardatapy}!module@\spxentry{module}}
\sphinxAtStartPar
This module provides microscopic, phenomenological and experimental data constraints.

\sphinxstepscope


\section{Miscelaneous}
\label{\detokenize{source/miscelaneous:miscelaneous}}\label{\detokenize{source/miscelaneous::doc}}

\subsection{Contributing}
\label{\detokenize{source/miscelaneous:contributing}}\label{\detokenize{source/miscelaneous:id1}}
\sphinxAtStartPar
For the moment, contributions are based on co\sphinxhyphen{}optation among the team.

\sphinxAtStartPar
To make contribution easy, we all work in the \sphinxtitleref{main} branch and we shall therefore remember to pull before working and pulling after, with a running version. For long developments, you can work in a local folder (in \sphinxtitleref{mylib} for instance) and copy your contribution to the GitHub folder once you are sure it is functionning. So the final step should last less than 5 minutes, and can be safely done between a pull and before a push. Since we are not numerous, we hope that no one will work in the same part of the code at the same time (i.e. between a pull and a push). It is probably the simpler way to proceed.

\sphinxAtStartPar
Once the toolkit is released, the rules to contribute will be changing. A team of developpers should be defined and a generic email to contact them should be created. Here is a suggestion to contribute after the release.

\sphinxAtStartPar
This file describes how new contributors to the project can start contributing.

\sphinxAtStartPar
\sphinxstylestrong{Two ways:}

\sphinxAtStartPar
You can provide your data and interacting with one of our developer.

\sphinxAtStartPar
You can also join the developing team and extend the functionality of this toolkit.

\sphinxAtStartPar
\sphinxstylestrong{Provide your data:}

\sphinxAtStartPar
Please contact the developer team directly by shooting an email to TBC.

\sphinxAtStartPar
Then you can interact directly with one of our developer and provide your data. You will not be able to push your data to the repository, but an updated version of the toolkit will contain your new data.

\sphinxAtStartPar
\sphinxstylestrong{Join the team:}

\sphinxAtStartPar
Please contact the developer team directly by shooting an email to TBC. Explain the reason why you wish to join the team and if you have ideas about extending the functionality of the toolkit.

\sphinxAtStartPar
Once in the team, a branch will be dedicated to your contribution. You could show it during our virtual meetings, and your contribution will be merged to the new version of the toolkit.


\subsection{License}
\label{\detokenize{source/miscelaneous:license}}\label{\detokenize{source/miscelaneous:id2}}
\sphinxAtStartPar
TBC.


\subsection{Report issues}
\label{\detokenize{source/miscelaneous:report-issues}}\label{\detokenize{source/miscelaneous:id3}}
\sphinxAtStartPar
For the current version, we report issues chatting among us.
Once this toolkit is released, we should setup a way that users could contact us and report issues or difficulties in installing or using the toolkit.


\subsection{Thanks}
\label{\detokenize{source/miscelaneous:thanks}}\label{\detokenize{source/miscelaneous:id4}}
\sphinxAtStartPar
A special thanks to all contributors who accepted to share their results in this toolkit.


\chapter{Complement}
\label{\detokenize{index:complement}}
\sphinxstepscope


\section{SetupEOSFFG}
\label{\detokenize{source/api/setup_eos_ffg:setupeosffg}}\label{\detokenize{source/api/setup_eos_ffg::doc}}\phantomsection\label{\detokenize{source/api/setup_eos_ffg:module-nucleardatapy.eos.setup_ffg}}\index{module@\spxentry{module}!nucleardatapy.eos.setup\_ffg@\spxentry{nucleardatapy.eos.setup\_ffg}}\index{nucleardatapy.eos.setup\_ffg@\spxentry{nucleardatapy.eos.setup\_ffg}!module@\spxentry{module}}\index{den() (in module nucleardatapy.eos.setup\_ffg)@\spxentry{den()}\spxextra{in module nucleardatapy.eos.setup\_ffg}}

\begin{fulllineitems}
\phantomsection\label{\detokenize{source/api/setup_eos_ffg:nucleardatapy.eos.setup_ffg.den}}
\pysigstartsignatures
\pysiglinewithargsret{\sphinxcode{\sphinxupquote{nucleardatapy.eos.setup\_ffg.}}\sphinxbfcode{\sphinxupquote{den}}}{\emph{\DUrole{n}{kf}}}{}
\pysigstopsignatures
\sphinxAtStartPar
Density as a function of the Fermi momentum.
\begin{quote}\begin{description}
\sphinxlineitem{Parameters}
\sphinxAtStartPar
\sphinxstyleliteralstrong{\sphinxupquote{kf\_n}} (\sphinxstyleliteralemphasis{\sphinxupquote{float}}\sphinxstyleliteralemphasis{\sphinxupquote{ or }}\sphinxstyleliteralemphasis{\sphinxupquote{numpy vector of real numbers.}}) \textendash{} Fermi momentum.

\end{description}\end{quote}

\end{fulllineitems}

\index{den\_n() (in module nucleardatapy.eos.setup\_ffg)@\spxentry{den\_n()}\spxextra{in module nucleardatapy.eos.setup\_ffg}}

\begin{fulllineitems}
\phantomsection\label{\detokenize{source/api/setup_eos_ffg:nucleardatapy.eos.setup_ffg.den_n}}
\pysigstartsignatures
\pysiglinewithargsret{\sphinxcode{\sphinxupquote{nucleardatapy.eos.setup\_ffg.}}\sphinxbfcode{\sphinxupquote{den\_n}}}{\emph{\DUrole{n}{kf\_n}}}{}
\pysigstopsignatures
\sphinxAtStartPar
Neutron density as a function of the neutron Fermi momentum.
\begin{quote}\begin{description}
\sphinxlineitem{Parameters}
\sphinxAtStartPar
\sphinxstyleliteralstrong{\sphinxupquote{kf\_n}} (\sphinxstyleliteralemphasis{\sphinxupquote{float}}\sphinxstyleliteralemphasis{\sphinxupquote{ or }}\sphinxstyleliteralemphasis{\sphinxupquote{numpy vector of real numbers.}}) \textendash{} neutron Fermi momentum.

\end{description}\end{quote}

\end{fulllineitems}

\index{eF\_n() (in module nucleardatapy.eos.setup\_ffg)@\spxentry{eF\_n()}\spxextra{in module nucleardatapy.eos.setup\_ffg}}

\begin{fulllineitems}
\phantomsection\label{\detokenize{source/api/setup_eos_ffg:nucleardatapy.eos.setup_ffg.eF_n}}
\pysigstartsignatures
\pysiglinewithargsret{\sphinxcode{\sphinxupquote{nucleardatapy.eos.setup\_ffg.}}\sphinxbfcode{\sphinxupquote{eF\_n}}}{\emph{\DUrole{n}{kf\_n}}}{}
\pysigstopsignatures
\sphinxAtStartPar
Neutron Fermi energy as a function of the neutron Fermi momentum.
\begin{quote}\begin{description}
\sphinxlineitem{Parameters}
\sphinxAtStartPar
\sphinxstyleliteralstrong{\sphinxupquote{kf\_n}} (\sphinxstyleliteralemphasis{\sphinxupquote{float}}\sphinxstyleliteralemphasis{\sphinxupquote{ or }}\sphinxstyleliteralemphasis{\sphinxupquote{numpy vector of real numbers.}}) \textendash{} neutron Fermi momentum.

\end{description}\end{quote}

\end{fulllineitems}

\index{eF\_n\_nr() (in module nucleardatapy.eos.setup\_ffg)@\spxentry{eF\_n\_nr()}\spxextra{in module nucleardatapy.eos.setup\_ffg}}

\begin{fulllineitems}
\phantomsection\label{\detokenize{source/api/setup_eos_ffg:nucleardatapy.eos.setup_ffg.eF_n_nr}}
\pysigstartsignatures
\pysiglinewithargsret{\sphinxcode{\sphinxupquote{nucleardatapy.eos.setup\_ffg.}}\sphinxbfcode{\sphinxupquote{eF\_n\_nr}}}{\emph{\DUrole{n}{kf\_n}}}{}
\pysigstopsignatures
\sphinxAtStartPar
Non\sphinxhyphen{}relativistic neutron Fermi energy as a function of the neutron Fermi momentum.
\begin{quote}\begin{description}
\sphinxlineitem{Parameters}
\sphinxAtStartPar
\sphinxstyleliteralstrong{\sphinxupquote{kf\_n}} (\sphinxstyleliteralemphasis{\sphinxupquote{float}}\sphinxstyleliteralemphasis{\sphinxupquote{ or }}\sphinxstyleliteralemphasis{\sphinxupquote{numpy vector of real numbers.}}) \textendash{} neutron Fermi momentum.

\end{description}\end{quote}

\end{fulllineitems}

\index{effg() (in module nucleardatapy.eos.setup\_ffg)@\spxentry{effg()}\spxextra{in module nucleardatapy.eos.setup\_ffg}}

\begin{fulllineitems}
\phantomsection\label{\detokenize{source/api/setup_eos_ffg:nucleardatapy.eos.setup_ffg.effg}}
\pysigstartsignatures
\pysiglinewithargsret{\sphinxcode{\sphinxupquote{nucleardatapy.eos.setup\_ffg.}}\sphinxbfcode{\sphinxupquote{effg}}}{\emph{\DUrole{n}{kf}}}{}
\pysigstopsignatures
\sphinxAtStartPar
Free Fermi gas energy as a function of the Fermi momentum.
\begin{quote}\begin{description}
\sphinxlineitem{Parameters}
\sphinxAtStartPar
\sphinxstyleliteralstrong{\sphinxupquote{kf}} (\sphinxstyleliteralemphasis{\sphinxupquote{float}}\sphinxstyleliteralemphasis{\sphinxupquote{ or }}\sphinxstyleliteralemphasis{\sphinxupquote{numpy vector of real numbers.}}) \textendash{} Fermi momentum.

\end{description}\end{quote}

\end{fulllineitems}

\index{effg\_NM() (in module nucleardatapy.eos.setup\_ffg)@\spxentry{effg\_NM()}\spxextra{in module nucleardatapy.eos.setup\_ffg}}

\begin{fulllineitems}
\phantomsection\label{\detokenize{source/api/setup_eos_ffg:nucleardatapy.eos.setup_ffg.effg_NM}}
\pysigstartsignatures
\pysiglinewithargsret{\sphinxcode{\sphinxupquote{nucleardatapy.eos.setup\_ffg.}}\sphinxbfcode{\sphinxupquote{effg\_NM}}}{\emph{\DUrole{n}{kf\_n}}}{}
\pysigstopsignatures
\sphinxAtStartPar
Free Fermi gas energy as a function of the neutron Fermi momentum.
\begin{quote}\begin{description}
\sphinxlineitem{Parameters}
\sphinxAtStartPar
\sphinxstyleliteralstrong{\sphinxupquote{kf\_n}} (\sphinxstyleliteralemphasis{\sphinxupquote{float}}\sphinxstyleliteralemphasis{\sphinxupquote{ or }}\sphinxstyleliteralemphasis{\sphinxupquote{numpy vector of real numbers.}}) \textendash{} neutron Fermi momentum.

\end{description}\end{quote}

\end{fulllineitems}

\index{effg\_SM() (in module nucleardatapy.eos.setup\_ffg)@\spxentry{effg\_SM()}\spxextra{in module nucleardatapy.eos.setup\_ffg}}

\begin{fulllineitems}
\phantomsection\label{\detokenize{source/api/setup_eos_ffg:nucleardatapy.eos.setup_ffg.effg_SM}}
\pysigstartsignatures
\pysiglinewithargsret{\sphinxcode{\sphinxupquote{nucleardatapy.eos.setup\_ffg.}}\sphinxbfcode{\sphinxupquote{effg\_SM}}}{\emph{\DUrole{n}{kf}}}{}
\pysigstopsignatures
\sphinxAtStartPar
Free Fermi gas energy as a function of the Fermi momentum in SM.
\begin{quote}\begin{description}
\sphinxlineitem{Parameters}
\sphinxAtStartPar
\sphinxstyleliteralstrong{\sphinxupquote{kf}} (\sphinxstyleliteralemphasis{\sphinxupquote{float}}\sphinxstyleliteralemphasis{\sphinxupquote{ or }}\sphinxstyleliteralemphasis{\sphinxupquote{numpy vector of real numbers.}}) \textendash{} neutron Fermi momentum.

\end{description}\end{quote}

\end{fulllineitems}

\index{esymffg() (in module nucleardatapy.eos.setup\_ffg)@\spxentry{esymffg()}\spxextra{in module nucleardatapy.eos.setup\_ffg}}

\begin{fulllineitems}
\phantomsection\label{\detokenize{source/api/setup_eos_ffg:nucleardatapy.eos.setup_ffg.esymffg}}
\pysigstartsignatures
\pysiglinewithargsret{\sphinxcode{\sphinxupquote{nucleardatapy.eos.setup\_ffg.}}\sphinxbfcode{\sphinxupquote{esymffg}}}{\emph{\DUrole{n}{kf}}}{}
\pysigstopsignatures
\sphinxAtStartPar
Free Fermi gas symmetry energy as a function of the Fermi momentum.
\begin{quote}\begin{description}
\sphinxlineitem{Parameters}
\sphinxAtStartPar
\sphinxstyleliteralstrong{\sphinxupquote{kf}} (\sphinxstyleliteralemphasis{\sphinxupquote{float}}\sphinxstyleliteralemphasis{\sphinxupquote{ or }}\sphinxstyleliteralemphasis{\sphinxupquote{numpy vector of real numbers.}}) \textendash{} Fermi momentum.

\end{description}\end{quote}

\end{fulllineitems}

\index{kf() (in module nucleardatapy.eos.setup\_ffg)@\spxentry{kf()}\spxextra{in module nucleardatapy.eos.setup\_ffg}}

\begin{fulllineitems}
\phantomsection\label{\detokenize{source/api/setup_eos_ffg:nucleardatapy.eos.setup_ffg.kf}}
\pysigstartsignatures
\pysiglinewithargsret{\sphinxcode{\sphinxupquote{nucleardatapy.eos.setup\_ffg.}}\sphinxbfcode{\sphinxupquote{kf}}}{\emph{\DUrole{n}{den}}}{}
\pysigstopsignatures
\sphinxAtStartPar
Fermi momentum as a function of the density.
\begin{quote}\begin{description}
\sphinxlineitem{Parameters}
\sphinxAtStartPar
\sphinxstyleliteralstrong{\sphinxupquote{den}} (\sphinxstyleliteralemphasis{\sphinxupquote{float}}\sphinxstyleliteralemphasis{\sphinxupquote{ or }}\sphinxstyleliteralemphasis{\sphinxupquote{numpy vector of real numbers.}}) \textendash{} density.

\end{description}\end{quote}

\end{fulllineitems}

\index{kf\_n() (in module nucleardatapy.eos.setup\_ffg)@\spxentry{kf\_n()}\spxextra{in module nucleardatapy.eos.setup\_ffg}}

\begin{fulllineitems}
\phantomsection\label{\detokenize{source/api/setup_eos_ffg:nucleardatapy.eos.setup_ffg.kf_n}}
\pysigstartsignatures
\pysiglinewithargsret{\sphinxcode{\sphinxupquote{nucleardatapy.eos.setup\_ffg.}}\sphinxbfcode{\sphinxupquote{kf\_n}}}{\emph{\DUrole{n}{den\_n}}}{}
\pysigstopsignatures
\sphinxAtStartPar
Neutron Fermi momentum as a function of the neutron density.
\begin{quote}\begin{description}
\sphinxlineitem{Parameters}
\sphinxAtStartPar
\sphinxstyleliteralstrong{\sphinxupquote{den\_n}} (\sphinxstyleliteralemphasis{\sphinxupquote{float}}\sphinxstyleliteralemphasis{\sphinxupquote{ or }}\sphinxstyleliteralemphasis{\sphinxupquote{numpy vector of real numbers.}}) \textendash{} neutron density.

\end{description}\end{quote}

\end{fulllineitems}

\index{setupFFG (class in nucleardatapy.eos.setup\_ffg)@\spxentry{setupFFG}\spxextra{class in nucleardatapy.eos.setup\_ffg}}

\begin{fulllineitems}
\phantomsection\label{\detokenize{source/api/setup_eos_ffg:nucleardatapy.eos.setup_ffg.setupFFG}}
\pysigstartsignatures
\pysiglinewithargsret{\sphinxbfcode{\sphinxupquote{class\DUrole{w}{  }}}\sphinxcode{\sphinxupquote{nucleardatapy.eos.setup\_ffg.}}\sphinxbfcode{\sphinxupquote{setupFFG}}}{\emph{\DUrole{n}{den}}, \emph{\DUrole{n}{delta}}}{}
\pysigstopsignatures
\sphinxAtStartPar
Instantiate the object with free Fermi gas (FFG) quantities.
\begin{quote}\begin{description}
\sphinxlineitem{Parameters}\begin{itemize}
\item {} 
\sphinxAtStartPar
\sphinxstyleliteralstrong{\sphinxupquote{den}} (\sphinxstyleliteralemphasis{\sphinxupquote{float}}\sphinxstyleliteralemphasis{\sphinxupquote{ or }}\sphinxstyleliteralemphasis{\sphinxupquote{numpy vector of floats.}}) \textendash{} density or densities for which the FFG quantities are calculated.

\item {} 
\sphinxAtStartPar
\sphinxstyleliteralstrong{\sphinxupquote{delta}} (\sphinxstyleliteralemphasis{\sphinxupquote{float}}\sphinxstyleliteralemphasis{\sphinxupquote{ or }}\sphinxstyleliteralemphasis{\sphinxupquote{numpy vector of floats.}}) \textendash{} isospin density or densities for which the FFG quantities are calculated.

\end{itemize}

\end{description}\end{quote}

\sphinxAtStartPar
\sphinxstylestrong{Attributes:}
\index{delta (nucleardatapy.eos.setup\_ffg.setupFFG attribute)@\spxentry{delta}\spxextra{nucleardatapy.eos.setup\_ffg.setupFFG attribute}}

\begin{fulllineitems}
\phantomsection\label{\detokenize{source/api/setup_eos_ffg:nucleardatapy.eos.setup_ffg.setupFFG.delta}}
\pysigstartsignatures
\pysigline{\sphinxbfcode{\sphinxupquote{delta}}}
\pysigstopsignatures
\sphinxAtStartPar
Attribute isospin parameter

\end{fulllineitems}

\index{den (nucleardatapy.eos.setup\_ffg.setupFFG attribute)@\spxentry{den}\spxextra{nucleardatapy.eos.setup\_ffg.setupFFG attribute}}

\begin{fulllineitems}
\phantomsection\label{\detokenize{source/api/setup_eos_ffg:nucleardatapy.eos.setup_ffg.setupFFG.den}}
\pysigstartsignatures
\pysigline{\sphinxbfcode{\sphinxupquote{den}}}
\pysigstopsignatures
\sphinxAtStartPar
Attribute isoscalar density

\end{fulllineitems}

\index{den\_n (nucleardatapy.eos.setup\_ffg.setupFFG attribute)@\spxentry{den\_n}\spxextra{nucleardatapy.eos.setup\_ffg.setupFFG attribute}}

\begin{fulllineitems}
\phantomsection\label{\detokenize{source/api/setup_eos_ffg:nucleardatapy.eos.setup_ffg.setupFFG.den_n}}
\pysigstartsignatures
\pysigline{\sphinxbfcode{\sphinxupquote{den\_n}}}
\pysigstopsignatures
\sphinxAtStartPar
Attribute neutron density

\end{fulllineitems}

\index{den\_p (nucleardatapy.eos.setup\_ffg.setupFFG attribute)@\spxentry{den\_p}\spxextra{nucleardatapy.eos.setup\_ffg.setupFFG attribute}}

\begin{fulllineitems}
\phantomsection\label{\detokenize{source/api/setup_eos_ffg:nucleardatapy.eos.setup_ffg.setupFFG.den_p}}
\pysigstartsignatures
\pysigline{\sphinxbfcode{\sphinxupquote{den\_p}}}
\pysigstopsignatures
\sphinxAtStartPar
Attribute proton density

\end{fulllineitems}

\index{e2a\_int\_nr (nucleardatapy.eos.setup\_ffg.setupFFG attribute)@\spxentry{e2a\_int\_nr}\spxextra{nucleardatapy.eos.setup\_ffg.setupFFG attribute}}

\begin{fulllineitems}
\phantomsection\label{\detokenize{source/api/setup_eos_ffg:nucleardatapy.eos.setup_ffg.setupFFG.e2a_int_nr}}
\pysigstartsignatures
\pysigline{\sphinxbfcode{\sphinxupquote{e2a\_int\_nr}}}
\pysigstopsignatures
\sphinxAtStartPar
Attribute FFG energy per particle

\end{fulllineitems}

\index{e2v\_int\_nr (nucleardatapy.eos.setup\_ffg.setupFFG attribute)@\spxentry{e2v\_int\_nr}\spxextra{nucleardatapy.eos.setup\_ffg.setupFFG attribute}}

\begin{fulllineitems}
\phantomsection\label{\detokenize{source/api/setup_eos_ffg:nucleardatapy.eos.setup_ffg.setupFFG.e2v_int_nr}}
\pysigstartsignatures
\pysigline{\sphinxbfcode{\sphinxupquote{e2v\_int\_nr}}}
\pysigstopsignatures
\sphinxAtStartPar
Attribute FFG energy per unit volum

\end{fulllineitems}

\index{eF\_n (nucleardatapy.eos.setup\_ffg.setupFFG attribute)@\spxentry{eF\_n}\spxextra{nucleardatapy.eos.setup\_ffg.setupFFG attribute}}

\begin{fulllineitems}
\phantomsection\label{\detokenize{source/api/setup_eos_ffg:nucleardatapy.eos.setup_ffg.setupFFG.eF_n}}
\pysigstartsignatures
\pysigline{\sphinxbfcode{\sphinxupquote{eF\_n}}}
\pysigstopsignatures
\sphinxAtStartPar
Attribute neutron Fermi energy

\end{fulllineitems}

\index{eF\_p (nucleardatapy.eos.setup\_ffg.setupFFG attribute)@\spxentry{eF\_p}\spxextra{nucleardatapy.eos.setup\_ffg.setupFFG attribute}}

\begin{fulllineitems}
\phantomsection\label{\detokenize{source/api/setup_eos_ffg:nucleardatapy.eos.setup_ffg.setupFFG.eF_p}}
\pysigstartsignatures
\pysigline{\sphinxbfcode{\sphinxupquote{eF\_p}}}
\pysigstopsignatures
\sphinxAtStartPar
Attribute proton Fermi energy

\end{fulllineitems}

\index{esym2\_nr (nucleardatapy.eos.setup\_ffg.setupFFG attribute)@\spxentry{esym2\_nr}\spxextra{nucleardatapy.eos.setup\_ffg.setupFFG attribute}}

\begin{fulllineitems}
\phantomsection\label{\detokenize{source/api/setup_eos_ffg:nucleardatapy.eos.setup_ffg.setupFFG.esym2_nr}}
\pysigstartsignatures
\pysigline{\sphinxbfcode{\sphinxupquote{esym2\_nr}}}
\pysigstopsignatures
\sphinxAtStartPar
Attribute FFG quadratic contribution to the symmetry energy

\end{fulllineitems}

\index{esym4\_nr (nucleardatapy.eos.setup\_ffg.setupFFG attribute)@\spxentry{esym4\_nr}\spxextra{nucleardatapy.eos.setup\_ffg.setupFFG attribute}}

\begin{fulllineitems}
\phantomsection\label{\detokenize{source/api/setup_eos_ffg:nucleardatapy.eos.setup_ffg.setupFFG.esym4_nr}}
\pysigstartsignatures
\pysigline{\sphinxbfcode{\sphinxupquote{esym4\_nr}}}
\pysigstopsignatures
\sphinxAtStartPar
Attribute FFG quartic contribution to the symmetry energy

\end{fulllineitems}

\index{esym\_nr (nucleardatapy.eos.setup\_ffg.setupFFG attribute)@\spxentry{esym\_nr}\spxextra{nucleardatapy.eos.setup\_ffg.setupFFG attribute}}

\begin{fulllineitems}
\phantomsection\label{\detokenize{source/api/setup_eos_ffg:nucleardatapy.eos.setup_ffg.setupFFG.esym_nr}}
\pysigstartsignatures
\pysigline{\sphinxbfcode{\sphinxupquote{esym\_nr}}}
\pysigstopsignatures
\sphinxAtStartPar
Attribute FFG symmetry energy

\end{fulllineitems}

\index{kf (nucleardatapy.eos.setup\_ffg.setupFFG attribute)@\spxentry{kf}\spxextra{nucleardatapy.eos.setup\_ffg.setupFFG attribute}}

\begin{fulllineitems}
\phantomsection\label{\detokenize{source/api/setup_eos_ffg:nucleardatapy.eos.setup_ffg.setupFFG.kf}}
\pysigstartsignatures
\pysigline{\sphinxbfcode{\sphinxupquote{kf}}}
\pysigstopsignatures
\sphinxAtStartPar
Attribute Fermi momentum

\end{fulllineitems}

\index{kf\_n (nucleardatapy.eos.setup\_ffg.setupFFG attribute)@\spxentry{kf\_n}\spxextra{nucleardatapy.eos.setup\_ffg.setupFFG attribute}}

\begin{fulllineitems}
\phantomsection\label{\detokenize{source/api/setup_eos_ffg:nucleardatapy.eos.setup_ffg.setupFFG.kf_n}}
\pysigstartsignatures
\pysigline{\sphinxbfcode{\sphinxupquote{kf\_n}}}
\pysigstopsignatures
\sphinxAtStartPar
Attribute neutron Fermi momentum

\end{fulllineitems}

\index{kf\_p (nucleardatapy.eos.setup\_ffg.setupFFG attribute)@\spxentry{kf\_p}\spxextra{nucleardatapy.eos.setup\_ffg.setupFFG attribute}}

\begin{fulllineitems}
\phantomsection\label{\detokenize{source/api/setup_eos_ffg:nucleardatapy.eos.setup_ffg.setupFFG.kf_p}}
\pysigstartsignatures
\pysigline{\sphinxbfcode{\sphinxupquote{kf\_p}}}
\pysigstopsignatures
\sphinxAtStartPar
Attribute proton Fermi momentum

\end{fulllineitems}

\index{label (nucleardatapy.eos.setup\_ffg.setupFFG attribute)@\spxentry{label}\spxextra{nucleardatapy.eos.setup\_ffg.setupFFG attribute}}

\begin{fulllineitems}
\phantomsection\label{\detokenize{source/api/setup_eos_ffg:nucleardatapy.eos.setup_ffg.setupFFG.label}}
\pysigstartsignatures
\pysigline{\sphinxbfcode{\sphinxupquote{label}}}
\pysigstopsignatures
\sphinxAtStartPar
Attribute providing the label the data is references for figures.

\end{fulllineitems}

\index{note (nucleardatapy.eos.setup\_ffg.setupFFG attribute)@\spxentry{note}\spxextra{nucleardatapy.eos.setup\_ffg.setupFFG attribute}}

\begin{fulllineitems}
\phantomsection\label{\detokenize{source/api/setup_eos_ffg:nucleardatapy.eos.setup_ffg.setupFFG.note}}
\pysigstartsignatures
\pysigline{\sphinxbfcode{\sphinxupquote{note}}}
\pysigstopsignatures
\sphinxAtStartPar
Attribute providing additional notes about the data.

\end{fulllineitems}

\index{pre\_nr (nucleardatapy.eos.setup\_ffg.setupFFG attribute)@\spxentry{pre\_nr}\spxextra{nucleardatapy.eos.setup\_ffg.setupFFG attribute}}

\begin{fulllineitems}
\phantomsection\label{\detokenize{source/api/setup_eos_ffg:nucleardatapy.eos.setup_ffg.setupFFG.pre_nr}}
\pysigstartsignatures
\pysigline{\sphinxbfcode{\sphinxupquote{pre\_nr}}}
\pysigstopsignatures
\sphinxAtStartPar
Attribute FFG pressure

\end{fulllineitems}

\index{print\_outputs() (nucleardatapy.eos.setup\_ffg.setupFFG method)@\spxentry{print\_outputs()}\spxextra{nucleardatapy.eos.setup\_ffg.setupFFG method}}

\begin{fulllineitems}
\phantomsection\label{\detokenize{source/api/setup_eos_ffg:nucleardatapy.eos.setup_ffg.setupFFG.print_outputs}}
\pysigstartsignatures
\pysiglinewithargsret{\sphinxbfcode{\sphinxupquote{print\_outputs}}}{}{}
\pysigstopsignatures
\sphinxAtStartPar
Method which print outputs on terminal’s screen.

\end{fulllineitems}


\end{fulllineitems}


\sphinxAtStartPar
Here are a set of figures which are produced with the Python sample: /nucleardatapy\_sample/plots/plot\_eos\_setupFFG.py

\begin{figure}[htbp]
\centering
\capstart

\noindent\sphinxincludegraphics[scale=0.7]{{plot_eos_setupFFG}.png}
\caption{This figure shows the free Fermi gas energy (top) and pressure (bottom) in symmetric matter (SM) (Blue solid line) and neutron matter (NM) (orange dashed line) as function of the particle density (left) and Fermi momentum (right).}\label{\detokenize{source/api/setup_eos_ffg:id1}}\end{figure}

\sphinxstepscope


\section{SetupEOSMicro}
\label{\detokenize{source/api/setup_eos_micro:setupeosmicro}}\label{\detokenize{source/api/setup_eos_micro::doc}}\phantomsection\label{\detokenize{source/api/setup_eos_micro:module-nucleardatapy.eos.setup_micro}}\index{module@\spxentry{module}!nucleardatapy.eos.setup\_micro@\spxentry{nucleardatapy.eos.setup\_micro}}\index{nucleardatapy.eos.setup\_micro@\spxentry{nucleardatapy.eos.setup\_micro}!module@\spxentry{module}}\index{micro\_models() (in module nucleardatapy.eos.setup\_micro)@\spxentry{micro\_models()}\spxextra{in module nucleardatapy.eos.setup\_micro}}

\begin{fulllineitems}
\phantomsection\label{\detokenize{source/api/setup_eos_micro:nucleardatapy.eos.setup_micro.micro_models}}
\pysigstartsignatures
\pysiglinewithargsret{\sphinxcode{\sphinxupquote{nucleardatapy.eos.setup\_micro.}}\sphinxbfcode{\sphinxupquote{micro\_models}}}{}{}
\pysigstopsignatures
\sphinxAtStartPar
Return a list with the name of the models available in this toolkit and     print them all on the prompt. These models are the following ones:     ‘1981\sphinxhyphen{}VAR\sphinxhyphen{}AM\sphinxhyphen{}FP’, ‘1998\sphinxhyphen{}VAR\sphinxhyphen{}AM\sphinxhyphen{}APR’, ‘1998\sphinxhyphen{}VAR\sphinxhyphen{}AM\sphinxhyphen{}APRfit’, ‘2006\sphinxhyphen{}BHF\sphinxhyphen{}AM*’,     ‘2012\sphinxhyphen{}AFDMC\sphinxhyphen{}NM\sphinxhyphen{}1’, ‘2012\sphinxhyphen{}AFDMC\sphinxhyphen{}NM\sphinxhyphen{}2’, ‘2012\sphinxhyphen{}AFDMC\sphinxhyphen{}NM\sphinxhyphen{}3’, ‘2012\sphinxhyphen{}AFDMC\sphinxhyphen{}NM\sphinxhyphen{}4’,     ‘2012\sphinxhyphen{}AFDMC\sphinxhyphen{}NM\sphinxhyphen{}5’, ‘2012\sphinxhyphen{}AFDMC\sphinxhyphen{}NM\sphinxhyphen{}6’, ‘2012\sphinxhyphen{}AFDMC\sphinxhyphen{}NM\sphinxhyphen{}7’,     ‘2008\sphinxhyphen{}QMC\sphinxhyphen{}NM\sphinxhyphen{}swave’, ‘2010\sphinxhyphen{}QMC\sphinxhyphen{}NM\sphinxhyphen{}AV4’, ‘2009\sphinxhyphen{}DLQMC\sphinxhyphen{}NM’, ‘2010\sphinxhyphen{}MBPT\sphinxhyphen{}NM’,     ‘2013\sphinxhyphen{}QMC\sphinxhyphen{}NM’, ‘2014\sphinxhyphen{}AFQMC\sphinxhyphen{}NM’, ‘2016\sphinxhyphen{}QMC\sphinxhyphen{}NM’, ‘2016\sphinxhyphen{}MBPT\sphinxhyphen{}AM’,     ‘2018\sphinxhyphen{}QMC\sphinxhyphen{}NM’, ‘2019\sphinxhyphen{}MBPT\sphinxhyphen{}AM\sphinxhyphen{}L59’, ‘2019\sphinxhyphen{}MBPT\sphinxhyphen{}AM\sphinxhyphen{}L69’,     ‘2020\sphinxhyphen{}MBPT\sphinxhyphen{}AM’, ‘2022\sphinxhyphen{}AFDMC\sphinxhyphen{}NM’, ‘2024\sphinxhyphen{}NLEFT\sphinxhyphen{}AM’,     ‘2024\sphinxhyphen{}BHF\sphinxhyphen{}AM\sphinxhyphen{}2BF\sphinxhyphen{}Av8p’, ‘2024\sphinxhyphen{}BHF\sphinxhyphen{}AM\sphinxhyphen{}2BF\sphinxhyphen{}Av18’, ‘2024\sphinxhyphen{}BHF\sphinxhyphen{}AM\sphinxhyphen{}2BF\sphinxhyphen{}BONN’, ‘2024\sphinxhyphen{}BHF\sphinxhyphen{}AM\sphinxhyphen{}2BF\sphinxhyphen{}CDBONN’,     ‘2024\sphinxhyphen{}BHF\sphinxhyphen{}AM\sphinxhyphen{}2BF\sphinxhyphen{}NSC97a’, ‘2024\sphinxhyphen{}BHF\sphinxhyphen{}AM\sphinxhyphen{}2BF\sphinxhyphen{}NSC97b’, ‘2024\sphinxhyphen{}BHF\sphinxhyphen{}AM\sphinxhyphen{}2BF\sphinxhyphen{}NSC97c’, ‘2024\sphinxhyphen{}BHF\sphinxhyphen{}AM\sphinxhyphen{}2BF\sphinxhyphen{}NSC97d’,     ‘2024\sphinxhyphen{}BHF\sphinxhyphen{}AM\sphinxhyphen{}2BF\sphinxhyphen{}NSC97e’, ‘2024\sphinxhyphen{}BHF\sphinxhyphen{}AM\sphinxhyphen{}2BF\sphinxhyphen{}NSC97f’, ‘2024\sphinxhyphen{}BHF\sphinxhyphen{}AM\sphinxhyphen{}2BF\sphinxhyphen{}SSCV14’,    ‘2024\sphinxhyphen{}BHF\sphinxhyphen{}AM\sphinxhyphen{}23BF\sphinxhyphen{}Av8p’, ‘2024\sphinxhyphen{}BHF\sphinxhyphen{}AM\sphinxhyphen{}23BF\sphinxhyphen{}Av18’, ‘2024\sphinxhyphen{}BHF\sphinxhyphen{}AM\sphinxhyphen{}23BF\sphinxhyphen{}BONN’, ‘2024\sphinxhyphen{}BHF\sphinxhyphen{}AM\sphinxhyphen{}23BF\sphinxhyphen{}CDBONN’,     ‘2024\sphinxhyphen{}BHF\sphinxhyphen{}AM\sphinxhyphen{}23BF\sphinxhyphen{}NSC97a’, ‘2024\sphinxhyphen{}BHF\sphinxhyphen{}AM\sphinxhyphen{}23BF\sphinxhyphen{}NSC97b’, ‘2024\sphinxhyphen{}BHF\sphinxhyphen{}AM\sphinxhyphen{}23BF\sphinxhyphen{}NSC97c’, ‘2024\sphinxhyphen{}BHF\sphinxhyphen{}AM\sphinxhyphen{}23BF\sphinxhyphen{}NSC97d’,     ‘2024\sphinxhyphen{}BHF\sphinxhyphen{}AM\sphinxhyphen{}23BF\sphinxhyphen{}NSC97e’, ‘2024\sphinxhyphen{}BHF\sphinxhyphen{}AM\sphinxhyphen{}23BF\sphinxhyphen{}NSC97f’, ‘2024\sphinxhyphen{}BHF\sphinxhyphen{}AM\sphinxhyphen{}23BF\sphinxhyphen{}SSCV14’,    ‘2024\sphinxhyphen{}BHF\sphinxhyphen{}AM\sphinxhyphen{}23BFmicro\sphinxhyphen{}Av18’, ‘2024\sphinxhyphen{}BHF\sphinxhyphen{}AM\sphinxhyphen{}23BFmicro\sphinxhyphen{}BONNB’, ‘2024\sphinxhyphen{}BHF\sphinxhyphen{}AM\sphinxhyphen{}23BFmicro\sphinxhyphen{}NSC93’,    ‘2024\sphinxhyphen{}DMC\sphinxhyphen{}NM’
\begin{quote}\begin{description}
\sphinxlineitem{Returns}
\sphinxAtStartPar
The list of models.

\sphinxlineitem{Return type}
\sphinxAtStartPar
list{[}str{]}.

\end{description}\end{quote}

\end{fulllineitems}

\index{micro\_models\_group\_NM() (in module nucleardatapy.eos.setup\_micro)@\spxentry{micro\_models\_group\_NM()}\spxextra{in module nucleardatapy.eos.setup\_micro}}

\begin{fulllineitems}
\phantomsection\label{\detokenize{source/api/setup_eos_micro:nucleardatapy.eos.setup_micro.micro_models_group_NM}}
\pysigstartsignatures
\pysiglinewithargsret{\sphinxcode{\sphinxupquote{nucleardatapy.eos.setup\_micro.}}\sphinxbfcode{\sphinxupquote{micro\_models\_group\_NM}}}{\emph{\DUrole{n}{group}}}{}
\pysigstopsignatures
\end{fulllineitems}

\index{micro\_models\_group\_SM() (in module nucleardatapy.eos.setup\_micro)@\spxentry{micro\_models\_group\_SM()}\spxextra{in module nucleardatapy.eos.setup\_micro}}

\begin{fulllineitems}
\phantomsection\label{\detokenize{source/api/setup_eos_micro:nucleardatapy.eos.setup_micro.micro_models_group_SM}}
\pysigstartsignatures
\pysiglinewithargsret{\sphinxcode{\sphinxupquote{nucleardatapy.eos.setup\_micro.}}\sphinxbfcode{\sphinxupquote{micro\_models\_group\_SM}}}{\emph{\DUrole{n}{group}}}{}
\pysigstopsignatures
\end{fulllineitems}

\index{setupMicro (class in nucleardatapy.eos.setup\_micro)@\spxentry{setupMicro}\spxextra{class in nucleardatapy.eos.setup\_micro}}

\begin{fulllineitems}
\phantomsection\label{\detokenize{source/api/setup_eos_micro:nucleardatapy.eos.setup_micro.setupMicro}}
\pysigstartsignatures
\pysiglinewithargsret{\sphinxbfcode{\sphinxupquote{class\DUrole{w}{  }}}\sphinxcode{\sphinxupquote{nucleardatapy.eos.setup\_micro.}}\sphinxbfcode{\sphinxupquote{setupMicro}}}{\emph{\DUrole{n}{model}\DUrole{o}{=}\DUrole{default_value}{\textquotesingle{}1998\sphinxhyphen{}VAR\sphinxhyphen{}AM\sphinxhyphen{}APR\textquotesingle{}}}, \emph{\DUrole{n}{var1}\DUrole{o}{=}\DUrole{default_value}{array({[}0.01, 0.03052632, 0.05105263, 0.07157895, 0.09210526, 0.11263158, 0.13315789, 0.15368421, 0.17421053, 0.19473684, 0.21526316, 0.23578947, 0.25631579, 0.27684211, 0.29736842, 0.31789474, 0.33842105, 0.35894737, 0.37947368, 0.4{]})}}, \emph{\DUrole{n}{var2}\DUrole{o}{=}\DUrole{default_value}{0.0}}}{}
\pysigstopsignatures
\sphinxAtStartPar
Instantiate the object with microscopic results choosen     by the toolkit practitioner.

\sphinxAtStartPar
This choice is defined in \sphinxtitleref{model}, which can chosen among     the following choices:     ‘1981\sphinxhyphen{}VAR\sphinxhyphen{}AM\sphinxhyphen{}FP’, ‘1998\sphinxhyphen{}VAR\sphinxhyphen{}AM\sphinxhyphen{}APR’, ‘1998\sphinxhyphen{}VAR\sphinxhyphen{}AM\sphinxhyphen{}iAPR’, ‘2006\sphinxhyphen{}BHF\sphinxhyphen{}AM*’,     ‘2008\sphinxhyphen{}QMC\sphinxhyphen{}NM\sphinxhyphen{}swave’, ‘2010\sphinxhyphen{}QMC\sphinxhyphen{}NM\sphinxhyphen{}AV4’, ‘2009\sphinxhyphen{}DLQMC\sphinxhyphen{}NM’, ‘2010\sphinxhyphen{}MBPT\sphinxhyphen{}NM’,     ‘2012\sphinxhyphen{}AFDMC\sphinxhyphen{}NM\sphinxhyphen{}1’, ‘2012\sphinxhyphen{}AFDMC\sphinxhyphen{}NM\sphinxhyphen{}2’, ‘2012\sphinxhyphen{}AFDMC\sphinxhyphen{}NM\sphinxhyphen{}3’, ‘2012\sphinxhyphen{}AFDMC\sphinxhyphen{}NM\sphinxhyphen{}4’,     ‘2012\sphinxhyphen{}AFDMC\sphinxhyphen{}NM\sphinxhyphen{}5’, ‘2012\sphinxhyphen{}AFDMC\sphinxhyphen{}NM\sphinxhyphen{}6’, ‘2012\sphinxhyphen{}AFDMC\sphinxhyphen{}NM\sphinxhyphen{}7’,     ‘2013\sphinxhyphen{}QMC\sphinxhyphen{}NM’, ‘2014\sphinxhyphen{}AFQMC\sphinxhyphen{}NM’, ‘2016\sphinxhyphen{}QMC\sphinxhyphen{}NM’, ‘2016\sphinxhyphen{}MBPT\sphinxhyphen{}AM’,     ‘2018\sphinxhyphen{}QMC\sphinxhyphen{}NM’, ‘2019\sphinxhyphen{}MBPT\sphinxhyphen{}AM\sphinxhyphen{}L59’, ‘2019\sphinxhyphen{}MBPT\sphinxhyphen{}AM\sphinxhyphen{}L69’,     ‘2020\sphinxhyphen{}MBPT\sphinxhyphen{}AM’, ‘2022\sphinxhyphen{}AFDMC\sphinxhyphen{}NM’, ‘2024\sphinxhyphen{}NLEFT\sphinxhyphen{}AM’,     ‘2024\sphinxhyphen{}BHF\sphinxhyphen{}AM\sphinxhyphen{}2BF\sphinxhyphen{}Av8p’, ‘2024\sphinxhyphen{}BHF\sphinxhyphen{}AM\sphinxhyphen{}2BF\sphinxhyphen{}Av18’, ‘2024\sphinxhyphen{}BHF\sphinxhyphen{}AM\sphinxhyphen{}2BF\sphinxhyphen{}BONN’, ‘2024\sphinxhyphen{}BHF\sphinxhyphen{}AM\sphinxhyphen{}2BF\sphinxhyphen{}CDBONN’,     ‘2024\sphinxhyphen{}BHF\sphinxhyphen{}AM\sphinxhyphen{}2BF\sphinxhyphen{}NSC97a’, ‘2024\sphinxhyphen{}BHF\sphinxhyphen{}AM\sphinxhyphen{}2BF\sphinxhyphen{}NSC97b’, ‘2024\sphinxhyphen{}BHF\sphinxhyphen{}AM\sphinxhyphen{}2BF\sphinxhyphen{}NSC97c’, ‘2024\sphinxhyphen{}BHF\sphinxhyphen{}AM\sphinxhyphen{}2BF\sphinxhyphen{}NSC97d’,     ‘2024\sphinxhyphen{}BHF\sphinxhyphen{}AM\sphinxhyphen{}2BF\sphinxhyphen{}NSC97e’, ‘2024\sphinxhyphen{}BHF\sphinxhyphen{}AM\sphinxhyphen{}2BF\sphinxhyphen{}NSC97f’, ‘2024\sphinxhyphen{}BHF\sphinxhyphen{}AM\sphinxhyphen{}2BF\sphinxhyphen{}SSCV14’,     ‘2024\sphinxhyphen{}BHF\sphinxhyphen{}AM\sphinxhyphen{}23BF\sphinxhyphen{}Av8p’, ‘2024\sphinxhyphen{}BHF\sphinxhyphen{}AM\sphinxhyphen{}23BF\sphinxhyphen{}Av18’, ‘2024\sphinxhyphen{}BHF\sphinxhyphen{}AM\sphinxhyphen{}23BF\sphinxhyphen{}BONN’, ‘2024\sphinxhyphen{}BHF\sphinxhyphen{}AM\sphinxhyphen{}23BF\sphinxhyphen{}CDBONN’,     ‘2024\sphinxhyphen{}BHF\sphinxhyphen{}AM\sphinxhyphen{}23BF\sphinxhyphen{}NSC97a’, ‘2024\sphinxhyphen{}BHF\sphinxhyphen{}AM\sphinxhyphen{}23BF\sphinxhyphen{}NSC97b’, ‘2024\sphinxhyphen{}BHF\sphinxhyphen{}AM\sphinxhyphen{}23BF\sphinxhyphen{}NSC97c’, ‘2024\sphinxhyphen{}BHF\sphinxhyphen{}AM\sphinxhyphen{}23BF\sphinxhyphen{}NSC97d’,     ‘2024\sphinxhyphen{}BHF\sphinxhyphen{}AM\sphinxhyphen{}23BF\sphinxhyphen{}NSC97e’, ‘2024\sphinxhyphen{}BHF\sphinxhyphen{}AM\sphinxhyphen{}23BF\sphinxhyphen{}NSC97f’, ‘2024\sphinxhyphen{}BHF\sphinxhyphen{}AM\sphinxhyphen{}23BF\sphinxhyphen{}SSCV14’, ‘2025\sphinxhyphen{}DMC\sphinxhyphen{}NM’
\begin{quote}\begin{description}
\sphinxlineitem{Parameters}
\sphinxAtStartPar
\sphinxstyleliteralstrong{\sphinxupquote{model}} (\sphinxstyleliteralemphasis{\sphinxupquote{str}}\sphinxstyleliteralemphasis{\sphinxupquote{, }}\sphinxstyleliteralemphasis{\sphinxupquote{optional.}}) \textendash{} Fix the name of model. Default value: ‘1998\sphinxhyphen{}VAR\sphinxhyphen{}AM\sphinxhyphen{}APR’.

\end{description}\end{quote}

\sphinxAtStartPar
\sphinxstylestrong{Attributes:}
\index{init\_self() (nucleardatapy.eos.setup\_micro.setupMicro method)@\spxentry{init\_self()}\spxextra{nucleardatapy.eos.setup\_micro.setupMicro method}}

\begin{fulllineitems}
\phantomsection\label{\detokenize{source/api/setup_eos_micro:nucleardatapy.eos.setup_micro.setupMicro.init_self}}
\pysigstartsignatures
\pysiglinewithargsret{\sphinxbfcode{\sphinxupquote{init\_self}}}{}{}
\pysigstopsignatures
\sphinxAtStartPar
Initialize variables in self.

\end{fulllineitems}

\index{model (nucleardatapy.eos.setup\_micro.setupMicro attribute)@\spxentry{model}\spxextra{nucleardatapy.eos.setup\_micro.setupMicro attribute}}

\begin{fulllineitems}
\phantomsection\label{\detokenize{source/api/setup_eos_micro:nucleardatapy.eos.setup_micro.setupMicro.model}}
\pysigstartsignatures
\pysigline{\sphinxbfcode{\sphinxupquote{model}}}
\pysigstopsignatures
\sphinxAtStartPar
Attribute model.

\end{fulllineitems}

\index{print\_outputs() (nucleardatapy.eos.setup\_micro.setupMicro method)@\spxentry{print\_outputs()}\spxextra{nucleardatapy.eos.setup\_micro.setupMicro method}}

\begin{fulllineitems}
\phantomsection\label{\detokenize{source/api/setup_eos_micro:nucleardatapy.eos.setup_micro.setupMicro.print_outputs}}
\pysigstartsignatures
\pysiglinewithargsret{\sphinxbfcode{\sphinxupquote{print\_outputs}}}{}{}
\pysigstopsignatures
\sphinxAtStartPar
Method which print outputs on terminal’s screen.

\end{fulllineitems}


\end{fulllineitems}


\sphinxAtStartPar
Here are a set of figures which are produced with the Python sample: /nucleardatapy\_sample/plots/plot\_eos\_setupMicro.py

\begin{figure}[htbp]
\centering
\capstart

\noindent\sphinxincludegraphics[scale=0.7]{{plot_eos_setupMicro_e2a_NM_VAR}.png}
\caption{This figure shows the energy in neutron matter (NM) over the free Fermi gas energy (top) and the energy per particle (bottom) as function of the density (left) and the neutron Fermi momentum (right) for the variational models available in the nucleardatapy toolkit.}\label{\detokenize{source/api/setup_eos_micro:id1}}\end{figure}

\begin{figure}[htbp]
\centering
\capstart

\noindent\sphinxincludegraphics[scale=0.7]{{plot_eos_setupMicro_e2a_NM_AFDMC}.png}
\caption{This figure shows the energy in neutron matter (NM) over the free Fermi gas energy (top) and the energy per particle (bottom) as function of the density (left) and the neutron Fermi momentum (right) for the AFDMC models available in the nucleardatapy toolkit.}\label{\detokenize{source/api/setup_eos_micro:id2}}\end{figure}

\begin{figure}[htbp]
\centering
\capstart

\noindent\sphinxincludegraphics[scale=0.7]{{plot_eos_setupMicro_e2a_NM_BHF}.png}
\caption{This figure shows the energy in neutron matter (NM) over the free Fermi gas energy (top) and the energy per particle (bottom) as function of the density (left) and the neutron Fermi momentum (right) for the BHF models available in the nucleardatapy toolkit.}\label{\detokenize{source/api/setup_eos_micro:id3}}\end{figure}

\begin{figure}[htbp]
\centering
\capstart

\noindent\sphinxincludegraphics[scale=0.7]{{plot_eos_setupMicro_e2a_NM_QMC}.png}
\caption{This figure shows the energy in neutron matter (NM) over the free Fermi gas energy (top) and the energy per particle (bottom) as function of the density (left) and the neutron Fermi momentum (right) for the QMC models available in the nucleardatapy toolkit.}\label{\detokenize{source/api/setup_eos_micro:id4}}\end{figure}

\begin{figure}[htbp]
\centering
\capstart

\noindent\sphinxincludegraphics[scale=0.7]{{plot_eos_setupMicro_e2a_NM_MBPT}.png}
\caption{This figure shows the energy in neutron matter (NM) over the free Fermi gas energy (top) and the energy per particle (bottom) as function of the density (left) and the neutron Fermi momentum (right) for the MBPT models available in the nucleardatapy toolkit.}\label{\detokenize{source/api/setup_eos_micro:id5}}\end{figure}

\sphinxstepscope


\section{SetupEOSMicroBand}
\label{\detokenize{source/api/setup_eos_micro_band:setupeosmicroband}}\label{\detokenize{source/api/setup_eos_micro_band::doc}}\phantomsection\label{\detokenize{source/api/setup_eos_micro_band:module-nucleardatapy.eos.setup_micro_band}}\index{module@\spxentry{module}!nucleardatapy.eos.setup\_micro\_band@\spxentry{nucleardatapy.eos.setup\_micro\_band}}\index{nucleardatapy.eos.setup\_micro\_band@\spxentry{nucleardatapy.eos.setup\_micro\_band}!module@\spxentry{module}}\index{setupMicroBand (class in nucleardatapy.eos.setup\_micro\_band)@\spxentry{setupMicroBand}\spxextra{class in nucleardatapy.eos.setup\_micro\_band}}

\begin{fulllineitems}
\phantomsection\label{\detokenize{source/api/setup_eos_micro_band:nucleardatapy.eos.setup_micro_band.setupMicroBand}}
\pysigstartsignatures
\pysiglinewithargsret{\sphinxbfcode{\sphinxupquote{class\DUrole{w}{  }}}\sphinxcode{\sphinxupquote{nucleardatapy.eos.setup\_micro\_band.}}\sphinxbfcode{\sphinxupquote{setupMicroBand}}}{\emph{\DUrole{n}{models}\DUrole{o}{=}\DUrole{default_value}{{[}\textquotesingle{}2016\sphinxhyphen{}MBPT\sphinxhyphen{}AM\textquotesingle{}{]}}}, \emph{\DUrole{n}{nden}\DUrole{o}{=}\DUrole{default_value}{10}}, \emph{\DUrole{n}{ne}\DUrole{o}{=}\DUrole{default_value}{200}}, \emph{\DUrole{n}{den}\DUrole{o}{=}\DUrole{default_value}{None}}, \emph{\DUrole{n}{matter}\DUrole{o}{=}\DUrole{default_value}{\textquotesingle{}NM\textquotesingle{}}}, \emph{\DUrole{n}{e2a\_min}\DUrole{o}{=}\DUrole{default_value}{\sphinxhyphen{} 20.0}}, \emph{\DUrole{n}{e2a\_max}\DUrole{o}{=}\DUrole{default_value}{50.0}}}{}
\pysigstopsignatures
\sphinxAtStartPar
Instantiate the object with statistical distributions averaging over
the models given as inputs and in NM.
\begin{quote}\begin{description}
\sphinxlineitem{Parameters}\begin{itemize}
\item {} 
\sphinxAtStartPar
\sphinxstyleliteralstrong{\sphinxupquote{models}} (\sphinxstyleliteralemphasis{\sphinxupquote{list.}}) \textendash{} The models given as inputs.

\item {} 
\sphinxAtStartPar
\sphinxstyleliteralstrong{\sphinxupquote{nden}} (\sphinxstyleliteralemphasis{\sphinxupquote{int}}\sphinxstyleliteralemphasis{\sphinxupquote{, }}\sphinxstyleliteralemphasis{\sphinxupquote{optional.}}) \textendash{} number of density points.

\item {} 
\sphinxAtStartPar
\sphinxstyleliteralstrong{\sphinxupquote{ne}} (\sphinxstyleliteralemphasis{\sphinxupquote{int}}\sphinxstyleliteralemphasis{\sphinxupquote{, }}\sphinxstyleliteralemphasis{\sphinxupquote{optional.}}) \textendash{} number of points along the energy axis.

\item {} 
\sphinxAtStartPar
\sphinxstyleliteralstrong{\sphinxupquote{den}} (\sphinxstyleliteralemphasis{\sphinxupquote{None}}\sphinxstyleliteralemphasis{\sphinxupquote{ or }}\sphinxstyleliteralemphasis{\sphinxupquote{numpy array}}\sphinxstyleliteralemphasis{\sphinxupquote{, }}\sphinxstyleliteralemphasis{\sphinxupquote{optional.}}) \textendash{} if not None (default), impose the densities.

\item {} 
\sphinxAtStartPar
\sphinxstyleliteralstrong{\sphinxupquote{matter}} (\sphinxstyleliteralemphasis{\sphinxupquote{str}}\sphinxstyleliteralemphasis{\sphinxupquote{, }}\sphinxstyleliteralemphasis{\sphinxupquote{optional.}}) \textendash{} can be ‘NM’ (default), ‘SM’ or ‘ESYM’.

\end{itemize}

\end{description}\end{quote}

\sphinxAtStartPar
\sphinxstylestrong{Attributes:}
\index{den (nucleardatapy.eos.setup\_micro\_band.setupMicroBand attribute)@\spxentry{den}\spxextra{nucleardatapy.eos.setup\_micro\_band.setupMicroBand attribute}}

\begin{fulllineitems}
\phantomsection\label{\detokenize{source/api/setup_eos_micro_band:nucleardatapy.eos.setup_micro_band.setupMicroBand.den}}
\pysigstartsignatures
\pysigline{\sphinxbfcode{\sphinxupquote{den}}}
\pysigstopsignatures
\sphinxAtStartPar
Attribute a set of density points.

\end{fulllineitems}

\index{init\_self() (nucleardatapy.eos.setup\_micro\_band.setupMicroBand method)@\spxentry{init\_self()}\spxextra{nucleardatapy.eos.setup\_micro\_band.setupMicroBand method}}

\begin{fulllineitems}
\phantomsection\label{\detokenize{source/api/setup_eos_micro_band:nucleardatapy.eos.setup_micro_band.setupMicroBand.init_self}}
\pysigstartsignatures
\pysiglinewithargsret{\sphinxbfcode{\sphinxupquote{init\_self}}}{}{}
\pysigstopsignatures
\sphinxAtStartPar
Initialize variables in self.

\end{fulllineitems}

\index{matter (nucleardatapy.eos.setup\_micro\_band.setupMicroBand attribute)@\spxentry{matter}\spxextra{nucleardatapy.eos.setup\_micro\_band.setupMicroBand attribute}}

\begin{fulllineitems}
\phantomsection\label{\detokenize{source/api/setup_eos_micro_band:nucleardatapy.eos.setup_micro_band.setupMicroBand.matter}}
\pysigstartsignatures
\pysigline{\sphinxbfcode{\sphinxupquote{matter}}}
\pysigstopsignatures
\sphinxAtStartPar
Attribute matter str.

\end{fulllineitems}

\index{models (nucleardatapy.eos.setup\_micro\_band.setupMicroBand attribute)@\spxentry{models}\spxextra{nucleardatapy.eos.setup\_micro\_band.setupMicroBand attribute}}

\begin{fulllineitems}
\phantomsection\label{\detokenize{source/api/setup_eos_micro_band:nucleardatapy.eos.setup_micro_band.setupMicroBand.models}}
\pysigstartsignatures
\pysigline{\sphinxbfcode{\sphinxupquote{models}}}
\pysigstopsignatures
\sphinxAtStartPar
Attribute model.

\end{fulllineitems}

\index{nden (nucleardatapy.eos.setup\_micro\_band.setupMicroBand attribute)@\spxentry{nden}\spxextra{nucleardatapy.eos.setup\_micro\_band.setupMicroBand attribute}}

\begin{fulllineitems}
\phantomsection\label{\detokenize{source/api/setup_eos_micro_band:nucleardatapy.eos.setup_micro_band.setupMicroBand.nden}}
\pysigstartsignatures
\pysigline{\sphinxbfcode{\sphinxupquote{nden}}}
\pysigstopsignatures
\sphinxAtStartPar
Attribute number of points in density.

\end{fulllineitems}

\index{print\_outputs() (nucleardatapy.eos.setup\_micro\_band.setupMicroBand method)@\spxentry{print\_outputs()}\spxextra{nucleardatapy.eos.setup\_micro\_band.setupMicroBand method}}

\begin{fulllineitems}
\phantomsection\label{\detokenize{source/api/setup_eos_micro_band:nucleardatapy.eos.setup_micro_band.setupMicroBand.print_outputs}}
\pysigstartsignatures
\pysiglinewithargsret{\sphinxbfcode{\sphinxupquote{print\_outputs}}}{}{}
\pysigstopsignatures
\sphinxAtStartPar
Method which print outputs on terminal’s screen.

\end{fulllineitems}


\end{fulllineitems}


\sphinxAtStartPar
Here are a set of figures which are produced with the Python sample: /nucleardatapy\_sample/plots/plot\_eos\_setupMicroBand.py

\begin{figure}[htbp]
\centering
\capstart

\noindent\sphinxincludegraphics[scale=0.7]{{plot_eos_setupMicroBand_NM}.png}
\caption{Uncertainty band in NM obtained from the analysis of different predictions: MBPT\sphinxhyphen{}2016, QMC\sphinxhyphen{}2016 and MBPT\sphinxhyphen{}2020.}\label{\detokenize{source/api/setup_eos_micro_band:id1}}\end{figure}

\begin{figure}[htbp]
\centering
\capstart

\noindent\sphinxincludegraphics[scale=0.7]{{plot_eos_setupMicroBand_SM}.png}
\caption{Uncertainty band in SM obtained from the analysis of different predictions: MBPT\sphinxhyphen{}2016 and MBPT\sphinxhyphen{}2020.}\label{\detokenize{source/api/setup_eos_micro_band:id2}}\end{figure}

\begin{figure}[htbp]
\centering
\capstart

\noindent\sphinxincludegraphics[scale=0.7]{{plot_eos_setupMicroBand_Esym}.png}
\caption{Uncertainty band for the symmetry energy obtained from the analysis of different predictions: MBPT\sphinxhyphen{}2016 and MBPT\sphinxhyphen{}2020.}\label{\detokenize{source/api/setup_eos_micro_band:id3}}\end{figure}

\sphinxstepscope


\section{SetupEOSMicroLP}
\label{\detokenize{source/api/setup_eos_micro_lp:setupeosmicrolp}}\label{\detokenize{source/api/setup_eos_micro_lp::doc}}\phantomsection\label{\detokenize{source/api/setup_eos_micro_lp:module-nucleardatapy.eos.setup_micro_lp}}\index{module@\spxentry{module}!nucleardatapy.eos.setup\_micro\_lp@\spxentry{nucleardatapy.eos.setup\_micro\_lp}}\index{nucleardatapy.eos.setup\_micro\_lp@\spxentry{nucleardatapy.eos.setup\_micro\_lp}!module@\spxentry{module}}\index{micro\_LP\_models() (in module nucleardatapy.eos.setup\_micro\_lp)@\spxentry{micro\_LP\_models()}\spxextra{in module nucleardatapy.eos.setup\_micro\_lp}}

\begin{fulllineitems}
\phantomsection\label{\detokenize{source/api/setup_eos_micro_lp:nucleardatapy.eos.setup_micro_lp.micro_LP_models}}
\pysigstartsignatures
\pysiglinewithargsret{\sphinxcode{\sphinxupquote{nucleardatapy.eos.setup\_micro\_lp.}}\sphinxbfcode{\sphinxupquote{micro\_LP\_models}}}{}{}
\pysigstopsignatures
\sphinxAtStartPar
Return a list with the name of the models available in this toolkit and     print them all on the prompt. These models are the following ones:     ‘1994\sphinxhyphen{}BHF\sphinxhyphen{}SM\sphinxhyphen{}LP\sphinxhyphen{}AV14\sphinxhyphen{}GAP’, ‘1994\sphinxhyphen{}BHF\sphinxhyphen{}SM\sphinxhyphen{}LP\sphinxhyphen{}AV14\sphinxhyphen{}CONT’,     ‘1994\sphinxhyphen{}BHF\sphinxhyphen{}SM\sphinxhyphen{}LP\sphinxhyphen{}REID\sphinxhyphen{}GAP’, ‘1994\sphinxhyphen{}BHF\sphinxhyphen{}SM\sphinxhyphen{}LP\sphinxhyphen{}REID\sphinxhyphen{}CONT’, ‘1994\sphinxhyphen{}BHF\sphinxhyphen{}SM\sphinxhyphen{}LP\sphinxhyphen{}AV14\sphinxhyphen{}CONT\sphinxhyphen{}0.7’.
\begin{quote}\begin{description}
\sphinxlineitem{Returns}
\sphinxAtStartPar
The list of models.

\sphinxlineitem{Return type}
\sphinxAtStartPar
list{[}str{]}.

\end{description}\end{quote}

\end{fulllineitems}

\index{setupMicroLP (class in nucleardatapy.eos.setup\_micro\_lp)@\spxentry{setupMicroLP}\spxextra{class in nucleardatapy.eos.setup\_micro\_lp}}

\begin{fulllineitems}
\phantomsection\label{\detokenize{source/api/setup_eos_micro_lp:nucleardatapy.eos.setup_micro_lp.setupMicroLP}}
\pysigstartsignatures
\pysiglinewithargsret{\sphinxbfcode{\sphinxupquote{class\DUrole{w}{  }}}\sphinxcode{\sphinxupquote{nucleardatapy.eos.setup\_micro\_lp.}}\sphinxbfcode{\sphinxupquote{setupMicroLP}}}{\emph{\DUrole{n}{model}\DUrole{o}{=}\DUrole{default_value}{\textquotesingle{}1994\sphinxhyphen{}BHF\sphinxhyphen{}SM\sphinxhyphen{}LP\sphinxhyphen{}AV14\sphinxhyphen{}GAP\textquotesingle{}}}}{}
\pysigstopsignatures
\sphinxAtStartPar
Instantiate the object with Landau parameters from microscopic calculations choosen     by the toolkit practitioner.

\sphinxAtStartPar
This choice is defined in \sphinxtitleref{model}, which can chosen among     the following choices:     ‘1994\sphinxhyphen{}BHF\sphinxhyphen{}SM\sphinxhyphen{}LP\sphinxhyphen{}AV14\sphinxhyphen{}GAP’, ‘1994\sphinxhyphen{}BHF\sphinxhyphen{}SM\sphinxhyphen{}LP\sphinxhyphen{}AV14\sphinxhyphen{}CONT’,     ‘1994\sphinxhyphen{}BHF\sphinxhyphen{}SM\sphinxhyphen{}LP\sphinxhyphen{}REID\sphinxhyphen{}GAP’, ‘1994\sphinxhyphen{}BHF\sphinxhyphen{}SM\sphinxhyphen{}LP\sphinxhyphen{}REID\sphinxhyphen{}CONT’, ‘1994\sphinxhyphen{}BHF\sphinxhyphen{}SM\sphinxhyphen{}LP\sphinxhyphen{}AV14\sphinxhyphen{}CONT\sphinxhyphen{}0.7’.
\begin{quote}\begin{description}
\sphinxlineitem{Parameters}
\sphinxAtStartPar
\sphinxstyleliteralstrong{\sphinxupquote{model}} (\sphinxstyleliteralemphasis{\sphinxupquote{str}}\sphinxstyleliteralemphasis{\sphinxupquote{, }}\sphinxstyleliteralemphasis{\sphinxupquote{optional.}}) \textendash{} Fix the name of model. Default value: ‘1994\sphinxhyphen{}BHF\sphinxhyphen{}LP’.

\end{description}\end{quote}

\sphinxAtStartPar
\sphinxstylestrong{Attributes:}
\index{init\_self() (nucleardatapy.eos.setup\_micro\_lp.setupMicroLP method)@\spxentry{init\_self()}\spxextra{nucleardatapy.eos.setup\_micro\_lp.setupMicroLP method}}

\begin{fulllineitems}
\phantomsection\label{\detokenize{source/api/setup_eos_micro_lp:nucleardatapy.eos.setup_micro_lp.setupMicroLP.init_self}}
\pysigstartsignatures
\pysiglinewithargsret{\sphinxbfcode{\sphinxupquote{init\_self}}}{}{}
\pysigstopsignatures
\sphinxAtStartPar
Initialize variables in self.

\end{fulllineitems}

\index{model (nucleardatapy.eos.setup\_micro\_lp.setupMicroLP attribute)@\spxentry{model}\spxextra{nucleardatapy.eos.setup\_micro\_lp.setupMicroLP attribute}}

\begin{fulllineitems}
\phantomsection\label{\detokenize{source/api/setup_eos_micro_lp:nucleardatapy.eos.setup_micro_lp.setupMicroLP.model}}
\pysigstartsignatures
\pysigline{\sphinxbfcode{\sphinxupquote{model}}}
\pysigstopsignatures
\sphinxAtStartPar
Attribute model.

\end{fulllineitems}

\index{print\_outputs() (nucleardatapy.eos.setup\_micro\_lp.setupMicroLP method)@\spxentry{print\_outputs()}\spxextra{nucleardatapy.eos.setup\_micro\_lp.setupMicroLP method}}

\begin{fulllineitems}
\phantomsection\label{\detokenize{source/api/setup_eos_micro_lp:nucleardatapy.eos.setup_micro_lp.setupMicroLP.print_outputs}}
\pysigstartsignatures
\pysiglinewithargsret{\sphinxbfcode{\sphinxupquote{print\_outputs}}}{}{}
\pysigstopsignatures
\sphinxAtStartPar
Method which print outputs on terminal’s screen.

\end{fulllineitems}


\end{fulllineitems}


\sphinxAtStartPar
Here are a set of figures which are produced with the Python sample: /nucleardatapy\_sample/plots/plot\_setupeosmicroLP.py

\begin{figure}[htbp]
\centering
\capstart

\noindent\sphinxincludegraphics[scale=0.7]{{plot_eos_setupMicroLP0_SM}.png}
\caption{This figure shows the L=0 Landau parameters in SM for different NN interactions obtained
from BHF calculations.}\label{\detokenize{source/api/setup_eos_micro_lp:id1}}\end{figure}

\begin{figure}[htbp]
\centering
\capstart

\noindent\sphinxincludegraphics[scale=0.7]{{plot_eos_setupMicroLP1_SM}.png}
\caption{This figure shows the L=1 Landau parameters in SM for different NN interactions obtained
from BHF calculations.}\label{\detokenize{source/api/setup_eos_micro_lp:id2}}\end{figure}

\sphinxstepscope


\section{SetupEOSMicroGap}
\label{\detokenize{source/api/setup_eos_micro_gap:setupeosmicrogap}}\label{\detokenize{source/api/setup_eos_micro_gap::doc}}\phantomsection\label{\detokenize{source/api/setup_eos_micro_gap:module-nucleardatapy.eos.setup_micro_gap}}\index{module@\spxentry{module}!nucleardatapy.eos.setup\_micro\_gap@\spxentry{nucleardatapy.eos.setup\_micro\_gap}}\index{nucleardatapy.eos.setup\_micro\_gap@\spxentry{nucleardatapy.eos.setup\_micro\_gap}!module@\spxentry{module}}\index{micro\_gap\_models() (in module nucleardatapy.eos.setup\_micro\_gap)@\spxentry{micro\_gap\_models()}\spxextra{in module nucleardatapy.eos.setup\_micro\_gap}}

\begin{fulllineitems}
\phantomsection\label{\detokenize{source/api/setup_eos_micro_gap:nucleardatapy.eos.setup_micro_gap.micro_gap_models}}
\pysigstartsignatures
\pysiglinewithargsret{\sphinxcode{\sphinxupquote{nucleardatapy.eos.setup\_micro\_gap.}}\sphinxbfcode{\sphinxupquote{micro\_gap\_models}}}{}{}
\pysigstopsignatures
\sphinxAtStartPar
Return a list with the name of the models available in this toolkit and     print them all on the prompt. These models are the following ones:     ‘2008\sphinxhyphen{}BCS\sphinxhyphen{}NM’, ‘2008\sphinxhyphen{}AFDMC\sphinxhyphen{}NM’,     ‘2008\sphinxhyphen{}QMC\sphinxhyphen{}NM\sphinxhyphen{}swave’, ‘2009\sphinxhyphen{}DLQMC\sphinxhyphen{}NM’, ‘2010\sphinxhyphen{}QMC\sphinxhyphen{}NM\sphinxhyphen{}AV4’, ‘2017\sphinxhyphen{}MBPT\sphinxhyphen{}NM\sphinxhyphen{}GAP\sphinxhyphen{}EMG\sphinxhyphen{}450\sphinxhyphen{}500\sphinxhyphen{}N2LO’,     ‘2017\sphinxhyphen{}MBPT\sphinxhyphen{}NM\sphinxhyphen{}GAP\sphinxhyphen{}EMG\sphinxhyphen{}450\sphinxhyphen{}500\sphinxhyphen{}N3LO’, ‘2017\sphinxhyphen{}MBPT\sphinxhyphen{}NM\sphinxhyphen{}GAP\sphinxhyphen{}EMG\sphinxhyphen{}450\sphinxhyphen{}700\sphinxhyphen{}N2LO’,     ‘2017\sphinxhyphen{}MBPT\sphinxhyphen{}NM\sphinxhyphen{}GAP\sphinxhyphen{}EMG\sphinxhyphen{}450\sphinxhyphen{}700\sphinxhyphen{}N3LO’, ‘2017\sphinxhyphen{}MBPT\sphinxhyphen{}NM\sphinxhyphen{}GAP\sphinxhyphen{}EM\sphinxhyphen{}500\sphinxhyphen{}N2LO’, ‘2017\sphinxhyphen{}MBPT\sphinxhyphen{}NM\sphinxhyphen{}GAP\sphinxhyphen{}EM\sphinxhyphen{}500\sphinxhyphen{}N3LO’,     ‘2022\sphinxhyphen{}AFDMC\sphinxhyphen{}NM’
:return: The list of models.
:rtype: list{[}str{]}.

\end{fulllineitems}

\index{setupMicroGap (class in nucleardatapy.eos.setup\_micro\_gap)@\spxentry{setupMicroGap}\spxextra{class in nucleardatapy.eos.setup\_micro\_gap}}

\begin{fulllineitems}
\phantomsection\label{\detokenize{source/api/setup_eos_micro_gap:nucleardatapy.eos.setup_micro_gap.setupMicroGap}}
\pysigstartsignatures
\pysiglinewithargsret{\sphinxbfcode{\sphinxupquote{class\DUrole{w}{  }}}\sphinxcode{\sphinxupquote{nucleardatapy.eos.setup\_micro\_gap.}}\sphinxbfcode{\sphinxupquote{setupMicroGap}}}{\emph{\DUrole{n}{model}\DUrole{o}{=}\DUrole{default_value}{\textquotesingle{}2008\sphinxhyphen{}BCS\sphinxhyphen{}NM\textquotesingle{}}}}{}
\pysigstopsignatures
\sphinxAtStartPar
Instantiate the object with microscopic results choosen     by the toolkit practitioner.

\sphinxAtStartPar
This choice is defined in \sphinxtitleref{model}, which can chosen among     the following choices:     ‘2008\sphinxhyphen{}BCS\sphinxhyphen{}NM’, ‘2008\sphinxhyphen{}AFDMC\sphinxhyphen{}NM’,     ‘2008\sphinxhyphen{}QMC\sphinxhyphen{}NM\sphinxhyphen{}swave’, ‘2009\sphinxhyphen{}DLQMC\sphinxhyphen{}NM’, ‘2010\sphinxhyphen{}QMC\sphinxhyphen{}NM\sphinxhyphen{}AV4’, ‘2017\sphinxhyphen{}MBPT\sphinxhyphen{}NM\sphinxhyphen{}GAP\sphinxhyphen{}EMG\sphinxhyphen{}450\sphinxhyphen{}500\sphinxhyphen{}N2LO’,     ‘2017\sphinxhyphen{}MBPT\sphinxhyphen{}NM\sphinxhyphen{}GAP\sphinxhyphen{}EMG\sphinxhyphen{}450\sphinxhyphen{}500\sphinxhyphen{}N3LO’, ‘2017\sphinxhyphen{}MBPT\sphinxhyphen{}NM\sphinxhyphen{}GAP\sphinxhyphen{}EMG\sphinxhyphen{}450\sphinxhyphen{}700\sphinxhyphen{}N2LO’,     ‘2017\sphinxhyphen{}MBPT\sphinxhyphen{}NM\sphinxhyphen{}GAP\sphinxhyphen{}EMG\sphinxhyphen{}450\sphinxhyphen{}700\sphinxhyphen{}N3LO’, ‘2017\sphinxhyphen{}MBPT\sphinxhyphen{}NM\sphinxhyphen{}GAP\sphinxhyphen{}EM\sphinxhyphen{}500\sphinxhyphen{}N2LO’, ‘2017\sphinxhyphen{}MBPT\sphinxhyphen{}NM\sphinxhyphen{}GAP\sphinxhyphen{}EM\sphinxhyphen{}500\sphinxhyphen{}N3LO’,     ‘2022\sphinxhyphen{}AFDMC\sphinxhyphen{}NM’
:param model: Fix the name of model. Default value: ‘2008\sphinxhyphen{}BCS\sphinxhyphen{}NM’.
:type model: str, optional.

\sphinxAtStartPar
\sphinxstylestrong{Attributes:}
\index{init\_self() (nucleardatapy.eos.setup\_micro\_gap.setupMicroGap method)@\spxentry{init\_self()}\spxextra{nucleardatapy.eos.setup\_micro\_gap.setupMicroGap method}}

\begin{fulllineitems}
\phantomsection\label{\detokenize{source/api/setup_eos_micro_gap:nucleardatapy.eos.setup_micro_gap.setupMicroGap.init_self}}
\pysigstartsignatures
\pysiglinewithargsret{\sphinxbfcode{\sphinxupquote{init\_self}}}{}{}
\pysigstopsignatures
\sphinxAtStartPar
Initialize variables in self.

\end{fulllineitems}

\index{model (nucleardatapy.eos.setup\_micro\_gap.setupMicroGap attribute)@\spxentry{model}\spxextra{nucleardatapy.eos.setup\_micro\_gap.setupMicroGap attribute}}

\begin{fulllineitems}
\phantomsection\label{\detokenize{source/api/setup_eos_micro_gap:nucleardatapy.eos.setup_micro_gap.setupMicroGap.model}}
\pysigstartsignatures
\pysigline{\sphinxbfcode{\sphinxupquote{model}}}
\pysigstopsignatures
\sphinxAtStartPar
Attribute model.

\end{fulllineitems}

\index{print\_outputs() (nucleardatapy.eos.setup\_micro\_gap.setupMicroGap method)@\spxentry{print\_outputs()}\spxextra{nucleardatapy.eos.setup\_micro\_gap.setupMicroGap method}}

\begin{fulllineitems}
\phantomsection\label{\detokenize{source/api/setup_eos_micro_gap:nucleardatapy.eos.setup_micro_gap.setupMicroGap.print_outputs}}
\pysigstartsignatures
\pysiglinewithargsret{\sphinxbfcode{\sphinxupquote{print\_outputs}}}{}{}
\pysigstopsignatures
\sphinxAtStartPar
Method which print outputs on terminal’s screen.

\end{fulllineitems}


\end{fulllineitems}


\sphinxAtStartPar
Here are a set of figures which are produced with the Python sample: /nucleardatapy\_sample/plots/plot\_eos\_setupMicroGap.py

\begin{figure}[htbp]
\centering
\capstart

\noindent\sphinxincludegraphics[scale=0.7]{{plot_eos_setupMicro_gap_1s0_NM}.png}
\caption{This figure shows the 1S0 pairing gap in neutron matter (NM) over the Fermi energy (top) and the pairing gap (bottom) as function of the density (left) and the neutron Fermi momentum (right) for the models available in the nucleardatapy toolkit.}\label{\detokenize{source/api/setup_eos_micro_gap:id1}}\end{figure}

\sphinxstepscope


\section{SetupEOSMicroEsym}
\label{\detokenize{source/api/setup_eos_micro_esym:setupeosmicroesym}}\label{\detokenize{source/api/setup_eos_micro_esym::doc}}\phantomsection\label{\detokenize{source/api/setup_eos_micro_esym:module-nucleardatapy.eos.setup_micro_esym}}\index{module@\spxentry{module}!nucleardatapy.eos.setup\_micro\_esym@\spxentry{nucleardatapy.eos.setup\_micro\_esym}}\index{nucleardatapy.eos.setup\_micro\_esym@\spxentry{nucleardatapy.eos.setup\_micro\_esym}!module@\spxentry{module}}\index{micro\_esym\_models() (in module nucleardatapy.eos.setup\_micro\_esym)@\spxentry{micro\_esym\_models()}\spxextra{in module nucleardatapy.eos.setup\_micro\_esym}}

\begin{fulllineitems}
\phantomsection\label{\detokenize{source/api/setup_eos_micro_esym:nucleardatapy.eos.setup_micro_esym.micro_esym_models}}
\pysigstartsignatures
\pysiglinewithargsret{\sphinxcode{\sphinxupquote{nucleardatapy.eos.setup\_micro\_esym.}}\sphinxbfcode{\sphinxupquote{micro\_esym\_models}}}{}{}
\pysigstopsignatures
\sphinxAtStartPar
Return a list with the name of the models available in this toolkit and     print them all on the prompt. These models are the following ones:     ‘1981\sphinxhyphen{}VAR\sphinxhyphen{}AM\sphinxhyphen{}FP’, ‘1998\sphinxhyphen{}VAR\sphinxhyphen{}AM\sphinxhyphen{}APR’, ‘1998\sphinxhyphen{}VAR\sphinxhyphen{}AM\sphinxhyphen{}APRfit’, ‘2006\sphinxhyphen{}BHF\sphinxhyphen{}AM*’,     2016\sphinxhyphen{}MBPT\sphinxhyphen{}AM’, 2019\sphinxhyphen{}MBPT\sphinxhyphen{}AM\sphinxhyphen{}L59’, ‘2019\sphinxhyphen{}MBPT\sphinxhyphen{}AM\sphinxhyphen{}L69’,     ‘2020\sphinxhyphen{}MBPT\sphinxhyphen{}AM’, ‘2024\sphinxhyphen{}NLEFT\sphinxhyphen{}AM’,     ‘2024\sphinxhyphen{}BHF\sphinxhyphen{}AM\sphinxhyphen{}2BF\sphinxhyphen{}Av8p’, ‘2024\sphinxhyphen{}BHF\sphinxhyphen{}AM\sphinxhyphen{}2BF\sphinxhyphen{}Av18’, ‘2024\sphinxhyphen{}BHF\sphinxhyphen{}AM\sphinxhyphen{}2BF\sphinxhyphen{}BONN’, ‘2024\sphinxhyphen{}BHF\sphinxhyphen{}AM\sphinxhyphen{}2BF\sphinxhyphen{}CDBONN’,     ‘2024\sphinxhyphen{}BHF\sphinxhyphen{}AM\sphinxhyphen{}2BF\sphinxhyphen{}NSC97a’, ‘2024\sphinxhyphen{}BHF\sphinxhyphen{}AM\sphinxhyphen{}2BF\sphinxhyphen{}NSC97b’, ‘2024\sphinxhyphen{}BHF\sphinxhyphen{}AM\sphinxhyphen{}2BF\sphinxhyphen{}NSC97c’, ‘2024\sphinxhyphen{}BHF\sphinxhyphen{}AM\sphinxhyphen{}2BF\sphinxhyphen{}NSC97d’,     ‘2024\sphinxhyphen{}BHF\sphinxhyphen{}AM\sphinxhyphen{}2BF\sphinxhyphen{}NSC97e’, ‘2024\sphinxhyphen{}BHF\sphinxhyphen{}AM\sphinxhyphen{}2BF\sphinxhyphen{}NSC97f’, ‘2024\sphinxhyphen{}BHF\sphinxhyphen{}AM\sphinxhyphen{}2BF\sphinxhyphen{}SSCV14’,    ‘2024\sphinxhyphen{}BHF\sphinxhyphen{}AM\sphinxhyphen{}23BF\sphinxhyphen{}Av8p’, ‘2024\sphinxhyphen{}BHF\sphinxhyphen{}AM\sphinxhyphen{}23BF\sphinxhyphen{}Av18’, ‘2024\sphinxhyphen{}BHF\sphinxhyphen{}AM\sphinxhyphen{}23BF\sphinxhyphen{}BONN’, ‘2024\sphinxhyphen{}BHF\sphinxhyphen{}AM\sphinxhyphen{}23BF\sphinxhyphen{}CDBONN’,     ‘2024\sphinxhyphen{}BHF\sphinxhyphen{}AM\sphinxhyphen{}23BF\sphinxhyphen{}NSC97a’, ‘2024\sphinxhyphen{}BHF\sphinxhyphen{}AM\sphinxhyphen{}23BF\sphinxhyphen{}NSC97b’, ‘2024\sphinxhyphen{}BHF\sphinxhyphen{}AM\sphinxhyphen{}23BF\sphinxhyphen{}NSC97c’, ‘2024\sphinxhyphen{}BHF\sphinxhyphen{}AM\sphinxhyphen{}23BF\sphinxhyphen{}NSC97d’,     ‘2024\sphinxhyphen{}BHF\sphinxhyphen{}AM\sphinxhyphen{}23BF\sphinxhyphen{}NSC97e’, ‘2024\sphinxhyphen{}BHF\sphinxhyphen{}AM\sphinxhyphen{}23BF\sphinxhyphen{}NSC97f’, ‘2024\sphinxhyphen{}BHF\sphinxhyphen{}AM\sphinxhyphen{}23BF\sphinxhyphen{}SSCV14’,    ‘2024\sphinxhyphen{}BHF\sphinxhyphen{}AM\sphinxhyphen{}23BFmicro\sphinxhyphen{}Av18’, ‘2024\sphinxhyphen{}BHF\sphinxhyphen{}AM\sphinxhyphen{}23BFmicro\sphinxhyphen{}BONNB’, ‘2024\sphinxhyphen{}BHF\sphinxhyphen{}AM\sphinxhyphen{}23BFmicro\sphinxhyphen{}NSC93’
:return: The list of models.
:rtype: list{[}str{]}.

\end{fulllineitems}

\index{setupMicroEsym (class in nucleardatapy.eos.setup\_micro\_esym)@\spxentry{setupMicroEsym}\spxextra{class in nucleardatapy.eos.setup\_micro\_esym}}

\begin{fulllineitems}
\phantomsection\label{\detokenize{source/api/setup_eos_micro_esym:nucleardatapy.eos.setup_micro_esym.setupMicroEsym}}
\pysigstartsignatures
\pysiglinewithargsret{\sphinxbfcode{\sphinxupquote{class\DUrole{w}{  }}}\sphinxcode{\sphinxupquote{nucleardatapy.eos.setup\_micro\_esym.}}\sphinxbfcode{\sphinxupquote{setupMicroEsym}}}{\emph{\DUrole{n}{model}\DUrole{o}{=}\DUrole{default_value}{\textquotesingle{}1998\sphinxhyphen{}VAR\sphinxhyphen{}AM\sphinxhyphen{}APR\textquotesingle{}}}, \emph{\DUrole{n}{var1}\DUrole{o}{=}\DUrole{default_value}{array({[}0.01, 0.03052632, 0.05105263, 0.07157895, 0.09210526, 0.11263158, 0.13315789, 0.15368421, 0.17421053, 0.19473684, 0.21526316, 0.23578947, 0.25631579, 0.27684211, 0.29736842, 0.31789474, 0.33842105, 0.35894737, 0.37947368, 0.4{]})}}, \emph{\DUrole{n}{var2}\DUrole{o}{=}\DUrole{default_value}{0.0}}}{}
\pysigstopsignatures
\sphinxAtStartPar
Instantiate the object with microscopic results choosen     by the toolkit practitioner.

\sphinxAtStartPar
This choice is defined in \sphinxtitleref{model}, which can chosen among     the following choices:     ‘1981\sphinxhyphen{}VAR\sphinxhyphen{}AM\sphinxhyphen{}FP’, ‘1998\sphinxhyphen{}VAR\sphinxhyphen{}AM\sphinxhyphen{}APR’, ‘1998\sphinxhyphen{}VAR\sphinxhyphen{}AM\sphinxhyphen{}iAPR’, ‘2006\sphinxhyphen{}BHF\sphinxhyphen{}AM*’,     ‘2016\sphinxhyphen{}MBPT\sphinxhyphen{}AM’, ‘2019\sphinxhyphen{}MBPT\sphinxhyphen{}AM\sphinxhyphen{}L59’, ‘2019\sphinxhyphen{}MBPT\sphinxhyphen{}AM\sphinxhyphen{}L69’,     ‘2020\sphinxhyphen{}MBPT\sphinxhyphen{}AM’, ‘2024\sphinxhyphen{}NLEFT\sphinxhyphen{}AM’,     ‘2024\sphinxhyphen{}BHF\sphinxhyphen{}AM\sphinxhyphen{}2BF\sphinxhyphen{}Av8p’, ‘2024\sphinxhyphen{}BHF\sphinxhyphen{}AM\sphinxhyphen{}2BF\sphinxhyphen{}Av18’, ‘2024\sphinxhyphen{}BHF\sphinxhyphen{}AM\sphinxhyphen{}2BF\sphinxhyphen{}BONN’, ‘2024\sphinxhyphen{}BHF\sphinxhyphen{}AM\sphinxhyphen{}2BF\sphinxhyphen{}CDBONN’,     ‘2024\sphinxhyphen{}BHF\sphinxhyphen{}AM\sphinxhyphen{}2BF\sphinxhyphen{}NSC97a’, ‘2024\sphinxhyphen{}BHF\sphinxhyphen{}AM\sphinxhyphen{}2BF\sphinxhyphen{}NSC97b’, ‘2024\sphinxhyphen{}BHF\sphinxhyphen{}AM\sphinxhyphen{}2BF\sphinxhyphen{}NSC97c’, ‘2024\sphinxhyphen{}BHF\sphinxhyphen{}AM\sphinxhyphen{}2BF\sphinxhyphen{}NSC97d’,     ‘2024\sphinxhyphen{}BHF\sphinxhyphen{}AM\sphinxhyphen{}2BF\sphinxhyphen{}NSC97e’, ‘2024\sphinxhyphen{}BHF\sphinxhyphen{}AM\sphinxhyphen{}2BF\sphinxhyphen{}NSC97f’, ‘2024\sphinxhyphen{}BHF\sphinxhyphen{}AM\sphinxhyphen{}2BF\sphinxhyphen{}SSCV14’,     ‘2024\sphinxhyphen{}BHF\sphinxhyphen{}AM\sphinxhyphen{}23BF\sphinxhyphen{}Av8p’, ‘2024\sphinxhyphen{}BHF\sphinxhyphen{}AM\sphinxhyphen{}23BF\sphinxhyphen{}Av18’, ‘2024\sphinxhyphen{}BHF\sphinxhyphen{}AM\sphinxhyphen{}23BF\sphinxhyphen{}BONN’, ‘2024\sphinxhyphen{}BHF\sphinxhyphen{}AM\sphinxhyphen{}23BF\sphinxhyphen{}CDBONN’,     ‘2024\sphinxhyphen{}BHF\sphinxhyphen{}AM\sphinxhyphen{}23BF\sphinxhyphen{}NSC97a’, ‘2024\sphinxhyphen{}BHF\sphinxhyphen{}AM\sphinxhyphen{}23BF\sphinxhyphen{}NSC97b’, ‘2024\sphinxhyphen{}BHF\sphinxhyphen{}AM\sphinxhyphen{}23BF\sphinxhyphen{}NSC97c’, ‘2024\sphinxhyphen{}BHF\sphinxhyphen{}AM\sphinxhyphen{}23BF\sphinxhyphen{}NSC97d’,     ‘2024\sphinxhyphen{}BHF\sphinxhyphen{}AM\sphinxhyphen{}23BF\sphinxhyphen{}NSC97e’, ‘2024\sphinxhyphen{}BHF\sphinxhyphen{}AM\sphinxhyphen{}23BF\sphinxhyphen{}NSC97f’, ‘2024\sphinxhyphen{}BHF\sphinxhyphen{}AM\sphinxhyphen{}23BF\sphinxhyphen{}SSCV14’
\begin{quote}\begin{description}
\sphinxlineitem{Parameters}
\sphinxAtStartPar
\sphinxstyleliteralstrong{\sphinxupquote{model}} (\sphinxstyleliteralemphasis{\sphinxupquote{str}}\sphinxstyleliteralemphasis{\sphinxupquote{, }}\sphinxstyleliteralemphasis{\sphinxupquote{optional.}}) \textendash{} Fix the name of model. Default value: ‘1998\sphinxhyphen{}VAR\sphinxhyphen{}AM\sphinxhyphen{}APR’.

\end{description}\end{quote}

\sphinxAtStartPar
\sphinxstylestrong{Attributes:}
\index{init\_self() (nucleardatapy.eos.setup\_micro\_esym.setupMicroEsym method)@\spxentry{init\_self()}\spxextra{nucleardatapy.eos.setup\_micro\_esym.setupMicroEsym method}}

\begin{fulllineitems}
\phantomsection\label{\detokenize{source/api/setup_eos_micro_esym:nucleardatapy.eos.setup_micro_esym.setupMicroEsym.init_self}}
\pysigstartsignatures
\pysiglinewithargsret{\sphinxbfcode{\sphinxupquote{init\_self}}}{}{}
\pysigstopsignatures
\sphinxAtStartPar
Initialize variables in self.

\end{fulllineitems}

\index{model (nucleardatapy.eos.setup\_micro\_esym.setupMicroEsym attribute)@\spxentry{model}\spxextra{nucleardatapy.eos.setup\_micro\_esym.setupMicroEsym attribute}}

\begin{fulllineitems}
\phantomsection\label{\detokenize{source/api/setup_eos_micro_esym:nucleardatapy.eos.setup_micro_esym.setupMicroEsym.model}}
\pysigstartsignatures
\pysigline{\sphinxbfcode{\sphinxupquote{model}}}
\pysigstopsignatures
\sphinxAtStartPar
Attribute model.

\end{fulllineitems}

\index{print\_outputs() (nucleardatapy.eos.setup\_micro\_esym.setupMicroEsym method)@\spxentry{print\_outputs()}\spxextra{nucleardatapy.eos.setup\_micro\_esym.setupMicroEsym method}}

\begin{fulllineitems}
\phantomsection\label{\detokenize{source/api/setup_eos_micro_esym:nucleardatapy.eos.setup_micro_esym.setupMicroEsym.print_outputs}}
\pysigstartsignatures
\pysiglinewithargsret{\sphinxbfcode{\sphinxupquote{print\_outputs}}}{}{}
\pysigstopsignatures
\sphinxAtStartPar
Method which print outputs on terminal’s screen.

\end{fulllineitems}


\end{fulllineitems}


\sphinxAtStartPar
Here are a set of figures which are produced with the Python sample: /nucleardatapy\_sample/plots/plot\_eos\_setupMicroEsym.py

\begin{figure}[htbp]
\centering
\capstart

\noindent\sphinxincludegraphics[scale=0.7]{{plot_eos_setupMicroEsym}.png}
\caption{This figure shows the symmetry energy as function of the density (left) and the neutron Fermi momentum (right) for the models available in the nucleardatapy toolkit.}\label{\detokenize{source/api/setup_eos_micro_esym:id1}}\end{figure}

\sphinxstepscope


\section{SetupEOSPheno}
\label{\detokenize{source/api/setup_eos_pheno:setupeospheno}}\label{\detokenize{source/api/setup_eos_pheno::doc}}\phantomsection\label{\detokenize{source/api/setup_eos_pheno:module-nucleardatapy.eos.setup_pheno}}\index{module@\spxentry{module}!nucleardatapy.eos.setup\_pheno@\spxentry{nucleardatapy.eos.setup\_pheno}}\index{nucleardatapy.eos.setup\_pheno@\spxentry{nucleardatapy.eos.setup\_pheno}!module@\spxentry{module}}\index{pheno\_models() (in module nucleardatapy.eos.setup\_pheno)@\spxentry{pheno\_models()}\spxextra{in module nucleardatapy.eos.setup\_pheno}}

\begin{fulllineitems}
\phantomsection\label{\detokenize{source/api/setup_eos_pheno:nucleardatapy.eos.setup_pheno.pheno_models}}
\pysigstartsignatures
\pysiglinewithargsret{\sphinxcode{\sphinxupquote{nucleardatapy.eos.setup\_pheno.}}\sphinxbfcode{\sphinxupquote{pheno\_models}}}{}{}
\pysigstopsignatures
\sphinxAtStartPar
Return a list of models available in this toolkit and print them all on the prompt.
\begin{quote}\begin{description}
\sphinxlineitem{Returns}
\sphinxAtStartPar
The list of models with can be ‘Skyrme’, ‘ESkyrme’, ‘NLRH’, ‘DDRH’, ‘DDRHF’.

\sphinxlineitem{Return type}
\sphinxAtStartPar
list{[}str{]}.

\end{description}\end{quote}

\end{fulllineitems}

\index{pheno\_params() (in module nucleardatapy.eos.setup\_pheno)@\spxentry{pheno\_params()}\spxextra{in module nucleardatapy.eos.setup\_pheno}}

\begin{fulllineitems}
\phantomsection\label{\detokenize{source/api/setup_eos_pheno:nucleardatapy.eos.setup_pheno.pheno_params}}
\pysigstartsignatures
\pysiglinewithargsret{\sphinxcode{\sphinxupquote{nucleardatapy.eos.setup\_pheno.}}\sphinxbfcode{\sphinxupquote{pheno\_params}}}{\emph{\DUrole{n}{model}}}{}
\pysigstopsignatures
\sphinxAtStartPar
Return a list with the parameterizations available in
this toolkit for a given model and print them all on the prompt.
\begin{quote}\begin{description}
\sphinxlineitem{Parameters}
\sphinxAtStartPar
\sphinxstyleliteralstrong{\sphinxupquote{model}} (\sphinxstyleliteralemphasis{\sphinxupquote{str.}}) \textendash{} The type of model for which there are parametrizations.     They should be chosen among the following options: ‘Skyrme’, ‘NLRH’,     ‘DDRH’, ‘DDRHF’.

\sphinxlineitem{Returns}
\sphinxAtStartPar
The list of parametrizations.     If \sphinxtitleref{models} == ‘Skyrme’: ‘BSK14’,     ‘BSK16’, ‘BSK17’, ‘BSK27’, ‘F\sphinxhyphen{}’, ‘F+’, ‘F0’, ‘FPL’, ‘LNS’, ‘LNS1’, ‘LNS5’,     ‘NRAPR’, ‘RATP’, ‘SAMI’, ‘SGII’, ‘SIII’, ‘SKGSIGMA’, ‘SKI2’, ‘SKI4’, ‘SKMP’,     ‘SKMS’, ‘SKO’, ‘SKOP’, ‘SKP’, ‘SKRSIGMA’, ‘SKX’, ‘Skz2’, ‘SLY4’, ‘SLY5’,     ‘SLY230A’, ‘SLY230B’, ‘SV’, ‘T6’, ‘T44’, ‘UNEDF0’, ‘UNEDF1’.     If \sphinxtitleref{models} == ‘ESkyrme’: ‘BSk22’, ‘BSk24’, ‘BSk25’, ‘BSk26’, ‘BSk31’, ‘BSk32’,     ‘BSkG1’, ‘BSkG2’, ‘BSkG3’.

\end{description}\end{quote}

\sphinxAtStartPar
If \sphinxtitleref{models} == ‘NLRH’: ‘NL\sphinxhyphen{}SH’, ‘NL3’, ‘NL3II’, ‘PK1’, ‘PK1R’, ‘TM1’.     If \sphinxtitleref{models} == ‘DDRH’: ‘DDME1’, ‘DDME2’, ‘DDMEd’, ‘PKDD’, ‘TW99’.     If \sphinxtitleref{models} == ‘DDRHF’: ‘PKA1’, ‘PKO1’, ‘PKO2’, ‘PKO3’.
:rtype: list{[}str{]}.

\end{fulllineitems}

\index{setupPheno (class in nucleardatapy.eos.setup\_pheno)@\spxentry{setupPheno}\spxextra{class in nucleardatapy.eos.setup\_pheno}}

\begin{fulllineitems}
\phantomsection\label{\detokenize{source/api/setup_eos_pheno:nucleardatapy.eos.setup_pheno.setupPheno}}
\pysigstartsignatures
\pysiglinewithargsret{\sphinxbfcode{\sphinxupquote{class\DUrole{w}{  }}}\sphinxcode{\sphinxupquote{nucleardatapy.eos.setup\_pheno.}}\sphinxbfcode{\sphinxupquote{setupPheno}}}{\emph{\DUrole{n}{model}\DUrole{o}{=}\DUrole{default_value}{\textquotesingle{}Skyrme\textquotesingle{}}}, \emph{\DUrole{n}{param}\DUrole{o}{=}\DUrole{default_value}{\textquotesingle{}SLY5\textquotesingle{}}}}{}
\pysigstopsignatures
\sphinxAtStartPar
Instantiate the object with results based on phenomenological    interactions and choosen by the toolkit practitioner.     This choice is defined in the variables \sphinxtitleref{model} and \sphinxtitleref{param}.

\sphinxAtStartPar
If \sphinxtitleref{models} == ‘Skyrme’, \sphinxtitleref{param} can be: ‘BSK14’,     ‘BSK16’, ‘BSK17’, ‘BSK27’, ‘F\sphinxhyphen{}’, ‘F+’, ‘F0’, ‘FPL’, ‘LNS’, ‘LNS1’, ‘LNS5’,     ‘NRAPR’, ‘RATP’, ‘SAMI’, ‘SGII’, ‘SIII’, ‘SKGSIGMA’, ‘SKI2’, ‘SKI4’, ‘SKMP’,     ‘SKMS’, ‘SKO’, ‘SKOP’, ‘SKP’, ‘SKRSIGMA’, ‘SKX’, ‘Skz2’, ‘SLY4’, ‘SLY5’,     ‘SLY230A’, ‘SLY230B’, ‘SV’, ‘T6’, ‘T44’, ‘UNEDF0’, ‘UNEDF1’.

\sphinxAtStartPar
If \sphinxtitleref{models} == ‘ESkyrme’, \sphinxtitleref{param} can be: ‘BSk22’, ‘BSk24’, ‘BSk25’,     ‘BSk26’, ‘BSk31’, ‘BSk32’, ‘BSkG1’, ‘BSkG2’, ‘BSkG3’.

\sphinxAtStartPar
If \sphinxtitleref{models} == ‘NLRH’, \sphinxtitleref{param} can be: ‘NL\sphinxhyphen{}SH’, ‘NL3’, ‘NL3II’, ‘PK1’, ‘PK1R’, ‘TM1’.

\sphinxAtStartPar
If \sphinxtitleref{models} == ‘DDRH’, \sphinxtitleref{param} can be: ‘DDME1’, ‘DDME2’, ‘DDMEd’, ‘PKDD’, ‘TW99’.

\sphinxAtStartPar
If \sphinxtitleref{models} == ‘DDRHF’, \sphinxtitleref{param} can be: ‘PKA1’, ‘PKO1’, ‘PKO2’, ‘PKO3’.
\begin{quote}\begin{description}
\sphinxlineitem{Parameters}\begin{itemize}
\item {} 
\sphinxAtStartPar
\sphinxstyleliteralstrong{\sphinxupquote{model}} (\sphinxstyleliteralemphasis{\sphinxupquote{str}}\sphinxstyleliteralemphasis{\sphinxupquote{, }}\sphinxstyleliteralemphasis{\sphinxupquote{optional.}}) \textendash{} Fix the name of model: ‘Skyrme’, ‘NLRH’,     ‘DDRH’, ‘DDRHF’. Default value: ‘Skyrme’.

\item {} 
\sphinxAtStartPar
\sphinxstyleliteralstrong{\sphinxupquote{param}} (\sphinxstyleliteralemphasis{\sphinxupquote{str}}\sphinxstyleliteralemphasis{\sphinxupquote{, }}\sphinxstyleliteralemphasis{\sphinxupquote{optional.}}) \textendash{} Fix the parameterization associated to model.     Default value: ‘SLY5’.

\end{itemize}

\end{description}\end{quote}

\sphinxAtStartPar
\sphinxstylestrong{Attributes:}
\index{init\_self() (nucleardatapy.eos.setup\_pheno.setupPheno method)@\spxentry{init\_self()}\spxextra{nucleardatapy.eos.setup\_pheno.setupPheno method}}

\begin{fulllineitems}
\phantomsection\label{\detokenize{source/api/setup_eos_pheno:nucleardatapy.eos.setup_pheno.setupPheno.init_self}}
\pysigstartsignatures
\pysiglinewithargsret{\sphinxbfcode{\sphinxupquote{init\_self}}}{}{}
\pysigstopsignatures
\sphinxAtStartPar
Initialize variables in self.

\end{fulllineitems}

\index{label (nucleardatapy.eos.setup\_pheno.setupPheno attribute)@\spxentry{label}\spxextra{nucleardatapy.eos.setup\_pheno.setupPheno attribute}}

\begin{fulllineitems}
\phantomsection\label{\detokenize{source/api/setup_eos_pheno:nucleardatapy.eos.setup_pheno.setupPheno.label}}
\pysigstartsignatures
\pysigline{\sphinxbfcode{\sphinxupquote{label}}}
\pysigstopsignatures
\sphinxAtStartPar
Attribute providing the label the data is references for figures.

\end{fulllineitems}

\index{model (nucleardatapy.eos.setup\_pheno.setupPheno attribute)@\spxentry{model}\spxextra{nucleardatapy.eos.setup\_pheno.setupPheno attribute}}

\begin{fulllineitems}
\phantomsection\label{\detokenize{source/api/setup_eos_pheno:nucleardatapy.eos.setup_pheno.setupPheno.model}}
\pysigstartsignatures
\pysigline{\sphinxbfcode{\sphinxupquote{model}}}
\pysigstopsignatures
\sphinxAtStartPar
Attribute model.

\end{fulllineitems}

\index{note (nucleardatapy.eos.setup\_pheno.setupPheno attribute)@\spxentry{note}\spxextra{nucleardatapy.eos.setup\_pheno.setupPheno attribute}}

\begin{fulllineitems}
\phantomsection\label{\detokenize{source/api/setup_eos_pheno:nucleardatapy.eos.setup_pheno.setupPheno.note}}
\pysigstartsignatures
\pysigline{\sphinxbfcode{\sphinxupquote{note}}}
\pysigstopsignatures
\sphinxAtStartPar
Attribute providing additional notes about the data.

\end{fulllineitems}

\index{param (nucleardatapy.eos.setup\_pheno.setupPheno attribute)@\spxentry{param}\spxextra{nucleardatapy.eos.setup\_pheno.setupPheno attribute}}

\begin{fulllineitems}
\phantomsection\label{\detokenize{source/api/setup_eos_pheno:nucleardatapy.eos.setup_pheno.setupPheno.param}}
\pysigstartsignatures
\pysigline{\sphinxbfcode{\sphinxupquote{param}}}
\pysigstopsignatures
\sphinxAtStartPar
Attribute param.

\end{fulllineitems}

\index{print\_outputs() (nucleardatapy.eos.setup\_pheno.setupPheno method)@\spxentry{print\_outputs()}\spxextra{nucleardatapy.eos.setup\_pheno.setupPheno method}}

\begin{fulllineitems}
\phantomsection\label{\detokenize{source/api/setup_eos_pheno:nucleardatapy.eos.setup_pheno.setupPheno.print_outputs}}
\pysigstartsignatures
\pysiglinewithargsret{\sphinxbfcode{\sphinxupquote{print\_outputs}}}{}{}
\pysigstopsignatures
\sphinxAtStartPar
Method which print outputs on terminal’s screen.

\end{fulllineitems}


\end{fulllineitems}


\sphinxAtStartPar
Here are a set of figures which are produced with the Python sample: /nucleardatapy\_sample/plots/plot\_eos\_setupPheno.py

\begin{figure}[htbp]
\centering
\capstart

\noindent\sphinxincludegraphics[scale=0.7]{{plot_eos_setupPheno-NLRH-E-NM}.png}
\caption{This figure shows the energy in neutron matter (NM) over the free Fermi gas energy (top) and the energy per particle (bottom) as function of the density (left) and the neutron Fermi momentum (right) for the complete list of phenomenological models based on non\sphinxhyphen{}linear meson(s) relativistic Hartree (NLRH) approach available in the nucleardatapy toolkit.}\label{\detokenize{source/api/setup_eos_pheno:id1}}\end{figure}

\begin{figure}[htbp]
\centering
\capstart

\noindent\sphinxincludegraphics[scale=0.7]{{plot_eos_setupPheno-NLRH-E-SM}.png}
\caption{This figure shows the energy in symmetric matter (SM) over the free Fermi gas energy (top) and the energy per particle (bottom) as function of the density (left) and the neutron Fermi momentum (right) for the complete list of phenomenological models based on non\sphinxhyphen{}linear meson(s) relativistic Hartree (NLRH) approach available in the nucleardatapy toolkit.}\label{\detokenize{source/api/setup_eos_pheno:id2}}\end{figure}

\begin{figure}[htbp]
\centering
\capstart

\noindent\sphinxincludegraphics[scale=0.7]{{plot_eos_setupPheno-DDRH-E-NM}.png}
\caption{This figure shows the energy in neutron matter (NM) over the free Fermi gas energy (top) and the energy per particle (bottom) as function of the density (left) and the neutron Fermi momentum (right) for the complete list of phenomenological models based on density\sphinxhyphen{}dependent relativistic Hartree (DDRH) approach available in the nucleardatapy toolkit.}\label{\detokenize{source/api/setup_eos_pheno:id3}}\end{figure}

\begin{figure}[htbp]
\centering
\capstart

\noindent\sphinxincludegraphics[scale=0.7]{{plot_eos_setupPheno-DDRH-E-SM}.png}
\caption{This figure shows the energy in symmetric matter (SM) over the free Fermi gas energy (top) and the energy per particle (bottom) as function of the density (left) and the neutron Fermi momentum (right) for the complete list of phenomenological models based on density\sphinxhyphen{}dependent relativistic Hartree (DDRH) approach available in the nucleardatapy toolkit.}\label{\detokenize{source/api/setup_eos_pheno:id4}}\end{figure}

\begin{figure}[htbp]
\centering
\capstart

\noindent\sphinxincludegraphics[scale=0.7]{{plot_eos_setupPheno-DDRHF-E-NM}.png}
\caption{This figure shows the energy in neutron matter (NM) over the free Fermi gas energy (top) and the energy per particle (bottom) as function of the density (left) and the neutron Fermi momentum (right) for the complete list of phenomenological models based on density\sphinxhyphen{}dependent relativistic Hartree\sphinxhyphen{}Fock (DDRHF) approach available in the nucleardatapy toolkit.}\label{\detokenize{source/api/setup_eos_pheno:id5}}\end{figure}

\begin{figure}[htbp]
\centering
\capstart

\noindent\sphinxincludegraphics[scale=0.7]{{plot_eos_setupPheno-DDRHF-E-SM}.png}
\caption{This figure shows the energy in symmetric matter (SM) over the free Fermi gas energy (top) and the energy per particle (bottom) as function of the density (left) and the neutron Fermi momentum (right) for the complete list of phenomenological models based on density\sphinxhyphen{}dependent relativistic Hartree\sphinxhyphen{}Fock (DDRHF) approach available in the nucleardatapy toolkit.}\label{\detokenize{source/api/setup_eos_pheno:id6}}\end{figure}

\begin{figure}[htbp]
\centering
\capstart

\noindent\sphinxincludegraphics[scale=0.7]{{plot_eos_setupPheno-Skyrme-E-NM}.png}
\caption{This figure shows the energy in neutron matter (NM) over the free Fermi gas energy (top) and the energy per particle (bottom) as function of the density (left) and the neutron Fermi momentum (right) for the complete list of phenomenological models based on the standard Skyrme interactions available in the nucleardatapy toolkit.}\label{\detokenize{source/api/setup_eos_pheno:id7}}\end{figure}

\begin{figure}[htbp]
\centering
\capstart

\noindent\sphinxincludegraphics[scale=0.7]{{plot_eos_setupPheno-Skyrme-E-SM}.png}
\caption{This figure shows the energy in symmetric matter (SM) over the free Fermi gas energy (top) and the energy per particle (bottom) as function of the density (left) and the neutron Fermi momentum (right) for the complete list of phenomenological models based on the standard Skyrme interactions available in the nucleardatapy toolkit.}\label{\detokenize{source/api/setup_eos_pheno:id8}}\end{figure}

\begin{figure}[htbp]
\centering
\capstart

\noindent\sphinxincludegraphics[scale=0.7]{{plot_eos_setupPheno-ESkyrme-E-NM}.png}
\caption{This figure shows the energy in neutron matter (NM) over the free Fermi gas energy (top) and the energy per particle (bottom) as function of the density (left) and the neutron Fermi momentum (right) for the complete list of phenomenological models based on the extended ESkyrme interactions available in the nucleardatapy toolkit.}\label{\detokenize{source/api/setup_eos_pheno:id9}}\end{figure}

\begin{figure}[htbp]
\centering
\capstart

\noindent\sphinxincludegraphics[scale=0.7]{{plot_eos_setupPheno-ESkyrme-E-SM}.png}
\caption{This figure shows the energy in symmetric matter (SM) over the free Fermi gas energy (top) and the energy per particle (bottom) as function of the density (left) and the neutron Fermi momentum (right) for the complete list of phenomenological models based on the extended ESkyrme interactions available in the nucleardatapy toolkit.}\label{\detokenize{source/api/setup_eos_pheno:id10}}\end{figure}

\begin{figure}[htbp]
\centering
\capstart

\noindent\sphinxincludegraphics[scale=0.7]{{plot_EOSNEP}.png}
\caption{Distribution of NEP for phenomenological models available in the nucleardatapy toolkit.}\label{\detokenize{source/api/setup_eos_pheno:id11}}\end{figure}

\sphinxstepscope


\section{SetupEOSPhenoEsym}
\label{\detokenize{source/api/setup_eos_pheno_esym:setupeosphenoesym}}\label{\detokenize{source/api/setup_eos_pheno_esym::doc}}\phantomsection\label{\detokenize{source/api/setup_eos_pheno_esym:module-nucleardatapy.eos.setup_pheno_esym}}\index{module@\spxentry{module}!nucleardatapy.eos.setup\_pheno\_esym@\spxentry{nucleardatapy.eos.setup\_pheno\_esym}}\index{nucleardatapy.eos.setup\_pheno\_esym@\spxentry{nucleardatapy.eos.setup\_pheno\_esym}!module@\spxentry{module}}\index{pheno\_esym\_models() (in module nucleardatapy.eos.setup\_pheno\_esym)@\spxentry{pheno\_esym\_models()}\spxextra{in module nucleardatapy.eos.setup\_pheno\_esym}}

\begin{fulllineitems}
\phantomsection\label{\detokenize{source/api/setup_eos_pheno_esym:nucleardatapy.eos.setup_pheno_esym.pheno_esym_models}}
\pysigstartsignatures
\pysiglinewithargsret{\sphinxcode{\sphinxupquote{nucleardatapy.eos.setup\_pheno\_esym.}}\sphinxbfcode{\sphinxupquote{pheno\_esym\_models}}}{}{}
\pysigstopsignatures
\sphinxAtStartPar
Return a list of models available in this toolkit and print them all on the prompt.
\begin{quote}\begin{description}
\sphinxlineitem{Returns}
\sphinxAtStartPar
The list of models with can be ‘Skyrme’, ‘ESkyrme’, ‘NLRH’, ‘DDRH’, ‘DDRHF’.

\sphinxlineitem{Return type}
\sphinxAtStartPar
list{[}str{]}.

\end{description}\end{quote}

\end{fulllineitems}

\index{pheno\_esym\_params() (in module nucleardatapy.eos.setup\_pheno\_esym)@\spxentry{pheno\_esym\_params()}\spxextra{in module nucleardatapy.eos.setup\_pheno\_esym}}

\begin{fulllineitems}
\phantomsection\label{\detokenize{source/api/setup_eos_pheno_esym:nucleardatapy.eos.setup_pheno_esym.pheno_esym_params}}
\pysigstartsignatures
\pysiglinewithargsret{\sphinxcode{\sphinxupquote{nucleardatapy.eos.setup\_pheno\_esym.}}\sphinxbfcode{\sphinxupquote{pheno\_esym\_params}}}{\emph{\DUrole{n}{model}}}{}
\pysigstopsignatures
\sphinxAtStartPar
Return a list with the parameterizations available in
this toolkit for a given model and print them all on the prompt.
\begin{quote}\begin{description}
\sphinxlineitem{Parameters}
\sphinxAtStartPar
\sphinxstyleliteralstrong{\sphinxupquote{model}} (\sphinxstyleliteralemphasis{\sphinxupquote{str.}}) \textendash{} The type of model for which there are parametrizations.     They should be chosen among the following options: ‘Skyrme’, ‘NLRH’,     ‘DDRH’, ‘DDRHF’.

\sphinxlineitem{Returns}
\sphinxAtStartPar
The list of parametrizations.     If \sphinxtitleref{models} == ‘skyrme’: ‘BSK14’,     ‘BSK16’, ‘BSK17’, ‘BSK27’, ‘F\sphinxhyphen{}’, ‘F+’, ‘F0’, ‘FPL’, ‘LNS’, ‘LNS1’, ‘LNS5’,     ‘NRAPR’, ‘RATP’, ‘SAMI’, ‘SGII’, ‘SIII’, ‘SKGSIGMA’, ‘SKI2’, ‘SKI4’, ‘SKMP’,     ‘SKMS’, ‘SKO’, ‘SKOP’, ‘SKP’, ‘SKRSIGMA’, ‘SKX’, ‘Skz2’, ‘SLY4’, ‘SLY5’,     ‘SLY230A’, ‘SLY230B’, ‘SV’, ‘T6’, ‘T44’, ‘UNEDF0’, ‘UNEDF1’.     If \sphinxtitleref{models} == ‘ESkyrme’: ‘BSk22’, ‘BSk24’, ‘BSk25’, ‘BSk26’, ‘BSk31’, ‘BSk32’,     ‘BSkG1’, ‘BSkG2’, ‘BSkG3’.

\end{description}\end{quote}

\sphinxAtStartPar
If \sphinxtitleref{models} == ‘NLRH’: ‘NL\sphinxhyphen{}SH’, ‘NL3’, ‘NL3II’, ‘PK1’, ‘PK1R’, ‘TM1’.     If \sphinxtitleref{models} == ‘DDRH’: ‘DDME1’, ‘DDME2’, ‘DDMEd’, ‘PKDD’, ‘TW99’.     If \sphinxtitleref{models} == ‘DDRHF’: ‘PKA1’, ‘PKO1’, ‘PKO2’, ‘PKO3’.
:rtype: list{[}str{]}.

\end{fulllineitems}

\index{setupPhenoEsym (class in nucleardatapy.eos.setup\_pheno\_esym)@\spxentry{setupPhenoEsym}\spxextra{class in nucleardatapy.eos.setup\_pheno\_esym}}

\begin{fulllineitems}
\phantomsection\label{\detokenize{source/api/setup_eos_pheno_esym:nucleardatapy.eos.setup_pheno_esym.setupPhenoEsym}}
\pysigstartsignatures
\pysiglinewithargsret{\sphinxbfcode{\sphinxupquote{class\DUrole{w}{  }}}\sphinxcode{\sphinxupquote{nucleardatapy.eos.setup\_pheno\_esym.}}\sphinxbfcode{\sphinxupquote{setupPhenoEsym}}}{\emph{\DUrole{n}{model}\DUrole{o}{=}\DUrole{default_value}{\textquotesingle{}Skyrme\textquotesingle{}}}, \emph{\DUrole{n}{param}\DUrole{o}{=}\DUrole{default_value}{\textquotesingle{}SLY5\textquotesingle{}}}}{}
\pysigstopsignatures
\sphinxAtStartPar
Instantiate the object with results based on phenomenological    interactions and choosen by the toolkit practitioner.     This choice is defined in the variables \sphinxtitleref{model} and \sphinxtitleref{param}.

\sphinxAtStartPar
If \sphinxtitleref{models} == ‘skyrme’, \sphinxtitleref{param} can be: ‘BSK14’,     ‘BSK16’, ‘BSK17’, ‘BSK27’, ‘F\sphinxhyphen{}’, ‘F+’, ‘F0’, ‘FPL’, ‘LNS’, ‘LNS1’, ‘LNS5’,     ‘NRAPR’, ‘RATP’, ‘SAMI’, ‘SGII’, ‘SIII’, ‘SKGSIGMA’, ‘SKI2’, ‘SKI4’, ‘SKMP’,     ‘SKMS’, ‘SKO’, ‘SKOP’, ‘SKP’, ‘SKRSIGMA’, ‘SKX’, ‘Skz2’, ‘SLY4’, ‘SLY5’,     ‘SLY230A’, ‘SLY230B’, ‘SV’, ‘T6’, ‘T44’, ‘UNEDF0’, ‘UNEDF1’.

\sphinxAtStartPar
If \sphinxtitleref{models} == ‘ESkyrme’, \sphinxtitleref{param} can be: ‘BSk22’, ‘BSk24’, ‘BSk25’,     ‘BSk26’, ‘BSk31’, ‘BSk32’, ‘BSkG1’, ‘BSkG2’, ‘BSkG3’.

\sphinxAtStartPar
If \sphinxtitleref{models} == ‘NLRH’, \sphinxtitleref{param} can be: ‘NL\sphinxhyphen{}SH’, ‘NL3’, ‘NL3II’, ‘PK1’, ‘PK1R’, ‘TM1’.

\sphinxAtStartPar
If \sphinxtitleref{models} == ‘DDRH’, \sphinxtitleref{param} can be: ‘DDME1’, ‘DDME2’, ‘DDMEd’, ‘PKDD’, ‘TW99’.

\sphinxAtStartPar
If \sphinxtitleref{models} == ‘DDRHF’, \sphinxtitleref{param} can be: ‘PKA1’, ‘PKO1’, ‘PKO2’, ‘PKO3’.
\begin{quote}\begin{description}
\sphinxlineitem{Parameters}\begin{itemize}
\item {} 
\sphinxAtStartPar
\sphinxstyleliteralstrong{\sphinxupquote{model}} (\sphinxstyleliteralemphasis{\sphinxupquote{str}}\sphinxstyleliteralemphasis{\sphinxupquote{, }}\sphinxstyleliteralemphasis{\sphinxupquote{optional.}}) \textendash{} Fix the name of model: ‘Skyrme’, ‘NLRH’,     ‘DDRH’, ‘DDRHF’. Default value: ‘Skyrme’.

\item {} 
\sphinxAtStartPar
\sphinxstyleliteralstrong{\sphinxupquote{param}} (\sphinxstyleliteralemphasis{\sphinxupquote{str}}\sphinxstyleliteralemphasis{\sphinxupquote{, }}\sphinxstyleliteralemphasis{\sphinxupquote{optional.}}) \textendash{} Fix the parameterization associated to model.     Default value: ‘SLY5’.

\end{itemize}

\end{description}\end{quote}

\sphinxAtStartPar
\sphinxstylestrong{Attributes:}
\index{init\_self() (nucleardatapy.eos.setup\_pheno\_esym.setupPhenoEsym method)@\spxentry{init\_self()}\spxextra{nucleardatapy.eos.setup\_pheno\_esym.setupPhenoEsym method}}

\begin{fulllineitems}
\phantomsection\label{\detokenize{source/api/setup_eos_pheno_esym:nucleardatapy.eos.setup_pheno_esym.setupPhenoEsym.init_self}}
\pysigstartsignatures
\pysiglinewithargsret{\sphinxbfcode{\sphinxupquote{init\_self}}}{}{}
\pysigstopsignatures
\sphinxAtStartPar
Initialize variables in self.

\end{fulllineitems}

\index{model (nucleardatapy.eos.setup\_pheno\_esym.setupPhenoEsym attribute)@\spxentry{model}\spxextra{nucleardatapy.eos.setup\_pheno\_esym.setupPhenoEsym attribute}}

\begin{fulllineitems}
\phantomsection\label{\detokenize{source/api/setup_eos_pheno_esym:nucleardatapy.eos.setup_pheno_esym.setupPhenoEsym.model}}
\pysigstartsignatures
\pysigline{\sphinxbfcode{\sphinxupquote{model}}}
\pysigstopsignatures
\sphinxAtStartPar
Attribute model.

\end{fulllineitems}

\index{param (nucleardatapy.eos.setup\_pheno\_esym.setupPhenoEsym attribute)@\spxentry{param}\spxextra{nucleardatapy.eos.setup\_pheno\_esym.setupPhenoEsym attribute}}

\begin{fulllineitems}
\phantomsection\label{\detokenize{source/api/setup_eos_pheno_esym:nucleardatapy.eos.setup_pheno_esym.setupPhenoEsym.param}}
\pysigstartsignatures
\pysigline{\sphinxbfcode{\sphinxupquote{param}}}
\pysigstopsignatures
\sphinxAtStartPar
Attribute param.

\end{fulllineitems}

\index{print\_outputs() (nucleardatapy.eos.setup\_pheno\_esym.setupPhenoEsym method)@\spxentry{print\_outputs()}\spxextra{nucleardatapy.eos.setup\_pheno\_esym.setupPhenoEsym method}}

\begin{fulllineitems}
\phantomsection\label{\detokenize{source/api/setup_eos_pheno_esym:nucleardatapy.eos.setup_pheno_esym.setupPhenoEsym.print_outputs}}
\pysigstartsignatures
\pysiglinewithargsret{\sphinxbfcode{\sphinxupquote{print\_outputs}}}{}{}
\pysigstopsignatures
\sphinxAtStartPar
Method which print outputs on terminal’s screen.

\end{fulllineitems}


\end{fulllineitems}


\sphinxAtStartPar
Here are a set of figures which are produced with the Python sample: /nucleardatapy\_sample/plots/plot\_eos\_setupPhenoEsym.py

\begin{figure}[htbp]
\centering
\capstart

\noindent\sphinxincludegraphics[scale=0.7]{{plot_eos_setupPhenoEsym}.png}
\caption{This figure shows the symmetry energy as function of the density (left) and the neutron Fermi momentum (right) for the models available in the nucleardatapy toolkit.}\label{\detokenize{source/api/setup_eos_pheno_esym:id1}}\end{figure}

\sphinxstepscope


\section{SetupEOSAM}
\label{\detokenize{source/api/setup_eos_am:setupeosam}}\label{\detokenize{source/api/setup_eos_am::doc}}\phantomsection\label{\detokenize{source/api/setup_eos_am:module-nucleardatapy.eos.setup_am}}\index{module@\spxentry{module}!nucleardatapy.eos.setup\_am@\spxentry{nucleardatapy.eos.setup\_am}}\index{nucleardatapy.eos.setup\_am@\spxentry{nucleardatapy.eos.setup\_am}!module@\spxentry{module}}\index{setupAM (class in nucleardatapy.eos.setup\_am)@\spxentry{setupAM}\spxextra{class in nucleardatapy.eos.setup\_am}}

\begin{fulllineitems}
\phantomsection\label{\detokenize{source/api/setup_eos_am:nucleardatapy.eos.setup_am.setupAM}}
\pysigstartsignatures
\pysiglinewithargsret{\sphinxbfcode{\sphinxupquote{class\DUrole{w}{  }}}\sphinxcode{\sphinxupquote{nucleardatapy.eos.setup\_am.}}\sphinxbfcode{\sphinxupquote{setupAM}}}{\emph{\DUrole{n}{model}\DUrole{o}{=}\DUrole{default_value}{\textquotesingle{}1998\sphinxhyphen{}VAR\sphinxhyphen{}AM\sphinxhyphen{}APR\textquotesingle{}}}, \emph{\DUrole{n}{param}\DUrole{o}{=}\DUrole{default_value}{None}}, \emph{\DUrole{n}{kind}\DUrole{o}{=}\DUrole{default_value}{\textquotesingle{}micro\textquotesingle{}}}, \emph{\DUrole{n}{asy}\DUrole{o}{=}\DUrole{default_value}{0.0}}, \emph{\DUrole{n}{var1}\DUrole{o}{=}\DUrole{default_value}{array({[}0.01, 0.01393939, 0.01787879, 0.02181818, 0.02575758, 0.02969697, 0.03363636, 0.03757576, 0.04151515, 0.04545455, 0.04939394, 0.05333333, 0.05727273, 0.06121212, 0.06515152, 0.06909091, 0.0730303, 0.0769697, 0.08090909, 0.08484848, 0.08878788, 0.09272727, 0.09666667, 0.10060606, 0.10454545, 0.10848485, 0.11242424, 0.11636364, 0.12030303, 0.12424242, 0.12818182, 0.13212121, 0.13606061, 0.14, 0.14393939, 0.14787879, 0.15181818, 0.15575758, 0.15969697, 0.16363636, 0.16757576, 0.17151515, 0.17545455, 0.17939394, 0.18333333, 0.18727273, 0.19121212, 0.19515152, 0.19909091, 0.2030303, 0.2069697, 0.21090909, 0.21484848, 0.21878788, 0.22272727, 0.22666667, 0.23060606, 0.23454545, 0.23848485, 0.24242424, 0.24636364, 0.25030303, 0.25424242, 0.25818182, 0.26212121, 0.26606061, 0.27, 0.27393939, 0.27787879, 0.28181818, 0.28575758, 0.28969697, 0.29363636, 0.29757576, 0.30151515, 0.30545455, 0.30939394, 0.31333333, 0.31727273, 0.32121212, 0.32515152, 0.32909091, 0.3330303, 0.3369697, 0.34090909, 0.34484848, 0.34878788, 0.35272727, 0.35666667, 0.36060606, 0.36454545, 0.36848485, 0.37242424, 0.37636364, 0.38030303, 0.38424242, 0.38818182, 0.39212121, 0.39606061, 0.4{]})}}}{}
\pysigstopsignatures
\sphinxAtStartPar
Instantiate the object with microscopic results choosen     by the toolkit practitioner.
\begin{quote}\begin{description}
\sphinxlineitem{Parameters}\begin{itemize}
\item {} 
\sphinxAtStartPar
\sphinxstyleliteralstrong{\sphinxupquote{model}} (\sphinxstyleliteralemphasis{\sphinxupquote{str}}\sphinxstyleliteralemphasis{\sphinxupquote{, }}\sphinxstyleliteralemphasis{\sphinxupquote{optional.}}) \textendash{} Fix the name of model. Default value: ‘1998\sphinxhyphen{}VAR\sphinxhyphen{}AM\sphinxhyphen{}APR’.

\item {} 
\sphinxAtStartPar
\sphinxstyleliteralstrong{\sphinxupquote{kind}} (\sphinxstyleliteralemphasis{\sphinxupquote{str}}\sphinxstyleliteralemphasis{\sphinxupquote{, }}\sphinxstyleliteralemphasis{\sphinxupquote{optional.}}) \textendash{} chose between ‘micro’ and ‘pheno’.

\end{itemize}

\end{description}\end{quote}

\sphinxAtStartPar
\sphinxstylestrong{Attributes:}
\index{init\_self() (nucleardatapy.eos.setup\_am.setupAM method)@\spxentry{init\_self()}\spxextra{nucleardatapy.eos.setup\_am.setupAM method}}

\begin{fulllineitems}
\phantomsection\label{\detokenize{source/api/setup_eos_am:nucleardatapy.eos.setup_am.setupAM.init_self}}
\pysigstartsignatures
\pysiglinewithargsret{\sphinxbfcode{\sphinxupquote{init\_self}}}{}{}
\pysigstopsignatures
\sphinxAtStartPar
Initialize variables in self.

\end{fulllineitems}

\index{model (nucleardatapy.eos.setup\_am.setupAM attribute)@\spxentry{model}\spxextra{nucleardatapy.eos.setup\_am.setupAM attribute}}

\begin{fulllineitems}
\phantomsection\label{\detokenize{source/api/setup_eos_am:nucleardatapy.eos.setup_am.setupAM.model}}
\pysigstartsignatures
\pysigline{\sphinxbfcode{\sphinxupquote{model}}}
\pysigstopsignatures
\sphinxAtStartPar
Attribute model.

\end{fulllineitems}

\index{print\_outputs() (nucleardatapy.eos.setup\_am.setupAM method)@\spxentry{print\_outputs()}\spxextra{nucleardatapy.eos.setup\_am.setupAM method}}

\begin{fulllineitems}
\phantomsection\label{\detokenize{source/api/setup_eos_am:nucleardatapy.eos.setup_am.setupAM.print_outputs}}
\pysigstartsignatures
\pysiglinewithargsret{\sphinxbfcode{\sphinxupquote{print\_outputs}}}{}{}
\pysigstopsignatures
\sphinxAtStartPar
Method which print outputs on terminal’s screen.

\end{fulllineitems}


\end{fulllineitems}


\sphinxAtStartPar
Here are a set of figures which are produced with the Python sample: /nucleardatapy\_sample/plots/plot\_eos\_setupAM.py

\begin{figure}[htbp]
\centering
\capstart

\noindent\sphinxincludegraphics[scale=0.7]{{plot_eos_setupAM}.png}
\caption{This figure shows the energy per particle in asymmetric matter (AM) with d=0.5. (left) Microscopic models and (right) phenomenological models available in the nucleardatapy toolkit.}\label{\detokenize{source/api/setup_eos_am:id1}}\end{figure}

\sphinxstepscope


\section{SetupEOSBeta}
\label{\detokenize{source/api/setup_eos_beta:setupeosbeta}}\label{\detokenize{source/api/setup_eos_beta::doc}}\phantomsection\label{\detokenize{source/api/setup_eos_beta:module-nucleardatapy.eos.setup_beta}}\index{module@\spxentry{module}!nucleardatapy.eos.setup\_beta@\spxentry{nucleardatapy.eos.setup\_beta}}\index{nucleardatapy.eos.setup\_beta@\spxentry{nucleardatapy.eos.setup\_beta}!module@\spxentry{module}}\index{setupBeta (class in nucleardatapy.eos.setup\_beta)@\spxentry{setupBeta}\spxextra{class in nucleardatapy.eos.setup\_beta}}

\begin{fulllineitems}
\phantomsection\label{\detokenize{source/api/setup_eos_beta:nucleardatapy.eos.setup_beta.setupBeta}}
\pysigstartsignatures
\pysiglinewithargsret{\sphinxbfcode{\sphinxupquote{class\DUrole{w}{  }}}\sphinxcode{\sphinxupquote{nucleardatapy.eos.setup\_beta.}}\sphinxbfcode{\sphinxupquote{setupBeta}}}{\emph{\DUrole{n}{model}\DUrole{o}{=}\DUrole{default_value}{\textquotesingle{}1998\sphinxhyphen{}VAR\sphinxhyphen{}AM\sphinxhyphen{}APR\textquotesingle{}}}, \emph{\DUrole{n}{param}\DUrole{o}{=}\DUrole{default_value}{None}}, \emph{\DUrole{n}{kind}\DUrole{o}{=}\DUrole{default_value}{\textquotesingle{}micro\textquotesingle{}}}, \emph{\DUrole{n}{var1}\DUrole{o}{=}\DUrole{default_value}{array({[}0.01, 0.01393939, 0.01787879, 0.02181818, 0.02575758, 0.02969697, 0.03363636, 0.03757576, 0.04151515, 0.04545455, 0.04939394, 0.05333333, 0.05727273, 0.06121212, 0.06515152, 0.06909091, 0.0730303, 0.0769697, 0.08090909, 0.08484848, 0.08878788, 0.09272727, 0.09666667, 0.10060606, 0.10454545, 0.10848485, 0.11242424, 0.11636364, 0.12030303, 0.12424242, 0.12818182, 0.13212121, 0.13606061, 0.14, 0.14393939, 0.14787879, 0.15181818, 0.15575758, 0.15969697, 0.16363636, 0.16757576, 0.17151515, 0.17545455, 0.17939394, 0.18333333, 0.18727273, 0.19121212, 0.19515152, 0.19909091, 0.2030303, 0.2069697, 0.21090909, 0.21484848, 0.21878788, 0.22272727, 0.22666667, 0.23060606, 0.23454545, 0.23848485, 0.24242424, 0.24636364, 0.25030303, 0.25424242, 0.25818182, 0.26212121, 0.26606061, 0.27, 0.27393939, 0.27787879, 0.28181818, 0.28575758, 0.28969697, 0.29363636, 0.29757576, 0.30151515, 0.30545455, 0.30939394, 0.31333333, 0.31727273, 0.32121212, 0.32515152, 0.32909091, 0.3330303, 0.3369697, 0.34090909, 0.34484848, 0.34878788, 0.35272727, 0.35666667, 0.36060606, 0.36454545, 0.36848485, 0.37242424, 0.37636364, 0.38030303, 0.38424242, 0.38818182, 0.39212121, 0.39606061, 0.4{]})}}}{}
\pysigstopsignatures
\sphinxAtStartPar
Instantiate the object with microscopic results choosen     by the toolkit practitioner.
\begin{quote}\begin{description}
\sphinxlineitem{Parameters}\begin{itemize}
\item {} 
\sphinxAtStartPar
\sphinxstyleliteralstrong{\sphinxupquote{model}} (\sphinxstyleliteralemphasis{\sphinxupquote{str}}\sphinxstyleliteralemphasis{\sphinxupquote{, }}\sphinxstyleliteralemphasis{\sphinxupquote{optional.}}) \textendash{} Fix the name of model. Default value: ‘1998\sphinxhyphen{}VAR\sphinxhyphen{}AM\sphinxhyphen{}APR’.

\item {} 
\sphinxAtStartPar
\sphinxstyleliteralstrong{\sphinxupquote{kind}} (\sphinxstyleliteralemphasis{\sphinxupquote{str}}\sphinxstyleliteralemphasis{\sphinxupquote{, }}\sphinxstyleliteralemphasis{\sphinxupquote{optional.}}) \textendash{} chose between ‘micro’ and ‘pheno’.

\end{itemize}

\end{description}\end{quote}

\sphinxAtStartPar
\sphinxstylestrong{Attributes:}
\index{init\_self() (nucleardatapy.eos.setup\_beta.setupBeta method)@\spxentry{init\_self()}\spxextra{nucleardatapy.eos.setup\_beta.setupBeta method}}

\begin{fulllineitems}
\phantomsection\label{\detokenize{source/api/setup_eos_beta:nucleardatapy.eos.setup_beta.setupBeta.init_self}}
\pysigstartsignatures
\pysiglinewithargsret{\sphinxbfcode{\sphinxupquote{init\_self}}}{}{}
\pysigstopsignatures
\sphinxAtStartPar
Initialize variables in self.

\end{fulllineitems}

\index{model (nucleardatapy.eos.setup\_beta.setupBeta attribute)@\spxentry{model}\spxextra{nucleardatapy.eos.setup\_beta.setupBeta attribute}}

\begin{fulllineitems}
\phantomsection\label{\detokenize{source/api/setup_eos_beta:nucleardatapy.eos.setup_beta.setupBeta.model}}
\pysigstartsignatures
\pysigline{\sphinxbfcode{\sphinxupquote{model}}}
\pysigstopsignatures
\sphinxAtStartPar
Attribute model.

\end{fulllineitems}

\index{print\_outputs() (nucleardatapy.eos.setup\_beta.setupBeta method)@\spxentry{print\_outputs()}\spxextra{nucleardatapy.eos.setup\_beta.setupBeta method}}

\begin{fulllineitems}
\phantomsection\label{\detokenize{source/api/setup_eos_beta:nucleardatapy.eos.setup_beta.setupBeta.print_outputs}}
\pysigstartsignatures
\pysiglinewithargsret{\sphinxbfcode{\sphinxupquote{print\_outputs}}}{}{}
\pysigstopsignatures
\sphinxAtStartPar
Method which print outputs on terminal’s screen.

\end{fulllineitems}


\end{fulllineitems}


\sphinxAtStartPar
Here are a set of figures which are produced with the Python sample: /nucleardatapy\_sample/plots/plot\_eos\_setupBeta.py

\begin{figure}[htbp]
\centering
\capstart

\noindent\sphinxincludegraphics[scale=0.7]{{plot_eos_setupBeta_xp}.png}
\caption{This figure shows the proton fraction at beta\sphinxhyphen{}equilibrium. (left) Microscopic models and (right) phenomenological models available in the nucleardatapy toolkit.}\label{\detokenize{source/api/setup_eos_beta:id1}}\end{figure}

\begin{figure}[htbp]
\centering
\capstart

\noindent\sphinxincludegraphics[scale=0.7]{{plot_eos_setupBeta_xe}.png}
\caption{This figure shows the electron fraction at beta\sphinxhyphen{}equilibrium. (left) Microscopic models and (right) phenomenological models available in the nucleardatapy toolkit.}\label{\detokenize{source/api/setup_eos_beta:id2}}\end{figure}

\begin{figure}[htbp]
\centering
\capstart

\noindent\sphinxincludegraphics[scale=0.7]{{plot_eos_setupBeta_xmu}.png}
\caption{This figure shows the muon fraction at beta\sphinxhyphen{}equilibrium. (left) Microscopic models and (right) phenomenological models available in the nucleardatapy toolkit.}\label{\detokenize{source/api/setup_eos_beta:id3}}\end{figure}

\sphinxstepscope


\section{SetupEOSHIC}
\label{\detokenize{source/api/setup_eos_hic:setupeoshic}}\label{\detokenize{source/api/setup_eos_hic::doc}}\phantomsection\label{\detokenize{source/api/setup_eos_hic:module-nucleardatapy.eos.setup_hic}}\index{module@\spxentry{module}!nucleardatapy.eos.setup\_hic@\spxentry{nucleardatapy.eos.setup\_hic}}\index{nucleardatapy.eos.setup\_hic@\spxentry{nucleardatapy.eos.setup\_hic}!module@\spxentry{module}}\index{hic\_constraints() (in module nucleardatapy.eos.setup\_hic)@\spxentry{hic\_constraints()}\spxextra{in module nucleardatapy.eos.setup\_hic}}

\begin{fulllineitems}
\phantomsection\label{\detokenize{source/api/setup_eos_hic:nucleardatapy.eos.setup_hic.hic_constraints}}
\pysigstartsignatures
\pysiglinewithargsret{\sphinxcode{\sphinxupquote{nucleardatapy.eos.setup\_hic.}}\sphinxbfcode{\sphinxupquote{hic\_constraints}}}{}{}
\pysigstopsignatures
\sphinxAtStartPar
Return a list of the HIC constraints available in this toolkit
for the equation of state in SM and NM and print them all on
the prompt. These constraints are the following
ones: {[} ‘DLL\sphinxhyphen{}2002’, ‘FOPI\sphinxhyphen{}2016’ {]}.
\begin{quote}\begin{description}
\sphinxlineitem{Returns}
\sphinxAtStartPar
The list of constraints.

\sphinxlineitem{Return type}
\sphinxAtStartPar
list{[}str{]}.

\end{description}\end{quote}

\end{fulllineitems}

\index{setupHIC (class in nucleardatapy.eos.setup\_hic)@\spxentry{setupHIC}\spxextra{class in nucleardatapy.eos.setup\_hic}}

\begin{fulllineitems}
\phantomsection\label{\detokenize{source/api/setup_eos_hic:nucleardatapy.eos.setup_hic.setupHIC}}
\pysigstartsignatures
\pysiglinewithargsret{\sphinxbfcode{\sphinxupquote{class\DUrole{w}{  }}}\sphinxcode{\sphinxupquote{nucleardatapy.eos.setup\_hic.}}\sphinxbfcode{\sphinxupquote{setupHIC}}}{\emph{\DUrole{n}{constraint}\DUrole{o}{=}\DUrole{default_value}{\textquotesingle{}DLL\sphinxhyphen{}2002\textquotesingle{}}}}{}
\pysigstopsignatures
\sphinxAtStartPar
Instantiate the constraints on the EOS from HIC.

\sphinxAtStartPar
This choice is defined in the variable \sphinxtitleref{constraint}.

\sphinxAtStartPar
\sphinxtitleref{constraint} can chosen among the following ones: {[} ‘DLL\sphinxhyphen{}2002’, ‘FOPI\sphinxhyphen{}2016’ {]}.
\begin{quote}\begin{description}
\sphinxlineitem{Parameters}
\sphinxAtStartPar
\sphinxstyleliteralstrong{\sphinxupquote{constraint}} (\sphinxstyleliteralemphasis{\sphinxupquote{str}}\sphinxstyleliteralemphasis{\sphinxupquote{, }}\sphinxstyleliteralemphasis{\sphinxupquote{optional.}}) \textendash{} Fix the name of \sphinxtitleref{constraint}. Default value: ‘DLL\sphinxhyphen{}2002’.

\end{description}\end{quote}

\sphinxAtStartPar
\sphinxstylestrong{Attributes:}
\index{init\_self() (nucleardatapy.eos.setup\_hic.setupHIC method)@\spxentry{init\_self()}\spxextra{nucleardatapy.eos.setup\_hic.setupHIC method}}

\begin{fulllineitems}
\phantomsection\label{\detokenize{source/api/setup_eos_hic:nucleardatapy.eos.setup_hic.setupHIC.init_self}}
\pysigstartsignatures
\pysiglinewithargsret{\sphinxbfcode{\sphinxupquote{init\_self}}}{}{}
\pysigstopsignatures
\sphinxAtStartPar
Initialize variables in self.

\end{fulllineitems}

\index{print\_outputs() (nucleardatapy.eos.setup\_hic.setupHIC method)@\spxentry{print\_outputs()}\spxextra{nucleardatapy.eos.setup\_hic.setupHIC method}}

\begin{fulllineitems}
\phantomsection\label{\detokenize{source/api/setup_eos_hic:nucleardatapy.eos.setup_hic.setupHIC.print_outputs}}
\pysigstartsignatures
\pysiglinewithargsret{\sphinxbfcode{\sphinxupquote{print\_outputs}}}{}{}
\pysigstopsignatures
\sphinxAtStartPar
Method which print outputs on terminal’s screen.

\end{fulllineitems}


\end{fulllineitems}


\sphinxAtStartPar
Here are a set of figures which are produced with the Python sample: /nucleardatapy\_sample/plots/plot\_eos\_setupHIC.py

\begin{figure}[htbp]
\centering
\capstart

\noindent\sphinxincludegraphics[scale=0.7]{{plot_eos_setupHIC}.png}
\caption{HIC Experimental constraints for the energy per particle (left) and pressure (right) in SM as a function of the particle density for different analyses available in the \sphinxtitleref{nuda} toolkit.}\label{\detokenize{source/api/setup_eos_hic:id1}}\end{figure}

\sphinxstepscope


\section{SetupEOSEsym}
\label{\detokenize{source/api/setup_eos_esym:setupeosesym}}\label{\detokenize{source/api/setup_eos_esym::doc}}\phantomsection\label{\detokenize{source/api/setup_eos_esym:module-nucleardatapy.eos.setup_esym}}\index{module@\spxentry{module}!nucleardatapy.eos.setup\_esym@\spxentry{nucleardatapy.eos.setup\_esym}}\index{nucleardatapy.eos.setup\_esym@\spxentry{nucleardatapy.eos.setup\_esym}!module@\spxentry{module}}\index{setupEsym (class in nucleardatapy.eos.setup\_esym)@\spxentry{setupEsym}\spxextra{class in nucleardatapy.eos.setup\_esym}}

\begin{fulllineitems}
\phantomsection\label{\detokenize{source/api/setup_eos_esym:nucleardatapy.eos.setup_esym.setupEsym}}
\pysigstartsignatures
\pysiglinewithargsret{\sphinxbfcode{\sphinxupquote{class\DUrole{w}{  }}}\sphinxcode{\sphinxupquote{nucleardatapy.eos.setup\_esym.}}\sphinxbfcode{\sphinxupquote{setupEsym}}}{\emph{\DUrole{n}{constraint}\DUrole{o}{=}\DUrole{default_value}{\textquotesingle{}2014\sphinxhyphen{}IAS\textquotesingle{}}}, \emph{\DUrole{n}{Ksym}\DUrole{o}{=}\DUrole{default_value}{0.0}}}{}
\pysigstopsignatures
\sphinxAtStartPar
Instantiate the values of Esym and Lsym from the constraint.
\begin{quote}\begin{description}
\sphinxlineitem{Parameters}
\sphinxAtStartPar
\sphinxstyleliteralstrong{\sphinxupquote{constraint}} (\sphinxstyleliteralemphasis{\sphinxupquote{str.}}) \textendash{} name of the model: ‘2014\sphinxhyphen{}IAS’, …

\sphinxlineitem{Returns}
\sphinxAtStartPar
constraint, ref, label, note, Esym, Lsym.

\end{description}\end{quote}
\index{constraint (nucleardatapy.eos.setup\_esym.setupEsym attribute)@\spxentry{constraint}\spxextra{nucleardatapy.eos.setup\_esym.setupEsym attribute}}

\begin{fulllineitems}
\phantomsection\label{\detokenize{source/api/setup_eos_esym:nucleardatapy.eos.setup_esym.setupEsym.constraint}}
\pysigstartsignatures
\pysigline{\sphinxbfcode{\sphinxupquote{constraint}}}
\pysigstopsignatures
\sphinxAtStartPar
Attribute the constraint

\end{fulllineitems}

\index{esym\_e2a\_max (nucleardatapy.eos.setup\_esym.setupEsym attribute)@\spxentry{esym\_e2a\_max}\spxextra{nucleardatapy.eos.setup\_esym.setupEsym attribute}}

\begin{fulllineitems}
\phantomsection\label{\detokenize{source/api/setup_eos_esym:nucleardatapy.eos.setup_esym.setupEsym.esym_e2a_max}}
\pysigstartsignatures
\pysigline{\sphinxbfcode{\sphinxupquote{esym\_e2a\_max}}}
\pysigstopsignatures
\sphinxAtStartPar
Attribute the maximal symmetry energy

\end{fulllineitems}

\index{esym\_e2a\_min (nucleardatapy.eos.setup\_esym.setupEsym attribute)@\spxentry{esym\_e2a\_min}\spxextra{nucleardatapy.eos.setup\_esym.setupEsym attribute}}

\begin{fulllineitems}
\phantomsection\label{\detokenize{source/api/setup_eos_esym:nucleardatapy.eos.setup_esym.setupEsym.esym_e2a_min}}
\pysigstartsignatures
\pysigline{\sphinxbfcode{\sphinxupquote{esym\_e2a\_min}}}
\pysigstopsignatures
\sphinxAtStartPar
Attribute the minimal symmetry energy

\end{fulllineitems}

\index{print\_outputs() (nucleardatapy.eos.setup\_esym.setupEsym method)@\spxentry{print\_outputs()}\spxextra{nucleardatapy.eos.setup\_esym.setupEsym method}}

\begin{fulllineitems}
\phantomsection\label{\detokenize{source/api/setup_eos_esym:nucleardatapy.eos.setup_esym.setupEsym.print_outputs}}
\pysigstartsignatures
\pysiglinewithargsret{\sphinxbfcode{\sphinxupquote{print\_outputs}}}{}{}
\pysigstopsignatures
\sphinxAtStartPar
Method which print outputs on terminal’s screen.

\end{fulllineitems}


\end{fulllineitems}


\sphinxAtStartPar
Here are a set of figures which are produced with the Python sample: /nucleardatapy\_sample/plots/plot\_eos\_setupEsym.py

\begin{figure}[htbp]
\centering
\capstart

\noindent\sphinxincludegraphics[scale=0.7]{{plot_eos_setupEsym_-200}.png}
\caption{Uncertainty band for Esym as a function of the density for Ksym=\sphinxhyphen{}200 MeV.}\label{\detokenize{source/api/setup_eos_esym:id1}}\end{figure}

\begin{figure}[htbp]
\centering
\capstart

\noindent\sphinxincludegraphics[scale=0.7]{{plot_eos_setupEsym_0}.png}
\caption{Uncertainty band for Esym as a function of the density for Ksym=0 MeV.}\label{\detokenize{source/api/setup_eos_esym:id2}}\end{figure}

\begin{figure}[htbp]
\centering
\capstart

\noindent\sphinxincludegraphics[scale=0.7]{{plot_eos_setupEsym_200}.png}
\caption{Uncertainty band for Esym as a function of the density for Ksym=200 MeV.}\label{\detokenize{source/api/setup_eos_esym:id3}}\end{figure}

\sphinxstepscope


\section{SetupNucBEExp}
\label{\detokenize{source/api/setup_nuc_be_exp:setupnucbeexp}}\label{\detokenize{source/api/setup_nuc_be_exp::doc}}\phantomsection\label{\detokenize{source/api/setup_nuc_be_exp:module-nucleardatapy.nuc.setup_be_exp}}\index{module@\spxentry{module}!nucleardatapy.nuc.setup\_be\_exp@\spxentry{nucleardatapy.nuc.setup\_be\_exp}}\index{nucleardatapy.nuc.setup\_be\_exp@\spxentry{nucleardatapy.nuc.setup\_be\_exp}!module@\spxentry{module}}\index{be\_exp\_tables() (in module nucleardatapy.nuc.setup\_be\_exp)@\spxentry{be\_exp\_tables()}\spxextra{in module nucleardatapy.nuc.setup\_be\_exp}}

\begin{fulllineitems}
\phantomsection\label{\detokenize{source/api/setup_nuc_be_exp:nucleardatapy.nuc.setup_be_exp.be_exp_tables}}
\pysigstartsignatures
\pysiglinewithargsret{\sphinxcode{\sphinxupquote{nucleardatapy.nuc.setup\_be\_exp.}}\sphinxbfcode{\sphinxupquote{be\_exp\_tables}}}{}{}
\pysigstopsignatures
\sphinxAtStartPar
Return a list of the tables available in this toolkit for the experimental masses and
print them all on the prompt. These tables are the following
ones: ‘AME’.
\begin{quote}\begin{description}
\sphinxlineitem{Returns}
\sphinxAtStartPar
The list of tables.

\sphinxlineitem{Return type}
\sphinxAtStartPar
list{[}str{]}.

\end{description}\end{quote}

\end{fulllineitems}

\index{be\_exp\_versions() (in module nucleardatapy.nuc.setup\_be\_exp)@\spxentry{be\_exp\_versions()}\spxextra{in module nucleardatapy.nuc.setup\_be\_exp}}

\begin{fulllineitems}
\phantomsection\label{\detokenize{source/api/setup_nuc_be_exp:nucleardatapy.nuc.setup_be_exp.be_exp_versions}}
\pysigstartsignatures
\pysiglinewithargsret{\sphinxcode{\sphinxupquote{nucleardatapy.nuc.setup\_be\_exp.}}\sphinxbfcode{\sphinxupquote{be\_exp\_versions}}}{\emph{\DUrole{n}{table}}}{}
\pysigstopsignatures
\sphinxAtStartPar
Return a list of versions of tables available in
this toolkit for a given model and print them all on the prompt.
\begin{quote}\begin{description}
\sphinxlineitem{Parameters}
\sphinxAtStartPar
\sphinxstyleliteralstrong{\sphinxupquote{table}} (\sphinxstyleliteralemphasis{\sphinxupquote{str.}}) \textendash{} The table for which there are different versions.

\sphinxlineitem{Returns}
\sphinxAtStartPar
The list of versions.     If table == ‘AME’: ‘2020’, ‘2016’, ‘2012’.

\sphinxlineitem{Return type}
\sphinxAtStartPar
list{[}str{]}.

\end{description}\end{quote}

\end{fulllineitems}

\index{setupBEExp (class in nucleardatapy.nuc.setup\_be\_exp)@\spxentry{setupBEExp}\spxextra{class in nucleardatapy.nuc.setup\_be\_exp}}

\begin{fulllineitems}
\phantomsection\label{\detokenize{source/api/setup_nuc_be_exp:nucleardatapy.nuc.setup_be_exp.setupBEExp}}
\pysigstartsignatures
\pysiglinewithargsret{\sphinxbfcode{\sphinxupquote{class\DUrole{w}{  }}}\sphinxcode{\sphinxupquote{nucleardatapy.nuc.setup\_be\_exp.}}\sphinxbfcode{\sphinxupquote{setupBEExp}}}{\emph{\DUrole{n}{table}\DUrole{o}{=}\DUrole{default_value}{\textquotesingle{}AME\textquotesingle{}}}, \emph{\DUrole{n}{version}\DUrole{o}{=}\DUrole{default_value}{\textquotesingle{}2020\textquotesingle{}}}}{}
\pysigstopsignatures
\sphinxAtStartPar
Instantiate the experimental nuclear masses from AME mass table.

\sphinxAtStartPar
This choice is defined in the variables \sphinxtitleref{table} and \sphinxtitleref{version}.

\sphinxAtStartPar
\sphinxtitleref{table} can chosen among the following ones: ‘AME’.

\sphinxAtStartPar
\sphinxtitleref{version} can be chosen among the following choices: ‘2020’, ‘2016’, ‘2012’.
\begin{quote}\begin{description}
\sphinxlineitem{Parameters}\begin{itemize}
\item {} 
\sphinxAtStartPar
\sphinxstyleliteralstrong{\sphinxupquote{table}} (\sphinxstyleliteralemphasis{\sphinxupquote{str}}\sphinxstyleliteralemphasis{\sphinxupquote{, }}\sphinxstyleliteralemphasis{\sphinxupquote{optional.}}) \textendash{} Fix the name of \sphinxtitleref{table}. Default value: ‘AME’.

\item {} 
\sphinxAtStartPar
\sphinxstyleliteralstrong{\sphinxupquote{version}} (\sphinxstyleliteralemphasis{\sphinxupquote{str}}\sphinxstyleliteralemphasis{\sphinxupquote{, }}\sphinxstyleliteralemphasis{\sphinxupquote{optional.}}) \textendash{} Fix the name of \sphinxtitleref{version}. Default value: 2020’.

\end{itemize}

\end{description}\end{quote}

\sphinxAtStartPar
\sphinxstylestrong{Attributes:}
\index{Zmax (nucleardatapy.nuc.setup\_be\_exp.setupBEExp attribute)@\spxentry{Zmax}\spxextra{nucleardatapy.nuc.setup\_be\_exp.setupBEExp attribute}}

\begin{fulllineitems}
\phantomsection\label{\detokenize{source/api/setup_nuc_be_exp:nucleardatapy.nuc.setup_be_exp.setupBEExp.Zmax}}
\pysigstartsignatures
\pysigline{\sphinxbfcode{\sphinxupquote{Zmax}}}
\pysigstopsignatures
\sphinxAtStartPar
maximum charge of nuclei present in the table.
\begin{quote}\begin{description}
\sphinxlineitem{Type}
\sphinxAtStartPar
Attribute Zmax

\end{description}\end{quote}

\end{fulllineitems}

\index{dist\_nbNuc (nucleardatapy.nuc.setup\_be\_exp.setupBEExp attribute)@\spxentry{dist\_nbNuc}\spxextra{nucleardatapy.nuc.setup\_be\_exp.setupBEExp attribute}}

\begin{fulllineitems}
\phantomsection\label{\detokenize{source/api/setup_nuc_be_exp:nucleardatapy.nuc.setup_be_exp.setupBEExp.dist_nbNuc}}
\pysigstartsignatures
\pysigline{\sphinxbfcode{\sphinxupquote{dist\_nbNuc}}}
\pysigstopsignatures
\sphinxAtStartPar
attribute number of nuclei discovered per year

\end{fulllineitems}

\index{dist\_year (nucleardatapy.nuc.setup\_be\_exp.setupBEExp attribute)@\spxentry{dist\_year}\spxextra{nucleardatapy.nuc.setup\_be\_exp.setupBEExp attribute}}

\begin{fulllineitems}
\phantomsection\label{\detokenize{source/api/setup_nuc_be_exp:nucleardatapy.nuc.setup_be_exp.setupBEExp.dist_year}}
\pysigstartsignatures
\pysigline{\sphinxbfcode{\sphinxupquote{dist\_year}}}
\pysigstopsignatures
\sphinxAtStartPar
attribute distribution of years

\end{fulllineitems}

\index{drip() (nucleardatapy.nuc.setup\_be\_exp.setupBEExp method)@\spxentry{drip()}\spxextra{nucleardatapy.nuc.setup\_be\_exp.setupBEExp method}}

\begin{fulllineitems}
\phantomsection\label{\detokenize{source/api/setup_nuc_be_exp:nucleardatapy.nuc.setup_be_exp.setupBEExp.drip}}
\pysigstartsignatures
\pysiglinewithargsret{\sphinxbfcode{\sphinxupquote{drip}}}{\emph{\DUrole{n}{Zmax}\DUrole{o}{=}\DUrole{default_value}{95}}}{}
\pysigstopsignatures
\sphinxAtStartPar
Method which find the drip\sphinxhyphen{}line nuclei (on the two sides).
\begin{quote}\begin{description}
\sphinxlineitem{Parameters}
\sphinxAtStartPar
\sphinxstyleliteralstrong{\sphinxupquote{Zmax}} (\sphinxstyleliteralemphasis{\sphinxupquote{int}}\sphinxstyleliteralemphasis{\sphinxupquote{, }}\sphinxstyleliteralemphasis{\sphinxupquote{optional. Default: 95.}}) \textendash{} Fix the maximum charge for the search of the drip line.

\end{description}\end{quote}

\sphinxAtStartPar
\sphinxstylestrong{Attributes:}

\end{fulllineitems}

\index{flagI (nucleardatapy.nuc.setup\_be\_exp.setupBEExp attribute)@\spxentry{flagI}\spxextra{nucleardatapy.nuc.setup\_be\_exp.setupBEExp attribute}}

\begin{fulllineitems}
\phantomsection\label{\detokenize{source/api/setup_nuc_be_exp:nucleardatapy.nuc.setup_be_exp.setupBEExp.flagI}}
\pysigstartsignatures
\pysigline{\sphinxbfcode{\sphinxupquote{flagI}}}
\pysigstopsignatures
\sphinxAtStartPar
Attribute I.

\end{fulllineitems}

\index{flagInterp (nucleardatapy.nuc.setup\_be\_exp.setupBEExp attribute)@\spxentry{flagInterp}\spxextra{nucleardatapy.nuc.setup\_be\_exp.setupBEExp attribute}}

\begin{fulllineitems}
\phantomsection\label{\detokenize{source/api/setup_nuc_be_exp:nucleardatapy.nuc.setup_be_exp.setupBEExp.flagInterp}}
\pysigstartsignatures
\pysigline{\sphinxbfcode{\sphinxupquote{flagInterp}}}
\pysigstopsignatures
\sphinxAtStartPar
Attribute Interp (interpolation). Interp=’y’ is the nucleushas not been measured but is in the table based on interpolation expressions.otherwise Interp = ‘n’ for nuclei produced in laboratory and measured.

\end{fulllineitems}

\index{label (nucleardatapy.nuc.setup\_be\_exp.setupBEExp attribute)@\spxentry{label}\spxextra{nucleardatapy.nuc.setup\_be\_exp.setupBEExp attribute}}

\begin{fulllineitems}
\phantomsection\label{\detokenize{source/api/setup_nuc_be_exp:nucleardatapy.nuc.setup_be_exp.setupBEExp.label}}
\pysigstartsignatures
\pysigline{\sphinxbfcode{\sphinxupquote{label}}}
\pysigstopsignatures
\sphinxAtStartPar
Attribute providing the label the data is references for figures.

\end{fulllineitems}

\index{nbLine (nucleardatapy.nuc.setup\_be\_exp.setupBEExp attribute)@\spxentry{nbLine}\spxextra{nucleardatapy.nuc.setup\_be\_exp.setupBEExp attribute}}

\begin{fulllineitems}
\phantomsection\label{\detokenize{source/api/setup_nuc_be_exp:nucleardatapy.nuc.setup_be_exp.setupBEExp.nbLine}}
\pysigstartsignatures
\pysigline{\sphinxbfcode{\sphinxupquote{nbLine}}}
\pysigstopsignatures
\sphinxAtStartPar
Attribute with the number of line in the file.

\end{fulllineitems}

\index{nbNuc (nucleardatapy.nuc.setup\_be\_exp.setupBEExp attribute)@\spxentry{nbNuc}\spxextra{nucleardatapy.nuc.setup\_be\_exp.setupBEExp attribute}}

\begin{fulllineitems}
\phantomsection\label{\detokenize{source/api/setup_nuc_be_exp:nucleardatapy.nuc.setup_be_exp.setupBEExp.nbNuc}}
\pysigstartsignatures
\pysigline{\sphinxbfcode{\sphinxupquote{nbNuc}}}
\pysigstopsignatures
\sphinxAtStartPar
Attribute with the number of nuclei read in the file.

\end{fulllineitems}

\index{note (nucleardatapy.nuc.setup\_be\_exp.setupBEExp attribute)@\spxentry{note}\spxextra{nucleardatapy.nuc.setup\_be\_exp.setupBEExp attribute}}

\begin{fulllineitems}
\phantomsection\label{\detokenize{source/api/setup_nuc_be_exp:nucleardatapy.nuc.setup_be_exp.setupBEExp.note}}
\pysigstartsignatures
\pysigline{\sphinxbfcode{\sphinxupquote{note}}}
\pysigstopsignatures
\sphinxAtStartPar
Attribute providing additional notes about the data.

\end{fulllineitems}

\index{nucA (nucleardatapy.nuc.setup\_be\_exp.setupBEExp attribute)@\spxentry{nucA}\spxextra{nucleardatapy.nuc.setup\_be\_exp.setupBEExp attribute}}

\begin{fulllineitems}
\phantomsection\label{\detokenize{source/api/setup_nuc_be_exp:nucleardatapy.nuc.setup_be_exp.setupBEExp.nucA}}
\pysigstartsignatures
\pysigline{\sphinxbfcode{\sphinxupquote{nucA}}}
\pysigstopsignatures
\sphinxAtStartPar
Attribute A (mass of the nucleus).

\end{fulllineitems}

\index{nucBE (nucleardatapy.nuc.setup\_be\_exp.setupBEExp attribute)@\spxentry{nucBE}\spxextra{nucleardatapy.nuc.setup\_be\_exp.setupBEExp attribute}}

\begin{fulllineitems}
\phantomsection\label{\detokenize{source/api/setup_nuc_be_exp:nucleardatapy.nuc.setup_be_exp.setupBEExp.nucBE}}
\pysigstartsignatures
\pysigline{\sphinxbfcode{\sphinxupquote{nucBE}}}
\pysigstopsignatures
\sphinxAtStartPar
Attribute BE (Binding Energy) of the nucleus.

\end{fulllineitems}

\index{nucBE\_err (nucleardatapy.nuc.setup\_be\_exp.setupBEExp attribute)@\spxentry{nucBE\_err}\spxextra{nucleardatapy.nuc.setup\_be\_exp.setupBEExp attribute}}

\begin{fulllineitems}
\phantomsection\label{\detokenize{source/api/setup_nuc_be_exp:nucleardatapy.nuc.setup_be_exp.setupBEExp.nucBE_err}}
\pysigstartsignatures
\pysigline{\sphinxbfcode{\sphinxupquote{nucBE\_err}}}
\pysigstopsignatures
\sphinxAtStartPar
Attribute uncertainty in the BE (Binding Energy) of the nucleus.

\end{fulllineitems}

\index{nucHT (nucleardatapy.nuc.setup\_be\_exp.setupBEExp attribute)@\spxentry{nucHT}\spxextra{nucleardatapy.nuc.setup\_be\_exp.setupBEExp attribute}}

\begin{fulllineitems}
\phantomsection\label{\detokenize{source/api/setup_nuc_be_exp:nucleardatapy.nuc.setup_be_exp.setupBEExp.nucHT}}
\pysigstartsignatures
\pysigline{\sphinxbfcode{\sphinxupquote{nucHT}}}
\pysigstopsignatures
\sphinxAtStartPar
Attribute HT (half\sphinxhyphen{}Time) of the nucleus.

\end{fulllineitems}

\index{nucN (nucleardatapy.nuc.setup\_be\_exp.setupBEExp attribute)@\spxentry{nucN}\spxextra{nucleardatapy.nuc.setup\_be\_exp.setupBEExp attribute}}

\begin{fulllineitems}
\phantomsection\label{\detokenize{source/api/setup_nuc_be_exp:nucleardatapy.nuc.setup_be_exp.setupBEExp.nucN}}
\pysigstartsignatures
\pysigline{\sphinxbfcode{\sphinxupquote{nucN}}}
\pysigstopsignatures
\sphinxAtStartPar
Attribute N (number of neutrons of the nucleus).

\end{fulllineitems}

\index{nucStbl (nucleardatapy.nuc.setup\_be\_exp.setupBEExp attribute)@\spxentry{nucStbl}\spxextra{nucleardatapy.nuc.setup\_be\_exp.setupBEExp attribute}}

\begin{fulllineitems}
\phantomsection\label{\detokenize{source/api/setup_nuc_be_exp:nucleardatapy.nuc.setup_be_exp.setupBEExp.nucStbl}}
\pysigstartsignatures
\pysigline{\sphinxbfcode{\sphinxupquote{nucStbl}}}
\pysigstopsignatures
\sphinxAtStartPar
Attribute stbl. stbl=’y’ if the nucleus is stable (according to the table). Otherwise stbl = ‘n’.

\end{fulllineitems}

\index{nucSymb (nucleardatapy.nuc.setup\_be\_exp.setupBEExp attribute)@\spxentry{nucSymb}\spxextra{nucleardatapy.nuc.setup\_be\_exp.setupBEExp attribute}}

\begin{fulllineitems}
\phantomsection\label{\detokenize{source/api/setup_nuc_be_exp:nucleardatapy.nuc.setup_be_exp.setupBEExp.nucSymb}}
\pysigstartsignatures
\pysigline{\sphinxbfcode{\sphinxupquote{nucSymb}}}
\pysigstopsignatures
\sphinxAtStartPar
Attribute symb (symbol) of the element, e.g., Fe.

\end{fulllineitems}

\index{nucYear (nucleardatapy.nuc.setup\_be\_exp.setupBEExp attribute)@\spxentry{nucYear}\spxextra{nucleardatapy.nuc.setup\_be\_exp.setupBEExp attribute}}

\begin{fulllineitems}
\phantomsection\label{\detokenize{source/api/setup_nuc_be_exp:nucleardatapy.nuc.setup_be_exp.setupBEExp.nucYear}}
\pysigstartsignatures
\pysigline{\sphinxbfcode{\sphinxupquote{nucYear}}}
\pysigstopsignatures
\sphinxAtStartPar
Attribute year of the discovery of the nucleus.

\end{fulllineitems}

\index{nucZ (nucleardatapy.nuc.setup\_be\_exp.setupBEExp attribute)@\spxentry{nucZ}\spxextra{nucleardatapy.nuc.setup\_be\_exp.setupBEExp attribute}}

\begin{fulllineitems}
\phantomsection\label{\detokenize{source/api/setup_nuc_be_exp:nucleardatapy.nuc.setup_be_exp.setupBEExp.nucZ}}
\pysigstartsignatures
\pysigline{\sphinxbfcode{\sphinxupquote{nucZ}}}
\pysigstopsignatures
\sphinxAtStartPar
Attribute Z (charge of the nucleus).

\end{fulllineitems}

\index{print\_outputs() (nucleardatapy.nuc.setup\_be\_exp.setupBEExp method)@\spxentry{print\_outputs()}\spxextra{nucleardatapy.nuc.setup\_be\_exp.setupBEExp method}}

\begin{fulllineitems}
\phantomsection\label{\detokenize{source/api/setup_nuc_be_exp:nucleardatapy.nuc.setup_be_exp.setupBEExp.print_outputs}}
\pysigstartsignatures
\pysiglinewithargsret{\sphinxbfcode{\sphinxupquote{print\_outputs}}}{}{}
\pysigstopsignatures
\sphinxAtStartPar
Method which print outputs on terminal’s screen.

\end{fulllineitems}

\index{ref (nucleardatapy.nuc.setup\_be\_exp.setupBEExp attribute)@\spxentry{ref}\spxextra{nucleardatapy.nuc.setup\_be\_exp.setupBEExp attribute}}

\begin{fulllineitems}
\phantomsection\label{\detokenize{source/api/setup_nuc_be_exp:nucleardatapy.nuc.setup_be_exp.setupBEExp.ref}}
\pysigstartsignatures
\pysigline{\sphinxbfcode{\sphinxupquote{ref}}}
\pysigstopsignatures
\sphinxAtStartPar
Attribute providing the full reference to the paper to be citted.

\end{fulllineitems}

\index{select() (nucleardatapy.nuc.setup\_be\_exp.setupBEExp method)@\spxentry{select()}\spxextra{nucleardatapy.nuc.setup\_be\_exp.setupBEExp method}}

\begin{fulllineitems}
\phantomsection\label{\detokenize{source/api/setup_nuc_be_exp:nucleardatapy.nuc.setup_be_exp.setupBEExp.select}}
\pysigstartsignatures
\pysiglinewithargsret{\sphinxbfcode{\sphinxupquote{select}}}{\emph{\DUrole{n}{Amin}\DUrole{o}{=}\DUrole{default_value}{0}}, \emph{\DUrole{n}{Zmin}\DUrole{o}{=}\DUrole{default_value}{0}}, \emph{\DUrole{n}{interp}\DUrole{o}{=}\DUrole{default_value}{\textquotesingle{}n\textquotesingle{}}}, \emph{\DUrole{n}{state}\DUrole{o}{=}\DUrole{default_value}{\textquotesingle{}gs\textquotesingle{}}}, \emph{\DUrole{n}{nucleus}\DUrole{o}{=}\DUrole{default_value}{\textquotesingle{}unstable\textquotesingle{}}}, \emph{\DUrole{n}{every}\DUrole{o}{=}\DUrole{default_value}{1}}}{}
\pysigstopsignatures
\sphinxAtStartPar
Method which select some nuclei from the table according to some criteria.
\begin{quote}\begin{description}
\sphinxlineitem{Parameters}\begin{itemize}
\item {} 
\sphinxAtStartPar
\sphinxstyleliteralstrong{\sphinxupquote{interp}} (\sphinxstyleliteralemphasis{\sphinxupquote{str}}\sphinxstyleliteralemphasis{\sphinxupquote{, }}\sphinxstyleliteralemphasis{\sphinxupquote{optional. Default = \textquotesingle{}n\textquotesingle{}.}}) \textendash{} If interp=’n’, exclude the interpolated nuclei from the selected ones.         If interp=’y’ consider them in the table, in addition to the others.

\item {} 
\sphinxAtStartPar
\sphinxstyleliteralstrong{\sphinxupquote{state}} (\sphinxstyleliteralemphasis{\sphinxupquote{str}}\sphinxstyleliteralemphasis{\sphinxupquote{, }}\sphinxstyleliteralemphasis{\sphinxupquote{optional. Default \textquotesingle{}gs\textquotesingle{}.}}) \textendash{} select the kind of state. If state=’gs’, select nuclei measured in their ground state.

\item {} 
\sphinxAtStartPar
\sphinxstyleliteralstrong{\sphinxupquote{nucleus}} (\sphinxstyleliteralemphasis{\sphinxupquote{str}}\sphinxstyleliteralemphasis{\sphinxupquote{, }}\sphinxstyleliteralemphasis{\sphinxupquote{optional. Default \textquotesingle{}unstable\textquotesingle{}.}}) \textendash{} ‘unstable’.

\end{itemize}

\end{description}\end{quote}

\sphinxAtStartPar
It can be set to ‘stable’, ‘longlive’ (with LT\textgreater{}10 min), ‘shortlive’ (with 10min\textgreater{}LT\textgreater{}1 ns),         ‘veryshortlive’ (with LT\textless{} 1ns)
:param every: consider only 1 out of \sphinxtitleref{every} nuclei in the table.
:type every: int, optional. Default every = 1.

\sphinxAtStartPar
\sphinxstylestrong{Attributes:}

\end{fulllineitems}

\index{select\_year() (nucleardatapy.nuc.setup\_be\_exp.setupBEExp method)@\spxentry{select\_year()}\spxextra{nucleardatapy.nuc.setup\_be\_exp.setupBEExp method}}

\begin{fulllineitems}
\phantomsection\label{\detokenize{source/api/setup_nuc_be_exp:nucleardatapy.nuc.setup_be_exp.setupBEExp.select_year}}
\pysigstartsignatures
\pysiglinewithargsret{\sphinxbfcode{\sphinxupquote{select\_year}}}{\emph{\DUrole{n}{year\_min}\DUrole{o}{=}\DUrole{default_value}{1940}}, \emph{\DUrole{n}{year\_max}\DUrole{o}{=}\DUrole{default_value}{1960}}, \emph{\DUrole{n}{state}\DUrole{o}{=}\DUrole{default_value}{\textquotesingle{}gs\textquotesingle{}}}}{}
\pysigstopsignatures
\sphinxAtStartPar
Method which select some nuclei from the table according to the discovery year.
\begin{quote}\begin{description}
\sphinxlineitem{Parameters}\begin{itemize}
\item {} 
\sphinxAtStartPar
\sphinxstyleliteralstrong{\sphinxupquote{year\_min}} \textendash{} 

\item {} 
\sphinxAtStartPar
\sphinxstyleliteralstrong{\sphinxupquote{year\_max}} \textendash{} 

\item {} 
\sphinxAtStartPar
\sphinxstyleliteralstrong{\sphinxupquote{state}} (\sphinxstyleliteralemphasis{\sphinxupquote{str}}\sphinxstyleliteralemphasis{\sphinxupquote{, }}\sphinxstyleliteralemphasis{\sphinxupquote{optional. Default \textquotesingle{}gs\textquotesingle{}.}}) \textendash{} select the kind of state. If state=’gs’, select nuclei measured in their ground state.

\end{itemize}

\end{description}\end{quote}

\sphinxAtStartPar
\sphinxstylestrong{Attributes:}

\end{fulllineitems}


\end{fulllineitems}


\sphinxAtStartPar
Here are a set of figures which are produced with the Python sample: /nucleardatapy\_sample/plots/plot\_nuc\_setupBEExp.py

\begin{figure}[htbp]
\centering
\capstart

\noindent\sphinxincludegraphics[scale=0.7]{{plot_nuc_setupBEExp_AME_2020}.png}
\caption{The nuclear chart based on AME 2020 table. The different colors correspond to the different measured half\sphinxhyphen{}times of nuclei.}\label{\detokenize{source/api/setup_nuc_be_exp:id1}}\end{figure}

\begin{figure}[htbp]
\centering
\capstart

\noindent\sphinxincludegraphics[scale=0.7]{{plot_nuc_setupBEExp_AME_2020_year}.png}
\caption{Histogram showing the distribution of nuclei per discovery year, since the first one discovered in 1851.}\label{\detokenize{source/api/setup_nuc_be_exp:id2}}\end{figure}

\sphinxstepscope


\section{SetupNucBETheo}
\label{\detokenize{source/api/setup_nuc_be_theo:setupnucbetheo}}\label{\detokenize{source/api/setup_nuc_be_theo::doc}}\phantomsection\label{\detokenize{source/api/setup_nuc_be_theo:module-nucleardatapy.nuc.setup_be_theo}}\index{module@\spxentry{module}!nucleardatapy.nuc.setup\_be\_theo@\spxentry{nucleardatapy.nuc.setup\_be\_theo}}\index{nucleardatapy.nuc.setup\_be\_theo@\spxentry{nucleardatapy.nuc.setup\_be\_theo}!module@\spxentry{module}}\index{be\_theo\_tables() (in module nucleardatapy.nuc.setup\_be\_theo)@\spxentry{be\_theo\_tables()}\spxextra{in module nucleardatapy.nuc.setup\_be\_theo}}

\begin{fulllineitems}
\phantomsection\label{\detokenize{source/api/setup_nuc_be_theo:nucleardatapy.nuc.setup_be_theo.be_theo_tables}}
\pysigstartsignatures
\pysiglinewithargsret{\sphinxcode{\sphinxupquote{nucleardatapy.nuc.setup\_be\_theo.}}\sphinxbfcode{\sphinxupquote{be\_theo\_tables}}}{}{}
\pysigstopsignatures
\sphinxAtStartPar
Return a list of the tables available in this toolkit for the masses
predicted by theoretical approaches and print them all on the prompt.
These tables are the following ones:     {[} ‘1988\sphinxhyphen{}MJ’, ‘1995\sphinxhyphen{}DZ’, ‘1995\sphinxhyphen{}ETFSI’, ‘1995\sphinxhyphen{}FRDM’,     ‘2005\sphinxhyphen{}KTUY’, ‘2007\sphinxhyphen{}HFB14’, ‘2010\sphinxhyphen{}WS3’, ‘2010\sphinxhyphen{}HFB21’, ‘2011\sphinxhyphen{}WS3’, ‘2013\sphinxhyphen{}HFB22’,     ‘2013\sphinxhyphen{}HFB23’, ‘2013\sphinxhyphen{}HFB24’, ‘2013\sphinxhyphen{}HFB25’, ‘2013\sphinxhyphen{}HFB26’, ‘2021\sphinxhyphen{}BSkG1’,     ‘2022\sphinxhyphen{}BSkG2’, ‘2023\sphinxhyphen{}BSkG3’ {]}
\begin{quote}\begin{description}
\sphinxlineitem{Returns}
\sphinxAtStartPar
The list of tables.

\sphinxlineitem{Return type}
\sphinxAtStartPar
list{[}str{]}.

\end{description}\end{quote}

\end{fulllineitems}

\index{conversionMBE() (in module nucleardatapy.nuc.setup\_be\_theo)@\spxentry{conversionMBE()}\spxextra{in module nucleardatapy.nuc.setup\_be\_theo}}

\begin{fulllineitems}
\phantomsection\label{\detokenize{source/api/setup_nuc_be_theo:nucleardatapy.nuc.setup_be_theo.conversionMBE}}
\pysigstartsignatures
\pysiglinewithargsret{\sphinxcode{\sphinxupquote{nucleardatapy.nuc.setup\_be\_theo.}}\sphinxbfcode{\sphinxupquote{conversionMBE}}}{\emph{\DUrole{n}{M}}, \emph{\DUrole{n}{N}}, \emph{\DUrole{n}{Z}}}{}
\pysigstopsignatures
\sphinxAtStartPar
Convert the mass excess of a nucleus to its binding energy.

\end{fulllineitems}

\index{setupBETheo (class in nucleardatapy.nuc.setup\_be\_theo)@\spxentry{setupBETheo}\spxextra{class in nucleardatapy.nuc.setup\_be\_theo}}

\begin{fulllineitems}
\phantomsection\label{\detokenize{source/api/setup_nuc_be_theo:nucleardatapy.nuc.setup_be_theo.setupBETheo}}
\pysigstartsignatures
\pysiglinewithargsret{\sphinxbfcode{\sphinxupquote{class\DUrole{w}{  }}}\sphinxcode{\sphinxupquote{nucleardatapy.nuc.setup\_be\_theo.}}\sphinxbfcode{\sphinxupquote{setupBETheo}}}{\emph{\DUrole{n}{table}\DUrole{o}{=}\DUrole{default_value}{\textquotesingle{}1995\sphinxhyphen{}DZ\textquotesingle{}}}}{}
\pysigstopsignatures
\sphinxAtStartPar
Instantiate the theory nuclear masses.

\sphinxAtStartPar
This choice is defined in the variable \sphinxtitleref{table}.

\sphinxAtStartPar
\sphinxtitleref{table} can chosen among the following ones:     {[} ‘1988\sphinxhyphen{}MJ’, ‘1995\sphinxhyphen{}DZ’, ‘1995\sphinxhyphen{}ETFSI’, ‘1995\sphinxhyphen{}FRDM’,     ‘2005\sphinxhyphen{}KTUY’, ‘2007\sphinxhyphen{}HFB14’, ‘2010\sphinxhyphen{}WS3’, ‘2010\sphinxhyphen{}HFB21’,’2011\sphinxhyphen{}WS3’, ‘2013\sphinxhyphen{}HFB26’, ‘2021\sphinxhyphen{}BSkG1’,     ‘2022\sphinxhyphen{}BSkG2’, ‘2023\sphinxhyphen{}BSkG3’ {]}
\begin{quote}\begin{description}
\sphinxlineitem{Parameters}
\sphinxAtStartPar
\sphinxstyleliteralstrong{\sphinxupquote{table}} (\sphinxstyleliteralemphasis{\sphinxupquote{str}}\sphinxstyleliteralemphasis{\sphinxupquote{, }}\sphinxstyleliteralemphasis{\sphinxupquote{optional.}}) \textendash{} Fix the name of \sphinxtitleref{table}. Default value: ‘1995\sphinxhyphen{}DZ’.

\end{description}\end{quote}

\sphinxAtStartPar
\sphinxstylestrong{Attributes:}
\index{diff() (nucleardatapy.nuc.setup\_be\_theo.setupBETheo method)@\spxentry{diff()}\spxextra{nucleardatapy.nuc.setup\_be\_theo.setupBETheo method}}

\begin{fulllineitems}
\phantomsection\label{\detokenize{source/api/setup_nuc_be_theo:nucleardatapy.nuc.setup_be_theo.setupBETheo.diff}}
\pysigstartsignatures
\pysiglinewithargsret{\sphinxbfcode{\sphinxupquote{diff}}}{\emph{\DUrole{n}{table}}, \emph{\DUrole{n}{Zref}\DUrole{o}{=}\DUrole{default_value}{50}}}{}
\pysigstopsignatures
\sphinxAtStartPar
Method calculates the difference between a given mass
model and table\_ref.
\begin{quote}\begin{description}
\sphinxlineitem{Parameters}\begin{itemize}
\item {} 
\sphinxAtStartPar
\sphinxstyleliteralstrong{\sphinxupquote{table}} (\sphinxstyleliteralemphasis{\sphinxupquote{str.}}) \textendash{} Fix the table to analyze.

\item {} 
\sphinxAtStartPar
\sphinxstyleliteralstrong{\sphinxupquote{Zref}} (\sphinxstyleliteralemphasis{\sphinxupquote{int}}\sphinxstyleliteralemphasis{\sphinxupquote{, }}\sphinxstyleliteralemphasis{\sphinxupquote{optional. Default: 50.}}) \textendash{} Fix the isotopic chain to study.

\end{itemize}

\end{description}\end{quote}

\sphinxAtStartPar
\sphinxstylestrong{Attributes:}

\end{fulllineitems}

\index{diff\_exp() (nucleardatapy.nuc.setup\_be\_theo.setupBETheo method)@\spxentry{diff\_exp()}\spxextra{nucleardatapy.nuc.setup\_be\_theo.setupBETheo method}}

\begin{fulllineitems}
\phantomsection\label{\detokenize{source/api/setup_nuc_be_theo:nucleardatapy.nuc.setup_be_theo.setupBETheo.diff_exp}}
\pysigstartsignatures
\pysiglinewithargsret{\sphinxbfcode{\sphinxupquote{diff\_exp}}}{\emph{\DUrole{n}{table\_exp}}, \emph{\DUrole{n}{version\_exp}}, \emph{\DUrole{n}{Zref}\DUrole{o}{=}\DUrole{default_value}{50}}}{}
\pysigstopsignatures
\sphinxAtStartPar
Method calculates the difference between a given experimental
mass (identified by \sphinxtitleref{table\_exp} and \sphinxtitleref{version\_exp}) and table\_ref.
\begin{quote}\begin{description}
\sphinxlineitem{Parameters}\begin{itemize}
\item {} 
\sphinxAtStartPar
\sphinxstyleliteralstrong{\sphinxupquote{table}} (\sphinxstyleliteralemphasis{\sphinxupquote{str.}}) \textendash{} Fix the table to analyze.

\item {} 
\sphinxAtStartPar
\sphinxstyleliteralstrong{\sphinxupquote{Zref}} (\sphinxstyleliteralemphasis{\sphinxupquote{int}}\sphinxstyleliteralemphasis{\sphinxupquote{, }}\sphinxstyleliteralemphasis{\sphinxupquote{optional. Default: 50.}}) \textendash{} Fix the isotopic chain to study.

\end{itemize}

\end{description}\end{quote}

\sphinxAtStartPar
\sphinxstylestrong{Attributes:}

\end{fulllineitems}

\index{drip() (nucleardatapy.nuc.setup\_be\_theo.setupBETheo method)@\spxentry{drip()}\spxextra{nucleardatapy.nuc.setup\_be\_theo.setupBETheo method}}

\begin{fulllineitems}
\phantomsection\label{\detokenize{source/api/setup_nuc_be_theo:nucleardatapy.nuc.setup_be_theo.setupBETheo.drip}}
\pysigstartsignatures
\pysiglinewithargsret{\sphinxbfcode{\sphinxupquote{drip}}}{\emph{\DUrole{n}{Zmax}\DUrole{o}{=}\DUrole{default_value}{95}}}{}
\pysigstopsignatures
\sphinxAtStartPar
Method which find the drip\sphinxhyphen{}line nuclei (on the two sides).
\begin{quote}\begin{description}
\sphinxlineitem{Parameters}
\sphinxAtStartPar
\sphinxstyleliteralstrong{\sphinxupquote{Zmax}} (\sphinxstyleliteralemphasis{\sphinxupquote{int}}\sphinxstyleliteralemphasis{\sphinxupquote{, }}\sphinxstyleliteralemphasis{\sphinxupquote{optional. Default: 95.}}) \textendash{} Fix the maximum charge for the search of the drip line.

\end{description}\end{quote}

\sphinxAtStartPar
\sphinxstylestrong{Attributes:}

\end{fulllineitems}

\index{init\_self() (nucleardatapy.nuc.setup\_be\_theo.setupBETheo method)@\spxentry{init\_self()}\spxextra{nucleardatapy.nuc.setup\_be\_theo.setupBETheo method}}

\begin{fulllineitems}
\phantomsection\label{\detokenize{source/api/setup_nuc_be_theo:nucleardatapy.nuc.setup_be_theo.setupBETheo.init_self}}
\pysigstartsignatures
\pysiglinewithargsret{\sphinxbfcode{\sphinxupquote{init\_self}}}{}{}
\pysigstopsignatures
\sphinxAtStartPar
Initialize variables in self.

\end{fulllineitems}

\index{print\_outputs() (nucleardatapy.nuc.setup\_be\_theo.setupBETheo method)@\spxentry{print\_outputs()}\spxextra{nucleardatapy.nuc.setup\_be\_theo.setupBETheo method}}

\begin{fulllineitems}
\phantomsection\label{\detokenize{source/api/setup_nuc_be_theo:nucleardatapy.nuc.setup_be_theo.setupBETheo.print_outputs}}
\pysigstartsignatures
\pysiglinewithargsret{\sphinxbfcode{\sphinxupquote{print\_outputs}}}{}{}
\pysigstopsignatures
\sphinxAtStartPar
Method which print outputs on terminal’s screen.

\end{fulllineitems}


\end{fulllineitems}


\sphinxAtStartPar
Here are a set of figures which are produced with the Python sample: /nucleardatapy\_sample/plots/plot\_nuc\_setupBETheo.py

\begin{figure}[htbp]
\centering
\capstart

\noindent\sphinxincludegraphics[scale=0.7]{{plot_nuc_setupBETheo_Zref20}.png}
\caption{Differences between binding energies predicted by different models with respect to the one predicted by Duflo\sphinxhyphen{}Zuker for Z = 20.}\label{\detokenize{source/api/setup_nuc_be_theo:id1}}\end{figure}

\begin{figure}[htbp]
\centering
\capstart

\noindent\sphinxincludegraphics[scale=0.7]{{plot_nuc_setupBETheo_Zref50}.png}
\caption{Differences between binding energies predicted by different models with respect to the one predicted by Duflo\sphinxhyphen{}Zuker for Z = 50.}\label{\detokenize{source/api/setup_nuc_be_theo:id2}}\end{figure}

\sphinxstepscope


\section{SetupNucRchExp}
\label{\detokenize{source/api/setup_nuc_rch_exp:setupnucrchexp}}\label{\detokenize{source/api/setup_nuc_rch_exp::doc}}\phantomsection\label{\detokenize{source/api/setup_nuc_rch_exp:module-nucleardatapy.nuc.setup_rch_exp}}\index{module@\spxentry{module}!nucleardatapy.nuc.setup\_rch\_exp@\spxentry{nucleardatapy.nuc.setup\_rch\_exp}}\index{nucleardatapy.nuc.setup\_rch\_exp@\spxentry{nucleardatapy.nuc.setup\_rch\_exp}!module@\spxentry{module}}\index{rch\_exp\_tables() (in module nucleardatapy.nuc.setup\_rch\_exp)@\spxentry{rch\_exp\_tables()}\spxextra{in module nucleardatapy.nuc.setup\_rch\_exp}}

\begin{fulllineitems}
\phantomsection\label{\detokenize{source/api/setup_nuc_rch_exp:nucleardatapy.nuc.setup_rch_exp.rch_exp_tables}}
\pysigstartsignatures
\pysiglinewithargsret{\sphinxcode{\sphinxupquote{nucleardatapy.nuc.setup\_rch\_exp.}}\sphinxbfcode{\sphinxupquote{rch\_exp\_tables}}}{}{}
\pysigstopsignatures
\sphinxAtStartPar
Return a list of the tables available in this toolkit for the charge radiuus and
print them all on the prompt.  These tables are the following
ones: ‘2013\sphinxhyphen{}Angeli’.
\begin{quote}\begin{description}
\sphinxlineitem{Returns}
\sphinxAtStartPar
The list of tables.

\sphinxlineitem{Return type}
\sphinxAtStartPar
list{[}str{]}.

\end{description}\end{quote}

\end{fulllineitems}

\index{setupRchExp (class in nucleardatapy.nuc.setup\_rch\_exp)@\spxentry{setupRchExp}\spxextra{class in nucleardatapy.nuc.setup\_rch\_exp}}

\begin{fulllineitems}
\phantomsection\label{\detokenize{source/api/setup_nuc_rch_exp:nucleardatapy.nuc.setup_rch_exp.setupRchExp}}
\pysigstartsignatures
\pysiglinewithargsret{\sphinxbfcode{\sphinxupquote{class\DUrole{w}{  }}}\sphinxcode{\sphinxupquote{nucleardatapy.nuc.setup\_rch\_exp.}}\sphinxbfcode{\sphinxupquote{setupRchExp}}}{\emph{\DUrole{n}{table}\DUrole{o}{=}\DUrole{default_value}{\textquotesingle{}2013\sphinxhyphen{}Angeli\textquotesingle{}}}}{}
\pysigstopsignatures
\sphinxAtStartPar
Instantiate the object with charge radii choosen    from a table.

\sphinxAtStartPar
This choice is defined in the variable \sphinxtitleref{table}.

\sphinxAtStartPar
The tables can chosen among the following ones:    ‘2013\sphinxhyphen{}Angeli’.
\begin{quote}\begin{description}
\sphinxlineitem{Parameters}
\sphinxAtStartPar
\sphinxstyleliteralstrong{\sphinxupquote{table}} (\sphinxstyleliteralemphasis{\sphinxupquote{str}}\sphinxstyleliteralemphasis{\sphinxupquote{, }}\sphinxstyleliteralemphasis{\sphinxupquote{optional.}}) \textendash{} Fix the name of \sphinxtitleref{table}. Default value: ‘2013\sphinxhyphen{}Angeli’.

\end{description}\end{quote}

\sphinxAtStartPar
\sphinxstylestrong{Attributes:}
\index{R\_unit (nucleardatapy.nuc.setup\_rch\_exp.setupRchExp attribute)@\spxentry{R\_unit}\spxextra{nucleardatapy.nuc.setup\_rch\_exp.setupRchExp attribute}}

\begin{fulllineitems}
\phantomsection\label{\detokenize{source/api/setup_nuc_rch_exp:nucleardatapy.nuc.setup_rch_exp.setupRchExp.R_unit}}
\pysigstartsignatures
\pysigline{\sphinxbfcode{\sphinxupquote{R\_unit}}}
\pysigstopsignatures
\sphinxAtStartPar
Attribute radius unit.

\end{fulllineitems}

\index{Rch\_isotopes() (nucleardatapy.nuc.setup\_rch\_exp.setupRchExp method)@\spxentry{Rch\_isotopes()}\spxextra{nucleardatapy.nuc.setup\_rch\_exp.setupRchExp method}}

\begin{fulllineitems}
\phantomsection\label{\detokenize{source/api/setup_nuc_rch_exp:nucleardatapy.nuc.setup_rch_exp.setupRchExp.Rch_isotopes}}
\pysigstartsignatures
\pysiglinewithargsret{\sphinxbfcode{\sphinxupquote{Rch\_isotopes}}}{\emph{\DUrole{n}{Zref}\DUrole{o}{=}\DUrole{default_value}{50}}}{}
\pysigstopsignatures
\sphinxAtStartPar
This method provide a list if radii for an isotopic chain defined by Zref.

\end{fulllineitems}

\index{label (nucleardatapy.nuc.setup\_rch\_exp.setupRchExp attribute)@\spxentry{label}\spxextra{nucleardatapy.nuc.setup\_rch\_exp.setupRchExp attribute}}

\begin{fulllineitems}
\phantomsection\label{\detokenize{source/api/setup_nuc_rch_exp:nucleardatapy.nuc.setup_rch_exp.setupRchExp.label}}
\pysigstartsignatures
\pysigline{\sphinxbfcode{\sphinxupquote{label}}}
\pysigstopsignatures
\sphinxAtStartPar
Attribute providing the label the data is references for figures.

\end{fulllineitems}

\index{note (nucleardatapy.nuc.setup\_rch\_exp.setupRchExp attribute)@\spxentry{note}\spxextra{nucleardatapy.nuc.setup\_rch\_exp.setupRchExp attribute}}

\begin{fulllineitems}
\phantomsection\label{\detokenize{source/api/setup_nuc_rch_exp:nucleardatapy.nuc.setup_rch_exp.setupRchExp.note}}
\pysigstartsignatures
\pysigline{\sphinxbfcode{\sphinxupquote{note}}}
\pysigstopsignatures
\sphinxAtStartPar
Attribute providing additional notes about the data.

\end{fulllineitems}

\index{nucA (nucleardatapy.nuc.setup\_rch\_exp.setupRchExp attribute)@\spxentry{nucA}\spxextra{nucleardatapy.nuc.setup\_rch\_exp.setupRchExp attribute}}

\begin{fulllineitems}
\phantomsection\label{\detokenize{source/api/setup_nuc_rch_exp:nucleardatapy.nuc.setup_rch_exp.setupRchExp.nucA}}
\pysigstartsignatures
\pysigline{\sphinxbfcode{\sphinxupquote{nucA}}}
\pysigstopsignatures
\sphinxAtStartPar
Attribute A (mass of the nucleus).

\end{fulllineitems}

\index{nucN (nucleardatapy.nuc.setup\_rch\_exp.setupRchExp attribute)@\spxentry{nucN}\spxextra{nucleardatapy.nuc.setup\_rch\_exp.setupRchExp attribute}}

\begin{fulllineitems}
\phantomsection\label{\detokenize{source/api/setup_nuc_rch_exp:nucleardatapy.nuc.setup_rch_exp.setupRchExp.nucN}}
\pysigstartsignatures
\pysigline{\sphinxbfcode{\sphinxupquote{nucN}}}
\pysigstopsignatures
\sphinxAtStartPar
Attribute N (number of neutrons of the nucleus).

\end{fulllineitems}

\index{nucRch (nucleardatapy.nuc.setup\_rch\_exp.setupRchExp attribute)@\spxentry{nucRch}\spxextra{nucleardatapy.nuc.setup\_rch\_exp.setupRchExp attribute}}

\begin{fulllineitems}
\phantomsection\label{\detokenize{source/api/setup_nuc_rch_exp:nucleardatapy.nuc.setup_rch_exp.setupRchExp.nucRch}}
\pysigstartsignatures
\pysigline{\sphinxbfcode{\sphinxupquote{nucRch}}}
\pysigstopsignatures
\sphinxAtStartPar
Attribue R\_ch (charge radius) in fm.

\end{fulllineitems}

\index{nucRch\_err (nucleardatapy.nuc.setup\_rch\_exp.setupRchExp attribute)@\spxentry{nucRch\_err}\spxextra{nucleardatapy.nuc.setup\_rch\_exp.setupRchExp attribute}}

\begin{fulllineitems}
\phantomsection\label{\detokenize{source/api/setup_nuc_rch_exp:nucleardatapy.nuc.setup_rch_exp.setupRchExp.nucRch_err}}
\pysigstartsignatures
\pysigline{\sphinxbfcode{\sphinxupquote{nucRch\_err}}}
\pysigstopsignatures
\sphinxAtStartPar
Attribue uncertainty in R\_ch (charge radius) in fm.

\end{fulllineitems}

\index{nucSymb (nucleardatapy.nuc.setup\_rch\_exp.setupRchExp attribute)@\spxentry{nucSymb}\spxextra{nucleardatapy.nuc.setup\_rch\_exp.setupRchExp attribute}}

\begin{fulllineitems}
\phantomsection\label{\detokenize{source/api/setup_nuc_rch_exp:nucleardatapy.nuc.setup_rch_exp.setupRchExp.nucSymb}}
\pysigstartsignatures
\pysigline{\sphinxbfcode{\sphinxupquote{nucSymb}}}
\pysigstopsignatures
\sphinxAtStartPar
Attribute symb (symbol) of the element, e.g., Fe.

\end{fulllineitems}

\index{nucZ (nucleardatapy.nuc.setup\_rch\_exp.setupRchExp attribute)@\spxentry{nucZ}\spxextra{nucleardatapy.nuc.setup\_rch\_exp.setupRchExp attribute}}

\begin{fulllineitems}
\phantomsection\label{\detokenize{source/api/setup_nuc_rch_exp:nucleardatapy.nuc.setup_rch_exp.setupRchExp.nucZ}}
\pysigstartsignatures
\pysigline{\sphinxbfcode{\sphinxupquote{nucZ}}}
\pysigstopsignatures
\sphinxAtStartPar
Attribute Z (charge of the nucleus).

\end{fulllineitems}

\index{print\_outputs() (nucleardatapy.nuc.setup\_rch\_exp.setupRchExp method)@\spxentry{print\_outputs()}\spxextra{nucleardatapy.nuc.setup\_rch\_exp.setupRchExp method}}

\begin{fulllineitems}
\phantomsection\label{\detokenize{source/api/setup_nuc_rch_exp:nucleardatapy.nuc.setup_rch_exp.setupRchExp.print_outputs}}
\pysigstartsignatures
\pysiglinewithargsret{\sphinxbfcode{\sphinxupquote{print\_outputs}}}{}{}
\pysigstopsignatures
\sphinxAtStartPar
Method which print outputs on terminal’s screen.

\end{fulllineitems}

\index{ref (nucleardatapy.nuc.setup\_rch\_exp.setupRchExp attribute)@\spxentry{ref}\spxextra{nucleardatapy.nuc.setup\_rch\_exp.setupRchExp attribute}}

\begin{fulllineitems}
\phantomsection\label{\detokenize{source/api/setup_nuc_rch_exp:nucleardatapy.nuc.setup_rch_exp.setupRchExp.ref}}
\pysigstartsignatures
\pysigline{\sphinxbfcode{\sphinxupquote{ref}}}
\pysigstopsignatures
\sphinxAtStartPar
Attribute providing the full reference to the paper to be citted.

\end{fulllineitems}


\end{fulllineitems}


\sphinxAtStartPar
Here are a set of figures which are produced with the Python sample: /nucleardatapy\_sample/plots/plot\_nuc\_setupRchExp.py

\begin{figure}[htbp]
\centering
\capstart

\noindent\sphinxincludegraphics[scale=0.7]{{plot_nuc_setupRchExp}.png}
\caption{Charge radii for Zn, Sn, and Pb isotopes and for the models available in the nuda toolkit.}\label{\detokenize{source/api/setup_nuc_rch_exp:id1}}\end{figure}

\sphinxstepscope


\section{SetupNucRchTheo}
\label{\detokenize{source/api/setup_nuc_rch_theo:setupnucrchtheo}}\label{\detokenize{source/api/setup_nuc_rch_theo::doc}}\phantomsection\label{\detokenize{source/api/setup_nuc_rch_theo:module-nucleardatapy.nuc.setup_rch_theo}}\index{module@\spxentry{module}!nucleardatapy.nuc.setup\_rch\_theo@\spxentry{nucleardatapy.nuc.setup\_rch\_theo}}\index{nucleardatapy.nuc.setup\_rch\_theo@\spxentry{nucleardatapy.nuc.setup\_rch\_theo}!module@\spxentry{module}}\index{rch\_theo\_tables() (in module nucleardatapy.nuc.setup\_rch\_theo)@\spxentry{rch\_theo\_tables()}\spxextra{in module nucleardatapy.nuc.setup\_rch\_theo}}

\begin{fulllineitems}
\phantomsection\label{\detokenize{source/api/setup_nuc_rch_theo:nucleardatapy.nuc.setup_rch_theo.rch_theo_tables}}
\pysigstartsignatures
\pysiglinewithargsret{\sphinxcode{\sphinxupquote{nucleardatapy.nuc.setup\_rch\_theo.}}\sphinxbfcode{\sphinxupquote{rch\_theo\_tables}}}{}{}
\pysigstopsignatures
\sphinxAtStartPar
Return a list of the tables available in this toolkit for the charge radiuus and
print them all on the prompt.  These tables are the following
ones: ‘2013\sphinxhyphen{}Angeli’.
\begin{quote}\begin{description}
\sphinxlineitem{Returns}
\sphinxAtStartPar
The list of tables.

\sphinxlineitem{Return type}
\sphinxAtStartPar
list{[}str{]}.

\end{description}\end{quote}

\end{fulllineitems}

\index{setupRchTheo (class in nucleardatapy.nuc.setup\_rch\_theo)@\spxentry{setupRchTheo}\spxextra{class in nucleardatapy.nuc.setup\_rch\_theo}}

\begin{fulllineitems}
\phantomsection\label{\detokenize{source/api/setup_nuc_rch_theo:nucleardatapy.nuc.setup_rch_theo.setupRchTheo}}
\pysigstartsignatures
\pysiglinewithargsret{\sphinxbfcode{\sphinxupquote{class\DUrole{w}{  }}}\sphinxcode{\sphinxupquote{nucleardatapy.nuc.setup\_rch\_theo.}}\sphinxbfcode{\sphinxupquote{setupRchTheo}}}{\emph{\DUrole{n}{table}\DUrole{o}{=}\DUrole{default_value}{\textquotesingle{}2021\sphinxhyphen{}BSkG1\textquotesingle{}}}}{}
\pysigstopsignatures
\sphinxAtStartPar
Instantiate the object with charge radii choosen    from a table.

\sphinxAtStartPar
This choice is defined in the variable \sphinxtitleref{table}.

\sphinxAtStartPar
The tables can chosen among the following ones:    ‘2013\sphinxhyphen{}Angeli’.
\begin{quote}\begin{description}
\sphinxlineitem{Parameters}
\sphinxAtStartPar
\sphinxstyleliteralstrong{\sphinxupquote{table}} (\sphinxstyleliteralemphasis{\sphinxupquote{str}}\sphinxstyleliteralemphasis{\sphinxupquote{, }}\sphinxstyleliteralemphasis{\sphinxupquote{optional.}}) \textendash{} Fix the name of \sphinxtitleref{table}. Default value: ‘2013\sphinxhyphen{}Angeli’.

\end{description}\end{quote}

\sphinxAtStartPar
\sphinxstylestrong{Attributes:}
\index{R\_unit (nucleardatapy.nuc.setup\_rch\_theo.setupRchTheo attribute)@\spxentry{R\_unit}\spxextra{nucleardatapy.nuc.setup\_rch\_theo.setupRchTheo attribute}}

\begin{fulllineitems}
\phantomsection\label{\detokenize{source/api/setup_nuc_rch_theo:nucleardatapy.nuc.setup_rch_theo.setupRchTheo.R_unit}}
\pysigstartsignatures
\pysigline{\sphinxbfcode{\sphinxupquote{R\_unit}}}
\pysigstopsignatures
\sphinxAtStartPar
Attribute radius unit.

\end{fulllineitems}

\index{Rch\_isotopes() (nucleardatapy.nuc.setup\_rch\_theo.setupRchTheo method)@\spxentry{Rch\_isotopes()}\spxextra{nucleardatapy.nuc.setup\_rch\_theo.setupRchTheo method}}

\begin{fulllineitems}
\phantomsection\label{\detokenize{source/api/setup_nuc_rch_theo:nucleardatapy.nuc.setup_rch_theo.setupRchTheo.Rch_isotopes}}
\pysigstartsignatures
\pysiglinewithargsret{\sphinxbfcode{\sphinxupquote{Rch\_isotopes}}}{\emph{\DUrole{n}{Zref}\DUrole{o}{=}\DUrole{default_value}{50}}}{}
\pysigstopsignatures
\sphinxAtStartPar
This method provide a list if radii for an isotopic chain defined by Zref.

\end{fulllineitems}

\index{label (nucleardatapy.nuc.setup\_rch\_theo.setupRchTheo attribute)@\spxentry{label}\spxextra{nucleardatapy.nuc.setup\_rch\_theo.setupRchTheo attribute}}

\begin{fulllineitems}
\phantomsection\label{\detokenize{source/api/setup_nuc_rch_theo:nucleardatapy.nuc.setup_rch_theo.setupRchTheo.label}}
\pysigstartsignatures
\pysigline{\sphinxbfcode{\sphinxupquote{label}}}
\pysigstopsignatures
\sphinxAtStartPar
Attribute providing the label the data is references for figures.

\end{fulllineitems}

\index{note (nucleardatapy.nuc.setup\_rch\_theo.setupRchTheo attribute)@\spxentry{note}\spxextra{nucleardatapy.nuc.setup\_rch\_theo.setupRchTheo attribute}}

\begin{fulllineitems}
\phantomsection\label{\detokenize{source/api/setup_nuc_rch_theo:nucleardatapy.nuc.setup_rch_theo.setupRchTheo.note}}
\pysigstartsignatures
\pysigline{\sphinxbfcode{\sphinxupquote{note}}}
\pysigstopsignatures
\sphinxAtStartPar
Attribute providing additional notes about the data.

\end{fulllineitems}

\index{nucA (nucleardatapy.nuc.setup\_rch\_theo.setupRchTheo attribute)@\spxentry{nucA}\spxextra{nucleardatapy.nuc.setup\_rch\_theo.setupRchTheo attribute}}

\begin{fulllineitems}
\phantomsection\label{\detokenize{source/api/setup_nuc_rch_theo:nucleardatapy.nuc.setup_rch_theo.setupRchTheo.nucA}}
\pysigstartsignatures
\pysigline{\sphinxbfcode{\sphinxupquote{nucA}}}
\pysigstopsignatures
\sphinxAtStartPar
Attribute A (mass of the nucleus).

\end{fulllineitems}

\index{nucN (nucleardatapy.nuc.setup\_rch\_theo.setupRchTheo attribute)@\spxentry{nucN}\spxextra{nucleardatapy.nuc.setup\_rch\_theo.setupRchTheo attribute}}

\begin{fulllineitems}
\phantomsection\label{\detokenize{source/api/setup_nuc_rch_theo:nucleardatapy.nuc.setup_rch_theo.setupRchTheo.nucN}}
\pysigstartsignatures
\pysigline{\sphinxbfcode{\sphinxupquote{nucN}}}
\pysigstopsignatures
\sphinxAtStartPar
Attribute N (number of neutrons of the nucleus).

\end{fulllineitems}

\index{nucRch (nucleardatapy.nuc.setup\_rch\_theo.setupRchTheo attribute)@\spxentry{nucRch}\spxextra{nucleardatapy.nuc.setup\_rch\_theo.setupRchTheo attribute}}

\begin{fulllineitems}
\phantomsection\label{\detokenize{source/api/setup_nuc_rch_theo:nucleardatapy.nuc.setup_rch_theo.setupRchTheo.nucRch}}
\pysigstartsignatures
\pysigline{\sphinxbfcode{\sphinxupquote{nucRch}}}
\pysigstopsignatures
\sphinxAtStartPar
Attribue R\_ch (charge radius) in fm.

\end{fulllineitems}

\index{nucSymb (nucleardatapy.nuc.setup\_rch\_theo.setupRchTheo attribute)@\spxentry{nucSymb}\spxextra{nucleardatapy.nuc.setup\_rch\_theo.setupRchTheo attribute}}

\begin{fulllineitems}
\phantomsection\label{\detokenize{source/api/setup_nuc_rch_theo:nucleardatapy.nuc.setup_rch_theo.setupRchTheo.nucSymb}}
\pysigstartsignatures
\pysigline{\sphinxbfcode{\sphinxupquote{nucSymb}}}
\pysigstopsignatures
\sphinxAtStartPar
Attribute symb (symbol) of the element, e.g., Fe.

\end{fulllineitems}

\index{nucZ (nucleardatapy.nuc.setup\_rch\_theo.setupRchTheo attribute)@\spxentry{nucZ}\spxextra{nucleardatapy.nuc.setup\_rch\_theo.setupRchTheo attribute}}

\begin{fulllineitems}
\phantomsection\label{\detokenize{source/api/setup_nuc_rch_theo:nucleardatapy.nuc.setup_rch_theo.setupRchTheo.nucZ}}
\pysigstartsignatures
\pysigline{\sphinxbfcode{\sphinxupquote{nucZ}}}
\pysigstopsignatures
\sphinxAtStartPar
Attribute Z (charge of the nucleus).

\end{fulllineitems}

\index{print\_outputs() (nucleardatapy.nuc.setup\_rch\_theo.setupRchTheo method)@\spxentry{print\_outputs()}\spxextra{nucleardatapy.nuc.setup\_rch\_theo.setupRchTheo method}}

\begin{fulllineitems}
\phantomsection\label{\detokenize{source/api/setup_nuc_rch_theo:nucleardatapy.nuc.setup_rch_theo.setupRchTheo.print_outputs}}
\pysigstartsignatures
\pysiglinewithargsret{\sphinxbfcode{\sphinxupquote{print\_outputs}}}{}{}
\pysigstopsignatures
\sphinxAtStartPar
Method which print outputs on terminal’s screen.

\end{fulllineitems}

\index{ref (nucleardatapy.nuc.setup\_rch\_theo.setupRchTheo attribute)@\spxentry{ref}\spxextra{nucleardatapy.nuc.setup\_rch\_theo.setupRchTheo attribute}}

\begin{fulllineitems}
\phantomsection\label{\detokenize{source/api/setup_nuc_rch_theo:nucleardatapy.nuc.setup_rch_theo.setupRchTheo.ref}}
\pysigstartsignatures
\pysigline{\sphinxbfcode{\sphinxupquote{ref}}}
\pysigstopsignatures
\sphinxAtStartPar
Attribute providing the full reference to the paper to be citted.

\end{fulllineitems}


\end{fulllineitems}


\sphinxAtStartPar
Here are a set of figures which are produced with the Python sample: /nucleardatapy\_sample/plots/plot\_nuc\_setupRchTheo.py

\begin{figure}[htbp]
\centering
\capstart

\noindent\sphinxincludegraphics[scale=0.7]{{plot_nuc_setupRchTheo}.png}
\caption{Charge radii for Zn, Sn, and Pb isotopes and for the theoretical models available in the nuda toolkit.}\label{\detokenize{source/api/setup_nuc_rch_theo:id1}}\end{figure}

\sphinxstepscope


\section{SetupNucISGMRExp}
\label{\detokenize{source/api/setup_nuc_isgmr_exp:setupnucisgmrexp}}\label{\detokenize{source/api/setup_nuc_isgmr_exp::doc}}\phantomsection\label{\detokenize{source/api/setup_nuc_isgmr_exp:module-nucleardatapy.nuc.setup_isgmr_exp}}\index{module@\spxentry{module}!nucleardatapy.nuc.setup\_isgmr\_exp@\spxentry{nucleardatapy.nuc.setup\_isgmr\_exp}}\index{nucleardatapy.nuc.setup\_isgmr\_exp@\spxentry{nucleardatapy.nuc.setup\_isgmr\_exp}!module@\spxentry{module}}\index{isgmr\_exp\_tables() (in module nucleardatapy.nuc.setup\_isgmr\_exp)@\spxentry{isgmr\_exp\_tables()}\spxextra{in module nucleardatapy.nuc.setup\_isgmr\_exp}}

\begin{fulllineitems}
\phantomsection\label{\detokenize{source/api/setup_nuc_isgmr_exp:nucleardatapy.nuc.setup_isgmr_exp.isgmr_exp_tables}}
\pysigstartsignatures
\pysiglinewithargsret{\sphinxcode{\sphinxupquote{nucleardatapy.nuc.setup\_isgmr\_exp.}}\sphinxbfcode{\sphinxupquote{isgmr\_exp\_tables}}}{}{}
\pysigstopsignatures
\sphinxAtStartPar
Return a list of tables available in this toolkit for the ISGMR energy and
print them all on the prompt. These tables are the following
ones: ‘2010\sphinxhyphen{}ISGMR\sphinxhyphen{}LI’, ‘2018\sphinxhyphen{}ISGMR\sphinxhyphen{}GARG’, ‘2018\sphinxhyphen{}ISGMR\sphinxhyphen{}GARG\sphinxhyphen{}LATEX’.
\begin{quote}\begin{description}
\sphinxlineitem{Returns}
\sphinxAtStartPar
The list of tables.

\sphinxlineitem{Return type}
\sphinxAtStartPar
list{[}str{]}.

\end{description}\end{quote}

\end{fulllineitems}

\index{setupISGMRExp (class in nucleardatapy.nuc.setup\_isgmr\_exp)@\spxentry{setupISGMRExp}\spxextra{class in nucleardatapy.nuc.setup\_isgmr\_exp}}

\begin{fulllineitems}
\phantomsection\label{\detokenize{source/api/setup_nuc_isgmr_exp:nucleardatapy.nuc.setup_isgmr_exp.setupISGMRExp}}
\pysigstartsignatures
\pysiglinewithargsret{\sphinxbfcode{\sphinxupquote{class\DUrole{w}{  }}}\sphinxcode{\sphinxupquote{nucleardatapy.nuc.setup\_isgmr\_exp.}}\sphinxbfcode{\sphinxupquote{setupISGMRExp}}}{\emph{\DUrole{n}{table}\DUrole{o}{=}\DUrole{default_value}{\textquotesingle{}2018\sphinxhyphen{}ISGMR\sphinxhyphen{}GARG\textquotesingle{}}}}{}
\pysigstopsignatures
\sphinxAtStartPar
Instantiate the object with microscopic results choosen    by the toolkit practitioner.
This choice is defined in the variable \sphinxtitleref{table}.

\sphinxAtStartPar
The \sphinxtitleref{table} can chosen among the following ones:    ‘2010\sphinxhyphen{}ISGMR\sphinxhyphen{}LI’, ‘2018\sphinxhyphen{}ISGMR\sphinxhyphen{}GARG’.
\begin{quote}\begin{description}
\sphinxlineitem{Parameters}
\sphinxAtStartPar
\sphinxstyleliteralstrong{\sphinxupquote{table}} (\sphinxstyleliteralemphasis{\sphinxupquote{str}}\sphinxstyleliteralemphasis{\sphinxupquote{, }}\sphinxstyleliteralemphasis{\sphinxupquote{optional.}}) \textendash{} Fix the name of \sphinxtitleref{table}. Default value: ‘2018\sphinxhyphen{}ISGMR\sphinxhyphen{}GARG’, ‘2018\sphinxhyphen{}ISGMR\sphinxhyphen{}GARG\sphinxhyphen{}LATEX’.

\end{description}\end{quote}

\sphinxAtStartPar
\sphinxstylestrong{Attributes:}
\index{E\_unit (nucleardatapy.nuc.setup\_isgmr\_exp.setupISGMRExp attribute)@\spxentry{E\_unit}\spxextra{nucleardatapy.nuc.setup\_isgmr\_exp.setupISGMRExp attribute}}

\begin{fulllineitems}
\phantomsection\label{\detokenize{source/api/setup_nuc_isgmr_exp:nucleardatapy.nuc.setup_isgmr_exp.setupISGMRExp.E_unit}}
\pysigstartsignatures
\pysigline{\sphinxbfcode{\sphinxupquote{E\_unit}}}
\pysigstopsignatures
\sphinxAtStartPar
Attribute energy unit.

\end{fulllineitems}

\index{label (nucleardatapy.nuc.setup\_isgmr\_exp.setupISGMRExp attribute)@\spxentry{label}\spxextra{nucleardatapy.nuc.setup\_isgmr\_exp.setupISGMRExp attribute}}

\begin{fulllineitems}
\phantomsection\label{\detokenize{source/api/setup_nuc_isgmr_exp:nucleardatapy.nuc.setup_isgmr_exp.setupISGMRExp.label}}
\pysigstartsignatures
\pysigline{\sphinxbfcode{\sphinxupquote{label}}}
\pysigstopsignatures
\sphinxAtStartPar
Attribute providing the label the data is references for figures.

\end{fulllineitems}

\index{note (nucleardatapy.nuc.setup\_isgmr\_exp.setupISGMRExp attribute)@\spxentry{note}\spxextra{nucleardatapy.nuc.setup\_isgmr\_exp.setupISGMRExp attribute}}

\begin{fulllineitems}
\phantomsection\label{\detokenize{source/api/setup_nuc_isgmr_exp:nucleardatapy.nuc.setup_isgmr_exp.setupISGMRExp.note}}
\pysigstartsignatures
\pysigline{\sphinxbfcode{\sphinxupquote{note}}}
\pysigstopsignatures
\sphinxAtStartPar
Attribute providing additional notes about the data.

\end{fulllineitems}

\index{nucA (nucleardatapy.nuc.setup\_isgmr\_exp.setupISGMRExp attribute)@\spxentry{nucA}\spxextra{nucleardatapy.nuc.setup\_isgmr\_exp.setupISGMRExp attribute}}

\begin{fulllineitems}
\phantomsection\label{\detokenize{source/api/setup_nuc_isgmr_exp:nucleardatapy.nuc.setup_isgmr_exp.setupISGMRExp.nucA}}
\pysigstartsignatures
\pysigline{\sphinxbfcode{\sphinxupquote{nucA}}}
\pysigstopsignatures
\sphinxAtStartPar
Attribute A (mass of the nucleus).

\end{fulllineitems}

\index{nucM12Mm1\_cent (nucleardatapy.nuc.setup\_isgmr\_exp.setupISGMRExp attribute)@\spxentry{nucM12Mm1\_cent}\spxextra{nucleardatapy.nuc.setup\_isgmr\_exp.setupISGMRExp attribute}}

\begin{fulllineitems}
\phantomsection\label{\detokenize{source/api/setup_nuc_isgmr_exp:nucleardatapy.nuc.setup_isgmr_exp.setupISGMRExp.nucM12Mm1_cent}}
\pysigstartsignatures
\pysigline{\sphinxbfcode{\sphinxupquote{nucM12Mm1\_cent}}}
\pysigstopsignatures
\sphinxAtStartPar
Attribute energy centroid.

\end{fulllineitems}

\index{nucM12Mm1\_errm (nucleardatapy.nuc.setup\_isgmr\_exp.setupISGMRExp attribute)@\spxentry{nucM12Mm1\_errm}\spxextra{nucleardatapy.nuc.setup\_isgmr\_exp.setupISGMRExp attribute}}

\begin{fulllineitems}
\phantomsection\label{\detokenize{source/api/setup_nuc_isgmr_exp:nucleardatapy.nuc.setup_isgmr_exp.setupISGMRExp.nucM12Mm1_errm}}
\pysigstartsignatures
\pysigline{\sphinxbfcode{\sphinxupquote{nucM12Mm1\_errm}}}
\pysigstopsignatures
\sphinxAtStartPar
Attribute (\sphinxhyphen{}) uncertainty in the energy centroid.

\end{fulllineitems}

\index{nucM12Mm1\_errp (nucleardatapy.nuc.setup\_isgmr\_exp.setupISGMRExp attribute)@\spxentry{nucM12Mm1\_errp}\spxextra{nucleardatapy.nuc.setup\_isgmr\_exp.setupISGMRExp attribute}}

\begin{fulllineitems}
\phantomsection\label{\detokenize{source/api/setup_nuc_isgmr_exp:nucleardatapy.nuc.setup_isgmr_exp.setupISGMRExp.nucM12Mm1_errp}}
\pysigstartsignatures
\pysigline{\sphinxbfcode{\sphinxupquote{nucM12Mm1\_errp}}}
\pysigstopsignatures
\sphinxAtStartPar
Attribute (+) uncertainty in the energy centroid.

\end{fulllineitems}

\index{nucSymbol (nucleardatapy.nuc.setup\_isgmr\_exp.setupISGMRExp attribute)@\spxentry{nucSymbol}\spxextra{nucleardatapy.nuc.setup\_isgmr\_exp.setupISGMRExp attribute}}

\begin{fulllineitems}
\phantomsection\label{\detokenize{source/api/setup_nuc_isgmr_exp:nucleardatapy.nuc.setup_isgmr_exp.setupISGMRExp.nucSymbol}}
\pysigstartsignatures
\pysigline{\sphinxbfcode{\sphinxupquote{nucSymbol}}}
\pysigstopsignatures
\sphinxAtStartPar
Attribute the symbol of the element.

\end{fulllineitems}

\index{nucZ (nucleardatapy.nuc.setup\_isgmr\_exp.setupISGMRExp attribute)@\spxentry{nucZ}\spxextra{nucleardatapy.nuc.setup\_isgmr\_exp.setupISGMRExp attribute}}

\begin{fulllineitems}
\phantomsection\label{\detokenize{source/api/setup_nuc_isgmr_exp:nucleardatapy.nuc.setup_isgmr_exp.setupISGMRExp.nucZ}}
\pysigstartsignatures
\pysigline{\sphinxbfcode{\sphinxupquote{nucZ}}}
\pysigstopsignatures
\sphinxAtStartPar
Attribute Z (charge of the nucleus).

\end{fulllineitems}

\index{print\_outputs() (nucleardatapy.nuc.setup\_isgmr\_exp.setupISGMRExp method)@\spxentry{print\_outputs()}\spxextra{nucleardatapy.nuc.setup\_isgmr\_exp.setupISGMRExp method}}

\begin{fulllineitems}
\phantomsection\label{\detokenize{source/api/setup_nuc_isgmr_exp:nucleardatapy.nuc.setup_isgmr_exp.setupISGMRExp.print_outputs}}
\pysigstartsignatures
\pysiglinewithargsret{\sphinxbfcode{\sphinxupquote{print\_outputs}}}{}{}
\pysigstopsignatures
\sphinxAtStartPar
Method which print outputs on terminal’s screen.

\end{fulllineitems}

\index{ref (nucleardatapy.nuc.setup\_isgmr\_exp.setupISGMRExp attribute)@\spxentry{ref}\spxextra{nucleardatapy.nuc.setup\_isgmr\_exp.setupISGMRExp attribute}}

\begin{fulllineitems}
\phantomsection\label{\detokenize{source/api/setup_nuc_isgmr_exp:nucleardatapy.nuc.setup_isgmr_exp.setupISGMRExp.ref}}
\pysigstartsignatures
\pysigline{\sphinxbfcode{\sphinxupquote{ref}}}
\pysigstopsignatures
\sphinxAtStartPar
Attribute providing the full reference to the paper to be citted.

\end{fulllineitems}

\index{table (nucleardatapy.nuc.setup\_isgmr\_exp.setupISGMRExp attribute)@\spxentry{table}\spxextra{nucleardatapy.nuc.setup\_isgmr\_exp.setupISGMRExp attribute}}

\begin{fulllineitems}
\phantomsection\label{\detokenize{source/api/setup_nuc_isgmr_exp:nucleardatapy.nuc.setup_isgmr_exp.setupISGMRExp.table}}
\pysigstartsignatures
\pysigline{\sphinxbfcode{\sphinxupquote{table}}}
\pysigstopsignatures
\sphinxAtStartPar
Attribute table.

\end{fulllineitems}


\end{fulllineitems}


\sphinxAtStartPar
Here are a set of figures which are produced with the Python sample: /nucleardatapy\_sample/plots/plot\_nuc\_setupISGMRExp.py

\begin{figure}[htbp]
\centering
\capstart

\noindent\sphinxincludegraphics[scale=0.7]{{plot_nuc_setupISGMRExp}.png}
\caption{Experimental ISGMR energies available in the nucleardatapy toolkit.}\label{\detokenize{source/api/setup_nuc_isgmr_exp:id1}}\end{figure}

\sphinxstepscope


\section{SetupCrust}
\label{\detokenize{source/api/setup_crust:setupcrust}}\label{\detokenize{source/api/setup_crust::doc}}\phantomsection\label{\detokenize{source/api/setup_crust:module-nucleardatapy.crust.setup_crust}}\index{module@\spxentry{module}!nucleardatapy.crust.setup\_crust@\spxentry{nucleardatapy.crust.setup\_crust}}\index{nucleardatapy.crust.setup\_crust@\spxentry{nucleardatapy.crust.setup\_crust}!module@\spxentry{module}}\index{crust\_models() (in module nucleardatapy.crust.setup\_crust)@\spxentry{crust\_models()}\spxextra{in module nucleardatapy.crust.setup\_crust}}

\begin{fulllineitems}
\phantomsection\label{\detokenize{source/api/setup_crust:nucleardatapy.crust.setup_crust.crust_models}}
\pysigstartsignatures
\pysiglinewithargsret{\sphinxcode{\sphinxupquote{nucleardatapy.crust.setup\_crust.}}\sphinxbfcode{\sphinxupquote{crust\_models}}}{}{}
\pysigstopsignatures
\sphinxAtStartPar
Return a list of the tables available in this toolkit for the experimental masses and
print them all on the prompt. These tables are the following
ones: ‘Negele\sphinxhyphen{}Vautheron\sphinxhyphen{}1973’.
\begin{quote}\begin{description}
\sphinxlineitem{Returns}
\sphinxAtStartPar
The list of tables.

\sphinxlineitem{Return type}
\sphinxAtStartPar
list{[}str{]}.

\end{description}\end{quote}

\end{fulllineitems}

\index{setupCrust (class in nucleardatapy.crust.setup\_crust)@\spxentry{setupCrust}\spxextra{class in nucleardatapy.crust.setup\_crust}}

\begin{fulllineitems}
\phantomsection\label{\detokenize{source/api/setup_crust:nucleardatapy.crust.setup_crust.setupCrust}}
\pysigstartsignatures
\pysiglinewithargsret{\sphinxbfcode{\sphinxupquote{class\DUrole{w}{  }}}\sphinxcode{\sphinxupquote{nucleardatapy.crust.setup\_crust.}}\sphinxbfcode{\sphinxupquote{setupCrust}}}{\emph{\DUrole{n}{model}\DUrole{o}{=}\DUrole{default_value}{\textquotesingle{}1973\sphinxhyphen{}Negele\sphinxhyphen{}Vautherin\textquotesingle{}}}}{}
\pysigstopsignatures
\sphinxAtStartPar
Instantiate the properties of the crust for the existing models.

\sphinxAtStartPar
This choice is defined in the variable \sphinxtitleref{crust}.

\sphinxAtStartPar
\sphinxtitleref{crust} can chosen among the following ones: ‘Negele\sphinxhyphen{}Vautherin\sphinxhyphen{}1973’.
\begin{quote}\begin{description}
\sphinxlineitem{Parameters}
\sphinxAtStartPar
\sphinxstyleliteralstrong{\sphinxupquote{crust}} (\sphinxstyleliteralemphasis{\sphinxupquote{str}}\sphinxstyleliteralemphasis{\sphinxupquote{, }}\sphinxstyleliteralemphasis{\sphinxupquote{optional.}}) \textendash{} Fix the name of \sphinxtitleref{crust}. Default value: ‘Negele\sphinxhyphen{}Vautherin\sphinxhyphen{}1973’.

\end{description}\end{quote}

\sphinxAtStartPar
\sphinxstylestrong{Attributes:}
\index{init\_self() (nucleardatapy.crust.setup\_crust.setupCrust method)@\spxentry{init\_self()}\spxextra{nucleardatapy.crust.setup\_crust.setupCrust method}}

\begin{fulllineitems}
\phantomsection\label{\detokenize{source/api/setup_crust:nucleardatapy.crust.setup_crust.setupCrust.init_self}}
\pysigstartsignatures
\pysiglinewithargsret{\sphinxbfcode{\sphinxupquote{init\_self}}}{}{}
\pysigstopsignatures
\sphinxAtStartPar
Initialize variables in self.

\end{fulllineitems}

\index{print\_outputs() (nucleardatapy.crust.setup\_crust.setupCrust method)@\spxentry{print\_outputs()}\spxextra{nucleardatapy.crust.setup\_crust.setupCrust method}}

\begin{fulllineitems}
\phantomsection\label{\detokenize{source/api/setup_crust:nucleardatapy.crust.setup_crust.setupCrust.print_outputs}}
\pysigstartsignatures
\pysiglinewithargsret{\sphinxbfcode{\sphinxupquote{print\_outputs}}}{}{}
\pysigstopsignatures
\sphinxAtStartPar
Method which print outputs on terminal’s screen.

\end{fulllineitems}


\end{fulllineitems}


\sphinxAtStartPar
Here are a set of figures which are produced with the Python sample: /nucleardatapy\_sample/plots/plot\_crust\_setupCrust.py

\begin{figure}[htbp]
\centering
\capstart

\noindent\sphinxincludegraphics[scale=0.7]{{plot_crust_setupCrust}.png}
\caption{Properties of the crust as given by the models available in the nuda toolkit.}\label{\detokenize{source/api/setup_crust:id1}}\end{figure}

\sphinxstepscope


\section{SetupAstroMasses}
\label{\detokenize{source/api/setup_astro_masses:setupastromasses}}\label{\detokenize{source/api/setup_astro_masses::doc}}\phantomsection\label{\detokenize{source/api/setup_astro_masses:module-nucleardatapy.astro.setup_masses}}\index{module@\spxentry{module}!nucleardatapy.astro.setup\_masses@\spxentry{nucleardatapy.astro.setup\_masses}}\index{nucleardatapy.astro.setup\_masses@\spxentry{nucleardatapy.astro.setup\_masses}!module@\spxentry{module}}\index{masses\_obss() (in module nucleardatapy.astro.setup\_masses)@\spxentry{masses\_obss()}\spxextra{in module nucleardatapy.astro.setup\_masses}}

\begin{fulllineitems}
\phantomsection\label{\detokenize{source/api/setup_astro_masses:nucleardatapy.astro.setup_masses.masses_obss}}
\pysigstartsignatures
\pysiglinewithargsret{\sphinxcode{\sphinxupquote{nucleardatapy.astro.setup\_masses.}}\sphinxbfcode{\sphinxupquote{masses\_obss}}}{\emph{\DUrole{n}{source}}}{}
\pysigstopsignatures
\sphinxAtStartPar
Return a list of observations for a given source and print them all on the prompt.
\begin{quote}\begin{description}
\sphinxlineitem{Parameters}
\sphinxAtStartPar
\sphinxstyleliteralstrong{\sphinxupquote{source}} (\sphinxstyleliteralemphasis{\sphinxupquote{str.}}) \textendash{} The source for which there are different observations.

\sphinxlineitem{Returns}
\sphinxAtStartPar
The list of observations.     If source == ‘J1614\textendash{}2230’: 1, 2, 3, 4, 5.

\sphinxlineitem{Return type}
\sphinxAtStartPar
list{[}str{]}.

\end{description}\end{quote}

\end{fulllineitems}

\index{masses\_sources() (in module nucleardatapy.astro.setup\_masses)@\spxentry{masses\_sources()}\spxextra{in module nucleardatapy.astro.setup\_masses}}

\begin{fulllineitems}
\phantomsection\label{\detokenize{source/api/setup_astro_masses:nucleardatapy.astro.setup_masses.masses_sources}}
\pysigstartsignatures
\pysiglinewithargsret{\sphinxcode{\sphinxupquote{nucleardatapy.astro.setup\_masses.}}\sphinxbfcode{\sphinxupquote{masses\_sources}}}{}{}
\pysigstopsignatures
\sphinxAtStartPar
Return a list of the astrophysical sources for which a mass is given
\begin{quote}\begin{description}
\sphinxlineitem{Returns}
\sphinxAtStartPar
The list of sources.

\sphinxlineitem{Return type}
\sphinxAtStartPar
list{[}str{]}.

\end{description}\end{quote}

\end{fulllineitems}

\index{setupMasses (class in nucleardatapy.astro.setup\_masses)@\spxentry{setupMasses}\spxextra{class in nucleardatapy.astro.setup\_masses}}

\begin{fulllineitems}
\phantomsection\label{\detokenize{source/api/setup_astro_masses:nucleardatapy.astro.setup_masses.setupMasses}}
\pysigstartsignatures
\pysiglinewithargsret{\sphinxbfcode{\sphinxupquote{class\DUrole{w}{  }}}\sphinxcode{\sphinxupquote{nucleardatapy.astro.setup\_masses.}}\sphinxbfcode{\sphinxupquote{setupMasses}}}{\emph{\DUrole{n}{source}\DUrole{o}{=}\DUrole{default_value}{\textquotesingle{}J1614\textendash{}2230\textquotesingle{}}}, \emph{\DUrole{n}{obs}\DUrole{o}{=}\DUrole{default_value}{1}}}{}
\pysigstopsignatures
\sphinxAtStartPar
Instantiate the observational mass for a given source and obs.

\sphinxAtStartPar
This choice is defined in the variables \sphinxtitleref{source} and \sphinxtitleref{obs}.

\sphinxAtStartPar
\sphinxtitleref{source} can chosen among the following ones: ‘J1614\textendash{}2230’.

\sphinxAtStartPar
\sphinxtitleref{obs} depends on the chosen source.
\begin{quote}\begin{description}
\sphinxlineitem{Parameters}\begin{itemize}
\item {} 
\sphinxAtStartPar
\sphinxstyleliteralstrong{\sphinxupquote{source}} (\sphinxstyleliteralemphasis{\sphinxupquote{str}}\sphinxstyleliteralemphasis{\sphinxupquote{, }}\sphinxstyleliteralemphasis{\sphinxupquote{optional.}}) \textendash{} Fix the name of \sphinxtitleref{source}. Default value: ‘J1614\textendash{}2230’.

\item {} 
\sphinxAtStartPar
\sphinxstyleliteralstrong{\sphinxupquote{obs}} (\sphinxstyleliteralemphasis{\sphinxupquote{str}}\sphinxstyleliteralemphasis{\sphinxupquote{, }}\sphinxstyleliteralemphasis{\sphinxupquote{optional.}}) \textendash{} Fix the \sphinxtitleref{obs}. Default value: 1.

\end{itemize}

\end{description}\end{quote}

\sphinxAtStartPar
\sphinxstylestrong{Attributes:}
\index{label (nucleardatapy.astro.setup\_masses.setupMasses attribute)@\spxentry{label}\spxextra{nucleardatapy.astro.setup\_masses.setupMasses attribute}}

\begin{fulllineitems}
\phantomsection\label{\detokenize{source/api/setup_astro_masses:nucleardatapy.astro.setup_masses.setupMasses.label}}
\pysigstartsignatures
\pysigline{\sphinxbfcode{\sphinxupquote{label}}}
\pysigstopsignatures
\sphinxAtStartPar
Attribute providing the label the data is references for figures.

\end{fulllineitems}

\index{latexCite (nucleardatapy.astro.setup\_masses.setupMasses attribute)@\spxentry{latexCite}\spxextra{nucleardatapy.astro.setup\_masses.setupMasses attribute}}

\begin{fulllineitems}
\phantomsection\label{\detokenize{source/api/setup_astro_masses:nucleardatapy.astro.setup_masses.setupMasses.latexCite}}
\pysigstartsignatures
\pysigline{\sphinxbfcode{\sphinxupquote{latexCite}}}
\pysigstopsignatures
\sphinxAtStartPar
Attribute latexCite.

\end{fulllineitems}

\index{mass (nucleardatapy.astro.setup\_masses.setupMasses attribute)@\spxentry{mass}\spxextra{nucleardatapy.astro.setup\_masses.setupMasses attribute}}

\begin{fulllineitems}
\phantomsection\label{\detokenize{source/api/setup_astro_masses:nucleardatapy.astro.setup_masses.setupMasses.mass}}
\pysigstartsignatures
\pysigline{\sphinxbfcode{\sphinxupquote{mass}}}
\pysigstopsignatures
\sphinxAtStartPar
Attribute the observational mass of the source.

\end{fulllineitems}

\index{note (nucleardatapy.astro.setup\_masses.setupMasses attribute)@\spxentry{note}\spxextra{nucleardatapy.astro.setup\_masses.setupMasses attribute}}

\begin{fulllineitems}
\phantomsection\label{\detokenize{source/api/setup_astro_masses:nucleardatapy.astro.setup_masses.setupMasses.note}}
\pysigstartsignatures
\pysigline{\sphinxbfcode{\sphinxupquote{note}}}
\pysigstopsignatures
\sphinxAtStartPar
Attribute providing additional notes about the observation.

\end{fulllineitems}

\index{print\_output() (nucleardatapy.astro.setup\_masses.setupMasses method)@\spxentry{print\_output()}\spxextra{nucleardatapy.astro.setup\_masses.setupMasses method}}

\begin{fulllineitems}
\phantomsection\label{\detokenize{source/api/setup_astro_masses:nucleardatapy.astro.setup_masses.setupMasses.print_output}}
\pysigstartsignatures
\pysiglinewithargsret{\sphinxbfcode{\sphinxupquote{print\_output}}}{}{}
\pysigstopsignatures
\sphinxAtStartPar
Method which print outputs on terminal’s screen.

\end{fulllineitems}

\index{print\_table() (nucleardatapy.astro.setup\_masses.setupMasses method)@\spxentry{print\_table()}\spxextra{nucleardatapy.astro.setup\_masses.setupMasses method}}

\begin{fulllineitems}
\phantomsection\label{\detokenize{source/api/setup_astro_masses:nucleardatapy.astro.setup_masses.setupMasses.print_table}}
\pysigstartsignatures
\pysiglinewithargsret{\sphinxbfcode{\sphinxupquote{print\_table}}}{}{}
\pysigstopsignatures
\sphinxAtStartPar
Method which print outputs in table format (latex) on terminal’s screen.

\end{fulllineitems}

\index{ref (nucleardatapy.astro.setup\_masses.setupMasses attribute)@\spxentry{ref}\spxextra{nucleardatapy.astro.setup\_masses.setupMasses attribute}}

\begin{fulllineitems}
\phantomsection\label{\detokenize{source/api/setup_astro_masses:nucleardatapy.astro.setup_masses.setupMasses.ref}}
\pysigstartsignatures
\pysigline{\sphinxbfcode{\sphinxupquote{ref}}}
\pysigstopsignatures
\sphinxAtStartPar
Attribute providing the full reference to the paper to be citted.

\end{fulllineitems}

\index{sig\_do (nucleardatapy.astro.setup\_masses.setupMasses attribute)@\spxentry{sig\_do}\spxextra{nucleardatapy.astro.setup\_masses.setupMasses attribute}}

\begin{fulllineitems}
\phantomsection\label{\detokenize{source/api/setup_astro_masses:nucleardatapy.astro.setup_masses.setupMasses.sig_do}}
\pysigstartsignatures
\pysigline{\sphinxbfcode{\sphinxupquote{sig\_do}}}
\pysigstopsignatures
\sphinxAtStartPar
Attribute the negative uncertainty.

\end{fulllineitems}

\index{sig\_up (nucleardatapy.astro.setup\_masses.setupMasses attribute)@\spxentry{sig\_up}\spxextra{nucleardatapy.astro.setup\_masses.setupMasses attribute}}

\begin{fulllineitems}
\phantomsection\label{\detokenize{source/api/setup_astro_masses:nucleardatapy.astro.setup_masses.setupMasses.sig_up}}
\pysigstartsignatures
\pysigline{\sphinxbfcode{\sphinxupquote{sig\_up}}}
\pysigstopsignatures
\sphinxAtStartPar
Attribute the positive uncertainty.

\end{fulllineitems}


\end{fulllineitems}

\index{setupMassesAverage (class in nucleardatapy.astro.setup\_masses)@\spxentry{setupMassesAverage}\spxextra{class in nucleardatapy.astro.setup\_masses}}

\begin{fulllineitems}
\phantomsection\label{\detokenize{source/api/setup_astro_masses:nucleardatapy.astro.setup_masses.setupMassesAverage}}
\pysigstartsignatures
\pysiglinewithargsret{\sphinxbfcode{\sphinxupquote{class\DUrole{w}{  }}}\sphinxcode{\sphinxupquote{nucleardatapy.astro.setup\_masses.}}\sphinxbfcode{\sphinxupquote{setupMassesAverage}}}{\emph{\DUrole{n}{source}\DUrole{o}{=}\DUrole{default_value}{\textquotesingle{}J1614\textendash{}2230\textquotesingle{}}}}{}
\pysigstopsignatures
\sphinxAtStartPar
Instantiate the observational mass for a given source and averaged over obs.

\sphinxAtStartPar
This choice is defined in the variable \sphinxtitleref{source}.

\sphinxAtStartPar
\sphinxtitleref{source} can chosen among the following ones: ‘J1614\textendash{}2230’.
\begin{quote}\begin{description}
\sphinxlineitem{Parameters}
\sphinxAtStartPar
\sphinxstyleliteralstrong{\sphinxupquote{source}} (\sphinxstyleliteralemphasis{\sphinxupquote{str}}\sphinxstyleliteralemphasis{\sphinxupquote{, }}\sphinxstyleliteralemphasis{\sphinxupquote{optional.}}) \textendash{} Fix the name of \sphinxtitleref{source}. Default value: ‘J1614\textendash{}2230’.

\end{description}\end{quote}

\sphinxAtStartPar
\sphinxstylestrong{Attributes:}
\index{print\_output() (nucleardatapy.astro.setup\_masses.setupMassesAverage method)@\spxentry{print\_output()}\spxextra{nucleardatapy.astro.setup\_masses.setupMassesAverage method}}

\begin{fulllineitems}
\phantomsection\label{\detokenize{source/api/setup_astro_masses:nucleardatapy.astro.setup_masses.setupMassesAverage.print_output}}
\pysigstartsignatures
\pysiglinewithargsret{\sphinxbfcode{\sphinxupquote{print\_output}}}{}{}
\pysigstopsignatures
\sphinxAtStartPar
Method which print outputs on terminal’s screen.

\end{fulllineitems}

\index{print\_table() (nucleardatapy.astro.setup\_masses.setupMassesAverage method)@\spxentry{print\_table()}\spxextra{nucleardatapy.astro.setup\_masses.setupMassesAverage method}}

\begin{fulllineitems}
\phantomsection\label{\detokenize{source/api/setup_astro_masses:nucleardatapy.astro.setup_masses.setupMassesAverage.print_table}}
\pysigstartsignatures
\pysiglinewithargsret{\sphinxbfcode{\sphinxupquote{print\_table}}}{}{}
\pysigstopsignatures
\sphinxAtStartPar
Method which print outputs in table format (latex) on terminal’s screen.

\end{fulllineitems}


\end{fulllineitems}


\sphinxAtStartPar
Here is a figure which is produced with the Python sample: /nucleardatapy\_sample/plots/plot\_astro\_setupMasses.py

\begin{figure}[htbp]
\centering
\capstart

\noindent\sphinxincludegraphics[scale=0.7]{{plot_astro_setupMasses}.png}
\caption{The masses measured for massive neutron stars is radio\sphinxhyphen{}astronomy. The different colors correspond to the different sources.}\label{\detokenize{source/api/setup_astro_masses:id1}}\end{figure}

\sphinxstepscope


\section{SetupAstroMup}
\label{\detokenize{source/api/setup_astro_mup:setupastromup}}\label{\detokenize{source/api/setup_astro_mup::doc}}\phantomsection\label{\detokenize{source/api/setup_astro_mup:module-nucleardatapy.astro.setup_mup}}\index{module@\spxentry{module}!nucleardatapy.astro.setup\_mup@\spxentry{nucleardatapy.astro.setup\_mup}}\index{nucleardatapy.astro.setup\_mup@\spxentry{nucleardatapy.astro.setup\_mup}!module@\spxentry{module}}\index{mup\_hyps() (in module nucleardatapy.astro.setup\_mup)@\spxentry{mup\_hyps()}\spxextra{in module nucleardatapy.astro.setup\_mup}}

\begin{fulllineitems}
\phantomsection\label{\detokenize{source/api/setup_astro_mup:nucleardatapy.astro.setup_mup.mup_hyps}}
\pysigstartsignatures
\pysiglinewithargsret{\sphinxcode{\sphinxupquote{nucleardatapy.astro.setup\_mup.}}\sphinxbfcode{\sphinxupquote{mup\_hyps}}}{\emph{\DUrole{n}{source}}}{}
\pysigstopsignatures
\sphinxAtStartPar
Return a list of observations for a given source and print them all on the prompt.
\begin{quote}\begin{description}
\sphinxlineitem{Parameters}
\sphinxAtStartPar
\sphinxstyleliteralstrong{\sphinxupquote{source}} (\sphinxstyleliteralemphasis{\sphinxupquote{str.}}) \textendash{} The source for which there are different observations.

\sphinxlineitem{Returns}
\sphinxAtStartPar
The list of observations.     If source == ‘J1614\textendash{}2230’: 1, 2, 3, 4, 5.

\sphinxlineitem{Return type}
\sphinxAtStartPar
list{[}str{]}.

\end{description}\end{quote}

\end{fulllineitems}

\index{mup\_sources() (in module nucleardatapy.astro.setup\_mup)@\spxentry{mup\_sources()}\spxextra{in module nucleardatapy.astro.setup\_mup}}

\begin{fulllineitems}
\phantomsection\label{\detokenize{source/api/setup_astro_mup:nucleardatapy.astro.setup_mup.mup_sources}}
\pysigstartsignatures
\pysiglinewithargsret{\sphinxcode{\sphinxupquote{nucleardatapy.astro.setup\_mup.}}\sphinxbfcode{\sphinxupquote{mup\_sources}}}{}{}
\pysigstopsignatures
\sphinxAtStartPar
Return a list of the astrophysical sources for which a mass is given
\begin{quote}\begin{description}
\sphinxlineitem{Returns}
\sphinxAtStartPar
The list of sources.

\sphinxlineitem{Return type}
\sphinxAtStartPar
list{[}str{]}.

\end{description}\end{quote}

\end{fulllineitems}

\index{setupMup (class in nucleardatapy.astro.setup\_mup)@\spxentry{setupMup}\spxextra{class in nucleardatapy.astro.setup\_mup}}

\begin{fulllineitems}
\phantomsection\label{\detokenize{source/api/setup_astro_mup:nucleardatapy.astro.setup_mup.setupMup}}
\pysigstartsignatures
\pysiglinewithargsret{\sphinxbfcode{\sphinxupquote{class\DUrole{w}{  }}}\sphinxcode{\sphinxupquote{nucleardatapy.astro.setup\_mup.}}\sphinxbfcode{\sphinxupquote{setupMup}}}{\emph{\DUrole{n}{source}\DUrole{o}{=}\DUrole{default_value}{\textquotesingle{}GW170817\textquotesingle{}}}, \emph{\DUrole{n}{hyp}\DUrole{o}{=}\DUrole{default_value}{1}}}{}
\pysigstopsignatures
\sphinxAtStartPar
Instantiate the upper mass for a given source and hyptheses.

\sphinxAtStartPar
This choice is defined in the variables \sphinxtitleref{source} and \sphinxtitleref{hyp}.

\sphinxAtStartPar
\sphinxtitleref{source} can chosen among the following ones: ‘GW170817’.

\sphinxAtStartPar
\sphinxtitleref{hyp} depends on the chosen hypotheses.
\begin{quote}\begin{description}
\sphinxlineitem{Parameters}\begin{itemize}
\item {} 
\sphinxAtStartPar
\sphinxstyleliteralstrong{\sphinxupquote{source}} (\sphinxstyleliteralemphasis{\sphinxupquote{str}}\sphinxstyleliteralemphasis{\sphinxupquote{, }}\sphinxstyleliteralemphasis{\sphinxupquote{optional.}}) \textendash{} Fix the name of \sphinxtitleref{source}. Default value: ‘GW170817’.

\item {} 
\sphinxAtStartPar
\sphinxstyleliteralstrong{\sphinxupquote{hyp}} (\sphinxstyleliteralemphasis{\sphinxupquote{str}}\sphinxstyleliteralemphasis{\sphinxupquote{, }}\sphinxstyleliteralemphasis{\sphinxupquote{optional.}}) \textendash{} Fix the \sphinxtitleref{hyp}. Default value: ‘low\sphinxhyphen{}spin+TaylorF2’.

\end{itemize}

\end{description}\end{quote}

\sphinxAtStartPar
\sphinxstylestrong{Attributes:}
\index{label (nucleardatapy.astro.setup\_mup.setupMup attribute)@\spxentry{label}\spxextra{nucleardatapy.astro.setup\_mup.setupMup attribute}}

\begin{fulllineitems}
\phantomsection\label{\detokenize{source/api/setup_astro_mup:nucleardatapy.astro.setup_mup.setupMup.label}}
\pysigstartsignatures
\pysigline{\sphinxbfcode{\sphinxupquote{label}}}
\pysigstopsignatures
\sphinxAtStartPar
Attribute providing the label the data is references for figures.

\end{fulllineitems}

\index{latexCite (nucleardatapy.astro.setup\_mup.setupMup attribute)@\spxentry{latexCite}\spxextra{nucleardatapy.astro.setup\_mup.setupMup attribute}}

\begin{fulllineitems}
\phantomsection\label{\detokenize{source/api/setup_astro_mup:nucleardatapy.astro.setup_mup.setupMup.latexCite}}
\pysigstartsignatures
\pysigline{\sphinxbfcode{\sphinxupquote{latexCite}}}
\pysigstopsignatures
\sphinxAtStartPar
Attribute latexCite.

\end{fulllineitems}

\index{mup (nucleardatapy.astro.setup\_mup.setupMup attribute)@\spxentry{mup}\spxextra{nucleardatapy.astro.setup\_mup.setupMup attribute}}

\begin{fulllineitems}
\phantomsection\label{\detokenize{source/api/setup_astro_mup:nucleardatapy.astro.setup_mup.setupMup.mup}}
\pysigstartsignatures
\pysigline{\sphinxbfcode{\sphinxupquote{mup}}}
\pysigstopsignatures
\sphinxAtStartPar
Attribute the observational mass of the source.

\end{fulllineitems}

\index{note (nucleardatapy.astro.setup\_mup.setupMup attribute)@\spxentry{note}\spxextra{nucleardatapy.astro.setup\_mup.setupMup attribute}}

\begin{fulllineitems}
\phantomsection\label{\detokenize{source/api/setup_astro_mup:nucleardatapy.astro.setup_mup.setupMup.note}}
\pysigstartsignatures
\pysigline{\sphinxbfcode{\sphinxupquote{note}}}
\pysigstopsignatures
\sphinxAtStartPar
Attribute providing additional notes about the observation.

\end{fulllineitems}

\index{print\_output() (nucleardatapy.astro.setup\_mup.setupMup method)@\spxentry{print\_output()}\spxextra{nucleardatapy.astro.setup\_mup.setupMup method}}

\begin{fulllineitems}
\phantomsection\label{\detokenize{source/api/setup_astro_mup:nucleardatapy.astro.setup_mup.setupMup.print_output}}
\pysigstartsignatures
\pysiglinewithargsret{\sphinxbfcode{\sphinxupquote{print\_output}}}{}{}
\pysigstopsignatures
\sphinxAtStartPar
Method which print outputs on terminal’s screen.

\end{fulllineitems}

\index{print\_table() (nucleardatapy.astro.setup\_mup.setupMup method)@\spxentry{print\_table()}\spxextra{nucleardatapy.astro.setup\_mup.setupMup method}}

\begin{fulllineitems}
\phantomsection\label{\detokenize{source/api/setup_astro_mup:nucleardatapy.astro.setup_mup.setupMup.print_table}}
\pysigstartsignatures
\pysiglinewithargsret{\sphinxbfcode{\sphinxupquote{print\_table}}}{}{}
\pysigstopsignatures
\sphinxAtStartPar
Method which print outputs in table format (latex) on terminal’s screen.

\end{fulllineitems}

\index{ref (nucleardatapy.astro.setup\_mup.setupMup attribute)@\spxentry{ref}\spxextra{nucleardatapy.astro.setup\_mup.setupMup attribute}}

\begin{fulllineitems}
\phantomsection\label{\detokenize{source/api/setup_astro_mup:nucleardatapy.astro.setup_mup.setupMup.ref}}
\pysigstartsignatures
\pysigline{\sphinxbfcode{\sphinxupquote{ref}}}
\pysigstopsignatures
\sphinxAtStartPar
Attribute providing the full reference to the paper to be citted.

\end{fulllineitems}

\index{sig\_do (nucleardatapy.astro.setup\_mup.setupMup attribute)@\spxentry{sig\_do}\spxextra{nucleardatapy.astro.setup\_mup.setupMup attribute}}

\begin{fulllineitems}
\phantomsection\label{\detokenize{source/api/setup_astro_mup:nucleardatapy.astro.setup_mup.setupMup.sig_do}}
\pysigstartsignatures
\pysigline{\sphinxbfcode{\sphinxupquote{sig\_do}}}
\pysigstopsignatures
\sphinxAtStartPar
Attribute the negative uncertainty.

\end{fulllineitems}

\index{sig\_up (nucleardatapy.astro.setup\_mup.setupMup attribute)@\spxentry{sig\_up}\spxextra{nucleardatapy.astro.setup\_mup.setupMup attribute}}

\begin{fulllineitems}
\phantomsection\label{\detokenize{source/api/setup_astro_mup:nucleardatapy.astro.setup_mup.setupMup.sig_up}}
\pysigstartsignatures
\pysigline{\sphinxbfcode{\sphinxupquote{sig\_up}}}
\pysigstopsignatures
\sphinxAtStartPar
Attribute the positive uncertainty.

\end{fulllineitems}


\end{fulllineitems}

\index{setupMupAverage (class in nucleardatapy.astro.setup\_mup)@\spxentry{setupMupAverage}\spxextra{class in nucleardatapy.astro.setup\_mup}}

\begin{fulllineitems}
\phantomsection\label{\detokenize{source/api/setup_astro_mup:nucleardatapy.astro.setup_mup.setupMupAverage}}
\pysigstartsignatures
\pysiglinewithargsret{\sphinxbfcode{\sphinxupquote{class\DUrole{w}{  }}}\sphinxcode{\sphinxupquote{nucleardatapy.astro.setup\_mup.}}\sphinxbfcode{\sphinxupquote{setupMupAverage}}}{\emph{\DUrole{n}{source}\DUrole{o}{=}\DUrole{default_value}{\textquotesingle{}GW170817\textquotesingle{}}}}{}
\pysigstopsignatures
\sphinxAtStartPar
Instantiate the upper mass for a given source and averaged over hypotheses.

\sphinxAtStartPar
This choice is defined in the variable \sphinxtitleref{source}.

\sphinxAtStartPar
\sphinxtitleref{source} can chosen among the following ones: ‘GW170817’.
\begin{quote}\begin{description}
\sphinxlineitem{Parameters}
\sphinxAtStartPar
\sphinxstyleliteralstrong{\sphinxupquote{source}} (\sphinxstyleliteralemphasis{\sphinxupquote{str}}\sphinxstyleliteralemphasis{\sphinxupquote{, }}\sphinxstyleliteralemphasis{\sphinxupquote{optional.}}) \textendash{} Fix the name of \sphinxtitleref{source}. Default value: ‘GW170817’.

\end{description}\end{quote}

\sphinxAtStartPar
\sphinxstylestrong{Attributes:}
\index{print\_output() (nucleardatapy.astro.setup\_mup.setupMupAverage method)@\spxentry{print\_output()}\spxextra{nucleardatapy.astro.setup\_mup.setupMupAverage method}}

\begin{fulllineitems}
\phantomsection\label{\detokenize{source/api/setup_astro_mup:nucleardatapy.astro.setup_mup.setupMupAverage.print_output}}
\pysigstartsignatures
\pysiglinewithargsret{\sphinxbfcode{\sphinxupquote{print\_output}}}{}{}
\pysigstopsignatures
\sphinxAtStartPar
Method which print outputs on terminal’s screen.

\end{fulllineitems}

\index{print\_table() (nucleardatapy.astro.setup\_mup.setupMupAverage method)@\spxentry{print\_table()}\spxextra{nucleardatapy.astro.setup\_mup.setupMupAverage method}}

\begin{fulllineitems}
\phantomsection\label{\detokenize{source/api/setup_astro_mup:nucleardatapy.astro.setup_mup.setupMupAverage.print_table}}
\pysigstartsignatures
\pysiglinewithargsret{\sphinxbfcode{\sphinxupquote{print\_table}}}{}{}
\pysigstopsignatures
\sphinxAtStartPar
Method which print outputs in table format (latex) on terminal’s screen.

\end{fulllineitems}


\end{fulllineitems}


\sphinxAtStartPar
Here is a figure which is produced with the Python sample: /nucleardatapy\_sample/plots/plot\_astro\_setupMup.py

\begin{figure}[htbp]
\centering
\capstart

\noindent\sphinxincludegraphics[scale=0.7]{{plot_astro_setupMup}.png}
\caption{The upper masses measured from GW observations. The different colors correspond to the different sources.}\label{\detokenize{source/api/setup_astro_mup:id1}}\end{figure}

\sphinxstepscope


\section{SetupAstroMtov}
\label{\detokenize{source/api/setup_astro_mtov:setupastromtov}}\label{\detokenize{source/api/setup_astro_mtov::doc}}\phantomsection\label{\detokenize{source/api/setup_astro_mtov:module-nucleardatapy.astro.setup_mtov}}\index{module@\spxentry{module}!nucleardatapy.astro.setup\_mtov@\spxentry{nucleardatapy.astro.setup\_mtov}}\index{nucleardatapy.astro.setup\_mtov@\spxentry{nucleardatapy.astro.setup\_mtov}!module@\spxentry{module}}\index{setupMtov (class in nucleardatapy.astro.setup\_mtov)@\spxentry{setupMtov}\spxextra{class in nucleardatapy.astro.setup\_mtov}}

\begin{fulllineitems}
\phantomsection\label{\detokenize{source/api/setup_astro_mtov:nucleardatapy.astro.setup_mtov.setupMtov}}
\pysigstartsignatures
\pysiglinewithargsret{\sphinxbfcode{\sphinxupquote{class\DUrole{w}{  }}}\sphinxcode{\sphinxupquote{nucleardatapy.astro.setup\_mtov.}}\sphinxbfcode{\sphinxupquote{setupMtov}}}{\emph{\DUrole{n}{sources\_do}\DUrole{o}{=}\DUrole{default_value}{array({[}\textquotesingle{}J1614\textendash{}2230\textquotesingle{}{]}, dtype=\textquotesingle{}\textless{}U10\textquotesingle{})}}, \emph{\DUrole{n}{sources\_up}\DUrole{o}{=}\DUrole{default_value}{array({[}\textquotesingle{}GW170817\textquotesingle{}{]}, dtype=\textquotesingle{}\textless{}U8\textquotesingle{})}}}{}
\pysigstopsignatures
\sphinxAtStartPar
Instantiate the observational mass for a given source and obs.

\sphinxAtStartPar
This choice is defined in the variable \sphinxtitleref{source}.

\sphinxAtStartPar
\sphinxtitleref{source} can chosen among the following ones: ‘J1614\textendash{}2230’.
\begin{quote}\begin{description}
\sphinxlineitem{Parameters}
\sphinxAtStartPar
\sphinxstyleliteralstrong{\sphinxupquote{source}} (\sphinxstyleliteralemphasis{\sphinxupquote{str}}\sphinxstyleliteralemphasis{\sphinxupquote{, }}\sphinxstyleliteralemphasis{\sphinxupquote{optional.}}) \textendash{} Fix the name of \sphinxtitleref{source}. Default value: ‘J1614\textendash{}2230’.

\end{description}\end{quote}

\sphinxAtStartPar
\sphinxstylestrong{Attributes:}
\index{print\_output() (nucleardatapy.astro.setup\_mtov.setupMtov method)@\spxentry{print\_output()}\spxextra{nucleardatapy.astro.setup\_mtov.setupMtov method}}

\begin{fulllineitems}
\phantomsection\label{\detokenize{source/api/setup_astro_mtov:nucleardatapy.astro.setup_mtov.setupMtov.print_output}}
\pysigstartsignatures
\pysiglinewithargsret{\sphinxbfcode{\sphinxupquote{print\_output}}}{}{}
\pysigstopsignatures
\sphinxAtStartPar
Method which print outputs on terminal’s screen.

\end{fulllineitems}

\index{print\_table() (nucleardatapy.astro.setup\_mtov.setupMtov method)@\spxentry{print\_table()}\spxextra{nucleardatapy.astro.setup\_mtov.setupMtov method}}

\begin{fulllineitems}
\phantomsection\label{\detokenize{source/api/setup_astro_mtov:nucleardatapy.astro.setup_mtov.setupMtov.print_table}}
\pysigstartsignatures
\pysiglinewithargsret{\sphinxbfcode{\sphinxupquote{print\_table}}}{}{}
\pysigstopsignatures
\sphinxAtStartPar
Method which print outputs in table format (latex) on terminal’s screen.

\end{fulllineitems}


\end{fulllineitems}


\sphinxAtStartPar
Here is a figure which is produced with the Python sample: /nucleardatapy\_sample/plots/plot\_astro\_setupMtov.py

\begin{figure}[htbp]
\centering
\capstart

\noindent\sphinxincludegraphics[scale=0.7]{{plot_astro_setupMtov}.png}
\caption{The probability distribution function for the TOV mass constructed from radio and gravitational\sphinxhyphen{}wave observations.
The different colors correspond to the different sources.}\label{\detokenize{source/api/setup_astro_mtov:id1}}\end{figure}

\sphinxstepscope


\section{SetupAstroMR}
\label{\detokenize{source/api/setup_astro_mr:setupastromr}}\label{\detokenize{source/api/setup_astro_mr::doc}}\phantomsection\label{\detokenize{source/api/setup_astro_mr:module-nucleardatapy.astro.setup_mr}}\index{module@\spxentry{module}!nucleardatapy.astro.setup\_mr@\spxentry{nucleardatapy.astro.setup\_mr}}\index{nucleardatapy.astro.setup\_mr@\spxentry{nucleardatapy.astro.setup\_mr}!module@\spxentry{module}}\index{mr\_obss() (in module nucleardatapy.astro.setup\_mr)@\spxentry{mr\_obss()}\spxextra{in module nucleardatapy.astro.setup\_mr}}

\begin{fulllineitems}
\phantomsection\label{\detokenize{source/api/setup_astro_mr:nucleardatapy.astro.setup_mr.mr_obss}}
\pysigstartsignatures
\pysiglinewithargsret{\sphinxcode{\sphinxupquote{nucleardatapy.astro.setup\_mr.}}\sphinxbfcode{\sphinxupquote{mr\_obss}}}{\emph{\DUrole{n}{source}}}{}
\pysigstopsignatures
\sphinxAtStartPar
Return a list of observations for a given source and print them all on the prompt.
\begin{quote}\begin{description}
\sphinxlineitem{Parameters}
\sphinxAtStartPar
\sphinxstyleliteralstrong{\sphinxupquote{source}} (\sphinxstyleliteralemphasis{\sphinxupquote{str.}}) \textendash{} The source for which there are different observations.

\sphinxlineitem{Returns}
\sphinxAtStartPar
The list of observations.     If source == ‘J1614\textendash{}2230’: 1, 2, 3, 4, 5.

\sphinxlineitem{Return type}
\sphinxAtStartPar
list{[}str{]}.

\end{description}\end{quote}

\end{fulllineitems}

\index{mr\_sources() (in module nucleardatapy.astro.setup\_mr)@\spxentry{mr\_sources()}\spxextra{in module nucleardatapy.astro.setup\_mr}}

\begin{fulllineitems}
\phantomsection\label{\detokenize{source/api/setup_astro_mr:nucleardatapy.astro.setup_mr.mr_sources}}
\pysigstartsignatures
\pysiglinewithargsret{\sphinxcode{\sphinxupquote{nucleardatapy.astro.setup\_mr.}}\sphinxbfcode{\sphinxupquote{mr\_sources}}}{}{}
\pysigstopsignatures
\sphinxAtStartPar
Return a list of the astrophysical sources for which a mass is given
\begin{quote}\begin{description}
\sphinxlineitem{Returns}
\sphinxAtStartPar
The list of sources.

\sphinxlineitem{Return type}
\sphinxAtStartPar
list{[}str{]}.

\end{description}\end{quote}

\end{fulllineitems}

\index{setupMR (class in nucleardatapy.astro.setup\_mr)@\spxentry{setupMR}\spxextra{class in nucleardatapy.astro.setup\_mr}}

\begin{fulllineitems}
\phantomsection\label{\detokenize{source/api/setup_astro_mr:nucleardatapy.astro.setup_mr.setupMR}}
\pysigstartsignatures
\pysiglinewithargsret{\sphinxbfcode{\sphinxupquote{class\DUrole{w}{  }}}\sphinxcode{\sphinxupquote{nucleardatapy.astro.setup\_mr.}}\sphinxbfcode{\sphinxupquote{setupMR}}}{\emph{\DUrole{n}{source}\DUrole{o}{=}\DUrole{default_value}{\textquotesingle{}J0030+0451\textquotesingle{}}}, \emph{\DUrole{n}{obs}\DUrole{o}{=}\DUrole{default_value}{1}}}{}
\pysigstopsignatures
\sphinxAtStartPar
Instantiate the observational mass for a given source and obs.

\sphinxAtStartPar
This choice is defined in the variables \sphinxtitleref{source} and \sphinxtitleref{obs}.

\sphinxAtStartPar
\sphinxtitleref{source} can chosen among the following ones: ‘J1614\textendash{}2230’.

\sphinxAtStartPar
\sphinxtitleref{obs} depends on the chosen source.
\begin{quote}\begin{description}
\sphinxlineitem{Parameters}\begin{itemize}
\item {} 
\sphinxAtStartPar
\sphinxstyleliteralstrong{\sphinxupquote{source}} (\sphinxstyleliteralemphasis{\sphinxupquote{str}}\sphinxstyleliteralemphasis{\sphinxupquote{, }}\sphinxstyleliteralemphasis{\sphinxupquote{optional.}}) \textendash{} Fix the name of \sphinxtitleref{source}. Default value: ‘J1614\textendash{}2230’.

\item {} 
\sphinxAtStartPar
\sphinxstyleliteralstrong{\sphinxupquote{obs}} (\sphinxstyleliteralemphasis{\sphinxupquote{str}}\sphinxstyleliteralemphasis{\sphinxupquote{, }}\sphinxstyleliteralemphasis{\sphinxupquote{optional.}}) \textendash{} Fix the \sphinxtitleref{obs}. Default value: 1.

\end{itemize}

\end{description}\end{quote}

\sphinxAtStartPar
\sphinxstylestrong{Attributes:}
\index{label (nucleardatapy.astro.setup\_mr.setupMR attribute)@\spxentry{label}\spxextra{nucleardatapy.astro.setup\_mr.setupMR attribute}}

\begin{fulllineitems}
\phantomsection\label{\detokenize{source/api/setup_astro_mr:nucleardatapy.astro.setup_mr.setupMR.label}}
\pysigstartsignatures
\pysigline{\sphinxbfcode{\sphinxupquote{label}}}
\pysigstopsignatures
\sphinxAtStartPar
Attribute providing the label the data is references for figures.

\end{fulllineitems}

\index{latexCite (nucleardatapy.astro.setup\_mr.setupMR attribute)@\spxentry{latexCite}\spxextra{nucleardatapy.astro.setup\_mr.setupMR attribute}}

\begin{fulllineitems}
\phantomsection\label{\detokenize{source/api/setup_astro_mr:nucleardatapy.astro.setup_mr.setupMR.latexCite}}
\pysigstartsignatures
\pysigline{\sphinxbfcode{\sphinxupquote{latexCite}}}
\pysigstopsignatures
\sphinxAtStartPar
Attribute latexCite.

\end{fulllineitems}

\index{mass (nucleardatapy.astro.setup\_mr.setupMR attribute)@\spxentry{mass}\spxextra{nucleardatapy.astro.setup\_mr.setupMR attribute}}

\begin{fulllineitems}
\phantomsection\label{\detokenize{source/api/setup_astro_mr:nucleardatapy.astro.setup_mr.setupMR.mass}}
\pysigstartsignatures
\pysigline{\sphinxbfcode{\sphinxupquote{mass}}}
\pysigstopsignatures
\sphinxAtStartPar
Attribute the observational mass of the source.

\end{fulllineitems}

\index{mass\_sig\_do (nucleardatapy.astro.setup\_mr.setupMR attribute)@\spxentry{mass\_sig\_do}\spxextra{nucleardatapy.astro.setup\_mr.setupMR attribute}}

\begin{fulllineitems}
\phantomsection\label{\detokenize{source/api/setup_astro_mr:nucleardatapy.astro.setup_mr.setupMR.mass_sig_do}}
\pysigstartsignatures
\pysigline{\sphinxbfcode{\sphinxupquote{mass\_sig\_do}}}
\pysigstopsignatures
\sphinxAtStartPar
Attribute the negative uncertainty.

\end{fulllineitems}

\index{mass\_sig\_up (nucleardatapy.astro.setup\_mr.setupMR attribute)@\spxentry{mass\_sig\_up}\spxextra{nucleardatapy.astro.setup\_mr.setupMR attribute}}

\begin{fulllineitems}
\phantomsection\label{\detokenize{source/api/setup_astro_mr:nucleardatapy.astro.setup_mr.setupMR.mass_sig_up}}
\pysigstartsignatures
\pysigline{\sphinxbfcode{\sphinxupquote{mass\_sig\_up}}}
\pysigstopsignatures
\sphinxAtStartPar
Attribute the positive uncertainty.

\end{fulllineitems}

\index{note (nucleardatapy.astro.setup\_mr.setupMR attribute)@\spxentry{note}\spxextra{nucleardatapy.astro.setup\_mr.setupMR attribute}}

\begin{fulllineitems}
\phantomsection\label{\detokenize{source/api/setup_astro_mr:nucleardatapy.astro.setup_mr.setupMR.note}}
\pysigstartsignatures
\pysigline{\sphinxbfcode{\sphinxupquote{note}}}
\pysigstopsignatures
\sphinxAtStartPar
Attribute providing additional notes about the observation.

\end{fulllineitems}

\index{print\_output() (nucleardatapy.astro.setup\_mr.setupMR method)@\spxentry{print\_output()}\spxextra{nucleardatapy.astro.setup\_mr.setupMR method}}

\begin{fulllineitems}
\phantomsection\label{\detokenize{source/api/setup_astro_mr:nucleardatapy.astro.setup_mr.setupMR.print_output}}
\pysigstartsignatures
\pysiglinewithargsret{\sphinxbfcode{\sphinxupquote{print\_output}}}{}{}
\pysigstopsignatures
\sphinxAtStartPar
Method which print outputs on terminal’s screen.

\end{fulllineitems}

\index{print\_table() (nucleardatapy.astro.setup\_mr.setupMR method)@\spxentry{print\_table()}\spxextra{nucleardatapy.astro.setup\_mr.setupMR method}}

\begin{fulllineitems}
\phantomsection\label{\detokenize{source/api/setup_astro_mr:nucleardatapy.astro.setup_mr.setupMR.print_table}}
\pysigstartsignatures
\pysiglinewithargsret{\sphinxbfcode{\sphinxupquote{print\_table}}}{}{}
\pysigstopsignatures
\sphinxAtStartPar
Method which print outputs in table format (latex) on terminal’s screen.

\end{fulllineitems}

\index{rad (nucleardatapy.astro.setup\_mr.setupMR attribute)@\spxentry{rad}\spxextra{nucleardatapy.astro.setup\_mr.setupMR attribute}}

\begin{fulllineitems}
\phantomsection\label{\detokenize{source/api/setup_astro_mr:nucleardatapy.astro.setup_mr.setupMR.rad}}
\pysigstartsignatures
\pysigline{\sphinxbfcode{\sphinxupquote{rad}}}
\pysigstopsignatures
\sphinxAtStartPar
Attribute the observational mass of the source.

\end{fulllineitems}

\index{rad\_sig\_do (nucleardatapy.astro.setup\_mr.setupMR attribute)@\spxentry{rad\_sig\_do}\spxextra{nucleardatapy.astro.setup\_mr.setupMR attribute}}

\begin{fulllineitems}
\phantomsection\label{\detokenize{source/api/setup_astro_mr:nucleardatapy.astro.setup_mr.setupMR.rad_sig_do}}
\pysigstartsignatures
\pysigline{\sphinxbfcode{\sphinxupquote{rad\_sig\_do}}}
\pysigstopsignatures
\sphinxAtStartPar
Attribute the negative uncertainty.

\end{fulllineitems}

\index{rad\_sig\_up (nucleardatapy.astro.setup\_mr.setupMR attribute)@\spxentry{rad\_sig\_up}\spxextra{nucleardatapy.astro.setup\_mr.setupMR attribute}}

\begin{fulllineitems}
\phantomsection\label{\detokenize{source/api/setup_astro_mr:nucleardatapy.astro.setup_mr.setupMR.rad_sig_up}}
\pysigstartsignatures
\pysigline{\sphinxbfcode{\sphinxupquote{rad\_sig\_up}}}
\pysigstopsignatures
\sphinxAtStartPar
Attribute the positive uncertainty.

\end{fulllineitems}

\index{ref (nucleardatapy.astro.setup\_mr.setupMR attribute)@\spxentry{ref}\spxextra{nucleardatapy.astro.setup\_mr.setupMR attribute}}

\begin{fulllineitems}
\phantomsection\label{\detokenize{source/api/setup_astro_mr:nucleardatapy.astro.setup_mr.setupMR.ref}}
\pysigstartsignatures
\pysigline{\sphinxbfcode{\sphinxupquote{ref}}}
\pysigstopsignatures
\sphinxAtStartPar
Attribute providing the full reference to the paper to be citted.

\end{fulllineitems}


\end{fulllineitems}

\index{setupMRAverage (class in nucleardatapy.astro.setup\_mr)@\spxentry{setupMRAverage}\spxextra{class in nucleardatapy.astro.setup\_mr}}

\begin{fulllineitems}
\phantomsection\label{\detokenize{source/api/setup_astro_mr:nucleardatapy.astro.setup_mr.setupMRAverage}}
\pysigstartsignatures
\pysiglinewithargsret{\sphinxbfcode{\sphinxupquote{class\DUrole{w}{  }}}\sphinxcode{\sphinxupquote{nucleardatapy.astro.setup\_mr.}}\sphinxbfcode{\sphinxupquote{setupMRAverage}}}{\emph{\DUrole{n}{source}\DUrole{o}{=}\DUrole{default_value}{\textquotesingle{}J1614\textendash{}2230\textquotesingle{}}}}{}
\pysigstopsignatures
\sphinxAtStartPar
Instantiate the observational mass for a given source and averaged over obs.

\sphinxAtStartPar
This choice is defined in the variable \sphinxtitleref{source}.

\sphinxAtStartPar
\sphinxtitleref{source} can chosen among the following ones: ‘J1614\textendash{}2230’.
\begin{quote}\begin{description}
\sphinxlineitem{Parameters}
\sphinxAtStartPar
\sphinxstyleliteralstrong{\sphinxupquote{source}} (\sphinxstyleliteralemphasis{\sphinxupquote{str}}\sphinxstyleliteralemphasis{\sphinxupquote{, }}\sphinxstyleliteralemphasis{\sphinxupquote{optional.}}) \textendash{} Fix the name of \sphinxtitleref{source}. Default value: ‘J1614\textendash{}2230’.

\end{description}\end{quote}

\sphinxAtStartPar
\sphinxstylestrong{Attributes:}
\index{print\_output() (nucleardatapy.astro.setup\_mr.setupMRAverage method)@\spxentry{print\_output()}\spxextra{nucleardatapy.astro.setup\_mr.setupMRAverage method}}

\begin{fulllineitems}
\phantomsection\label{\detokenize{source/api/setup_astro_mr:nucleardatapy.astro.setup_mr.setupMRAverage.print_output}}
\pysigstartsignatures
\pysiglinewithargsret{\sphinxbfcode{\sphinxupquote{print\_output}}}{}{}
\pysigstopsignatures
\sphinxAtStartPar
Method which print outputs on terminal’s screen.

\end{fulllineitems}

\index{print\_table() (nucleardatapy.astro.setup\_mr.setupMRAverage method)@\spxentry{print\_table()}\spxextra{nucleardatapy.astro.setup\_mr.setupMRAverage method}}

\begin{fulllineitems}
\phantomsection\label{\detokenize{source/api/setup_astro_mr:nucleardatapy.astro.setup_mr.setupMRAverage.print_table}}
\pysigstartsignatures
\pysiglinewithargsret{\sphinxbfcode{\sphinxupquote{print\_table}}}{}{}
\pysigstopsignatures
\sphinxAtStartPar
Method which print outputs in table format (latex) on terminal’s screen.

\end{fulllineitems}


\end{fulllineitems}


\sphinxAtStartPar
Here is a figure which is produced with the Python sample: /nucleardatapy\_sample/plots/plot\_astro\_setupMR.py

\begin{figure}[htbp]
\centering
\capstart

\noindent\sphinxincludegraphics[scale=0.7]{{plot_astro_setupMR}.png}
\caption{The masses and radii measured by NICER. The different colors correspond to the different sources.}\label{\detokenize{source/api/setup_astro_mr:id1}}\end{figure}

\sphinxstepscope


\section{SetupAstroGW}
\label{\detokenize{source/api/setup_astro_gw:setupastrogw}}\label{\detokenize{source/api/setup_astro_gw::doc}}\phantomsection\label{\detokenize{source/api/setup_astro_gw:module-nucleardatapy.astro.setup_gw}}\index{module@\spxentry{module}!nucleardatapy.astro.setup\_gw@\spxentry{nucleardatapy.astro.setup\_gw}}\index{nucleardatapy.astro.setup\_gw@\spxentry{nucleardatapy.astro.setup\_gw}!module@\spxentry{module}}\index{gw\_hyps() (in module nucleardatapy.astro.setup\_gw)@\spxentry{gw\_hyps()}\spxextra{in module nucleardatapy.astro.setup\_gw}}

\begin{fulllineitems}
\phantomsection\label{\detokenize{source/api/setup_astro_gw:nucleardatapy.astro.setup_gw.gw_hyps}}
\pysigstartsignatures
\pysiglinewithargsret{\sphinxcode{\sphinxupquote{nucleardatapy.astro.setup\_gw.}}\sphinxbfcode{\sphinxupquote{gw\_hyps}}}{\emph{\DUrole{n}{source}}}{}
\pysigstopsignatures
\sphinxAtStartPar
Return a list of observations for a given source and print them all on the prompt.
\begin{quote}\begin{description}
\sphinxlineitem{Parameters}
\sphinxAtStartPar
\sphinxstyleliteralstrong{\sphinxupquote{source}} (\sphinxstyleliteralemphasis{\sphinxupquote{str.}}) \textendash{} The source for which there are different hypotheses.

\sphinxlineitem{Returns}
\sphinxAtStartPar
The list of hypotheses.     If source == ‘GW170817’: 1, 2, 3, 4, 5.

\sphinxlineitem{Return type}
\sphinxAtStartPar
list{[}str{]}.

\end{description}\end{quote}

\end{fulllineitems}

\index{gw\_sources() (in module nucleardatapy.astro.setup\_gw)@\spxentry{gw\_sources()}\spxextra{in module nucleardatapy.astro.setup\_gw}}

\begin{fulllineitems}
\phantomsection\label{\detokenize{source/api/setup_astro_gw:nucleardatapy.astro.setup_gw.gw_sources}}
\pysigstartsignatures
\pysiglinewithargsret{\sphinxcode{\sphinxupquote{nucleardatapy.astro.setup\_gw.}}\sphinxbfcode{\sphinxupquote{gw\_sources}}}{}{}
\pysigstopsignatures
\sphinxAtStartPar
Return a list of the astrophysical sources for which a mass is given
\begin{quote}\begin{description}
\sphinxlineitem{Returns}
\sphinxAtStartPar
The list of sources.

\sphinxlineitem{Return type}
\sphinxAtStartPar
list{[}str{]}.

\end{description}\end{quote}

\end{fulllineitems}

\index{setupGW (class in nucleardatapy.astro.setup\_gw)@\spxentry{setupGW}\spxextra{class in nucleardatapy.astro.setup\_gw}}

\begin{fulllineitems}
\phantomsection\label{\detokenize{source/api/setup_astro_gw:nucleardatapy.astro.setup_gw.setupGW}}
\pysigstartsignatures
\pysiglinewithargsret{\sphinxbfcode{\sphinxupquote{class\DUrole{w}{  }}}\sphinxcode{\sphinxupquote{nucleardatapy.astro.setup\_gw.}}\sphinxbfcode{\sphinxupquote{setupGW}}}{\emph{\DUrole{n}{source}\DUrole{o}{=}\DUrole{default_value}{\textquotesingle{}GW170817\textquotesingle{}}}, \emph{\DUrole{n}{hyp}\DUrole{o}{=}\DUrole{default_value}{1}}}{}
\pysigstopsignatures
\sphinxAtStartPar
Instantiate the tidal deformability for a given source and obs.

\sphinxAtStartPar
This choice is defined in the variables \sphinxtitleref{source} and \sphinxtitleref{obs}.

\sphinxAtStartPar
\sphinxtitleref{source} can chosen among the following ones: ‘GW170817’.

\sphinxAtStartPar
\sphinxtitleref{obs} depends on the chosen source.
\begin{quote}\begin{description}
\sphinxlineitem{Parameters}\begin{itemize}
\item {} 
\sphinxAtStartPar
\sphinxstyleliteralstrong{\sphinxupquote{source}} (\sphinxstyleliteralemphasis{\sphinxupquote{str}}\sphinxstyleliteralemphasis{\sphinxupquote{, }}\sphinxstyleliteralemphasis{\sphinxupquote{optional.}}) \textendash{} Fix the name of \sphinxtitleref{source}. Default value: ‘GW170817’.

\item {} 
\sphinxAtStartPar
\sphinxstyleliteralstrong{\sphinxupquote{obs}} (\sphinxstyleliteralemphasis{\sphinxupquote{str}}\sphinxstyleliteralemphasis{\sphinxupquote{, }}\sphinxstyleliteralemphasis{\sphinxupquote{optional.}}) \textendash{} Fix the \sphinxtitleref{obs}. Default value: 1.

\end{itemize}

\end{description}\end{quote}

\sphinxAtStartPar
\sphinxstylestrong{Attributes:}
\index{init\_self() (nucleardatapy.astro.setup\_gw.setupGW method)@\spxentry{init\_self()}\spxextra{nucleardatapy.astro.setup\_gw.setupGW method}}

\begin{fulllineitems}
\phantomsection\label{\detokenize{source/api/setup_astro_gw:nucleardatapy.astro.setup_gw.setupGW.init_self}}
\pysigstartsignatures
\pysiglinewithargsret{\sphinxbfcode{\sphinxupquote{init\_self}}}{}{}
\pysigstopsignatures
\sphinxAtStartPar
Initialize variables in self.

\end{fulllineitems}

\index{label (nucleardatapy.astro.setup\_gw.setupGW attribute)@\spxentry{label}\spxextra{nucleardatapy.astro.setup\_gw.setupGW attribute}}

\begin{fulllineitems}
\phantomsection\label{\detokenize{source/api/setup_astro_gw:nucleardatapy.astro.setup_gw.setupGW.label}}
\pysigstartsignatures
\pysigline{\sphinxbfcode{\sphinxupquote{label}}}
\pysigstopsignatures
\sphinxAtStartPar
Attribute providing the label the data is references for figures.

\end{fulllineitems}

\index{note (nucleardatapy.astro.setup\_gw.setupGW attribute)@\spxentry{note}\spxextra{nucleardatapy.astro.setup\_gw.setupGW attribute}}

\begin{fulllineitems}
\phantomsection\label{\detokenize{source/api/setup_astro_gw:nucleardatapy.astro.setup_gw.setupGW.note}}
\pysigstartsignatures
\pysigline{\sphinxbfcode{\sphinxupquote{note}}}
\pysigstopsignatures
\sphinxAtStartPar
Attribute providing additional notes about the data.

\end{fulllineitems}

\index{print\_output() (nucleardatapy.astro.setup\_gw.setupGW method)@\spxentry{print\_output()}\spxextra{nucleardatapy.astro.setup\_gw.setupGW method}}

\begin{fulllineitems}
\phantomsection\label{\detokenize{source/api/setup_astro_gw:nucleardatapy.astro.setup_gw.setupGW.print_output}}
\pysigstartsignatures
\pysiglinewithargsret{\sphinxbfcode{\sphinxupquote{print\_output}}}{}{}
\pysigstopsignatures
\sphinxAtStartPar
Method which print outputs on terminal’s screen.

\end{fulllineitems}

\index{print\_table() (nucleardatapy.astro.setup\_gw.setupGW method)@\spxentry{print\_table()}\spxextra{nucleardatapy.astro.setup\_gw.setupGW method}}

\begin{fulllineitems}
\phantomsection\label{\detokenize{source/api/setup_astro_gw:nucleardatapy.astro.setup_gw.setupGW.print_table}}
\pysigstartsignatures
\pysiglinewithargsret{\sphinxbfcode{\sphinxupquote{print\_table}}}{}{}
\pysigstopsignatures
\sphinxAtStartPar
Method which print outputs in table format (latex) on terminal’s screen.

\end{fulllineitems}

\index{ref (nucleardatapy.astro.setup\_gw.setupGW attribute)@\spxentry{ref}\spxextra{nucleardatapy.astro.setup\_gw.setupGW attribute}}

\begin{fulllineitems}
\phantomsection\label{\detokenize{source/api/setup_astro_gw:nucleardatapy.astro.setup_gw.setupGW.ref}}
\pysigstartsignatures
\pysigline{\sphinxbfcode{\sphinxupquote{ref}}}
\pysigstopsignatures
\sphinxAtStartPar
Attribute providing the full reference to the paper to be citted.

\end{fulllineitems}


\end{fulllineitems}

\index{setupGWAverage (class in nucleardatapy.astro.setup\_gw)@\spxentry{setupGWAverage}\spxextra{class in nucleardatapy.astro.setup\_gw}}

\begin{fulllineitems}
\phantomsection\label{\detokenize{source/api/setup_astro_gw:nucleardatapy.astro.setup_gw.setupGWAverage}}
\pysigstartsignatures
\pysiglinewithargsret{\sphinxbfcode{\sphinxupquote{class\DUrole{w}{  }}}\sphinxcode{\sphinxupquote{nucleardatapy.astro.setup\_gw.}}\sphinxbfcode{\sphinxupquote{setupGWAverage}}}{\emph{\DUrole{n}{source}\DUrole{o}{=}\DUrole{default_value}{\textquotesingle{}GW170817\textquotesingle{}}}}{}
\pysigstopsignatures
\sphinxAtStartPar
Instantiate the total mass for a given source and averaged over hypotheses.

\sphinxAtStartPar
This choice is defined in the variable \sphinxtitleref{source}.

\sphinxAtStartPar
\sphinxtitleref{source} can chosen among the following ones: ‘GW170817’.
\begin{quote}\begin{description}
\sphinxlineitem{Parameters}
\sphinxAtStartPar
\sphinxstyleliteralstrong{\sphinxupquote{source}} (\sphinxstyleliteralemphasis{\sphinxupquote{str}}\sphinxstyleliteralemphasis{\sphinxupquote{, }}\sphinxstyleliteralemphasis{\sphinxupquote{optional.}}) \textendash{} Fix the name of \sphinxtitleref{source}. Default value: ‘GW170817’.

\end{description}\end{quote}

\sphinxAtStartPar
\sphinxstylestrong{Attributes:}
\index{init\_self() (nucleardatapy.astro.setup\_gw.setupGWAverage method)@\spxentry{init\_self()}\spxextra{nucleardatapy.astro.setup\_gw.setupGWAverage method}}

\begin{fulllineitems}
\phantomsection\label{\detokenize{source/api/setup_astro_gw:nucleardatapy.astro.setup_gw.setupGWAverage.init_self}}
\pysigstartsignatures
\pysiglinewithargsret{\sphinxbfcode{\sphinxupquote{init\_self}}}{}{}
\pysigstopsignatures
\sphinxAtStartPar
Initialize variables in self.

\end{fulllineitems}

\index{print\_output() (nucleardatapy.astro.setup\_gw.setupGWAverage method)@\spxentry{print\_output()}\spxextra{nucleardatapy.astro.setup\_gw.setupGWAverage method}}

\begin{fulllineitems}
\phantomsection\label{\detokenize{source/api/setup_astro_gw:nucleardatapy.astro.setup_gw.setupGWAverage.print_output}}
\pysigstartsignatures
\pysiglinewithargsret{\sphinxbfcode{\sphinxupquote{print\_output}}}{}{}
\pysigstopsignatures
\sphinxAtStartPar
Method which print outputs on terminal’s screen.

\end{fulllineitems}

\index{print\_table() (nucleardatapy.astro.setup\_gw.setupGWAverage method)@\spxentry{print\_table()}\spxextra{nucleardatapy.astro.setup\_gw.setupGWAverage method}}

\begin{fulllineitems}
\phantomsection\label{\detokenize{source/api/setup_astro_gw:nucleardatapy.astro.setup_gw.setupGWAverage.print_table}}
\pysigstartsignatures
\pysiglinewithargsret{\sphinxbfcode{\sphinxupquote{print\_table}}}{}{}
\pysigstopsignatures
\sphinxAtStartPar
Method which print outputs in table format (latex) on terminal’s screen.

\end{fulllineitems}


\end{fulllineitems}


\sphinxAtStartPar
Here is a figure which is produced with the Python sample: /nucleardatapy\_sample/plots/plot\_astro\_setupGW.py

\begin{figure}[htbp]
\centering
\capstart

\noindent\sphinxincludegraphics[scale=0.7]{{plot_astro_setupGW}.png}
\caption{Estimation of the effective tidal deformability from difference sources (different colors) and different hypotheses on the pulsar spin and the waveform employed for match filtering.}\label{\detokenize{source/api/setup_astro_gw:id1}}\end{figure}

\sphinxstepscope


\section{SetupCorrEsymLsym}
\label{\detokenize{source/api/setup_corr_EsymLsym:setupcorresymlsym}}\label{\detokenize{source/api/setup_corr_EsymLsym::doc}}\phantomsection\label{\detokenize{source/api/setup_corr_EsymLsym:module-nucleardatapy.corr.setup_EsymLsym}}\index{module@\spxentry{module}!nucleardatapy.corr.setup\_EsymLsym@\spxentry{nucleardatapy.corr.setup\_EsymLsym}}\index{nucleardatapy.corr.setup\_EsymLsym@\spxentry{nucleardatapy.corr.setup\_EsymLsym}!module@\spxentry{module}}\index{EsymLsym\_constraints() (in module nucleardatapy.corr.setup\_EsymLsym)@\spxentry{EsymLsym\_constraints()}\spxextra{in module nucleardatapy.corr.setup\_EsymLsym}}

\begin{fulllineitems}
\phantomsection\label{\detokenize{source/api/setup_corr_EsymLsym:nucleardatapy.corr.setup_EsymLsym.EsymLsym_constraints}}
\pysigstartsignatures
\pysiglinewithargsret{\sphinxcode{\sphinxupquote{nucleardatapy.corr.setup\_EsymLsym.}}\sphinxbfcode{\sphinxupquote{EsymLsym\_constraints}}}{}{}
\pysigstopsignatures
\sphinxAtStartPar
Return a list of constraints available in this toolkit in the     following list: ‘2009\sphinxhyphen{}HIC’, ‘2010\sphinxhyphen{}RNP’, ‘2012\sphinxhyphen{}FRDM’, ‘2013\sphinxhyphen{}NS’,     ‘2014\sphinxhyphen{}IAS’, ‘2014\sphinxhyphen{}IAS+RNP’, ‘2015\sphinxhyphen{}POL\sphinxhyphen{}208PB’, ‘2015\sphinxhyphen{}POL\sphinxhyphen{}120SN’,     ‘2015\sphinxhyphen{}POL\sphinxhyphen{}68NI’, ‘2017\sphinxhyphen{}UG’, ‘2021\sphinxhyphen{}PREXII\sphinxhyphen{}Reed’,     ‘2021\sphinxhyphen{}PREXII\sphinxhyphen{}Reinhard’, ‘2023\sphinxhyphen{}PREXII+CREX\sphinxhyphen{}Zhang’; and     print them all on the prompt.
\begin{quote}\begin{description}
\sphinxlineitem{Returns}
\sphinxAtStartPar
The list of constraints.

\sphinxlineitem{Return type}
\sphinxAtStartPar
list{[}str{]}.

\end{description}\end{quote}

\end{fulllineitems}

\index{setupEsymLsym (class in nucleardatapy.corr.setup\_EsymLsym)@\spxentry{setupEsymLsym}\spxextra{class in nucleardatapy.corr.setup\_EsymLsym}}

\begin{fulllineitems}
\phantomsection\label{\detokenize{source/api/setup_corr_EsymLsym:nucleardatapy.corr.setup_EsymLsym.setupEsymLsym}}
\pysigstartsignatures
\pysiglinewithargsret{\sphinxbfcode{\sphinxupquote{class\DUrole{w}{  }}}\sphinxcode{\sphinxupquote{nucleardatapy.corr.setup\_EsymLsym.}}\sphinxbfcode{\sphinxupquote{setupEsymLsym}}}{\emph{\DUrole{n}{constraint}\DUrole{o}{=}\DUrole{default_value}{\textquotesingle{}2014\sphinxhyphen{}IAS\textquotesingle{}}}}{}
\pysigstopsignatures
\sphinxAtStartPar
Instantiate the values of Esym and Lsym from the constraint.

\sphinxAtStartPar
The name of the constraint to be chosen in the     following list: ‘2009\sphinxhyphen{}HIC’, ‘2010\sphinxhyphen{}RNP’, ‘2012\sphinxhyphen{}FRDM’, ‘2013\sphinxhyphen{}NS’,     ‘2014\sphinxhyphen{}IAS’, ‘2014\sphinxhyphen{}IAS+RNP’, ‘2015\sphinxhyphen{}POL\sphinxhyphen{}208PB’, ‘2015\sphinxhyphen{}POL\sphinxhyphen{}120SN’,     ‘2015\sphinxhyphen{}POL\sphinxhyphen{}68NI’, ‘2017\sphinxhyphen{}UG’, ‘2021\sphinxhyphen{}PREXII\sphinxhyphen{}Reed’,     ‘2021\sphinxhyphen{}PREXII\sphinxhyphen{}Reinhard’, ‘2021\sphinxhyphen{}PREXII+CREX\sphinxhyphen{}Zhang’.
\begin{quote}\begin{description}
\sphinxlineitem{Parameters}
\sphinxAtStartPar
\sphinxstyleliteralstrong{\sphinxupquote{constraint}} (\sphinxstyleliteralemphasis{\sphinxupquote{str}}\sphinxstyleliteralemphasis{\sphinxupquote{, }}\sphinxstyleliteralemphasis{\sphinxupquote{optional.}}) \textendash{} Fix the name of \sphinxtitleref{constraint}. Default value: ‘2014\sphinxhyphen{}IAS’.

\end{description}\end{quote}

\sphinxAtStartPar
\sphinxstylestrong{Attributes:}
\index{constraint (nucleardatapy.corr.setup\_EsymLsym.setupEsymLsym attribute)@\spxentry{constraint}\spxextra{nucleardatapy.corr.setup\_EsymLsym.setupEsymLsym attribute}}

\begin{fulllineitems}
\phantomsection\label{\detokenize{source/api/setup_corr_EsymLsym:nucleardatapy.corr.setup_EsymLsym.setupEsymLsym.constraint}}
\pysigstartsignatures
\pysigline{\sphinxbfcode{\sphinxupquote{constraint}}}
\pysigstopsignatures
\sphinxAtStartPar
Attribute constraint.

\end{fulllineitems}

\index{init\_self() (nucleardatapy.corr.setup\_EsymLsym.setupEsymLsym method)@\spxentry{init\_self()}\spxextra{nucleardatapy.corr.setup\_EsymLsym.setupEsymLsym method}}

\begin{fulllineitems}
\phantomsection\label{\detokenize{source/api/setup_corr_EsymLsym:nucleardatapy.corr.setup_EsymLsym.setupEsymLsym.init_self}}
\pysigstartsignatures
\pysiglinewithargsret{\sphinxbfcode{\sphinxupquote{init\_self}}}{}{}
\pysigstopsignatures
\sphinxAtStartPar
Initialize variables in self.

\end{fulllineitems}

\index{label (nucleardatapy.corr.setup\_EsymLsym.setupEsymLsym attribute)@\spxentry{label}\spxextra{nucleardatapy.corr.setup\_EsymLsym.setupEsymLsym attribute}}

\begin{fulllineitems}
\phantomsection\label{\detokenize{source/api/setup_corr_EsymLsym:nucleardatapy.corr.setup_EsymLsym.setupEsymLsym.label}}
\pysigstartsignatures
\pysigline{\sphinxbfcode{\sphinxupquote{label}}}
\pysigstopsignatures
\sphinxAtStartPar
Attribute providing the label the data is references for figures.

\end{fulllineitems}

\index{note (nucleardatapy.corr.setup\_EsymLsym.setupEsymLsym attribute)@\spxentry{note}\spxextra{nucleardatapy.corr.setup\_EsymLsym.setupEsymLsym attribute}}

\begin{fulllineitems}
\phantomsection\label{\detokenize{source/api/setup_corr_EsymLsym:nucleardatapy.corr.setup_EsymLsym.setupEsymLsym.note}}
\pysigstartsignatures
\pysigline{\sphinxbfcode{\sphinxupquote{note}}}
\pysigstopsignatures
\sphinxAtStartPar
Attribute providing additional notes about the constraint.

\end{fulllineitems}

\index{print\_outputs() (nucleardatapy.corr.setup\_EsymLsym.setupEsymLsym method)@\spxentry{print\_outputs()}\spxextra{nucleardatapy.corr.setup\_EsymLsym.setupEsymLsym method}}

\begin{fulllineitems}
\phantomsection\label{\detokenize{source/api/setup_corr_EsymLsym:nucleardatapy.corr.setup_EsymLsym.setupEsymLsym.print_outputs}}
\pysigstartsignatures
\pysiglinewithargsret{\sphinxbfcode{\sphinxupquote{print\_outputs}}}{}{}
\pysigstopsignatures
\sphinxAtStartPar
Method which print outputs on terminal’s screen.

\end{fulllineitems}

\index{ref (nucleardatapy.corr.setup\_EsymLsym.setupEsymLsym attribute)@\spxentry{ref}\spxextra{nucleardatapy.corr.setup\_EsymLsym.setupEsymLsym attribute}}

\begin{fulllineitems}
\phantomsection\label{\detokenize{source/api/setup_corr_EsymLsym:nucleardatapy.corr.setup_EsymLsym.setupEsymLsym.ref}}
\pysigstartsignatures
\pysigline{\sphinxbfcode{\sphinxupquote{ref}}}
\pysigstopsignatures
\sphinxAtStartPar
Attribute providing the full reference to the paper to be citted.

\end{fulllineitems}


\end{fulllineitems}


\sphinxAtStartPar
Here are a set of figures which are produced with the Python sample: /nucleardatapy\_sample/plots/plot\_corr\_setupEsymLsym.py

\begin{figure}[htbp]
\centering
\capstart

\noindent\sphinxincludegraphics[scale=0.7]{{plot_corr_setupEsymLsym}.png}
\caption{This figure shows the Esym,2 versus Lsym,2 correlation for the different constraints availble in the nucleardatapy toolkit.}\label{\detokenize{source/api/setup_corr_EsymLsym:id1}}\end{figure}


\chapter{Indices and tables}
\label{\detokenize{index:indices-and-tables}}\begin{itemize}
\item {} 
\sphinxAtStartPar
\DUrole{xref,std,std-ref}{genindex}

\item {} 
\sphinxAtStartPar
\DUrole{xref,std,std-ref}{modindex}

\item {} 
\sphinxAtStartPar
\DUrole{xref,std,std-ref}{search}

\end{itemize}


\renewcommand{\indexname}{Python Module Index}
\begin{sphinxtheindex}
\let\bigletter\sphinxstyleindexlettergroup
\bigletter{n}
\item\relax\sphinxstyleindexentry{nucleardatapy}\sphinxstyleindexpageref{source/generated/nucleardatapy:\detokenize{module-nucleardatapy}}
\item\relax\sphinxstyleindexentry{nucleardatapy.astro.setup\_gw}\sphinxstyleindexpageref{source/api/setup_astro_gw:\detokenize{module-nucleardatapy.astro.setup_gw}}
\item\relax\sphinxstyleindexentry{nucleardatapy.astro.setup\_masses}\sphinxstyleindexpageref{source/api/setup_astro_masses:\detokenize{module-nucleardatapy.astro.setup_masses}}
\item\relax\sphinxstyleindexentry{nucleardatapy.astro.setup\_mr}\sphinxstyleindexpageref{source/api/setup_astro_mr:\detokenize{module-nucleardatapy.astro.setup_mr}}
\item\relax\sphinxstyleindexentry{nucleardatapy.astro.setup\_mtov}\sphinxstyleindexpageref{source/api/setup_astro_mtov:\detokenize{module-nucleardatapy.astro.setup_mtov}}
\item\relax\sphinxstyleindexentry{nucleardatapy.astro.setup\_mup}\sphinxstyleindexpageref{source/api/setup_astro_mup:\detokenize{module-nucleardatapy.astro.setup_mup}}
\item\relax\sphinxstyleindexentry{nucleardatapy.corr.setup\_EsymLsym}\sphinxstyleindexpageref{source/api/setup_corr_EsymLsym:\detokenize{module-nucleardatapy.corr.setup_EsymLsym}}
\item\relax\sphinxstyleindexentry{nucleardatapy.crust.setup\_crust}\sphinxstyleindexpageref{source/api/setup_crust:\detokenize{module-nucleardatapy.crust.setup_crust}}
\item\relax\sphinxstyleindexentry{nucleardatapy.eos.setup\_am}\sphinxstyleindexpageref{source/api/setup_eos_am:\detokenize{module-nucleardatapy.eos.setup_am}}
\item\relax\sphinxstyleindexentry{nucleardatapy.eos.setup\_beta}\sphinxstyleindexpageref{source/api/setup_eos_beta:\detokenize{module-nucleardatapy.eos.setup_beta}}
\item\relax\sphinxstyleindexentry{nucleardatapy.eos.setup\_esym}\sphinxstyleindexpageref{source/api/setup_eos_esym:\detokenize{module-nucleardatapy.eos.setup_esym}}
\item\relax\sphinxstyleindexentry{nucleardatapy.eos.setup\_ffg}\sphinxstyleindexpageref{source/api/setup_eos_ffg:\detokenize{module-nucleardatapy.eos.setup_ffg}}
\item\relax\sphinxstyleindexentry{nucleardatapy.eos.setup\_hic}\sphinxstyleindexpageref{source/api/setup_eos_hic:\detokenize{module-nucleardatapy.eos.setup_hic}}
\item\relax\sphinxstyleindexentry{nucleardatapy.eos.setup\_micro}\sphinxstyleindexpageref{source/api/setup_eos_micro:\detokenize{module-nucleardatapy.eos.setup_micro}}
\item\relax\sphinxstyleindexentry{nucleardatapy.eos.setup\_micro\_band}\sphinxstyleindexpageref{source/api/setup_eos_micro_band:\detokenize{module-nucleardatapy.eos.setup_micro_band}}
\item\relax\sphinxstyleindexentry{nucleardatapy.eos.setup\_micro\_esym}\sphinxstyleindexpageref{source/api/setup_eos_micro_esym:\detokenize{module-nucleardatapy.eos.setup_micro_esym}}
\item\relax\sphinxstyleindexentry{nucleardatapy.eos.setup\_micro\_gap}\sphinxstyleindexpageref{source/api/setup_eos_micro_gap:\detokenize{module-nucleardatapy.eos.setup_micro_gap}}
\item\relax\sphinxstyleindexentry{nucleardatapy.eos.setup\_micro\_lp}\sphinxstyleindexpageref{source/api/setup_eos_micro_lp:\detokenize{module-nucleardatapy.eos.setup_micro_lp}}
\item\relax\sphinxstyleindexentry{nucleardatapy.eos.setup\_pheno}\sphinxstyleindexpageref{source/api/setup_eos_pheno:\detokenize{module-nucleardatapy.eos.setup_pheno}}
\item\relax\sphinxstyleindexentry{nucleardatapy.eos.setup\_pheno\_esym}\sphinxstyleindexpageref{source/api/setup_eos_pheno_esym:\detokenize{module-nucleardatapy.eos.setup_pheno_esym}}
\item\relax\sphinxstyleindexentry{nucleardatapy.nuc.setup\_be\_exp}\sphinxstyleindexpageref{source/api/setup_nuc_be_exp:\detokenize{module-nucleardatapy.nuc.setup_be_exp}}
\item\relax\sphinxstyleindexentry{nucleardatapy.nuc.setup\_be\_theo}\sphinxstyleindexpageref{source/api/setup_nuc_be_theo:\detokenize{module-nucleardatapy.nuc.setup_be_theo}}
\item\relax\sphinxstyleindexentry{nucleardatapy.nuc.setup\_isgmr\_exp}\sphinxstyleindexpageref{source/api/setup_nuc_isgmr_exp:\detokenize{module-nucleardatapy.nuc.setup_isgmr_exp}}
\item\relax\sphinxstyleindexentry{nucleardatapy.nuc.setup\_rch\_exp}\sphinxstyleindexpageref{source/api/setup_nuc_rch_exp:\detokenize{module-nucleardatapy.nuc.setup_rch_exp}}
\item\relax\sphinxstyleindexentry{nucleardatapy.nuc.setup\_rch\_theo}\sphinxstyleindexpageref{source/api/setup_nuc_rch_theo:\detokenize{module-nucleardatapy.nuc.setup_rch_theo}}
\end{sphinxtheindex}

\renewcommand{\indexname}{Index}
\printindex
\end{document}