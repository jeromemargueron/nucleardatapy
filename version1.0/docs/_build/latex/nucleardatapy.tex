%% Generated by Sphinx.
\def\sphinxdocclass{report}
\documentclass[letterpaper,10pt,english]{sphinxmanual}
\ifdefined\pdfpxdimen
   \let\sphinxpxdimen\pdfpxdimen\else\newdimen\sphinxpxdimen
\fi \sphinxpxdimen=.75bp\relax
\ifdefined\pdfimageresolution
    \pdfimageresolution= \numexpr \dimexpr1in\relax/\sphinxpxdimen\relax
\fi
%% let collapsible pdf bookmarks panel have high depth per default
\PassOptionsToPackage{bookmarksdepth=5}{hyperref}

\PassOptionsToPackage{booktabs}{sphinx}
\PassOptionsToPackage{colorrows}{sphinx}

\PassOptionsToPackage{warn}{textcomp}
\usepackage[utf8]{inputenc}
\ifdefined\DeclareUnicodeCharacter
% support both utf8 and utf8x syntaxes
  \ifdefined\DeclareUnicodeCharacterAsOptional
    \def\sphinxDUC#1{\DeclareUnicodeCharacter{"#1}}
  \else
    \let\sphinxDUC\DeclareUnicodeCharacter
  \fi
  \sphinxDUC{00A0}{\nobreakspace}
  \sphinxDUC{2500}{\sphinxunichar{2500}}
  \sphinxDUC{2502}{\sphinxunichar{2502}}
  \sphinxDUC{2514}{\sphinxunichar{2514}}
  \sphinxDUC{251C}{\sphinxunichar{251C}}
  \sphinxDUC{2572}{\textbackslash}
\fi
\usepackage{cmap}
\usepackage[T1]{fontenc}
\usepackage{amsmath,amssymb,amstext}
\usepackage{babel}



\usepackage{tgtermes}
\usepackage{tgheros}
\renewcommand{\ttdefault}{txtt}



\usepackage[Bjarne]{fncychap}
\usepackage{sphinx}

\fvset{fontsize=auto}
\usepackage{geometry}


% Include hyperref last.
\usepackage{hyperref}
% Fix anchor placement for figures with captions.
\usepackage{hypcap}% it must be loaded after hyperref.
% Set up styles of URL: it should be placed after hyperref.
\urlstyle{same}


\usepackage{sphinxmessages}




\title{nucleardatapy}
\date{Nov 07, 2024}
\release{0.1}
\author{Jérôme Margueron, IRL NPA, USA}
\newcommand{\sphinxlogo}{\vbox{}}
\renewcommand{\releasename}{Release}
\makeindex
\begin{document}

\ifdefined\shorthandoff
  \ifnum\catcode`\=\string=\active\shorthandoff{=}\fi
  \ifnum\catcode`\"=\active\shorthandoff{"}\fi
\fi

\pagestyle{empty}
\sphinxmaketitle
\pagestyle{plain}
\sphinxtableofcontents
\pagestyle{normal}
\phantomsection\label{\detokenize{index::doc}}


\sphinxAtStartPar
\sphinxstylestrong{nucleardatapy} (/in short nuda/) is a Python library for nuclear physicists facilitating the access to theoretical or experimental nuclear data. It is specificaly designed for equation of state practitionners interested in the modeling of neutron stars, and it offers \sphinxstyleemphasis{simple} and \sphinxstyleemphasis{intuitive} APIs.

\sphinxAtStartPar
All data are provided with their reference, so when using these data in a scientific paper, reference to data should be provided explicitely. The reference to this toolkit could be given, but it should not mask the reference to data.

\sphinxAtStartPar
This python toolkit is designed to provide:
1) microscopic calculations in nuclear matter,
2) phenomenological predictions in nuclear matter,
3) experimental data for finite nuclei.

\sphinxAtStartPar
Check out the {\hyperref[\detokenize{source/usage::doc}]{\sphinxcrossref{\DUrole{doc}{Usage}}}} section for further information, including how to
{\hyperref[\detokenize{source/usage:installation}]{\sphinxcrossref{\DUrole{std}{\DUrole{std-ref}{install}}}}} the project.

\begin{sphinxadmonition}{note}{Note:}
\sphinxAtStartPar
This project is under active development.
\end{sphinxadmonition}


\chapter{Contents}
\label{\detokenize{index:contents}}
\sphinxstepscope


\section{Usage}
\label{\detokenize{source/usage:usage}}\label{\detokenize{source/usage::doc}}

\subsection{Installation}
\label{\detokenize{source/usage:installation}}\label{\detokenize{source/usage:id1}}
\sphinxAtStartPar
To use nucleardatapy, first download the .zip file from the git repository, or clone it in your local computer:

\begin{sphinxVerbatim}[commandchars=\\\{\}]
\PYG{g+gp}{\PYGZdl{} }git\PYG{+w}{ }clone\PYG{+w}{ }https://github.com/jeromemargueron/nucleardatapy
\end{sphinxVerbatim}

\sphinxAtStartPar
If you have downloaded the .zip file, you can unzip it anywhere in your local computer:

\begin{sphinxVerbatim}[commandchars=\\\{\}]
\PYG{g+gp}{\PYGZdl{} }unzip\PYG{+w}{ }nucleardatapy.zip
\end{sphinxVerbatim}

\sphinxAtStartPar
Then, in all cases, you shall enter into the new folder \sphinxtitleref{/nucleardatapy}:

\begin{sphinxVerbatim}[commandchars=\\\{\}]
\PYG{g+gp}{\PYGZdl{} }\PYG{n+nb}{cd}\PYG{+w}{ }nucleardatapy
\end{sphinxVerbatim}

\sphinxAtStartPar
and launch the install script:

\begin{sphinxVerbatim}[commandchars=\\\{\}]
\PYG{g+gp}{\PYGZdl{} }bash\PYG{+w}{ }install.sh
\end{sphinxVerbatim}

\sphinxAtStartPar
This will copy the Python toolkit into \$HOME/mylib/ as well as a few samples. It will also give you the content of the global variable NUCLEARDATAPY\_TK. If you edit install.sh, you can change the version (by default it is set to the latest one) as well as the destination folder (by default it is \$HOME/mylib).

\sphinxAtStartPar
Finally, you will have to create the global variable NUCLEARDATAPY\_TK with its right content. If you do not want to create it each time you open a new terminal, then you can define it in your .profile or .zprofil or .bash file as:

\begin{sphinxVerbatim}[commandchars=\\\{\}]
\PYG{g+go}{export NUCLEARDATAPY\PYGZus{}TK=\PYGZdl{}HOME/mylib/nucleardatapy}
\end{sphinxVerbatim}

\begin{sphinxadmonition}{note}{Note:}
\sphinxAtStartPar
The exact path to write above is given at the end of the installation.
\end{sphinxadmonition}


\subsection{Use nucleardatapy}
\label{\detokenize{source/usage:use-nucleardatapy}}\label{\detokenize{source/usage:use}}
\sphinxAtStartPar
Go to the folder \sphinxtitleref{mylib/nucleardatapy/samples/nucleardatapy\_samples/} and try that:

\begin{sphinxVerbatim}[commandchars=\\\{\}]
\PYG{g+gp}{\PYGZdl{} }python3\PYG{+w}{ }sample\PYGZus{}SetupMicroMatter.py
\end{sphinxVerbatim}


\subsection{Test nucleardatapy}
\label{\detokenize{source/usage:test-nucleardatapy}}\label{\detokenize{source/usage:test}}
\sphinxAtStartPar
A set of tests can be easily performed. They are stored in tests/ folder.

\begin{sphinxVerbatim}[commandchars=\\\{\}]
\PYG{g+gp}{\PYGZdl{} }bash\PYG{+w}{ }run\PYGZus{}tests.sh
\end{sphinxVerbatim}


\subsection{Get started}
\label{\detokenize{source/usage:get-started}}\label{\detokenize{source/usage:id2}}
\sphinxAtStartPar
How to obtain microscopic results for APR equation of state:

\begin{sphinxVerbatim}[commandchars=\\\{\}]
\PYG{k+kn}{import} \PYG{n+nn}{os}
\PYG{n}{nucleardatapy\PYGZus{}tk} \PYG{o}{=} \PYG{n}{os}\PYG{o}{.}\PYG{n}{getenv}\PYG{p}{(}\PYG{l+s+s1}{\PYGZsq{}}\PYG{l+s+s1}{NUCLEARDATAPY\PYGZus{}TK}\PYG{l+s+s1}{\PYGZsq{}}\PYG{p}{)}
\PYG{n}{sys}\PYG{o}{.}\PYG{n}{path}\PYG{o}{.}\PYG{n}{insert}\PYG{p}{(}\PYG{l+m+mi}{0}\PYG{p}{,} \PYG{n}{nucleardatapy\PYGZus{}tk}\PYG{p}{)}

\PYG{k+kn}{import} \PYG{n+nn}{nucleardatapy} \PYG{k}{as} \PYG{n+nn}{nuda}

\PYG{n}{mic} \PYG{o}{=} \PYG{n}{nuda}\PYG{o}{.}\PYG{n}{SetupMicroMatter}\PYG{p}{(} \PYG{n}{model} \PYG{o}{=} \PYG{l+s+s1}{\PYGZsq{}}\PYG{l+s+s1}{1998\PYGZhy{}VAR\PYGZhy{}AM\PYGZhy{}APR}\PYG{l+s+s1}{\PYGZsq{}} \PYG{p}{)}

\PYG{n}{mic}\PYG{o}{.}\PYG{n}{print\PYGZus{}outputs}\PYG{p}{(} \PYG{p}{)}
\end{sphinxVerbatim}

\sphinxstepscope


\section{API}
\label{\detokenize{source/api:api}}\label{\detokenize{source/api::doc}}

\begin{savenotes}\sphinxattablestart
\sphinxthistablewithglobalstyle
\sphinxthistablewithnovlinesstyle
\centering
\begin{tabulary}{\linewidth}[t]{\X{1}{2}\X{1}{2}}
\sphinxtoprule
\sphinxtableatstartofbodyhook
\sphinxAtStartPar
{\hyperref[\detokenize{source/generated/nucleardatapy:module-nucleardatapy}]{\sphinxcrossref{\sphinxcode{\sphinxupquote{nucleardatapy}}}}}
&
\sphinxAtStartPar
This module provides microscopic, phenomenological and experimental data constraints.
\\
\sphinxbottomrule
\end{tabulary}
\sphinxtableafterendhook\par
\sphinxattableend\end{savenotes}

\sphinxstepscope


\subsection{nucleardatapy}
\label{\detokenize{source/generated/nucleardatapy:module-nucleardatapy}}\label{\detokenize{source/generated/nucleardatapy:nucleardatapy}}\label{\detokenize{source/generated/nucleardatapy::doc}}\index{module@\spxentry{module}!nucleardatapy@\spxentry{nucleardatapy}}\index{nucleardatapy@\spxentry{nucleardatapy}!module@\spxentry{module}}
\sphinxAtStartPar
This module provides microscopic, phenomenological and experimental data constraints.

\sphinxstepscope


\section{Miscelaneous}
\label{\detokenize{source/miscelaneous:miscelaneous}}\label{\detokenize{source/miscelaneous::doc}}

\subsection{Contributing}
\label{\detokenize{source/miscelaneous:contributing}}\label{\detokenize{source/miscelaneous:id1}}
\sphinxAtStartPar
For the moment, contributions are based on co\sphinxhyphen{}optation among the team.

\sphinxAtStartPar
To make contribution easy, we all work in the \sphinxtitleref{main} branch and we shall therefore remember to pull before working and pulling after, with a running version. For long developments, you can work in a local folder (in \sphinxtitleref{mylib} for instance) and copy your contribution to the GitHub folder once you are sure it is functionning. So the final step should last less than 5 minutes, and can be safely done between a pull and before a push. Since we are not numerous, we hope that no one will work in the same part of the code at the same time (i.e. between a pull and a push). It is probably the simpler way to proceed.

\sphinxAtStartPar
Once the toolkit is released, the rules to contribute will be changing. A team of developpers should be defined and a generic email to contact them should be created. Here is a suggestion to contribute after the release.

\sphinxAtStartPar
This file describes how new contributors to the project can start contributing.

\sphinxAtStartPar
\sphinxstylestrong{Two ways:}

\sphinxAtStartPar
You can provide your data and interacting with one of our developer.

\sphinxAtStartPar
You can also join the developing team and extend the functionality of this toolkit.

\sphinxAtStartPar
\sphinxstylestrong{Provide your data:}

\sphinxAtStartPar
Please contact the developer team directly by shooting an email to TBC.

\sphinxAtStartPar
Then you can interact directly with one of our developer and provide your data. You will not be able to push your data to the repository, but an updated version of the toolkit will contain your new data.

\sphinxAtStartPar
\sphinxstylestrong{Join the team:}

\sphinxAtStartPar
Please contact the developer team directly by shooting an email to TBC. Explain the reason why you wish to join the team and if you have ideas about extending the functionality of the toolkit.

\sphinxAtStartPar
Once in the team, a branch will be dedicated to your contribution. You could show it during our virtual meetings, and your contribution will be merged to the new version of the toolkit.


\subsection{License}
\label{\detokenize{source/miscelaneous:license}}\label{\detokenize{source/miscelaneous:id2}}
\sphinxAtStartPar
TBC.


\subsection{Report issues}
\label{\detokenize{source/miscelaneous:report-issues}}\label{\detokenize{source/miscelaneous:id3}}
\sphinxAtStartPar
For the current version, we report issues chatting among us.
Once this toolkit is released, we should setup a way that users could contact us and report issues or difficulties in installing or using the toolkit.


\subsection{Thanks}
\label{\detokenize{source/miscelaneous:thanks}}\label{\detokenize{source/miscelaneous:id4}}
\sphinxAtStartPar
A special thanks to all contributors who accepted to share their results in this toolkit.


\chapter{Complement}
\label{\detokenize{index:complement}}
\sphinxstepscope


\section{SetupMicroMatter}
\label{\detokenize{source/api/setup_micro_matter:setupmicromatter}}\label{\detokenize{source/api/setup_micro_matter::doc}}\index{module@\spxentry{module}!nucleardatapy.setup\_micro\_matter@\spxentry{nucleardatapy.setup\_micro\_matter}}\index{nucleardatapy.setup\_micro\_matter@\spxentry{nucleardatapy.setup\_micro\_matter}!module@\spxentry{module}}\index{SetupMicroMatter (class in nucleardatapy.setup\_micro\_matter)@\spxentry{SetupMicroMatter}\spxextra{class in nucleardatapy.setup\_micro\_matter}}\phantomsection\label{\detokenize{source/api/setup_micro_matter:module-nucleardatapy.setup_micro_matter}}

\begin{fulllineitems}
\phantomsection\label{\detokenize{source/api/setup_micro_matter:nucleardatapy.setup_micro_matter.SetupMicroMatter}}
\pysigstartsignatures
\pysiglinewithargsret
{\sphinxbfcode{\sphinxupquote{class\DUrole{w}{ }}}\sphinxcode{\sphinxupquote{nucleardatapy.setup\_micro\_matter.}}\sphinxbfcode{\sphinxupquote{SetupMicroMatter}}}
{\sphinxparam{\DUrole{n}{model}\DUrole{o}{=}\DUrole{default_value}{\textquotesingle{}1998\sphinxhyphen{}VAR\sphinxhyphen{}AM\sphinxhyphen{}APR\textquotesingle{}}}}
{}
\pysigstopsignatures
\sphinxAtStartPar
Instantiate the object with microscopic results choosen     by the toolkit practitioner.

\sphinxAtStartPar
This choice is defined in \sphinxtitleref{model}, which can chosen among     the following choices:     ‘1981\sphinxhyphen{}VAR\sphinxhyphen{}AM\sphinxhyphen{}FP’, ‘1998\sphinxhyphen{}VAR\sphinxhyphen{}AM\sphinxhyphen{}APR’, ‘2006\sphinxhyphen{}BHF\sphinxhyphen{}AM*’, ‘2008\sphinxhyphen{}BCS\sphinxhyphen{}NM’, ‘2008\sphinxhyphen{}AFDMC\sphinxhyphen{}NM’,     ‘2008\sphinxhyphen{}QMC\sphinxhyphen{}NM\sphinxhyphen{}swave’, ‘2010\sphinxhyphen{}QMC\sphinxhyphen{}NM\sphinxhyphen{}AV4’, ‘2009\sphinxhyphen{}DLQMC\sphinxhyphen{}NM’, ‘2010\sphinxhyphen{}MBPT\sphinxhyphen{}NM’,     ‘2012\sphinxhyphen{}AFDMC\sphinxhyphen{}NM\sphinxhyphen{}1’, ‘2012\sphinxhyphen{}AFDMC\sphinxhyphen{}NM\sphinxhyphen{}2’, ‘2012\sphinxhyphen{}AFDMC\sphinxhyphen{}NM\sphinxhyphen{}3’, ‘2012\sphinxhyphen{}AFDMC\sphinxhyphen{}NM\sphinxhyphen{}4’,     ‘2012\sphinxhyphen{}AFDMC\sphinxhyphen{}NM\sphinxhyphen{}5’, ‘2012\sphinxhyphen{}AFDMC\sphinxhyphen{}NM\sphinxhyphen{}6’, ‘2012\sphinxhyphen{}AFDMC\sphinxhyphen{}NM\sphinxhyphen{}7’,     ‘2013\sphinxhyphen{}QMC\sphinxhyphen{}NM’, ‘2014\sphinxhyphen{}AFQMC\sphinxhyphen{}NM’, ‘2016\sphinxhyphen{}QMC\sphinxhyphen{}NM’, ‘2016\sphinxhyphen{}MBPT\sphinxhyphen{}AM’,     ‘2018\sphinxhyphen{}QMC\sphinxhyphen{}NM’, ‘2019\sphinxhyphen{}MBPT\sphinxhyphen{}AM\sphinxhyphen{}L59’, ‘2019\sphinxhyphen{}MBPT\sphinxhyphen{}AM\sphinxhyphen{}L69’,     ‘2020\sphinxhyphen{}MBPT\sphinxhyphen{}AM’, ‘2022\sphinxhyphen{}AFDMC\sphinxhyphen{}NM’, ‘2024\sphinxhyphen{}NLEFT\sphinxhyphen{}AM’,     ‘2024\sphinxhyphen{}BHF\sphinxhyphen{}AM\sphinxhyphen{}2BF\sphinxhyphen{}Av8p’, ‘2024\sphinxhyphen{}BHF\sphinxhyphen{}AM\sphinxhyphen{}2BF\sphinxhyphen{}Av18’, ‘2024\sphinxhyphen{}BHF\sphinxhyphen{}AM\sphinxhyphen{}2BF\sphinxhyphen{}BONN’, ‘2024\sphinxhyphen{}BHF\sphinxhyphen{}AM\sphinxhyphen{}2BF\sphinxhyphen{}CDBONN’,     ‘2024\sphinxhyphen{}BHF\sphinxhyphen{}AM\sphinxhyphen{}2BF\sphinxhyphen{}NSC97a’, ‘2024\sphinxhyphen{}BHF\sphinxhyphen{}AM\sphinxhyphen{}2BF\sphinxhyphen{}NSC97b’, ‘2024\sphinxhyphen{}BHF\sphinxhyphen{}AM\sphinxhyphen{}2BF\sphinxhyphen{}NSC97c’, ‘2024\sphinxhyphen{}BHF\sphinxhyphen{}AM\sphinxhyphen{}2BF\sphinxhyphen{}NSC97d’,     ‘2024\sphinxhyphen{}BHF\sphinxhyphen{}AM\sphinxhyphen{}2BF\sphinxhyphen{}NSC97e’, ‘2024\sphinxhyphen{}BHF\sphinxhyphen{}AM\sphinxhyphen{}2BF\sphinxhyphen{}NSC97f’, ‘2024\sphinxhyphen{}BHF\sphinxhyphen{}AM\sphinxhyphen{}2BF\sphinxhyphen{}SSCV14’,     ‘2024\sphinxhyphen{}BHF\sphinxhyphen{}AM\sphinxhyphen{}23BF\sphinxhyphen{}Av8p’, ‘2024\sphinxhyphen{}BHF\sphinxhyphen{}AM\sphinxhyphen{}23BF\sphinxhyphen{}Av18’, ‘2024\sphinxhyphen{}BHF\sphinxhyphen{}AM\sphinxhyphen{}23BF\sphinxhyphen{}BONN’, ‘2024\sphinxhyphen{}BHF\sphinxhyphen{}AM\sphinxhyphen{}23BF\sphinxhyphen{}CDBONN’,     ‘2024\sphinxhyphen{}BHF\sphinxhyphen{}AM\sphinxhyphen{}23BF\sphinxhyphen{}NSC97a’, ‘2024\sphinxhyphen{}BHF\sphinxhyphen{}AM\sphinxhyphen{}23BF\sphinxhyphen{}NSC97b’, ‘2024\sphinxhyphen{}BHF\sphinxhyphen{}AM\sphinxhyphen{}23BF\sphinxhyphen{}NSC97c’, ‘2024\sphinxhyphen{}BHF\sphinxhyphen{}AM\sphinxhyphen{}23BF\sphinxhyphen{}NSC97d’,     ‘2024\sphinxhyphen{}BHF\sphinxhyphen{}AM\sphinxhyphen{}23BF\sphinxhyphen{}NSC97e’, ‘2024\sphinxhyphen{}BHF\sphinxhyphen{}AM\sphinxhyphen{}23BF\sphinxhyphen{}NSC97f’, ‘2024\sphinxhyphen{}BHF\sphinxhyphen{}AM\sphinxhyphen{}23BF\sphinxhyphen{}SSCV14’
\begin{quote}\begin{description}
\sphinxlineitem{Parameters}
\sphinxAtStartPar
\sphinxstyleliteralstrong{\sphinxupquote{model}} (\sphinxstyleliteralemphasis{\sphinxupquote{str}}\sphinxstyleliteralemphasis{\sphinxupquote{, }}\sphinxstyleliteralemphasis{\sphinxupquote{optional.}}) \textendash{} Fix the name of model. Default value: ‘1998\sphinxhyphen{}VAR\sphinxhyphen{}AM\sphinxhyphen{}APR’.

\end{description}\end{quote}

\sphinxAtStartPar
\sphinxstylestrong{Attributes:}
\index{init\_self() (nucleardatapy.setup\_micro\_matter.SetupMicroMatter method)@\spxentry{init\_self()}\spxextra{nucleardatapy.setup\_micro\_matter.SetupMicroMatter method}}

\begin{fulllineitems}
\phantomsection\label{\detokenize{source/api/setup_micro_matter:nucleardatapy.setup_micro_matter.SetupMicroMatter.init_self}}
\pysigstartsignatures
\pysiglinewithargsret
{\sphinxbfcode{\sphinxupquote{init\_self}}}
{}
{}
\pysigstopsignatures
\sphinxAtStartPar
Initialize variables in self.

\end{fulllineitems}

\index{model (nucleardatapy.setup\_micro\_matter.SetupMicroMatter attribute)@\spxentry{model}\spxextra{nucleardatapy.setup\_micro\_matter.SetupMicroMatter attribute}}

\begin{fulllineitems}
\phantomsection\label{\detokenize{source/api/setup_micro_matter:nucleardatapy.setup_micro_matter.SetupMicroMatter.model}}
\pysigstartsignatures
\pysigline
{\sphinxbfcode{\sphinxupquote{model}}}
\pysigstopsignatures
\sphinxAtStartPar
Attribute model.

\end{fulllineitems}

\index{print\_outputs() (nucleardatapy.setup\_micro\_matter.SetupMicroMatter method)@\spxentry{print\_outputs()}\spxextra{nucleardatapy.setup\_micro\_matter.SetupMicroMatter method}}

\begin{fulllineitems}
\phantomsection\label{\detokenize{source/api/setup_micro_matter:nucleardatapy.setup_micro_matter.SetupMicroMatter.print_outputs}}
\pysigstartsignatures
\pysiglinewithargsret
{\sphinxbfcode{\sphinxupquote{print\_outputs}}}
{}
{}
\pysigstopsignatures
\sphinxAtStartPar
Method which print outputs on terminal’s screen.

\end{fulllineitems}


\end{fulllineitems}

\index{models\_micro\_matter() (in module nucleardatapy.setup\_micro\_matter)@\spxentry{models\_micro\_matter()}\spxextra{in module nucleardatapy.setup\_micro\_matter}}

\begin{fulllineitems}
\phantomsection\label{\detokenize{source/api/setup_micro_matter:nucleardatapy.setup_micro_matter.models_micro_matter}}
\pysigstartsignatures
\pysiglinewithargsret
{\sphinxcode{\sphinxupquote{nucleardatapy.setup\_micro\_matter.}}\sphinxbfcode{\sphinxupquote{models\_micro\_matter}}}
{}
{}
\pysigstopsignatures
\sphinxAtStartPar
Return a list with the name of the models available in this toolkit and     print them all on the prompt. These models are the following ones:     ‘1981\sphinxhyphen{}VAR\sphinxhyphen{}AM\sphinxhyphen{}FP’, ‘1998\sphinxhyphen{}VAR\sphinxhyphen{}AM\sphinxhyphen{}APR’, ‘2006\sphinxhyphen{}BHF\sphinxhyphen{}AM*’, ‘2008\sphinxhyphen{}BCS\sphinxhyphen{}NM’, ‘2008\sphinxhyphen{}AFDMC\sphinxhyphen{}NM’,     ‘2012\sphinxhyphen{}AFDMC\sphinxhyphen{}NM\sphinxhyphen{}1’, ‘2012\sphinxhyphen{}AFDMC\sphinxhyphen{}NM\sphinxhyphen{}2’, ‘2012\sphinxhyphen{}AFDMC\sphinxhyphen{}NM\sphinxhyphen{}3’, ‘2012\sphinxhyphen{}AFDMC\sphinxhyphen{}NM\sphinxhyphen{}4’,     ‘2012\sphinxhyphen{}AFDMC\sphinxhyphen{}NM\sphinxhyphen{}5’, ‘2012\sphinxhyphen{}AFDMC\sphinxhyphen{}NM\sphinxhyphen{}6’, ‘2012\sphinxhyphen{}AFDMC\sphinxhyphen{}NM\sphinxhyphen{}7’,     ‘2008\sphinxhyphen{}QMC\sphinxhyphen{}NM\sphinxhyphen{}swave’, ‘2010\sphinxhyphen{}QMC\sphinxhyphen{}NM\sphinxhyphen{}AV4’, ‘2009\sphinxhyphen{}DLQMC\sphinxhyphen{}NM’, ‘2010\sphinxhyphen{}MBPT\sphinxhyphen{}NM’,     ‘2013\sphinxhyphen{}QMC\sphinxhyphen{}NM’, ‘2014\sphinxhyphen{}AFQMC\sphinxhyphen{}NM’, ‘2016\sphinxhyphen{}QMC\sphinxhyphen{}NM’, ‘2016\sphinxhyphen{}MBPT\sphinxhyphen{}AM’,     ‘2018\sphinxhyphen{}QMC\sphinxhyphen{}NM’, ‘2019\sphinxhyphen{}MBPT\sphinxhyphen{}AM\sphinxhyphen{}L59’, ‘2019\sphinxhyphen{}MBPT\sphinxhyphen{}AM\sphinxhyphen{}L69’,     ‘2020\sphinxhyphen{}MBPT\sphinxhyphen{}AM’, ‘2022\sphinxhyphen{}AFDMC\sphinxhyphen{}NM’, ‘2024\sphinxhyphen{}NLEFT\sphinxhyphen{}AM’,     ‘2024\sphinxhyphen{}BHF\sphinxhyphen{}AM\sphinxhyphen{}2BF\sphinxhyphen{}Av8p’, ‘2024\sphinxhyphen{}BHF\sphinxhyphen{}AM\sphinxhyphen{}2BF\sphinxhyphen{}Av18’, ‘2024\sphinxhyphen{}BHF\sphinxhyphen{}AM\sphinxhyphen{}2BF\sphinxhyphen{}BONN’, ‘2024\sphinxhyphen{}BHF\sphinxhyphen{}AM\sphinxhyphen{}2BF\sphinxhyphen{}CDBONN’,     ‘2024\sphinxhyphen{}BHF\sphinxhyphen{}AM\sphinxhyphen{}2BF\sphinxhyphen{}NSC97a’, ‘2024\sphinxhyphen{}BHF\sphinxhyphen{}AM\sphinxhyphen{}2BF\sphinxhyphen{}NSC97b’, ‘2024\sphinxhyphen{}BHF\sphinxhyphen{}AM\sphinxhyphen{}2BF\sphinxhyphen{}NSC97c’, ‘2024\sphinxhyphen{}BHF\sphinxhyphen{}AM\sphinxhyphen{}2BF\sphinxhyphen{}NSC97d’,     ‘2024\sphinxhyphen{}BHF\sphinxhyphen{}AM\sphinxhyphen{}2BF\sphinxhyphen{}NSC97e’, ‘2024\sphinxhyphen{}BHF\sphinxhyphen{}AM\sphinxhyphen{}2BF\sphinxhyphen{}NSC97f’, ‘2024\sphinxhyphen{}BHF\sphinxhyphen{}AM\sphinxhyphen{}2BF\sphinxhyphen{}SSCV14’,    ‘2024\sphinxhyphen{}BHF\sphinxhyphen{}AM\sphinxhyphen{}23BF\sphinxhyphen{}Av8p’, ‘2024\sphinxhyphen{}BHF\sphinxhyphen{}AM\sphinxhyphen{}23BF\sphinxhyphen{}Av18’, ‘2024\sphinxhyphen{}BHF\sphinxhyphen{}AM\sphinxhyphen{}23BF\sphinxhyphen{}BONN’, ‘2024\sphinxhyphen{}BHF\sphinxhyphen{}AM\sphinxhyphen{}23BF\sphinxhyphen{}CDBONN’,     ‘2024\sphinxhyphen{}BHF\sphinxhyphen{}AM\sphinxhyphen{}23BF\sphinxhyphen{}NSC97a’, ‘2024\sphinxhyphen{}BHF\sphinxhyphen{}AM\sphinxhyphen{}23BF\sphinxhyphen{}NSC97b’, ‘2024\sphinxhyphen{}BHF\sphinxhyphen{}AM\sphinxhyphen{}23BF\sphinxhyphen{}NSC97c’, ‘2024\sphinxhyphen{}BHF\sphinxhyphen{}AM\sphinxhyphen{}23BF\sphinxhyphen{}NSC97d’,     ‘2024\sphinxhyphen{}BHF\sphinxhyphen{}AM\sphinxhyphen{}23BF\sphinxhyphen{}NSC97e’, ‘2024\sphinxhyphen{}BHF\sphinxhyphen{}AM\sphinxhyphen{}23BF\sphinxhyphen{}NSC97f’, ‘2024\sphinxhyphen{}BHF\sphinxhyphen{}AM\sphinxhyphen{}23BF\sphinxhyphen{}SSCV14’,    ‘2024\sphinxhyphen{}BHF\sphinxhyphen{}AM\sphinxhyphen{}23BFmicro\sphinxhyphen{}Av18’, ‘2024\sphinxhyphen{}BHF\sphinxhyphen{}AM\sphinxhyphen{}23BFmicro\sphinxhyphen{}BONNB’, ‘2024\sphinxhyphen{}BHF\sphinxhyphen{}AM\sphinxhyphen{}23BFmicro\sphinxhyphen{}NSC93’
:return: The list of models.
:rtype: list{[}str{]}.

\end{fulllineitems}


\sphinxAtStartPar
Here are a set of figures which are produced with the Python sample: /sample/nucleardatapy\_plots/plot\_setupMicro.py

\begin{figure}[htbp]
\centering
\capstart

\noindent\sphinxincludegraphics[scale=0.7]{{plot_SetupMicroMatter_e2a_NM_VAR}.png}
\caption{This figure shows the energy in neutron matter (NM) over the free Fermi gas energy (top) and the energy per particle (bottom) as function of the density (left) and the neutron Fermi momentum (right) for the variational models available in the nucleardatapy toolkit.}\label{\detokenize{source/api/setup_micro_matter:id1}}\end{figure}

\begin{figure}[htbp]
\centering
\capstart

\noindent\sphinxincludegraphics[scale=0.7]{{plot_SetupMicroMatter_e2a_NM_AFDMC}.png}
\caption{This figure shows the energy in neutron matter (NM) over the free Fermi gas energy (top) and the energy per particle (bottom) as function of the density (left) and the neutron Fermi momentum (right) for the AFDMC models available in the nucleardatapy toolkit.}\label{\detokenize{source/api/setup_micro_matter:id2}}\end{figure}

\begin{figure}[htbp]
\centering
\capstart

\noindent\sphinxincludegraphics[scale=0.7]{{plot_SetupMicroMatter_e2a_NM_BHF}.png}
\caption{This figure shows the energy in neutron matter (NM) over the free Fermi gas energy (top) and the energy per particle (bottom) as function of the density (left) and the neutron Fermi momentum (right) for the BHF models available in the nucleardatapy toolkit.}\label{\detokenize{source/api/setup_micro_matter:id3}}\end{figure}

\begin{figure}[htbp]
\centering
\capstart

\noindent\sphinxincludegraphics[scale=0.7]{{plot_SetupMicroMatter_e2a_NM_QMC}.png}
\caption{This figure shows the energy in neutron matter (NM) over the free Fermi gas energy (top) and the energy per particle (bottom) as function of the density (left) and the neutron Fermi momentum (right) for the QMC models available in the nucleardatapy toolkit.}\label{\detokenize{source/api/setup_micro_matter:id4}}\end{figure}

\begin{figure}[htbp]
\centering
\capstart

\noindent\sphinxincludegraphics[scale=0.7]{{plot_SetupMicroMatter_e2a_NM_MBPT}.png}
\caption{This figure shows the energy in neutron matter (NM) over the free Fermi gas energy (top) and the energy per particle (bottom) as function of the density (left) and the neutron Fermi momentum (right) for the MBPT models available in the nucleardatapy toolkit.}\label{\detokenize{source/api/setup_micro_matter:id5}}\end{figure}

\begin{figure}[htbp]
\centering
\capstart

\noindent\sphinxincludegraphics[scale=0.7]{{plot_SetupMicroMatter_gap_NM}.png}
\caption{This figure shows the pairing gap in neutron matter (NM) over the Fermi energy (top) and the pairing gap (bottom) as function of the density (left) and the neutron Fermi momentum (right) for the models available in the nucleardatapy toolkit.}\label{\detokenize{source/api/setup_micro_matter:id6}}\end{figure}

\sphinxstepscope


\section{SetupMicroMatterBand}
\label{\detokenize{source/api/setup_micro_matter_band:setupmicromatterband}}\label{\detokenize{source/api/setup_micro_matter_band::doc}}\index{module@\spxentry{module}!nucleardatapy.setup\_micro\_matter\_band@\spxentry{nucleardatapy.setup\_micro\_matter\_band}}\index{nucleardatapy.setup\_micro\_matter\_band@\spxentry{nucleardatapy.setup\_micro\_matter\_band}!module@\spxentry{module}}\index{SetupMicroMatterBand (class in nucleardatapy.setup\_micro\_matter\_band)@\spxentry{SetupMicroMatterBand}\spxextra{class in nucleardatapy.setup\_micro\_matter\_band}}\phantomsection\label{\detokenize{source/api/setup_micro_matter_band:module-nucleardatapy.setup_micro_matter_band}}

\begin{fulllineitems}
\phantomsection\label{\detokenize{source/api/setup_micro_matter_band:nucleardatapy.setup_micro_matter_band.SetupMicroMatterBand}}
\pysigstartsignatures
\pysiglinewithargsret
{\sphinxbfcode{\sphinxupquote{class\DUrole{w}{ }}}\sphinxcode{\sphinxupquote{nucleardatapy.setup\_micro\_matter\_band.}}\sphinxbfcode{\sphinxupquote{SetupMicroMatterBand}}}
{\sphinxparam{\DUrole{n}{models}\DUrole{o}{=}\DUrole{default_value}{{[}\textquotesingle{}2016\sphinxhyphen{}MBPT\sphinxhyphen{}AM\textquotesingle{}{]}}}\sphinxparamcomma \sphinxparam{\DUrole{n}{nden}\DUrole{o}{=}\DUrole{default_value}{10}}\sphinxparamcomma \sphinxparam{\DUrole{n}{ne}\DUrole{o}{=}\DUrole{default_value}{200}}\sphinxparamcomma \sphinxparam{\DUrole{n}{den}\DUrole{o}{=}\DUrole{default_value}{None}}\sphinxparamcomma \sphinxparam{\DUrole{n}{matter}\DUrole{o}{=}\DUrole{default_value}{\textquotesingle{}NM\textquotesingle{}}}\sphinxparamcomma \sphinxparam{\DUrole{n}{e2a\_min}\DUrole{o}{=}\DUrole{default_value}{\sphinxhyphen{}20.0}}\sphinxparamcomma \sphinxparam{\DUrole{n}{e2a\_max}\DUrole{o}{=}\DUrole{default_value}{50.0}}}
{}
\pysigstopsignatures
\sphinxAtStartPar
Instantiate the object with statistical distributions averaging over
the models given as inputs and in NM.
\begin{quote}\begin{description}
\sphinxlineitem{Parameters}\begin{itemize}
\item {} 
\sphinxAtStartPar
\sphinxstyleliteralstrong{\sphinxupquote{models}} (\sphinxstyleliteralemphasis{\sphinxupquote{list.}}) \textendash{} The models given as inputs.

\item {} 
\sphinxAtStartPar
\sphinxstyleliteralstrong{\sphinxupquote{nden}} (\sphinxstyleliteralemphasis{\sphinxupquote{int}}\sphinxstyleliteralemphasis{\sphinxupquote{, }}\sphinxstyleliteralemphasis{\sphinxupquote{optional.}}) \textendash{} number of density points.

\item {} 
\sphinxAtStartPar
\sphinxstyleliteralstrong{\sphinxupquote{ne}} (\sphinxstyleliteralemphasis{\sphinxupquote{int}}\sphinxstyleliteralemphasis{\sphinxupquote{, }}\sphinxstyleliteralemphasis{\sphinxupquote{optional.}}) \textendash{} number of points along the energy axis.

\item {} 
\sphinxAtStartPar
\sphinxstyleliteralstrong{\sphinxupquote{den}} (\sphinxstyleliteralemphasis{\sphinxupquote{None}}\sphinxstyleliteralemphasis{\sphinxupquote{ or }}\sphinxstyleliteralemphasis{\sphinxupquote{numpy array}}\sphinxstyleliteralemphasis{\sphinxupquote{, }}\sphinxstyleliteralemphasis{\sphinxupquote{optional.}}) \textendash{} if not None (default), impose the densities.

\item {} 
\sphinxAtStartPar
\sphinxstyleliteralstrong{\sphinxupquote{matter}} (\sphinxstyleliteralemphasis{\sphinxupquote{str}}\sphinxstyleliteralemphasis{\sphinxupquote{, }}\sphinxstyleliteralemphasis{\sphinxupquote{optional.}}) \textendash{} can be ‘NM’ (default), ‘SM’ or ‘ESYM’.

\end{itemize}

\end{description}\end{quote}

\sphinxAtStartPar
\sphinxstylestrong{Attributes:}
\index{den (nucleardatapy.setup\_micro\_matter\_band.SetupMicroMatterBand attribute)@\spxentry{den}\spxextra{nucleardatapy.setup\_micro\_matter\_band.SetupMicroMatterBand attribute}}

\begin{fulllineitems}
\phantomsection\label{\detokenize{source/api/setup_micro_matter_band:nucleardatapy.setup_micro_matter_band.SetupMicroMatterBand.den}}
\pysigstartsignatures
\pysigline
{\sphinxbfcode{\sphinxupquote{den}}}
\pysigstopsignatures
\sphinxAtStartPar
Attribute a set of density points.

\end{fulllineitems}

\index{init\_self() (nucleardatapy.setup\_micro\_matter\_band.SetupMicroMatterBand method)@\spxentry{init\_self()}\spxextra{nucleardatapy.setup\_micro\_matter\_band.SetupMicroMatterBand method}}

\begin{fulllineitems}
\phantomsection\label{\detokenize{source/api/setup_micro_matter_band:nucleardatapy.setup_micro_matter_band.SetupMicroMatterBand.init_self}}
\pysigstartsignatures
\pysiglinewithargsret
{\sphinxbfcode{\sphinxupquote{init\_self}}}
{}
{}
\pysigstopsignatures
\sphinxAtStartPar
Initialize variables in self.

\end{fulllineitems}

\index{matter (nucleardatapy.setup\_micro\_matter\_band.SetupMicroMatterBand attribute)@\spxentry{matter}\spxextra{nucleardatapy.setup\_micro\_matter\_band.SetupMicroMatterBand attribute}}

\begin{fulllineitems}
\phantomsection\label{\detokenize{source/api/setup_micro_matter_band:nucleardatapy.setup_micro_matter_band.SetupMicroMatterBand.matter}}
\pysigstartsignatures
\pysigline
{\sphinxbfcode{\sphinxupquote{matter}}}
\pysigstopsignatures
\sphinxAtStartPar
Attribute matter str.

\end{fulllineitems}

\index{models (nucleardatapy.setup\_micro\_matter\_band.SetupMicroMatterBand attribute)@\spxentry{models}\spxextra{nucleardatapy.setup\_micro\_matter\_band.SetupMicroMatterBand attribute}}

\begin{fulllineitems}
\phantomsection\label{\detokenize{source/api/setup_micro_matter_band:nucleardatapy.setup_micro_matter_band.SetupMicroMatterBand.models}}
\pysigstartsignatures
\pysigline
{\sphinxbfcode{\sphinxupquote{models}}}
\pysigstopsignatures
\sphinxAtStartPar
Attribute model.

\end{fulllineitems}

\index{nden (nucleardatapy.setup\_micro\_matter\_band.SetupMicroMatterBand attribute)@\spxentry{nden}\spxextra{nucleardatapy.setup\_micro\_matter\_band.SetupMicroMatterBand attribute}}

\begin{fulllineitems}
\phantomsection\label{\detokenize{source/api/setup_micro_matter_band:nucleardatapy.setup_micro_matter_band.SetupMicroMatterBand.nden}}
\pysigstartsignatures
\pysigline
{\sphinxbfcode{\sphinxupquote{nden}}}
\pysigstopsignatures
\sphinxAtStartPar
Attribute number of points in density.

\end{fulllineitems}

\index{print\_outputs() (nucleardatapy.setup\_micro\_matter\_band.SetupMicroMatterBand method)@\spxentry{print\_outputs()}\spxextra{nucleardatapy.setup\_micro\_matter\_band.SetupMicroMatterBand method}}

\begin{fulllineitems}
\phantomsection\label{\detokenize{source/api/setup_micro_matter_band:nucleardatapy.setup_micro_matter_band.SetupMicroMatterBand.print_outputs}}
\pysigstartsignatures
\pysiglinewithargsret
{\sphinxbfcode{\sphinxupquote{print\_outputs}}}
{}
{}
\pysigstopsignatures
\sphinxAtStartPar
Method which print outputs on terminal’s screen.

\end{fulllineitems}


\end{fulllineitems}


\begin{figure}[htbp]
\centering
\capstart

\noindent\sphinxincludegraphics[scale=0.7]{{plot_SetupMicroMatterBand_NM}.png}
\caption{Uncertainty band in NM obtained from the analysis of different predictions: MBPT\sphinxhyphen{}2016, QMC\sphinxhyphen{}2016 and MBPT\sphinxhyphen{}2020.}\label{\detokenize{source/api/setup_micro_matter_band:id1}}\end{figure}

\begin{figure}[htbp]
\centering
\capstart

\noindent\sphinxincludegraphics[scale=0.7]{{plot_SetupMicroMatterBand_SM}.png}
\caption{Uncertainty band in SM obtained from the analysis of different predictions: MBPT\sphinxhyphen{}2016 and MBPT\sphinxhyphen{}2020.}\label{\detokenize{source/api/setup_micro_matter_band:id2}}\end{figure}

\begin{figure}[htbp]
\centering
\capstart

\noindent\sphinxincludegraphics[scale=0.7]{{plot_SetupMicroMatterBand_Esym}.png}
\caption{Uncertainty band for the symmetry energy obtained from the analysis of different predictions: MBPT\sphinxhyphen{}2016 and MBPT\sphinxhyphen{}2020.}\label{\detokenize{source/api/setup_micro_matter_band:id3}}\end{figure}

\sphinxstepscope


\section{SetupMicroLP}
\label{\detokenize{source/api/setup_micro_LP:setupmicrolp}}\label{\detokenize{source/api/setup_micro_LP::doc}}\index{module@\spxentry{module}!nucleardatapy.setup\_micro\_LP@\spxentry{nucleardatapy.setup\_micro\_LP}}\index{nucleardatapy.setup\_micro\_LP@\spxentry{nucleardatapy.setup\_micro\_LP}!module@\spxentry{module}}\index{SetupMicroLP (class in nucleardatapy.setup\_micro\_LP)@\spxentry{SetupMicroLP}\spxextra{class in nucleardatapy.setup\_micro\_LP}}\phantomsection\label{\detokenize{source/api/setup_micro_LP:module-nucleardatapy.setup_micro_LP}}

\begin{fulllineitems}
\phantomsection\label{\detokenize{source/api/setup_micro_LP:nucleardatapy.setup_micro_LP.SetupMicroLP}}
\pysigstartsignatures
\pysiglinewithargsret
{\sphinxbfcode{\sphinxupquote{class\DUrole{w}{ }}}\sphinxcode{\sphinxupquote{nucleardatapy.setup\_micro\_LP.}}\sphinxbfcode{\sphinxupquote{SetupMicroLP}}}
{\sphinxparam{\DUrole{n}{model}\DUrole{o}{=}\DUrole{default_value}{\textquotesingle{}1994\sphinxhyphen{}BHF\sphinxhyphen{}SM\sphinxhyphen{}LP\sphinxhyphen{}AV14\sphinxhyphen{}GAP\textquotesingle{}}}}
{}
\pysigstopsignatures
\sphinxAtStartPar
Instantiate the object with Landau parameters from microscopic calculations choosen     by the toolkit practitioner.

\sphinxAtStartPar
This choice is defined in \sphinxtitleref{model}, which can chosen among     the following choices:     ‘1994\sphinxhyphen{}BHF\sphinxhyphen{}SM\sphinxhyphen{}LP\sphinxhyphen{}AV14\sphinxhyphen{}GAP’, ‘1994\sphinxhyphen{}BHF\sphinxhyphen{}SM\sphinxhyphen{}LP\sphinxhyphen{}AV14\sphinxhyphen{}CONT’,     ‘1994\sphinxhyphen{}BHF\sphinxhyphen{}SM\sphinxhyphen{}LP\sphinxhyphen{}REID\sphinxhyphen{}GAP’, ‘1994\sphinxhyphen{}BHF\sphinxhyphen{}SM\sphinxhyphen{}LP\sphinxhyphen{}REID\sphinxhyphen{}CONT’, ‘1994\sphinxhyphen{}BHF\sphinxhyphen{}SM\sphinxhyphen{}LP\sphinxhyphen{}AV14\sphinxhyphen{}CONT\sphinxhyphen{}0.7’.
\begin{quote}\begin{description}
\sphinxlineitem{Parameters}
\sphinxAtStartPar
\sphinxstyleliteralstrong{\sphinxupquote{model}} (\sphinxstyleliteralemphasis{\sphinxupquote{str}}\sphinxstyleliteralemphasis{\sphinxupquote{, }}\sphinxstyleliteralemphasis{\sphinxupquote{optional.}}) \textendash{} Fix the name of model. Default value: ‘1994\sphinxhyphen{}BHF\sphinxhyphen{}LP’.

\end{description}\end{quote}

\sphinxAtStartPar
\sphinxstylestrong{Attributes:}
\index{init\_self() (nucleardatapy.setup\_micro\_LP.SetupMicroLP method)@\spxentry{init\_self()}\spxextra{nucleardatapy.setup\_micro\_LP.SetupMicroLP method}}

\begin{fulllineitems}
\phantomsection\label{\detokenize{source/api/setup_micro_LP:nucleardatapy.setup_micro_LP.SetupMicroLP.init_self}}
\pysigstartsignatures
\pysiglinewithargsret
{\sphinxbfcode{\sphinxupquote{init\_self}}}
{}
{}
\pysigstopsignatures
\sphinxAtStartPar
Initialize variables in self.

\end{fulllineitems}

\index{model (nucleardatapy.setup\_micro\_LP.SetupMicroLP attribute)@\spxentry{model}\spxextra{nucleardatapy.setup\_micro\_LP.SetupMicroLP attribute}}

\begin{fulllineitems}
\phantomsection\label{\detokenize{source/api/setup_micro_LP:nucleardatapy.setup_micro_LP.SetupMicroLP.model}}
\pysigstartsignatures
\pysigline
{\sphinxbfcode{\sphinxupquote{model}}}
\pysigstopsignatures
\sphinxAtStartPar
Attribute model.

\end{fulllineitems}

\index{print\_outputs() (nucleardatapy.setup\_micro\_LP.SetupMicroLP method)@\spxentry{print\_outputs()}\spxextra{nucleardatapy.setup\_micro\_LP.SetupMicroLP method}}

\begin{fulllineitems}
\phantomsection\label{\detokenize{source/api/setup_micro_LP:nucleardatapy.setup_micro_LP.SetupMicroLP.print_outputs}}
\pysigstartsignatures
\pysiglinewithargsret
{\sphinxbfcode{\sphinxupquote{print\_outputs}}}
{}
{}
\pysigstopsignatures
\sphinxAtStartPar
Method which print outputs on terminal’s screen.

\end{fulllineitems}


\end{fulllineitems}

\index{models\_micro\_LP() (in module nucleardatapy.setup\_micro\_LP)@\spxentry{models\_micro\_LP()}\spxextra{in module nucleardatapy.setup\_micro\_LP}}

\begin{fulllineitems}
\phantomsection\label{\detokenize{source/api/setup_micro_LP:nucleardatapy.setup_micro_LP.models_micro_LP}}
\pysigstartsignatures
\pysiglinewithargsret
{\sphinxcode{\sphinxupquote{nucleardatapy.setup\_micro\_LP.}}\sphinxbfcode{\sphinxupquote{models\_micro\_LP}}}
{}
{}
\pysigstopsignatures
\sphinxAtStartPar
Return a list with the name of the models available in this toolkit and     print them all on the prompt. These models are the following ones:     ‘1994\sphinxhyphen{}BHF\sphinxhyphen{}SM\sphinxhyphen{}LP\sphinxhyphen{}AV14\sphinxhyphen{}GAP’, ‘1994\sphinxhyphen{}BHF\sphinxhyphen{}SM\sphinxhyphen{}LP\sphinxhyphen{}AV14\sphinxhyphen{}CONT’,     ‘1994\sphinxhyphen{}BHF\sphinxhyphen{}SM\sphinxhyphen{}LP\sphinxhyphen{}REID\sphinxhyphen{}GAP’, ‘1994\sphinxhyphen{}BHF\sphinxhyphen{}SM\sphinxhyphen{}LP\sphinxhyphen{}REID\sphinxhyphen{}CONT’, ‘1994\sphinxhyphen{}BHF\sphinxhyphen{}SM\sphinxhyphen{}LP\sphinxhyphen{}AV14\sphinxhyphen{}CONT\sphinxhyphen{}0.7’.
\begin{quote}\begin{description}
\sphinxlineitem{Returns}
\sphinxAtStartPar
The list of models.

\sphinxlineitem{Return type}
\sphinxAtStartPar
list{[}str{]}.

\end{description}\end{quote}

\end{fulllineitems}


\sphinxAtStartPar
Here are a set of figures which are produced with the Python sample: /sample/nucleardatapy\_plots/plot\_setupLP.py

\begin{figure}[htbp]
\centering
\capstart

\noindent\sphinxincludegraphics[scale=0.7]{{plot_SetupMicroLP0_SM}.png}
\caption{This figure shows the L=0 Landau parameters in SM for different NN interactions obtained
from BHF calculations.}\label{\detokenize{source/api/setup_micro_LP:id1}}\end{figure}

\begin{figure}[htbp]
\centering
\capstart

\noindent\sphinxincludegraphics[scale=0.7]{{plot_SetupMicroLP1_SM}.png}
\caption{This figure shows the L=1 Landau parameters in SM for different NN interactions obtained
from BHF calculations.}\label{\detokenize{source/api/setup_micro_LP:id2}}\end{figure}

\sphinxstepscope


\section{SetupPhenoMatter}
\label{\detokenize{source/api/setup_pheno_matter:setupphenomatter}}\label{\detokenize{source/api/setup_pheno_matter::doc}}\index{module@\spxentry{module}!nucleardatapy.setup\_pheno\_matter@\spxentry{nucleardatapy.setup\_pheno\_matter}}\index{nucleardatapy.setup\_pheno\_matter@\spxentry{nucleardatapy.setup\_pheno\_matter}!module@\spxentry{module}}\index{SetupPhenoMatter (class in nucleardatapy.setup\_pheno\_matter)@\spxentry{SetupPhenoMatter}\spxextra{class in nucleardatapy.setup\_pheno\_matter}}\phantomsection\label{\detokenize{source/api/setup_pheno_matter:module-nucleardatapy.setup_pheno_matter}}

\begin{fulllineitems}
\phantomsection\label{\detokenize{source/api/setup_pheno_matter:nucleardatapy.setup_pheno_matter.SetupPhenoMatter}}
\pysigstartsignatures
\pysiglinewithargsret
{\sphinxbfcode{\sphinxupquote{class\DUrole{w}{ }}}\sphinxcode{\sphinxupquote{nucleardatapy.setup\_pheno\_matter.}}\sphinxbfcode{\sphinxupquote{SetupPhenoMatter}}}
{\sphinxparam{\DUrole{n}{model}\DUrole{o}{=}\DUrole{default_value}{\textquotesingle{}Skyrme\textquotesingle{}}}\sphinxparamcomma \sphinxparam{\DUrole{n}{param}\DUrole{o}{=}\DUrole{default_value}{\textquotesingle{}SLY5\textquotesingle{}}}}
{}
\pysigstopsignatures
\sphinxAtStartPar
Instantiate the object with results based on phenomenological    interactions and choosen by the toolkit practitioner.     This choice is defined in the variables \sphinxtitleref{model} and \sphinxtitleref{param}.

\sphinxAtStartPar
If \sphinxtitleref{models} == ‘skyrme’, \sphinxtitleref{param} can be: ‘BSK14’,     ‘BSK16’, ‘BSK17’, ‘BSK27’, ‘F\sphinxhyphen{}’, ‘F+’, ‘F0’, ‘FPL’, ‘LNS’, ‘LNS1’, ‘LNS5’,     ‘NRAPR’, ‘RATP’, ‘SAMI’, ‘SGII’, ‘SIII’, ‘SKGSIGMA’, ‘SKI2’, ‘SKI4’, ‘SKMP’,     ‘SKMS’, ‘SKO’, ‘SKOP’, ‘SKP’, ‘SKRSIGMA’, ‘SKX’, ‘Skz2’, ‘SLY4’, ‘SLY5’,     ‘SLY230A’, ‘SLY230B’, ‘SV’, ‘T6’, ‘T44’, ‘UNEDF0’, ‘UNEDF1’.

\sphinxAtStartPar
If \sphinxtitleref{models} == ‘NLRH’, \sphinxtitleref{param} can be: ‘NL\sphinxhyphen{}SH’, ‘NL3’, ‘NL3II’, ‘PK1’, ‘PK1R’, ‘TM1’.

\sphinxAtStartPar
If \sphinxtitleref{models} == ‘DDRH’, \sphinxtitleref{param} can be: ‘DDME1’, ‘DDME2’, ‘DDMEd’, ‘PKDD’, ‘TW99’.

\sphinxAtStartPar
If \sphinxtitleref{models} == ‘DDRHF’, \sphinxtitleref{param} can be: ‘PKA1’, ‘PKO1’, ‘PKO2’, ‘PKO3’.
\begin{quote}\begin{description}
\sphinxlineitem{Parameters}\begin{itemize}
\item {} 
\sphinxAtStartPar
\sphinxstyleliteralstrong{\sphinxupquote{model}} (\sphinxstyleliteralemphasis{\sphinxupquote{str}}\sphinxstyleliteralemphasis{\sphinxupquote{, }}\sphinxstyleliteralemphasis{\sphinxupquote{optional.}}) \textendash{} Fix the name of model: ‘Skyrme’, ‘NLRH’,     ‘DDRH’, ‘DDRHF’. Default value: ‘Skyrme’.

\item {} 
\sphinxAtStartPar
\sphinxstyleliteralstrong{\sphinxupquote{param}} (\sphinxstyleliteralemphasis{\sphinxupquote{str}}\sphinxstyleliteralemphasis{\sphinxupquote{, }}\sphinxstyleliteralemphasis{\sphinxupquote{optional.}}) \textendash{} Fix the parameterization associated to model.     Default value: ‘SLY5’.

\end{itemize}

\end{description}\end{quote}

\sphinxAtStartPar
\sphinxstylestrong{Attributes:}
\index{Esat (nucleardatapy.setup\_pheno\_matter.SetupPhenoMatter attribute)@\spxentry{Esat}\spxextra{nucleardatapy.setup\_pheno\_matter.SetupPhenoMatter attribute}}

\begin{fulllineitems}
\phantomsection\label{\detokenize{source/api/setup_pheno_matter:nucleardatapy.setup_pheno_matter.SetupPhenoMatter.Esat}}
\pysigstartsignatures
\pysigline
{\sphinxbfcode{\sphinxupquote{Esat}}}
\pysigstopsignatures
\sphinxAtStartPar
Attribute the NEP.

\end{fulllineitems}

\index{esym\_den (nucleardatapy.setup\_pheno\_matter.SetupPhenoMatter attribute)@\spxentry{esym\_den}\spxextra{nucleardatapy.setup\_pheno\_matter.SetupPhenoMatter attribute}}

\begin{fulllineitems}
\phantomsection\label{\detokenize{source/api/setup_pheno_matter:nucleardatapy.setup_pheno_matter.SetupPhenoMatter.esym_den}}
\pysigstartsignatures
\pysigline
{\sphinxbfcode{\sphinxupquote{esym\_den}}}
\pysigstopsignatures
\sphinxAtStartPar
Attribute the density for the symmetry energy.

\end{fulllineitems}

\index{esym\_e2a (nucleardatapy.setup\_pheno\_matter.SetupPhenoMatter attribute)@\spxentry{esym\_e2a}\spxextra{nucleardatapy.setup\_pheno\_matter.SetupPhenoMatter attribute}}

\begin{fulllineitems}
\phantomsection\label{\detokenize{source/api/setup_pheno_matter:nucleardatapy.setup_pheno_matter.SetupPhenoMatter.esym_e2a}}
\pysigstartsignatures
\pysigline
{\sphinxbfcode{\sphinxupquote{esym\_e2a}}}
\pysigstopsignatures
\sphinxAtStartPar
Attribute the symmetry energy.

\end{fulllineitems}

\index{esym\_kf (nucleardatapy.setup\_pheno\_matter.SetupPhenoMatter attribute)@\spxentry{esym\_kf}\spxextra{nucleardatapy.setup\_pheno\_matter.SetupPhenoMatter attribute}}

\begin{fulllineitems}
\phantomsection\label{\detokenize{source/api/setup_pheno_matter:nucleardatapy.setup_pheno_matter.SetupPhenoMatter.esym_kf}}
\pysigstartsignatures
\pysigline
{\sphinxbfcode{\sphinxupquote{esym\_kf}}}
\pysigstopsignatures
\sphinxAtStartPar
Attribute the Fermi momentum for the symmetry energy.

\end{fulllineitems}

\index{label (nucleardatapy.setup\_pheno\_matter.SetupPhenoMatter attribute)@\spxentry{label}\spxextra{nucleardatapy.setup\_pheno\_matter.SetupPhenoMatter attribute}}

\begin{fulllineitems}
\phantomsection\label{\detokenize{source/api/setup_pheno_matter:nucleardatapy.setup_pheno_matter.SetupPhenoMatter.label}}
\pysigstartsignatures
\pysigline
{\sphinxbfcode{\sphinxupquote{label}}}
\pysigstopsignatures
\sphinxAtStartPar
Attribute providing the label the data is references for figures.

\end{fulllineitems}

\index{model (nucleardatapy.setup\_pheno\_matter.SetupPhenoMatter attribute)@\spxentry{model}\spxextra{nucleardatapy.setup\_pheno\_matter.SetupPhenoMatter attribute}}

\begin{fulllineitems}
\phantomsection\label{\detokenize{source/api/setup_pheno_matter:nucleardatapy.setup_pheno_matter.SetupPhenoMatter.model}}
\pysigstartsignatures
\pysigline
{\sphinxbfcode{\sphinxupquote{model}}}
\pysigstopsignatures
\sphinxAtStartPar
Attribute model.

\end{fulllineitems}

\index{nm\_cs2 (nucleardatapy.setup\_pheno\_matter.SetupPhenoMatter attribute)@\spxentry{nm\_cs2}\spxextra{nucleardatapy.setup\_pheno\_matter.SetupPhenoMatter attribute}}

\begin{fulllineitems}
\phantomsection\label{\detokenize{source/api/setup_pheno_matter:nucleardatapy.setup_pheno_matter.SetupPhenoMatter.nm_cs2}}
\pysigstartsignatures
\pysigline
{\sphinxbfcode{\sphinxupquote{nm\_cs2}}}
\pysigstopsignatures
\sphinxAtStartPar
Attribute the neutron matter sound speed (c\_s/c)\textasciicircum{}2.

\end{fulllineitems}

\index{nm\_den (nucleardatapy.setup\_pheno\_matter.SetupPhenoMatter attribute)@\spxentry{nm\_den}\spxextra{nucleardatapy.setup\_pheno\_matter.SetupPhenoMatter attribute}}

\begin{fulllineitems}
\phantomsection\label{\detokenize{source/api/setup_pheno_matter:nucleardatapy.setup_pheno_matter.SetupPhenoMatter.nm_den}}
\pysigstartsignatures
\pysigline
{\sphinxbfcode{\sphinxupquote{nm\_den}}}
\pysigstopsignatures
\sphinxAtStartPar
Attribute the neutron matter density.

\end{fulllineitems}

\index{nm\_e2a (nucleardatapy.setup\_pheno\_matter.SetupPhenoMatter attribute)@\spxentry{nm\_e2a}\spxextra{nucleardatapy.setup\_pheno\_matter.SetupPhenoMatter attribute}}

\begin{fulllineitems}
\phantomsection\label{\detokenize{source/api/setup_pheno_matter:nucleardatapy.setup_pheno_matter.SetupPhenoMatter.nm_e2a}}
\pysigstartsignatures
\pysigline
{\sphinxbfcode{\sphinxupquote{nm\_e2a}}}
\pysigstopsignatures
\sphinxAtStartPar
Attribute the neutron matter energy per particle.

\end{fulllineitems}

\index{nm\_gap (nucleardatapy.setup\_pheno\_matter.SetupPhenoMatter attribute)@\spxentry{nm\_gap}\spxextra{nucleardatapy.setup\_pheno\_matter.SetupPhenoMatter attribute}}

\begin{fulllineitems}
\phantomsection\label{\detokenize{source/api/setup_pheno_matter:nucleardatapy.setup_pheno_matter.SetupPhenoMatter.nm_gap}}
\pysigstartsignatures
\pysigline
{\sphinxbfcode{\sphinxupquote{nm\_gap}}}
\pysigstopsignatures
\sphinxAtStartPar
Attribute the neutron matter pairing gap.

\end{fulllineitems}

\index{nm\_kfn (nucleardatapy.setup\_pheno\_matter.SetupPhenoMatter attribute)@\spxentry{nm\_kfn}\spxextra{nucleardatapy.setup\_pheno\_matter.SetupPhenoMatter attribute}}

\begin{fulllineitems}
\phantomsection\label{\detokenize{source/api/setup_pheno_matter:nucleardatapy.setup_pheno_matter.SetupPhenoMatter.nm_kfn}}
\pysigstartsignatures
\pysigline
{\sphinxbfcode{\sphinxupquote{nm\_kfn}}}
\pysigstopsignatures
\sphinxAtStartPar
Attribute the neutron matter neutron Fermi momentum.

\end{fulllineitems}

\index{nm\_pre (nucleardatapy.setup\_pheno\_matter.SetupPhenoMatter attribute)@\spxentry{nm\_pre}\spxextra{nucleardatapy.setup\_pheno\_matter.SetupPhenoMatter attribute}}

\begin{fulllineitems}
\phantomsection\label{\detokenize{source/api/setup_pheno_matter:nucleardatapy.setup_pheno_matter.SetupPhenoMatter.nm_pre}}
\pysigstartsignatures
\pysigline
{\sphinxbfcode{\sphinxupquote{nm\_pre}}}
\pysigstopsignatures
\sphinxAtStartPar
Attribute the neutron matter pressure.

\end{fulllineitems}

\index{note (nucleardatapy.setup\_pheno\_matter.SetupPhenoMatter attribute)@\spxentry{note}\spxextra{nucleardatapy.setup\_pheno\_matter.SetupPhenoMatter attribute}}

\begin{fulllineitems}
\phantomsection\label{\detokenize{source/api/setup_pheno_matter:nucleardatapy.setup_pheno_matter.SetupPhenoMatter.note}}
\pysigstartsignatures
\pysigline
{\sphinxbfcode{\sphinxupquote{note}}}
\pysigstopsignatures
\sphinxAtStartPar
Attribute providing additional notes about the data.

\end{fulllineitems}

\index{param (nucleardatapy.setup\_pheno\_matter.SetupPhenoMatter attribute)@\spxentry{param}\spxextra{nucleardatapy.setup\_pheno\_matter.SetupPhenoMatter attribute}}

\begin{fulllineitems}
\phantomsection\label{\detokenize{source/api/setup_pheno_matter:nucleardatapy.setup_pheno_matter.SetupPhenoMatter.param}}
\pysigstartsignatures
\pysigline
{\sphinxbfcode{\sphinxupquote{param}}}
\pysigstopsignatures
\sphinxAtStartPar
Attribute param.

\end{fulllineitems}

\index{print\_outputs() (nucleardatapy.setup\_pheno\_matter.SetupPhenoMatter method)@\spxentry{print\_outputs()}\spxextra{nucleardatapy.setup\_pheno\_matter.SetupPhenoMatter method}}

\begin{fulllineitems}
\phantomsection\label{\detokenize{source/api/setup_pheno_matter:nucleardatapy.setup_pheno_matter.SetupPhenoMatter.print_outputs}}
\pysigstartsignatures
\pysiglinewithargsret
{\sphinxbfcode{\sphinxupquote{print\_outputs}}}
{}
{}
\pysigstopsignatures
\sphinxAtStartPar
Method which print outputs on terminal’s screen.

\end{fulllineitems}

\index{ref (nucleardatapy.setup\_pheno\_matter.SetupPhenoMatter attribute)@\spxentry{ref}\spxextra{nucleardatapy.setup\_pheno\_matter.SetupPhenoMatter attribute}}

\begin{fulllineitems}
\phantomsection\label{\detokenize{source/api/setup_pheno_matter:nucleardatapy.setup_pheno_matter.SetupPhenoMatter.ref}}
\pysigstartsignatures
\pysigline
{\sphinxbfcode{\sphinxupquote{ref}}}
\pysigstopsignatures
\sphinxAtStartPar
Attribute providing the full reference to the paper to be citted.

\end{fulllineitems}

\index{sm\_cs2 (nucleardatapy.setup\_pheno\_matter.SetupPhenoMatter attribute)@\spxentry{sm\_cs2}\spxextra{nucleardatapy.setup\_pheno\_matter.SetupPhenoMatter attribute}}

\begin{fulllineitems}
\phantomsection\label{\detokenize{source/api/setup_pheno_matter:nucleardatapy.setup_pheno_matter.SetupPhenoMatter.sm_cs2}}
\pysigstartsignatures
\pysigline
{\sphinxbfcode{\sphinxupquote{sm\_cs2}}}
\pysigstopsignatures
\sphinxAtStartPar
Attribute the symmetric matter sound speed (c\_s/c)\textasciicircum{}2.

\end{fulllineitems}

\index{sm\_den (nucleardatapy.setup\_pheno\_matter.SetupPhenoMatter attribute)@\spxentry{sm\_den}\spxextra{nucleardatapy.setup\_pheno\_matter.SetupPhenoMatter attribute}}

\begin{fulllineitems}
\phantomsection\label{\detokenize{source/api/setup_pheno_matter:nucleardatapy.setup_pheno_matter.SetupPhenoMatter.sm_den}}
\pysigstartsignatures
\pysigline
{\sphinxbfcode{\sphinxupquote{sm\_den}}}
\pysigstopsignatures
\sphinxAtStartPar
Attribute the symmetric matter density.

\end{fulllineitems}

\index{sm\_e2a (nucleardatapy.setup\_pheno\_matter.SetupPhenoMatter attribute)@\spxentry{sm\_e2a}\spxextra{nucleardatapy.setup\_pheno\_matter.SetupPhenoMatter attribute}}

\begin{fulllineitems}
\phantomsection\label{\detokenize{source/api/setup_pheno_matter:nucleardatapy.setup_pheno_matter.SetupPhenoMatter.sm_e2a}}
\pysigstartsignatures
\pysigline
{\sphinxbfcode{\sphinxupquote{sm\_e2a}}}
\pysigstopsignatures
\sphinxAtStartPar
Attribute the symmetric matter energy per particle.

\end{fulllineitems}

\index{sm\_gap (nucleardatapy.setup\_pheno\_matter.SetupPhenoMatter attribute)@\spxentry{sm\_gap}\spxextra{nucleardatapy.setup\_pheno\_matter.SetupPhenoMatter attribute}}

\begin{fulllineitems}
\phantomsection\label{\detokenize{source/api/setup_pheno_matter:nucleardatapy.setup_pheno_matter.SetupPhenoMatter.sm_gap}}
\pysigstartsignatures
\pysigline
{\sphinxbfcode{\sphinxupquote{sm\_gap}}}
\pysigstopsignatures
\sphinxAtStartPar
Attribute the symmetric matter pairing gap.

\end{fulllineitems}

\index{sm\_kf (nucleardatapy.setup\_pheno\_matter.SetupPhenoMatter attribute)@\spxentry{sm\_kf}\spxextra{nucleardatapy.setup\_pheno\_matter.SetupPhenoMatter attribute}}

\begin{fulllineitems}
\phantomsection\label{\detokenize{source/api/setup_pheno_matter:nucleardatapy.setup_pheno_matter.SetupPhenoMatter.sm_kf}}
\pysigstartsignatures
\pysigline
{\sphinxbfcode{\sphinxupquote{sm\_kf}}}
\pysigstopsignatures
\sphinxAtStartPar
Attribute the symmetric matter Fermi momentum.

\end{fulllineitems}

\index{sm\_kfn (nucleardatapy.setup\_pheno\_matter.SetupPhenoMatter attribute)@\spxentry{sm\_kfn}\spxextra{nucleardatapy.setup\_pheno\_matter.SetupPhenoMatter attribute}}

\begin{fulllineitems}
\phantomsection\label{\detokenize{source/api/setup_pheno_matter:nucleardatapy.setup_pheno_matter.SetupPhenoMatter.sm_kfn}}
\pysigstartsignatures
\pysigline
{\sphinxbfcode{\sphinxupquote{sm\_kfn}}}
\pysigstopsignatures
\sphinxAtStartPar
Attribute the symmetric matter neutron Fermi momentum.

\end{fulllineitems}

\index{sm\_pre (nucleardatapy.setup\_pheno\_matter.SetupPhenoMatter attribute)@\spxentry{sm\_pre}\spxextra{nucleardatapy.setup\_pheno\_matter.SetupPhenoMatter attribute}}

\begin{fulllineitems}
\phantomsection\label{\detokenize{source/api/setup_pheno_matter:nucleardatapy.setup_pheno_matter.SetupPhenoMatter.sm_pre}}
\pysigstartsignatures
\pysigline
{\sphinxbfcode{\sphinxupquote{sm\_pre}}}
\pysigstopsignatures
\sphinxAtStartPar
Attribute the symmetric matter pressure.

\end{fulllineitems}


\end{fulllineitems}

\index{models\_pheno\_matter() (in module nucleardatapy.setup\_pheno\_matter)@\spxentry{models\_pheno\_matter()}\spxextra{in module nucleardatapy.setup\_pheno\_matter}}

\begin{fulllineitems}
\phantomsection\label{\detokenize{source/api/setup_pheno_matter:nucleardatapy.setup_pheno_matter.models_pheno_matter}}
\pysigstartsignatures
\pysiglinewithargsret
{\sphinxcode{\sphinxupquote{nucleardatapy.setup\_pheno\_matter.}}\sphinxbfcode{\sphinxupquote{models\_pheno\_matter}}}
{}
{}
\pysigstopsignatures
\sphinxAtStartPar
Return a list of models available in this toolkit and print them all on the prompt.
\begin{quote}\begin{description}
\sphinxlineitem{Returns}
\sphinxAtStartPar
The list of models with can be ‘Skyrme’, ‘NLRH’, ‘DDRH’, ‘DDRHF’.

\sphinxlineitem{Return type}
\sphinxAtStartPar
list{[}str{]}.

\end{description}\end{quote}

\end{fulllineitems}

\index{params\_pheno\_matter() (in module nucleardatapy.setup\_pheno\_matter)@\spxentry{params\_pheno\_matter()}\spxextra{in module nucleardatapy.setup\_pheno\_matter}}

\begin{fulllineitems}
\phantomsection\label{\detokenize{source/api/setup_pheno_matter:nucleardatapy.setup_pheno_matter.params_pheno_matter}}
\pysigstartsignatures
\pysiglinewithargsret
{\sphinxcode{\sphinxupquote{nucleardatapy.setup\_pheno\_matter.}}\sphinxbfcode{\sphinxupquote{params\_pheno\_matter}}}
{\sphinxparam{\DUrole{n}{model}}}
{}
\pysigstopsignatures
\sphinxAtStartPar
Return a list with the parameterizations available in
this toolkit for a given model and print them all on the prompt.
\begin{quote}\begin{description}
\sphinxlineitem{Parameters}
\sphinxAtStartPar
\sphinxstyleliteralstrong{\sphinxupquote{model}} (\sphinxstyleliteralemphasis{\sphinxupquote{str.}}) \textendash{} The type of model for which there are parametrizations.     They should be chosen among the following options: ‘Skyrme’, ‘NLRH’,     ‘DDRH’, ‘DDRHF’.

\sphinxlineitem{Returns}
\sphinxAtStartPar
The list of parametrizations.     If \sphinxtitleref{models} == ‘skyrme’: ‘BSK14’,     ‘BSK16’, ‘BSK17’, ‘BSK27’, ‘F\sphinxhyphen{}’, ‘F+’, ‘F0’, ‘FPL’, ‘LNS’, ‘LNS1’, ‘LNS5’,     ‘NRAPR’, ‘RATP’, ‘SAMI’, ‘SGII’, ‘SIII’, ‘SKGSIGMA’, ‘SKI2’, ‘SKI4’, ‘SKMP’,     ‘SKMS’, ‘SKO’, ‘SKOP’, ‘SKP’, ‘SKRSIGMA’, ‘SKX’, ‘Skz2’, ‘SLY4’, ‘SLY5’,     ‘SLY230A’, ‘SLY230B’, ‘SV’, ‘T6’, ‘T44’, ‘UNEDF0’, ‘UNEDF1’.     If \sphinxtitleref{models} == ‘NLRH’: ‘NL\sphinxhyphen{}SH’, ‘NL3’, ‘NL3II’, ‘PK1’, ‘PK1R’, ‘TM1’.     If \sphinxtitleref{models} == ‘DDRH’: ‘DDME1’, ‘DDME2’, ‘DDMEd’, ‘PKDD’, ‘TW99’.     If \sphinxtitleref{models} == ‘DDRHF’: ‘PKA1’, ‘PKO1’, ‘PKO2’, ‘PKO3’.

\sphinxlineitem{Return type}
\sphinxAtStartPar
list{[}str{]}.

\end{description}\end{quote}

\end{fulllineitems}


\sphinxAtStartPar
Here are a set of figures which are produced with the Python sample: /sample/nucleardatapy\_plots/plot\_setupPhenoMatter.py

\begin{figure}[htbp]
\centering
\capstart

\noindent\sphinxincludegraphics[scale=0.7]{{plot_SetupPhenoMatter-NLRH-E-NM}.png}
\caption{This figure shows the energy in neutron matter (NM) over the free Fermi gas energy (top) and the energy per particle (bottom) as function of the density (left) and the neutron Fermi momentum (right) for the complete list of phenomenological models based on non\sphinxhyphen{}linear meson(s) relativistic Hartree (NLRH) approach available in the nucleardatapy toolkit.}\label{\detokenize{source/api/setup_pheno_matter:id1}}\end{figure}

\begin{figure}[htbp]
\centering
\capstart

\noindent\sphinxincludegraphics[scale=0.7]{{plot_SetupPhenoMatter-NLRH-E-SM}.png}
\caption{This figure shows the energy in symmetric matter (SM) over the free Fermi gas energy (top) and the energy per particle (bottom) as function of the density (left) and the neutron Fermi momentum (right) for the complete list of phenomenological models based on non\sphinxhyphen{}linear meson(s) relativistic Hartree (NLRH) approach available in the nucleardatapy toolkit.}\label{\detokenize{source/api/setup_pheno_matter:id2}}\end{figure}

\begin{figure}[htbp]
\centering
\capstart

\noindent\sphinxincludegraphics[scale=0.7]{{plot_SetupPhenoMatter-DDRH-E-NM}.png}
\caption{This figure shows the energy in neutron matter (NM) over the free Fermi gas energy (top) and the energy per particle (bottom) as function of the density (left) and the neutron Fermi momentum (right) for the complete list of phenomenological models based on density\sphinxhyphen{}dependent relativistic Hartree (DDRH) approach available in the nucleardatapy toolkit.}\label{\detokenize{source/api/setup_pheno_matter:id3}}\end{figure}

\begin{figure}[htbp]
\centering
\capstart

\noindent\sphinxincludegraphics[scale=0.7]{{plot_SetupPhenoMatter-DDRH-E-SM}.png}
\caption{This figure shows the energy in symmetric matter (SM) over the free Fermi gas energy (top) and the energy per particle (bottom) as function of the density (left) and the neutron Fermi momentum (right) for the complete list of phenomenological models based on density\sphinxhyphen{}dependent relativistic Hartree (DDRH) approach available in the nucleardatapy toolkit.}\label{\detokenize{source/api/setup_pheno_matter:id4}}\end{figure}

\begin{figure}[htbp]
\centering
\capstart

\noindent\sphinxincludegraphics[scale=0.7]{{plot_SetupPhenoMatter-DDRHF-E-NM}.png}
\caption{This figure shows the energy in neutron matter (NM) over the free Fermi gas energy (top) and the energy per particle (bottom) as function of the density (left) and the neutron Fermi momentum (right) for the complete list of phenomenological models based on density\sphinxhyphen{}dependent relativistic Hartree\sphinxhyphen{}Fock (DDRHF) approach available in the nucleardatapy toolkit.}\label{\detokenize{source/api/setup_pheno_matter:id5}}\end{figure}

\begin{figure}[htbp]
\centering
\capstart

\noindent\sphinxincludegraphics[scale=0.7]{{plot_SetupPhenoMatter-DDRHF-E-SM}.png}
\caption{This figure shows the energy in symmetric matter (SM) over the free Fermi gas energy (top) and the energy per particle (bottom) as function of the density (left) and the neutron Fermi momentum (right) for the complete list of phenomenological models based on density\sphinxhyphen{}dependent relativistic Hartree\sphinxhyphen{}Fock (DDRHF) approach available in the nucleardatapy toolkit.}\label{\detokenize{source/api/setup_pheno_matter:id6}}\end{figure}

\begin{figure}[htbp]
\centering
\capstart

\noindent\sphinxincludegraphics[scale=0.7]{{plot_SetupPhenoMatter-Skyrme-E-NM}.png}
\caption{This figure shows the energy in neutron matter (NM) over the free Fermi gas energy (top) and the energy per particle (bottom) as function of the density (left) and the neutron Fermi momentum (right) for the complete list of phenomenological models based on the standard Skyrme interaction available in the nucleardatapy toolkit.}\label{\detokenize{source/api/setup_pheno_matter:id7}}\end{figure}

\begin{figure}[htbp]
\centering
\capstart

\noindent\sphinxincludegraphics[scale=0.7]{{plot_SetupPhenoMatter-Skyrme-E-SM}.png}
\caption{This figure shows the energy in symmetric matter (SM) over the free Fermi gas energy (top) and the energy per particle (bottom) as function of the density (left) and the neutron Fermi momentum (right) for the complete list of phenomenological models based on the standard Skyrme interaction available in the nucleardatapy toolkit.}\label{\detokenize{source/api/setup_pheno_matter:id8}}\end{figure}

\begin{figure}[htbp]
\centering
\capstart

\noindent\sphinxincludegraphics[scale=0.7]{{plot_NEP}.png}
\caption{Distribution of NEP for phenomenological models available in the nucleardatapy toolkit.}\label{\detokenize{source/api/setup_pheno_matter:id9}}\end{figure}

\sphinxstepscope


\section{SetupHICMatter}
\label{\detokenize{source/api/setup_hic_matter:setuphicmatter}}\label{\detokenize{source/api/setup_hic_matter::doc}}\index{module@\spxentry{module}!nucleardatapy.setup\_hic\_matter@\spxentry{nucleardatapy.setup\_hic\_matter}}\index{nucleardatapy.setup\_hic\_matter@\spxentry{nucleardatapy.setup\_hic\_matter}!module@\spxentry{module}}\index{SetupHICMatter (class in nucleardatapy.setup\_hic\_matter)@\spxentry{SetupHICMatter}\spxextra{class in nucleardatapy.setup\_hic\_matter}}\phantomsection\label{\detokenize{source/api/setup_hic_matter:module-nucleardatapy.setup_hic_matter}}

\begin{fulllineitems}
\phantomsection\label{\detokenize{source/api/setup_hic_matter:nucleardatapy.setup_hic_matter.SetupHICMatter}}
\pysigstartsignatures
\pysiglinewithargsret
{\sphinxbfcode{\sphinxupquote{class\DUrole{w}{ }}}\sphinxcode{\sphinxupquote{nucleardatapy.setup\_hic\_matter.}}\sphinxbfcode{\sphinxupquote{SetupHICMatter}}}
{\sphinxparam{\DUrole{n}{constraint}\DUrole{o}{=}\DUrole{default_value}{\textquotesingle{}DLL\sphinxhyphen{}2002\textquotesingle{}}}}
{}
\pysigstopsignatures
\sphinxAtStartPar
Instantiate the constraints on the EOS from HIC.

\sphinxAtStartPar
This choice is defined in the variable \sphinxtitleref{constraint}.

\sphinxAtStartPar
\sphinxtitleref{constraint} can chosen among the following ones: {[} ‘DLL\sphinxhyphen{}2002’, ‘FOPI\sphinxhyphen{}2016’ {]}.
\begin{quote}\begin{description}
\sphinxlineitem{Parameters}
\sphinxAtStartPar
\sphinxstyleliteralstrong{\sphinxupquote{constraint}} (\sphinxstyleliteralemphasis{\sphinxupquote{str}}\sphinxstyleliteralemphasis{\sphinxupquote{, }}\sphinxstyleliteralemphasis{\sphinxupquote{optional.}}) \textendash{} Fix the name of \sphinxtitleref{constraint}. Default value: ‘DLL\sphinxhyphen{}2002’.

\end{description}\end{quote}

\sphinxAtStartPar
\sphinxstylestrong{Attributes:}
\index{init\_self() (nucleardatapy.setup\_hic\_matter.SetupHICMatter method)@\spxentry{init\_self()}\spxextra{nucleardatapy.setup\_hic\_matter.SetupHICMatter method}}

\begin{fulllineitems}
\phantomsection\label{\detokenize{source/api/setup_hic_matter:nucleardatapy.setup_hic_matter.SetupHICMatter.init_self}}
\pysigstartsignatures
\pysiglinewithargsret
{\sphinxbfcode{\sphinxupquote{init\_self}}}
{}
{}
\pysigstopsignatures
\sphinxAtStartPar
Initialize variables in self.

\end{fulllineitems}

\index{print\_outputs() (nucleardatapy.setup\_hic\_matter.SetupHICMatter method)@\spxentry{print\_outputs()}\spxextra{nucleardatapy.setup\_hic\_matter.SetupHICMatter method}}

\begin{fulllineitems}
\phantomsection\label{\detokenize{source/api/setup_hic_matter:nucleardatapy.setup_hic_matter.SetupHICMatter.print_outputs}}
\pysigstartsignatures
\pysiglinewithargsret
{\sphinxbfcode{\sphinxupquote{print\_outputs}}}
{}
{}
\pysigstopsignatures
\sphinxAtStartPar
Method which print outputs on terminal’s screen.

\end{fulllineitems}


\end{fulllineitems}

\index{constraints\_HIC\_matter() (in module nucleardatapy.setup\_hic\_matter)@\spxentry{constraints\_HIC\_matter()}\spxextra{in module nucleardatapy.setup\_hic\_matter}}

\begin{fulllineitems}
\phantomsection\label{\detokenize{source/api/setup_hic_matter:nucleardatapy.setup_hic_matter.constraints_HIC_matter}}
\pysigstartsignatures
\pysiglinewithargsret
{\sphinxcode{\sphinxupquote{nucleardatapy.setup\_hic\_matter.}}\sphinxbfcode{\sphinxupquote{constraints\_HIC\_matter}}}
{}
{}
\pysigstopsignatures
\sphinxAtStartPar
Return a list of the HIC constraints available in this toolkit
for the equation of state in SM and NM and print them all on
the prompt. These constraints are the following
ones: {[} ‘DLL\sphinxhyphen{}2002’, ‘FOPI\sphinxhyphen{}2016’ {]}.
\begin{quote}\begin{description}
\sphinxlineitem{Returns}
\sphinxAtStartPar
The list of constraints.

\sphinxlineitem{Return type}
\sphinxAtStartPar
list{[}str{]}.

\end{description}\end{quote}

\end{fulllineitems}


\begin{figure}[htbp]
\centering
\capstart

\noindent\sphinxincludegraphics[scale=0.7]{{plot_SetupHICMatter}.png}
\caption{HIC Experimental constraints for the energy per particle (left) and pressure (right) in SM as a function of the particle density for different analyses available in the \sphinxtitleref{nuda} toolkit.}\label{\detokenize{source/api/setup_hic_matter:id1}}\end{figure}

\sphinxstepscope


\section{SetupCrust}
\label{\detokenize{source/api/setup_crust:setupcrust}}\label{\detokenize{source/api/setup_crust::doc}}\index{module@\spxentry{module}!nucleardatapy.setup\_crust@\spxentry{nucleardatapy.setup\_crust}}\index{nucleardatapy.setup\_crust@\spxentry{nucleardatapy.setup\_crust}!module@\spxentry{module}}\index{SetupCrust (class in nucleardatapy.setup\_crust)@\spxentry{SetupCrust}\spxextra{class in nucleardatapy.setup\_crust}}\phantomsection\label{\detokenize{source/api/setup_crust:module-nucleardatapy.setup_crust}}

\begin{fulllineitems}
\phantomsection\label{\detokenize{source/api/setup_crust:nucleardatapy.setup_crust.SetupCrust}}
\pysigstartsignatures
\pysiglinewithargsret
{\sphinxbfcode{\sphinxupquote{class\DUrole{w}{ }}}\sphinxcode{\sphinxupquote{nucleardatapy.setup\_crust.}}\sphinxbfcode{\sphinxupquote{SetupCrust}}}
{\sphinxparam{\DUrole{n}{modcrust}\DUrole{o}{=}\DUrole{default_value}{\textquotesingle{}1973\sphinxhyphen{}Negele\sphinxhyphen{}Vautherin\textquotesingle{}}}}
{}
\pysigstopsignatures
\sphinxAtStartPar
Instantiate the properties of the crust for the existing models.

\sphinxAtStartPar
This choice is defined in the variable \sphinxtitleref{crust}.

\sphinxAtStartPar
\sphinxtitleref{crust} can chosen among the following ones: ‘Negele\sphinxhyphen{}Vautherin\sphinxhyphen{}1973’.
\begin{quote}\begin{description}
\sphinxlineitem{Parameters}
\sphinxAtStartPar
\sphinxstyleliteralstrong{\sphinxupquote{crust}} (\sphinxstyleliteralemphasis{\sphinxupquote{str}}\sphinxstyleliteralemphasis{\sphinxupquote{, }}\sphinxstyleliteralemphasis{\sphinxupquote{optional.}}) \textendash{} Fix the name of \sphinxtitleref{crust}. Default value: ‘Negele\sphinxhyphen{}Vautherin\sphinxhyphen{}1973’.

\end{description}\end{quote}

\sphinxAtStartPar
\sphinxstylestrong{Attributes:}
\index{A (nucleardatapy.setup\_crust.SetupCrust attribute)@\spxentry{A}\spxextra{nucleardatapy.setup\_crust.SetupCrust attribute}}

\begin{fulllineitems}
\phantomsection\label{\detokenize{source/api/setup_crust:nucleardatapy.setup_crust.SetupCrust.A}}
\pysigstartsignatures
\pysigline
{\sphinxbfcode{\sphinxupquote{A}}}
\pysigstopsignatures
\sphinxAtStartPar
Attribute A (mass of the nucleus).

\end{fulllineitems}

\index{N (nucleardatapy.setup\_crust.SetupCrust attribute)@\spxentry{N}\spxextra{nucleardatapy.setup\_crust.SetupCrust attribute}}

\begin{fulllineitems}
\phantomsection\label{\detokenize{source/api/setup_crust:nucleardatapy.setup_crust.SetupCrust.N}}
\pysigstartsignatures
\pysigline
{\sphinxbfcode{\sphinxupquote{N}}}
\pysigstopsignatures
\sphinxAtStartPar
Attribute N (total number of neutrons of the WS cell).

\end{fulllineitems}

\index{N\_bound (nucleardatapy.setup\_crust.SetupCrust attribute)@\spxentry{N\_bound}\spxextra{nucleardatapy.setup\_crust.SetupCrust attribute}}

\begin{fulllineitems}
\phantomsection\label{\detokenize{source/api/setup_crust:nucleardatapy.setup_crust.SetupCrust.N_bound}}
\pysigstartsignatures
\pysigline
{\sphinxbfcode{\sphinxupquote{N\_bound}}}
\pysigstopsignatures
\sphinxAtStartPar
Attribute N\_bound (number of bound neutrons).

\end{fulllineitems}

\index{N\_g (nucleardatapy.setup\_crust.SetupCrust attribute)@\spxentry{N\_g}\spxextra{nucleardatapy.setup\_crust.SetupCrust attribute}}

\begin{fulllineitems}
\phantomsection\label{\detokenize{source/api/setup_crust:nucleardatapy.setup_crust.SetupCrust.N_g}}
\pysigstartsignatures
\pysigline
{\sphinxbfcode{\sphinxupquote{N\_g}}}
\pysigstopsignatures
\sphinxAtStartPar
Attribute N\_g (number of neutrons in the gas).

\end{fulllineitems}

\index{RWS (nucleardatapy.setup\_crust.SetupCrust attribute)@\spxentry{RWS}\spxextra{nucleardatapy.setup\_crust.SetupCrust attribute}}

\begin{fulllineitems}
\phantomsection\label{\detokenize{source/api/setup_crust:nucleardatapy.setup_crust.SetupCrust.RWS}}
\pysigstartsignatures
\pysigline
{\sphinxbfcode{\sphinxupquote{RWS}}}
\pysigstopsignatures
\sphinxAtStartPar
Attribute the radius of the WS cell (in fm).

\end{fulllineitems}

\index{Z (nucleardatapy.setup\_crust.SetupCrust attribute)@\spxentry{Z}\spxextra{nucleardatapy.setup\_crust.SetupCrust attribute}}

\begin{fulllineitems}
\phantomsection\label{\detokenize{source/api/setup_crust:nucleardatapy.setup_crust.SetupCrust.Z}}
\pysigstartsignatures
\pysigline
{\sphinxbfcode{\sphinxupquote{Z}}}
\pysigstopsignatures
\sphinxAtStartPar
Attribute Z (charge of the nucleus).

\end{fulllineitems}

\index{den (nucleardatapy.setup\_crust.SetupCrust attribute)@\spxentry{den}\spxextra{nucleardatapy.setup\_crust.SetupCrust attribute}}

\begin{fulllineitems}
\phantomsection\label{\detokenize{source/api/setup_crust:nucleardatapy.setup_crust.SetupCrust.den}}
\pysigstartsignatures
\pysigline
{\sphinxbfcode{\sphinxupquote{den}}}
\pysigstopsignatures
\sphinxAtStartPar
Attribute the density of the system (in fm\textasciicircum{}\sphinxhyphen{}3).

\end{fulllineitems}

\index{den\_cgs (nucleardatapy.setup\_crust.SetupCrust attribute)@\spxentry{den\_cgs}\spxextra{nucleardatapy.setup\_crust.SetupCrust attribute}}

\begin{fulllineitems}
\phantomsection\label{\detokenize{source/api/setup_crust:nucleardatapy.setup_crust.SetupCrust.den_cgs}}
\pysigstartsignatures
\pysigline
{\sphinxbfcode{\sphinxupquote{den\_cgs}}}
\pysigstopsignatures
\sphinxAtStartPar
Attribute the density of the system (in cm\textasciicircum{}\sphinxhyphen{}3).

\end{fulllineitems}

\index{den\_g (nucleardatapy.setup\_crust.SetupCrust attribute)@\spxentry{den\_g}\spxextra{nucleardatapy.setup\_crust.SetupCrust attribute}}

\begin{fulllineitems}
\phantomsection\label{\detokenize{source/api/setup_crust:nucleardatapy.setup_crust.SetupCrust.den_g}}
\pysigstartsignatures
\pysigline
{\sphinxbfcode{\sphinxupquote{den\_g}}}
\pysigstopsignatures
\sphinxAtStartPar
Attribute the approximate density of neutron in the gas (in fm\sphinxhyphen{}3).

\end{fulllineitems}

\index{e2a\_int (nucleardatapy.setup\_crust.SetupCrust attribute)@\spxentry{e2a\_int}\spxextra{nucleardatapy.setup\_crust.SetupCrust attribute}}

\begin{fulllineitems}
\phantomsection\label{\detokenize{source/api/setup_crust:nucleardatapy.setup_crust.SetupCrust.e2a_int}}
\pysigstartsignatures
\pysigline
{\sphinxbfcode{\sphinxupquote{e2a\_int}}}
\pysigstopsignatures
\sphinxAtStartPar
Attribute the internal energy (in MeV).

\end{fulllineitems}

\index{e2a\_int2 (nucleardatapy.setup\_crust.SetupCrust attribute)@\spxentry{e2a\_int2}\spxextra{nucleardatapy.setup\_crust.SetupCrust attribute}}

\begin{fulllineitems}
\phantomsection\label{\detokenize{source/api/setup_crust:nucleardatapy.setup_crust.SetupCrust.e2a_int2}}
\pysigstartsignatures
\pysigline
{\sphinxbfcode{\sphinxupquote{e2a\_int2}}}
\pysigstopsignatures
\sphinxAtStartPar
Attribute the energy minus the neutron mass (in MeV).

\end{fulllineitems}

\index{e2a\_int\_g (nucleardatapy.setup\_crust.SetupCrust attribute)@\spxentry{e2a\_int\_g}\spxextra{nucleardatapy.setup\_crust.SetupCrust attribute}}

\begin{fulllineitems}
\phantomsection\label{\detokenize{source/api/setup_crust:nucleardatapy.setup_crust.SetupCrust.e2a_int_g}}
\pysigstartsignatures
\pysigline
{\sphinxbfcode{\sphinxupquote{e2a\_int\_g}}}
\pysigstopsignatures
\sphinxAtStartPar
Attribute the internal energy of the gas component (in MeV).

\end{fulllineitems}

\index{e2a\_rm (nucleardatapy.setup\_crust.SetupCrust attribute)@\spxentry{e2a\_rm}\spxextra{nucleardatapy.setup\_crust.SetupCrust attribute}}

\begin{fulllineitems}
\phantomsection\label{\detokenize{source/api/setup_crust:nucleardatapy.setup_crust.SetupCrust.e2a_rm}}
\pysigstartsignatures
\pysigline
{\sphinxbfcode{\sphinxupquote{e2a\_rm}}}
\pysigstopsignatures
\sphinxAtStartPar
Attribute the rest mass energy (in MeV).

\end{fulllineitems}

\index{mu\_n (nucleardatapy.setup\_crust.SetupCrust attribute)@\spxentry{mu\_n}\spxextra{nucleardatapy.setup\_crust.SetupCrust attribute}}

\begin{fulllineitems}
\phantomsection\label{\detokenize{source/api/setup_crust:nucleardatapy.setup_crust.SetupCrust.mu_n}}
\pysigstartsignatures
\pysigline
{\sphinxbfcode{\sphinxupquote{mu\_n}}}
\pysigstopsignatures
\sphinxAtStartPar
Attribute the neutron chemical potential (in MeV).

\end{fulllineitems}

\index{mu\_p (nucleardatapy.setup\_crust.SetupCrust attribute)@\spxentry{mu\_p}\spxextra{nucleardatapy.setup\_crust.SetupCrust attribute}}

\begin{fulllineitems}
\phantomsection\label{\detokenize{source/api/setup_crust:nucleardatapy.setup_crust.SetupCrust.mu_p}}
\pysigstartsignatures
\pysigline
{\sphinxbfcode{\sphinxupquote{mu\_p}}}
\pysigstopsignatures
\sphinxAtStartPar
Attribute the proton chemical potential (in MeV).

\end{fulllineitems}

\index{print\_outputs() (nucleardatapy.setup\_crust.SetupCrust method)@\spxentry{print\_outputs()}\spxextra{nucleardatapy.setup\_crust.SetupCrust method}}

\begin{fulllineitems}
\phantomsection\label{\detokenize{source/api/setup_crust:nucleardatapy.setup_crust.SetupCrust.print_outputs}}
\pysigstartsignatures
\pysiglinewithargsret
{\sphinxbfcode{\sphinxupquote{print\_outputs}}}
{}
{}
\pysigstopsignatures
\sphinxAtStartPar
Method which print outputs on terminal’s screen.

\end{fulllineitems}

\index{xn (nucleardatapy.setup\_crust.SetupCrust attribute)@\spxentry{xn}\spxextra{nucleardatapy.setup\_crust.SetupCrust attribute}}

\begin{fulllineitems}
\phantomsection\label{\detokenize{source/api/setup_crust:nucleardatapy.setup_crust.SetupCrust.xn}}
\pysigstartsignatures
\pysigline
{\sphinxbfcode{\sphinxupquote{xn}}}
\pysigstopsignatures
\sphinxAtStartPar
Attribute the fraction of neutrons.

\end{fulllineitems}

\index{xn\_bound (nucleardatapy.setup\_crust.SetupCrust attribute)@\spxentry{xn\_bound}\spxextra{nucleardatapy.setup\_crust.SetupCrust attribute}}

\begin{fulllineitems}
\phantomsection\label{\detokenize{source/api/setup_crust:nucleardatapy.setup_crust.SetupCrust.xn_bound}}
\pysigstartsignatures
\pysigline
{\sphinxbfcode{\sphinxupquote{xn\_bound}}}
\pysigstopsignatures
\sphinxAtStartPar
Attribute the fraction of bound neutrons.

\end{fulllineitems}

\index{xp (nucleardatapy.setup\_crust.SetupCrust attribute)@\spxentry{xp}\spxextra{nucleardatapy.setup\_crust.SetupCrust attribute}}

\begin{fulllineitems}
\phantomsection\label{\detokenize{source/api/setup_crust:nucleardatapy.setup_crust.SetupCrust.xp}}
\pysigstartsignatures
\pysigline
{\sphinxbfcode{\sphinxupquote{xp}}}
\pysigstopsignatures
\sphinxAtStartPar
Attribute the fraction of protons.

\end{fulllineitems}

\index{xpn\_bound (nucleardatapy.setup\_crust.SetupCrust attribute)@\spxentry{xpn\_bound}\spxextra{nucleardatapy.setup\_crust.SetupCrust attribute}}

\begin{fulllineitems}
\phantomsection\label{\detokenize{source/api/setup_crust:nucleardatapy.setup_crust.SetupCrust.xpn_bound}}
\pysigstartsignatures
\pysigline
{\sphinxbfcode{\sphinxupquote{xpn\_bound}}}
\pysigstopsignatures
\sphinxAtStartPar
Attribute the approximate ratio of proton to neutron in the nucleus.

\end{fulllineitems}


\end{fulllineitems}

\index{models\_crust() (in module nucleardatapy.setup\_crust)@\spxentry{models\_crust()}\spxextra{in module nucleardatapy.setup\_crust}}

\begin{fulllineitems}
\phantomsection\label{\detokenize{source/api/setup_crust:nucleardatapy.setup_crust.models_crust}}
\pysigstartsignatures
\pysiglinewithargsret
{\sphinxcode{\sphinxupquote{nucleardatapy.setup\_crust.}}\sphinxbfcode{\sphinxupquote{models\_crust}}}
{}
{}
\pysigstopsignatures
\sphinxAtStartPar
Return a list of the tables available in this toolkit for the experimental masses and
print them all on the prompt. These tables are the following
ones: ‘Negele\sphinxhyphen{}Vautheron\sphinxhyphen{}1973’.
\begin{quote}\begin{description}
\sphinxlineitem{Returns}
\sphinxAtStartPar
The list of tables.

\sphinxlineitem{Return type}
\sphinxAtStartPar
list{[}str{]}.

\end{description}\end{quote}

\end{fulllineitems}


\begin{figure}[htbp]
\centering
\capstart

\noindent\sphinxincludegraphics[scale=0.7]{{plot_SetupCrust}.png}
\caption{Properties of the crust as given by the models available in the nuda toolkit.}\label{\detokenize{source/api/setup_crust:id1}}\end{figure}

\sphinxstepscope


\section{SetupMassesExp}
\label{\detokenize{source/api/setup_masses_exp:setupmassesexp}}\label{\detokenize{source/api/setup_masses_exp::doc}}\index{module@\spxentry{module}!nucleardatapy.setup\_masses\_exp@\spxentry{nucleardatapy.setup\_masses\_exp}}\index{nucleardatapy.setup\_masses\_exp@\spxentry{nucleardatapy.setup\_masses\_exp}!module@\spxentry{module}}\index{SetupMassesExp (class in nucleardatapy.setup\_masses\_exp)@\spxentry{SetupMassesExp}\spxextra{class in nucleardatapy.setup\_masses\_exp}}\phantomsection\label{\detokenize{source/api/setup_masses_exp:module-nucleardatapy.setup_masses_exp}}

\begin{fulllineitems}
\phantomsection\label{\detokenize{source/api/setup_masses_exp:nucleardatapy.setup_masses_exp.SetupMassesExp}}
\pysigstartsignatures
\pysiglinewithargsret
{\sphinxbfcode{\sphinxupquote{class\DUrole{w}{ }}}\sphinxcode{\sphinxupquote{nucleardatapy.setup\_masses\_exp.}}\sphinxbfcode{\sphinxupquote{SetupMassesExp}}}
{\sphinxparam{\DUrole{n}{table}\DUrole{o}{=}\DUrole{default_value}{\textquotesingle{}AME\textquotesingle{}}}\sphinxparamcomma \sphinxparam{\DUrole{n}{version}\DUrole{o}{=}\DUrole{default_value}{\textquotesingle{}2020\textquotesingle{}}}}
{}
\pysigstopsignatures
\sphinxAtStartPar
Instantiate the experimental nuclear masses from AME mass table.

\sphinxAtStartPar
This choice is defined in the variables \sphinxtitleref{table} and \sphinxtitleref{version}.

\sphinxAtStartPar
\sphinxtitleref{table} can chosen among the following ones: ‘AME’.

\sphinxAtStartPar
\sphinxtitleref{version} can be chosen among the following choices: ‘2020’, ‘2016’, ‘2012’.
\begin{quote}\begin{description}
\sphinxlineitem{Parameters}\begin{itemize}
\item {} 
\sphinxAtStartPar
\sphinxstyleliteralstrong{\sphinxupquote{table}} (\sphinxstyleliteralemphasis{\sphinxupquote{str}}\sphinxstyleliteralemphasis{\sphinxupquote{, }}\sphinxstyleliteralemphasis{\sphinxupquote{optional.}}) \textendash{} Fix the name of \sphinxtitleref{table}. Default value: ‘AME’.

\item {} 
\sphinxAtStartPar
\sphinxstyleliteralstrong{\sphinxupquote{version}} (\sphinxstyleliteralemphasis{\sphinxupquote{str}}\sphinxstyleliteralemphasis{\sphinxupquote{, }}\sphinxstyleliteralemphasis{\sphinxupquote{optional.}}) \textendash{} Fix the name of \sphinxtitleref{version}. Default value: 2020’.

\end{itemize}

\end{description}\end{quote}

\sphinxAtStartPar
\sphinxstylestrong{Attributes:}
\index{Zmax (nucleardatapy.setup\_masses\_exp.SetupMassesExp attribute)@\spxentry{Zmax}\spxextra{nucleardatapy.setup\_masses\_exp.SetupMassesExp attribute}}

\begin{fulllineitems}
\phantomsection\label{\detokenize{source/api/setup_masses_exp:nucleardatapy.setup_masses_exp.SetupMassesExp.Zmax}}
\pysigstartsignatures
\pysigline
{\sphinxbfcode{\sphinxupquote{Zmax}}}
\pysigstopsignatures
\sphinxAtStartPar
maximum charge of nuclei present in the table.
\begin{quote}\begin{description}
\sphinxlineitem{Type}
\sphinxAtStartPar
Attribute Zmax

\end{description}\end{quote}

\end{fulllineitems}

\index{dist\_nbNuc (nucleardatapy.setup\_masses\_exp.SetupMassesExp attribute)@\spxentry{dist\_nbNuc}\spxextra{nucleardatapy.setup\_masses\_exp.SetupMassesExp attribute}}

\begin{fulllineitems}
\phantomsection\label{\detokenize{source/api/setup_masses_exp:nucleardatapy.setup_masses_exp.SetupMassesExp.dist_nbNuc}}
\pysigstartsignatures
\pysigline
{\sphinxbfcode{\sphinxupquote{dist\_nbNuc}}}
\pysigstopsignatures
\sphinxAtStartPar
attribute number of nuclei discovered per year

\end{fulllineitems}

\index{dist\_year (nucleardatapy.setup\_masses\_exp.SetupMassesExp attribute)@\spxentry{dist\_year}\spxextra{nucleardatapy.setup\_masses\_exp.SetupMassesExp attribute}}

\begin{fulllineitems}
\phantomsection\label{\detokenize{source/api/setup_masses_exp:nucleardatapy.setup_masses_exp.SetupMassesExp.dist_year}}
\pysigstartsignatures
\pysigline
{\sphinxbfcode{\sphinxupquote{dist\_year}}}
\pysigstopsignatures
\sphinxAtStartPar
attribute distribution of years

\end{fulllineitems}

\index{drip() (nucleardatapy.setup\_masses\_exp.SetupMassesExp method)@\spxentry{drip()}\spxextra{nucleardatapy.setup\_masses\_exp.SetupMassesExp method}}

\begin{fulllineitems}
\phantomsection\label{\detokenize{source/api/setup_masses_exp:nucleardatapy.setup_masses_exp.SetupMassesExp.drip}}
\pysigstartsignatures
\pysiglinewithargsret
{\sphinxbfcode{\sphinxupquote{drip}}}
{\sphinxparam{\DUrole{n}{Zmax}\DUrole{o}{=}\DUrole{default_value}{95}}}
{}
\pysigstopsignatures
\sphinxAtStartPar
Method which find the drip\sphinxhyphen{}line nuclei (on the two sides).
\begin{quote}\begin{description}
\sphinxlineitem{Parameters}
\sphinxAtStartPar
\sphinxstyleliteralstrong{\sphinxupquote{Zmax}} (\sphinxstyleliteralemphasis{\sphinxupquote{int}}\sphinxstyleliteralemphasis{\sphinxupquote{, }}\sphinxstyleliteralemphasis{\sphinxupquote{optional. Default: 95.}}) \textendash{} Fix the maximum charge for the search of the drip line.

\end{description}\end{quote}

\sphinxAtStartPar
\sphinxstylestrong{Attributes:}

\end{fulllineitems}

\index{flagI (nucleardatapy.setup\_masses\_exp.SetupMassesExp attribute)@\spxentry{flagI}\spxextra{nucleardatapy.setup\_masses\_exp.SetupMassesExp attribute}}

\begin{fulllineitems}
\phantomsection\label{\detokenize{source/api/setup_masses_exp:nucleardatapy.setup_masses_exp.SetupMassesExp.flagI}}
\pysigstartsignatures
\pysigline
{\sphinxbfcode{\sphinxupquote{flagI}}}
\pysigstopsignatures
\sphinxAtStartPar
Attribute I.

\end{fulllineitems}

\index{flagInterp (nucleardatapy.setup\_masses\_exp.SetupMassesExp attribute)@\spxentry{flagInterp}\spxextra{nucleardatapy.setup\_masses\_exp.SetupMassesExp attribute}}

\begin{fulllineitems}
\phantomsection\label{\detokenize{source/api/setup_masses_exp:nucleardatapy.setup_masses_exp.SetupMassesExp.flagInterp}}
\pysigstartsignatures
\pysigline
{\sphinxbfcode{\sphinxupquote{flagInterp}}}
\pysigstopsignatures
\sphinxAtStartPar
Attribute Interp (interpolation). Interp=’y’ is the nucleushas not been measured but is in the table based on interpolation expressions.otherwise Interp = ‘n’ for nuclei produced in laboratory and measured.

\end{fulllineitems}

\index{label (nucleardatapy.setup\_masses\_exp.SetupMassesExp attribute)@\spxentry{label}\spxextra{nucleardatapy.setup\_masses\_exp.SetupMassesExp attribute}}

\begin{fulllineitems}
\phantomsection\label{\detokenize{source/api/setup_masses_exp:nucleardatapy.setup_masses_exp.SetupMassesExp.label}}
\pysigstartsignatures
\pysigline
{\sphinxbfcode{\sphinxupquote{label}}}
\pysigstopsignatures
\sphinxAtStartPar
Attribute providing the label the data is references for figures.

\end{fulllineitems}

\index{nbLine (nucleardatapy.setup\_masses\_exp.SetupMassesExp attribute)@\spxentry{nbLine}\spxextra{nucleardatapy.setup\_masses\_exp.SetupMassesExp attribute}}

\begin{fulllineitems}
\phantomsection\label{\detokenize{source/api/setup_masses_exp:nucleardatapy.setup_masses_exp.SetupMassesExp.nbLine}}
\pysigstartsignatures
\pysigline
{\sphinxbfcode{\sphinxupquote{nbLine}}}
\pysigstopsignatures
\sphinxAtStartPar
Attribute with the number of line in the file.

\end{fulllineitems}

\index{nbNuc (nucleardatapy.setup\_masses\_exp.SetupMassesExp attribute)@\spxentry{nbNuc}\spxextra{nucleardatapy.setup\_masses\_exp.SetupMassesExp attribute}}

\begin{fulllineitems}
\phantomsection\label{\detokenize{source/api/setup_masses_exp:nucleardatapy.setup_masses_exp.SetupMassesExp.nbNuc}}
\pysigstartsignatures
\pysigline
{\sphinxbfcode{\sphinxupquote{nbNuc}}}
\pysigstopsignatures
\sphinxAtStartPar
Attribute with the number of nuclei read in the file.

\end{fulllineitems}

\index{note (nucleardatapy.setup\_masses\_exp.SetupMassesExp attribute)@\spxentry{note}\spxextra{nucleardatapy.setup\_masses\_exp.SetupMassesExp attribute}}

\begin{fulllineitems}
\phantomsection\label{\detokenize{source/api/setup_masses_exp:nucleardatapy.setup_masses_exp.SetupMassesExp.note}}
\pysigstartsignatures
\pysigline
{\sphinxbfcode{\sphinxupquote{note}}}
\pysigstopsignatures
\sphinxAtStartPar
Attribute providing additional notes about the data.

\end{fulllineitems}

\index{nucA (nucleardatapy.setup\_masses\_exp.SetupMassesExp attribute)@\spxentry{nucA}\spxextra{nucleardatapy.setup\_masses\_exp.SetupMassesExp attribute}}

\begin{fulllineitems}
\phantomsection\label{\detokenize{source/api/setup_masses_exp:nucleardatapy.setup_masses_exp.SetupMassesExp.nucA}}
\pysigstartsignatures
\pysigline
{\sphinxbfcode{\sphinxupquote{nucA}}}
\pysigstopsignatures
\sphinxAtStartPar
Attribute A (mass of the nucleus).

\end{fulllineitems}

\index{nucBE (nucleardatapy.setup\_masses\_exp.SetupMassesExp attribute)@\spxentry{nucBE}\spxextra{nucleardatapy.setup\_masses\_exp.SetupMassesExp attribute}}

\begin{fulllineitems}
\phantomsection\label{\detokenize{source/api/setup_masses_exp:nucleardatapy.setup_masses_exp.SetupMassesExp.nucBE}}
\pysigstartsignatures
\pysigline
{\sphinxbfcode{\sphinxupquote{nucBE}}}
\pysigstopsignatures
\sphinxAtStartPar
Attribute BE (Binding Energy) of the nucleus.

\end{fulllineitems}

\index{nucBE\_err (nucleardatapy.setup\_masses\_exp.SetupMassesExp attribute)@\spxentry{nucBE\_err}\spxextra{nucleardatapy.setup\_masses\_exp.SetupMassesExp attribute}}

\begin{fulllineitems}
\phantomsection\label{\detokenize{source/api/setup_masses_exp:nucleardatapy.setup_masses_exp.SetupMassesExp.nucBE_err}}
\pysigstartsignatures
\pysigline
{\sphinxbfcode{\sphinxupquote{nucBE\_err}}}
\pysigstopsignatures
\sphinxAtStartPar
Attribute uncertainty in the BE (Binding Energy) of the nucleus.

\end{fulllineitems}

\index{nucHT (nucleardatapy.setup\_masses\_exp.SetupMassesExp attribute)@\spxentry{nucHT}\spxextra{nucleardatapy.setup\_masses\_exp.SetupMassesExp attribute}}

\begin{fulllineitems}
\phantomsection\label{\detokenize{source/api/setup_masses_exp:nucleardatapy.setup_masses_exp.SetupMassesExp.nucHT}}
\pysigstartsignatures
\pysigline
{\sphinxbfcode{\sphinxupquote{nucHT}}}
\pysigstopsignatures
\sphinxAtStartPar
Attribute HT (half\sphinxhyphen{}Time) of the nucleus.

\end{fulllineitems}

\index{nucN (nucleardatapy.setup\_masses\_exp.SetupMassesExp attribute)@\spxentry{nucN}\spxextra{nucleardatapy.setup\_masses\_exp.SetupMassesExp attribute}}

\begin{fulllineitems}
\phantomsection\label{\detokenize{source/api/setup_masses_exp:nucleardatapy.setup_masses_exp.SetupMassesExp.nucN}}
\pysigstartsignatures
\pysigline
{\sphinxbfcode{\sphinxupquote{nucN}}}
\pysigstopsignatures
\sphinxAtStartPar
Attribute N (number of neutrons of the nucleus).

\end{fulllineitems}

\index{nucStbl (nucleardatapy.setup\_masses\_exp.SetupMassesExp attribute)@\spxentry{nucStbl}\spxextra{nucleardatapy.setup\_masses\_exp.SetupMassesExp attribute}}

\begin{fulllineitems}
\phantomsection\label{\detokenize{source/api/setup_masses_exp:nucleardatapy.setup_masses_exp.SetupMassesExp.nucStbl}}
\pysigstartsignatures
\pysigline
{\sphinxbfcode{\sphinxupquote{nucStbl}}}
\pysigstopsignatures
\sphinxAtStartPar
Attribute stbl. stbl=’y’ if the nucleus is stable (according to the table). Otherwise stbl = ‘n’.

\end{fulllineitems}

\index{nucSymb (nucleardatapy.setup\_masses\_exp.SetupMassesExp attribute)@\spxentry{nucSymb}\spxextra{nucleardatapy.setup\_masses\_exp.SetupMassesExp attribute}}

\begin{fulllineitems}
\phantomsection\label{\detokenize{source/api/setup_masses_exp:nucleardatapy.setup_masses_exp.SetupMassesExp.nucSymb}}
\pysigstartsignatures
\pysigline
{\sphinxbfcode{\sphinxupquote{nucSymb}}}
\pysigstopsignatures
\sphinxAtStartPar
Attribute symb (symbol) of the element, e.g., Fe.

\end{fulllineitems}

\index{nucYear (nucleardatapy.setup\_masses\_exp.SetupMassesExp attribute)@\spxentry{nucYear}\spxextra{nucleardatapy.setup\_masses\_exp.SetupMassesExp attribute}}

\begin{fulllineitems}
\phantomsection\label{\detokenize{source/api/setup_masses_exp:nucleardatapy.setup_masses_exp.SetupMassesExp.nucYear}}
\pysigstartsignatures
\pysigline
{\sphinxbfcode{\sphinxupquote{nucYear}}}
\pysigstopsignatures
\sphinxAtStartPar
Attribute year of the discovery of the nucleus.

\end{fulllineitems}

\index{nucZ (nucleardatapy.setup\_masses\_exp.SetupMassesExp attribute)@\spxentry{nucZ}\spxextra{nucleardatapy.setup\_masses\_exp.SetupMassesExp attribute}}

\begin{fulllineitems}
\phantomsection\label{\detokenize{source/api/setup_masses_exp:nucleardatapy.setup_masses_exp.SetupMassesExp.nucZ}}
\pysigstartsignatures
\pysigline
{\sphinxbfcode{\sphinxupquote{nucZ}}}
\pysigstopsignatures
\sphinxAtStartPar
Attribute Z (charge of the nucleus).

\end{fulllineitems}

\index{print\_outputs() (nucleardatapy.setup\_masses\_exp.SetupMassesExp method)@\spxentry{print\_outputs()}\spxextra{nucleardatapy.setup\_masses\_exp.SetupMassesExp method}}

\begin{fulllineitems}
\phantomsection\label{\detokenize{source/api/setup_masses_exp:nucleardatapy.setup_masses_exp.SetupMassesExp.print_outputs}}
\pysigstartsignatures
\pysiglinewithargsret
{\sphinxbfcode{\sphinxupquote{print\_outputs}}}
{}
{}
\pysigstopsignatures
\sphinxAtStartPar
Method which print outputs on terminal’s screen.

\end{fulllineitems}

\index{ref (nucleardatapy.setup\_masses\_exp.SetupMassesExp attribute)@\spxentry{ref}\spxextra{nucleardatapy.setup\_masses\_exp.SetupMassesExp attribute}}

\begin{fulllineitems}
\phantomsection\label{\detokenize{source/api/setup_masses_exp:nucleardatapy.setup_masses_exp.SetupMassesExp.ref}}
\pysigstartsignatures
\pysigline
{\sphinxbfcode{\sphinxupquote{ref}}}
\pysigstopsignatures
\sphinxAtStartPar
Attribute providing the full reference to the paper to be citted.

\end{fulllineitems}

\index{select() (nucleardatapy.setup\_masses\_exp.SetupMassesExp method)@\spxentry{select()}\spxextra{nucleardatapy.setup\_masses\_exp.SetupMassesExp method}}

\begin{fulllineitems}
\phantomsection\label{\detokenize{source/api/setup_masses_exp:nucleardatapy.setup_masses_exp.SetupMassesExp.select}}
\pysigstartsignatures
\pysiglinewithargsret
{\sphinxbfcode{\sphinxupquote{select}}}
{\sphinxparam{\DUrole{n}{Amin}\DUrole{o}{=}\DUrole{default_value}{0}}\sphinxparamcomma \sphinxparam{\DUrole{n}{Zmin}\DUrole{o}{=}\DUrole{default_value}{0}}\sphinxparamcomma \sphinxparam{\DUrole{n}{interp}\DUrole{o}{=}\DUrole{default_value}{\textquotesingle{}n\textquotesingle{}}}\sphinxparamcomma \sphinxparam{\DUrole{n}{state}\DUrole{o}{=}\DUrole{default_value}{\textquotesingle{}gs\textquotesingle{}}}\sphinxparamcomma \sphinxparam{\DUrole{n}{nucleus}\DUrole{o}{=}\DUrole{default_value}{\textquotesingle{}unstable\textquotesingle{}}}\sphinxparamcomma \sphinxparam{\DUrole{n}{every}\DUrole{o}{=}\DUrole{default_value}{1}}}
{}
\pysigstopsignatures
\sphinxAtStartPar
Method which select some nuclei from the table according to some criteria.
\begin{quote}\begin{description}
\sphinxlineitem{Parameters}\begin{itemize}
\item {} 
\sphinxAtStartPar
\sphinxstyleliteralstrong{\sphinxupquote{interp}} (\sphinxstyleliteralemphasis{\sphinxupquote{str}}\sphinxstyleliteralemphasis{\sphinxupquote{, }}\sphinxstyleliteralemphasis{\sphinxupquote{optional. Default = \textquotesingle{}n\textquotesingle{}.}}) \textendash{} If interp=’n’, exclude the interpolated nuclei from the selected ones.         If interp=’y’ consider them in the table, in addition to the others.

\item {} 
\sphinxAtStartPar
\sphinxstyleliteralstrong{\sphinxupquote{state}} (\sphinxstyleliteralemphasis{\sphinxupquote{str}}\sphinxstyleliteralemphasis{\sphinxupquote{, }}\sphinxstyleliteralemphasis{\sphinxupquote{optional. Default \textquotesingle{}gs\textquotesingle{}.}}) \textendash{} select the kind of state. If state=’gs’, select nuclei measured in their ground state.

\item {} 
\sphinxAtStartPar
\sphinxstyleliteralstrong{\sphinxupquote{nucleus}} (\sphinxstyleliteralemphasis{\sphinxupquote{str}}\sphinxstyleliteralemphasis{\sphinxupquote{, }}\sphinxstyleliteralemphasis{\sphinxupquote{optional. Default \textquotesingle{}unstable\textquotesingle{}.}}) \textendash{} ‘unstable’.

\end{itemize}

\end{description}\end{quote}

\sphinxAtStartPar
It can be set to ‘stable’, ‘longlive’ (with LT\textgreater{}10 min), ‘shortlive’ (with 10min\textgreater{}LT\textgreater{}1 ns),         ‘veryshortlive’ (with LT\textless{} 1ns)
:param every: consider only 1 out of \sphinxtitleref{every} nuclei in the table.
:type every: int, optional. Default every = 1.

\sphinxAtStartPar
\sphinxstylestrong{Attributes:}

\end{fulllineitems}

\index{select\_year() (nucleardatapy.setup\_masses\_exp.SetupMassesExp method)@\spxentry{select\_year()}\spxextra{nucleardatapy.setup\_masses\_exp.SetupMassesExp method}}

\begin{fulllineitems}
\phantomsection\label{\detokenize{source/api/setup_masses_exp:nucleardatapy.setup_masses_exp.SetupMassesExp.select_year}}
\pysigstartsignatures
\pysiglinewithargsret
{\sphinxbfcode{\sphinxupquote{select\_year}}}
{\sphinxparam{\DUrole{n}{year\_min}\DUrole{o}{=}\DUrole{default_value}{1940}}\sphinxparamcomma \sphinxparam{\DUrole{n}{year\_max}\DUrole{o}{=}\DUrole{default_value}{1960}}\sphinxparamcomma \sphinxparam{\DUrole{n}{state}\DUrole{o}{=}\DUrole{default_value}{\textquotesingle{}gs\textquotesingle{}}}}
{}
\pysigstopsignatures
\sphinxAtStartPar
Method which select some nuclei from the table according to the discovery year.
\begin{quote}\begin{description}
\sphinxlineitem{Parameters}\begin{itemize}
\item {} 
\sphinxAtStartPar
\sphinxstyleliteralstrong{\sphinxupquote{year\_min}}

\item {} 
\sphinxAtStartPar
\sphinxstyleliteralstrong{\sphinxupquote{year\_max}}

\item {} 
\sphinxAtStartPar
\sphinxstyleliteralstrong{\sphinxupquote{state}} (\sphinxstyleliteralemphasis{\sphinxupquote{str}}\sphinxstyleliteralemphasis{\sphinxupquote{, }}\sphinxstyleliteralemphasis{\sphinxupquote{optional. Default \textquotesingle{}gs\textquotesingle{}.}}) \textendash{} select the kind of state. If state=’gs’, select nuclei measured in their ground state.

\end{itemize}

\end{description}\end{quote}

\sphinxAtStartPar
\sphinxstylestrong{Attributes:}

\end{fulllineitems}


\end{fulllineitems}

\index{tables\_masses\_exp() (in module nucleardatapy.setup\_masses\_exp)@\spxentry{tables\_masses\_exp()}\spxextra{in module nucleardatapy.setup\_masses\_exp}}

\begin{fulllineitems}
\phantomsection\label{\detokenize{source/api/setup_masses_exp:nucleardatapy.setup_masses_exp.tables_masses_exp}}
\pysigstartsignatures
\pysiglinewithargsret
{\sphinxcode{\sphinxupquote{nucleardatapy.setup\_masses\_exp.}}\sphinxbfcode{\sphinxupquote{tables\_masses\_exp}}}
{}
{}
\pysigstopsignatures
\sphinxAtStartPar
Return a list of the tables available in this toolkit for the experimental masses and
print them all on the prompt. These tables are the following
ones: ‘AME’.
\begin{quote}\begin{description}
\sphinxlineitem{Returns}
\sphinxAtStartPar
The list of tables.

\sphinxlineitem{Return type}
\sphinxAtStartPar
list{[}str{]}.

\end{description}\end{quote}

\end{fulllineitems}

\index{versions\_masses\_exp() (in module nucleardatapy.setup\_masses\_exp)@\spxentry{versions\_masses\_exp()}\spxextra{in module nucleardatapy.setup\_masses\_exp}}

\begin{fulllineitems}
\phantomsection\label{\detokenize{source/api/setup_masses_exp:nucleardatapy.setup_masses_exp.versions_masses_exp}}
\pysigstartsignatures
\pysiglinewithargsret
{\sphinxcode{\sphinxupquote{nucleardatapy.setup\_masses\_exp.}}\sphinxbfcode{\sphinxupquote{versions\_masses\_exp}}}
{\sphinxparam{\DUrole{n}{table}}}
{}
\pysigstopsignatures
\sphinxAtStartPar
Return a list of versions of tables available in
this toolkit for a given model and print them all on the prompt.
\begin{quote}\begin{description}
\sphinxlineitem{Parameters}
\sphinxAtStartPar
\sphinxstyleliteralstrong{\sphinxupquote{table}} (\sphinxstyleliteralemphasis{\sphinxupquote{str.}}) \textendash{} The table for which there are different versions.

\sphinxlineitem{Returns}
\sphinxAtStartPar
The list of versions.     If table == ‘AME’: ‘2020’, ‘2016’, ‘2012’.

\sphinxlineitem{Return type}
\sphinxAtStartPar
list{[}str{]}.

\end{description}\end{quote}

\end{fulllineitems}


\sphinxAtStartPar
Here are a set of figures which are produced with the Python sample: /sample/nucleardatapy\_plots/plot\_setupMassesExp.py

\begin{figure}[htbp]
\centering
\capstart

\noindent\sphinxincludegraphics[scale=0.7]{{plot_SetupMassesExp_AME_2020}.png}
\caption{The nuclear chart based on AME 2020 table. The different colors correspond to the different measured half\sphinxhyphen{}times of nuclei.}\label{\detokenize{source/api/setup_masses_exp:id1}}\end{figure}

\begin{figure}[htbp]
\centering
\capstart

\noindent\sphinxincludegraphics[scale=0.7]{{plot_SetupMassesExp_AME_2020_year}.png}
\caption{Histogram showing the distribution of nuclei per discovery year, since the first one discovered in 1851.}\label{\detokenize{source/api/setup_masses_exp:id2}}\end{figure}

\sphinxstepscope


\section{SetupMassesTheory}
\label{\detokenize{source/api/setup_masses_theory:setupmassestheory}}\label{\detokenize{source/api/setup_masses_theory::doc}}\index{module@\spxentry{module}!nucleardatapy.setup\_masses\_theory@\spxentry{nucleardatapy.setup\_masses\_theory}}\index{nucleardatapy.setup\_masses\_theory@\spxentry{nucleardatapy.setup\_masses\_theory}!module@\spxentry{module}}\index{SetupMassesTheory (class in nucleardatapy.setup\_masses\_theory)@\spxentry{SetupMassesTheory}\spxextra{class in nucleardatapy.setup\_masses\_theory}}\phantomsection\label{\detokenize{source/api/setup_masses_theory:module-nucleardatapy.setup_masses_theory}}

\begin{fulllineitems}
\phantomsection\label{\detokenize{source/api/setup_masses_theory:nucleardatapy.setup_masses_theory.SetupMassesTheory}}
\pysigstartsignatures
\pysiglinewithargsret
{\sphinxbfcode{\sphinxupquote{class\DUrole{w}{ }}}\sphinxcode{\sphinxupquote{nucleardatapy.setup\_masses\_theory.}}\sphinxbfcode{\sphinxupquote{SetupMassesTheory}}}
{\sphinxparam{\DUrole{n}{table}\DUrole{o}{=}\DUrole{default_value}{\textquotesingle{}1995\sphinxhyphen{}DZ\textquotesingle{}}}}
{}
\pysigstopsignatures
\sphinxAtStartPar
Instantiate the theory nuclear masses.

\sphinxAtStartPar
This choice is defined in the variable \sphinxtitleref{table}.

\sphinxAtStartPar
\sphinxtitleref{table} can chosen among the following ones:     {[} ‘1988\sphinxhyphen{}MJ’, ‘1995\sphinxhyphen{}DZ’, ‘1995\sphinxhyphen{}ETFSI’, ‘1995\sphinxhyphen{}FRDM’,     ‘2005\sphinxhyphen{}KTUY’, ‘2007\sphinxhyphen{}HFB14’, ‘2010\sphinxhyphen{}WS3’, ‘2010\sphinxhyphen{}HFB21’,’2011\sphinxhyphen{}WS3’, ‘2013\sphinxhyphen{}HFB26’ {]}
\begin{quote}\begin{description}
\sphinxlineitem{Parameters}
\sphinxAtStartPar
\sphinxstyleliteralstrong{\sphinxupquote{table}} (\sphinxstyleliteralemphasis{\sphinxupquote{str}}\sphinxstyleliteralemphasis{\sphinxupquote{, }}\sphinxstyleliteralemphasis{\sphinxupquote{optional.}}) \textendash{} Fix the name of \sphinxtitleref{table}. Default value: ‘1995\sphinxhyphen{}DZ’.

\end{description}\end{quote}

\sphinxAtStartPar
\sphinxstylestrong{Attributes:}
\index{diff() (nucleardatapy.setup\_masses\_theory.SetupMassesTheory method)@\spxentry{diff()}\spxextra{nucleardatapy.setup\_masses\_theory.SetupMassesTheory method}}

\begin{fulllineitems}
\phantomsection\label{\detokenize{source/api/setup_masses_theory:nucleardatapy.setup_masses_theory.SetupMassesTheory.diff}}
\pysigstartsignatures
\pysiglinewithargsret
{\sphinxbfcode{\sphinxupquote{diff}}}
{\sphinxparam{\DUrole{n}{table}}\sphinxparamcomma \sphinxparam{\DUrole{n}{Zref}\DUrole{o}{=}\DUrole{default_value}{50}}}
{}
\pysigstopsignatures
\sphinxAtStartPar
Method calculates the difference between a given mass
model and table\_ref.
\begin{quote}\begin{description}
\sphinxlineitem{Parameters}\begin{itemize}
\item {} 
\sphinxAtStartPar
\sphinxstyleliteralstrong{\sphinxupquote{table}} (\sphinxstyleliteralemphasis{\sphinxupquote{str.}}) \textendash{} Fix the table to analyze.

\item {} 
\sphinxAtStartPar
\sphinxstyleliteralstrong{\sphinxupquote{Zref}} (\sphinxstyleliteralemphasis{\sphinxupquote{int}}\sphinxstyleliteralemphasis{\sphinxupquote{, }}\sphinxstyleliteralemphasis{\sphinxupquote{optional. Default: 50.}}) \textendash{} Fix the isotopic chain to study.

\end{itemize}

\end{description}\end{quote}

\sphinxAtStartPar
\sphinxstylestrong{Attributes:}

\end{fulllineitems}

\index{diff\_exp() (nucleardatapy.setup\_masses\_theory.SetupMassesTheory method)@\spxentry{diff\_exp()}\spxextra{nucleardatapy.setup\_masses\_theory.SetupMassesTheory method}}

\begin{fulllineitems}
\phantomsection\label{\detokenize{source/api/setup_masses_theory:nucleardatapy.setup_masses_theory.SetupMassesTheory.diff_exp}}
\pysigstartsignatures
\pysiglinewithargsret
{\sphinxbfcode{\sphinxupquote{diff\_exp}}}
{\sphinxparam{\DUrole{n}{table\_exp}}\sphinxparamcomma \sphinxparam{\DUrole{n}{version\_exp}}\sphinxparamcomma \sphinxparam{\DUrole{n}{Zref}\DUrole{o}{=}\DUrole{default_value}{50}}}
{}
\pysigstopsignatures
\sphinxAtStartPar
Method calculates the difference between a given experimental
mass (identified by \sphinxtitleref{table\_exp} and \sphinxtitleref{version\_exp}) and table\_ref.
\begin{quote}\begin{description}
\sphinxlineitem{Parameters}\begin{itemize}
\item {} 
\sphinxAtStartPar
\sphinxstyleliteralstrong{\sphinxupquote{table}} (\sphinxstyleliteralemphasis{\sphinxupquote{str.}}) \textendash{} Fix the table to analyze.

\item {} 
\sphinxAtStartPar
\sphinxstyleliteralstrong{\sphinxupquote{Zref}} (\sphinxstyleliteralemphasis{\sphinxupquote{int}}\sphinxstyleliteralemphasis{\sphinxupquote{, }}\sphinxstyleliteralemphasis{\sphinxupquote{optional. Default: 50.}}) \textendash{} Fix the isotopic chain to study.

\end{itemize}

\end{description}\end{quote}

\sphinxAtStartPar
\sphinxstylestrong{Attributes:}

\end{fulllineitems}

\index{drip() (nucleardatapy.setup\_masses\_theory.SetupMassesTheory method)@\spxentry{drip()}\spxextra{nucleardatapy.setup\_masses\_theory.SetupMassesTheory method}}

\begin{fulllineitems}
\phantomsection\label{\detokenize{source/api/setup_masses_theory:nucleardatapy.setup_masses_theory.SetupMassesTheory.drip}}
\pysigstartsignatures
\pysiglinewithargsret
{\sphinxbfcode{\sphinxupquote{drip}}}
{\sphinxparam{\DUrole{n}{Zmax}\DUrole{o}{=}\DUrole{default_value}{95}}}
{}
\pysigstopsignatures
\sphinxAtStartPar
Method which find the drip\sphinxhyphen{}line nuclei (on the two sides).
\begin{quote}\begin{description}
\sphinxlineitem{Parameters}
\sphinxAtStartPar
\sphinxstyleliteralstrong{\sphinxupquote{Zmax}} (\sphinxstyleliteralemphasis{\sphinxupquote{int}}\sphinxstyleliteralemphasis{\sphinxupquote{, }}\sphinxstyleliteralemphasis{\sphinxupquote{optional. Default: 95.}}) \textendash{} Fix the maximum charge for the search of the drip line.

\end{description}\end{quote}

\sphinxAtStartPar
\sphinxstylestrong{Attributes:}

\end{fulllineitems}

\index{init\_self() (nucleardatapy.setup\_masses\_theory.SetupMassesTheory method)@\spxentry{init\_self()}\spxextra{nucleardatapy.setup\_masses\_theory.SetupMassesTheory method}}

\begin{fulllineitems}
\phantomsection\label{\detokenize{source/api/setup_masses_theory:nucleardatapy.setup_masses_theory.SetupMassesTheory.init_self}}
\pysigstartsignatures
\pysiglinewithargsret
{\sphinxbfcode{\sphinxupquote{init\_self}}}
{}
{}
\pysigstopsignatures
\sphinxAtStartPar
Initialize variables in self.

\end{fulllineitems}

\index{print\_outputs() (nucleardatapy.setup\_masses\_theory.SetupMassesTheory method)@\spxentry{print\_outputs()}\spxextra{nucleardatapy.setup\_masses\_theory.SetupMassesTheory method}}

\begin{fulllineitems}
\phantomsection\label{\detokenize{source/api/setup_masses_theory:nucleardatapy.setup_masses_theory.SetupMassesTheory.print_outputs}}
\pysigstartsignatures
\pysiglinewithargsret
{\sphinxbfcode{\sphinxupquote{print\_outputs}}}
{}
{}
\pysigstopsignatures
\sphinxAtStartPar
Method which print outputs on terminal’s screen.

\end{fulllineitems}


\end{fulllineitems}

\index{tables\_masses\_theory() (in module nucleardatapy.setup\_masses\_theory)@\spxentry{tables\_masses\_theory()}\spxextra{in module nucleardatapy.setup\_masses\_theory}}

\begin{fulllineitems}
\phantomsection\label{\detokenize{source/api/setup_masses_theory:nucleardatapy.setup_masses_theory.tables_masses_theory}}
\pysigstartsignatures
\pysiglinewithargsret
{\sphinxcode{\sphinxupquote{nucleardatapy.setup\_masses\_theory.}}\sphinxbfcode{\sphinxupquote{tables\_masses\_theory}}}
{}
{}
\pysigstopsignatures
\sphinxAtStartPar
Return a list of the tables available in this toolkit for the masses
predicted by theoretical approaches and print them all on the prompt.
These tables are the following ones:     {[} ‘1988\sphinxhyphen{}MJ’, ‘1995\sphinxhyphen{}DZ’, ‘1995\sphinxhyphen{}ETFSI’, ‘1995\sphinxhyphen{}FRDM’,     ‘2005\sphinxhyphen{}KTUY’, ‘2007\sphinxhyphen{}HFB14’, ‘2010\sphinxhyphen{}WS3’, ‘2010\sphinxhyphen{}HFB21’, ‘2011\sphinxhyphen{}WS3’, ‘2013\sphinxhyphen{}HFB26’ {]}
\begin{quote}\begin{description}
\sphinxlineitem{Returns}
\sphinxAtStartPar
The list of tables.

\sphinxlineitem{Return type}
\sphinxAtStartPar
list{[}str{]}.

\end{description}\end{quote}

\end{fulllineitems}


\sphinxAtStartPar
Here are a set of figures which are produced with the Python sample: /sample/nucleardatapy\_plots/plot\_setupMassesTheory.py

\begin{figure}[htbp]
\centering
\capstart

\noindent\sphinxincludegraphics[scale=0.7]{{plot_SetupMassesTheory_Zref20}.png}
\caption{Differences between binding energies predicted by different models with respect to the one predicted by Duflo\sphinxhyphen{}Zuker for Z = 20.}\label{\detokenize{source/api/setup_masses_theory:id1}}\end{figure}

\begin{figure}[htbp]
\centering
\capstart

\noindent\sphinxincludegraphics[scale=0.7]{{plot_SetupMassesTheory_Zref50}.png}
\caption{Differences between binding energies predicted by different models with respect to the one predicted by Duflo\sphinxhyphen{}Zuker for Z = 50.}\label{\detokenize{source/api/setup_masses_theory:id2}}\end{figure}

\sphinxstepscope


\section{SetupRadCh}
\label{\detokenize{source/api/setup_rad_ch:setupradch}}\label{\detokenize{source/api/setup_rad_ch::doc}}\index{module@\spxentry{module}!nucleardatapy.setup\_rad\_ch@\spxentry{nucleardatapy.setup\_rad\_ch}}\index{nucleardatapy.setup\_rad\_ch@\spxentry{nucleardatapy.setup\_rad\_ch}!module@\spxentry{module}}\index{SetupRadCh (class in nucleardatapy.setup\_rad\_ch)@\spxentry{SetupRadCh}\spxextra{class in nucleardatapy.setup\_rad\_ch}}\phantomsection\label{\detokenize{source/api/setup_rad_ch:module-nucleardatapy.setup_rad_ch}}

\begin{fulllineitems}
\phantomsection\label{\detokenize{source/api/setup_rad_ch:nucleardatapy.setup_rad_ch.SetupRadCh}}
\pysigstartsignatures
\pysiglinewithargsret
{\sphinxbfcode{\sphinxupquote{class\DUrole{w}{ }}}\sphinxcode{\sphinxupquote{nucleardatapy.setup\_rad\_ch.}}\sphinxbfcode{\sphinxupquote{SetupRadCh}}}
{\sphinxparam{\DUrole{n}{table}\DUrole{o}{=}\DUrole{default_value}{\textquotesingle{}2013\sphinxhyphen{}Angeli\textquotesingle{}}}}
{}
\pysigstopsignatures
\sphinxAtStartPar
Instantiate the object with charge radii choosen    from a table.

\sphinxAtStartPar
This choice is defined in the variable \sphinxtitleref{table}.

\sphinxAtStartPar
The tables can chosen among the following ones:    ‘2013\sphinxhyphen{}Angeli’.
\begin{quote}\begin{description}
\sphinxlineitem{Parameters}
\sphinxAtStartPar
\sphinxstyleliteralstrong{\sphinxupquote{table}} (\sphinxstyleliteralemphasis{\sphinxupquote{str}}\sphinxstyleliteralemphasis{\sphinxupquote{, }}\sphinxstyleliteralemphasis{\sphinxupquote{optional.}}) \textendash{} Fix the name of \sphinxtitleref{table}. Default value: ‘2013\sphinxhyphen{}Angeli’.

\end{description}\end{quote}

\sphinxAtStartPar
\sphinxstylestrong{Attributes:}
\index{R\_unit (nucleardatapy.setup\_rad\_ch.SetupRadCh attribute)@\spxentry{R\_unit}\spxextra{nucleardatapy.setup\_rad\_ch.SetupRadCh attribute}}

\begin{fulllineitems}
\phantomsection\label{\detokenize{source/api/setup_rad_ch:nucleardatapy.setup_rad_ch.SetupRadCh.R_unit}}
\pysigstartsignatures
\pysigline
{\sphinxbfcode{\sphinxupquote{R\_unit}}}
\pysigstopsignatures
\sphinxAtStartPar
Attribute radius unit.

\end{fulllineitems}

\index{RadCh\_isotopes() (nucleardatapy.setup\_rad\_ch.SetupRadCh method)@\spxentry{RadCh\_isotopes()}\spxextra{nucleardatapy.setup\_rad\_ch.SetupRadCh method}}

\begin{fulllineitems}
\phantomsection\label{\detokenize{source/api/setup_rad_ch:nucleardatapy.setup_rad_ch.SetupRadCh.RadCh_isotopes}}
\pysigstartsignatures
\pysiglinewithargsret
{\sphinxbfcode{\sphinxupquote{RadCh\_isotopes}}}
{\sphinxparam{\DUrole{n}{Zref}\DUrole{o}{=}\DUrole{default_value}{50}}}
{}
\pysigstopsignatures
\sphinxAtStartPar
This method provide a list if radii for an isotopic chain defined by Zref.

\end{fulllineitems}

\index{label (nucleardatapy.setup\_rad\_ch.SetupRadCh attribute)@\spxentry{label}\spxextra{nucleardatapy.setup\_rad\_ch.SetupRadCh attribute}}

\begin{fulllineitems}
\phantomsection\label{\detokenize{source/api/setup_rad_ch:nucleardatapy.setup_rad_ch.SetupRadCh.label}}
\pysigstartsignatures
\pysigline
{\sphinxbfcode{\sphinxupquote{label}}}
\pysigstopsignatures
\sphinxAtStartPar
Attribute providing the label the data is references for figures.

\end{fulllineitems}

\index{note (nucleardatapy.setup\_rad\_ch.SetupRadCh attribute)@\spxentry{note}\spxextra{nucleardatapy.setup\_rad\_ch.SetupRadCh attribute}}

\begin{fulllineitems}
\phantomsection\label{\detokenize{source/api/setup_rad_ch:nucleardatapy.setup_rad_ch.SetupRadCh.note}}
\pysigstartsignatures
\pysigline
{\sphinxbfcode{\sphinxupquote{note}}}
\pysigstopsignatures
\sphinxAtStartPar
Attribute providing additional notes about the data.

\end{fulllineitems}

\index{nucA (nucleardatapy.setup\_rad\_ch.SetupRadCh attribute)@\spxentry{nucA}\spxextra{nucleardatapy.setup\_rad\_ch.SetupRadCh attribute}}

\begin{fulllineitems}
\phantomsection\label{\detokenize{source/api/setup_rad_ch:nucleardatapy.setup_rad_ch.SetupRadCh.nucA}}
\pysigstartsignatures
\pysigline
{\sphinxbfcode{\sphinxupquote{nucA}}}
\pysigstopsignatures
\sphinxAtStartPar
Attribute A (mass of the nucleus).

\end{fulllineitems}

\index{nucN (nucleardatapy.setup\_rad\_ch.SetupRadCh attribute)@\spxentry{nucN}\spxextra{nucleardatapy.setup\_rad\_ch.SetupRadCh attribute}}

\begin{fulllineitems}
\phantomsection\label{\detokenize{source/api/setup_rad_ch:nucleardatapy.setup_rad_ch.SetupRadCh.nucN}}
\pysigstartsignatures
\pysigline
{\sphinxbfcode{\sphinxupquote{nucN}}}
\pysigstopsignatures
\sphinxAtStartPar
Attribute N (number of neutrons of the nucleus).

\end{fulllineitems}

\index{nucRch (nucleardatapy.setup\_rad\_ch.SetupRadCh attribute)@\spxentry{nucRch}\spxextra{nucleardatapy.setup\_rad\_ch.SetupRadCh attribute}}

\begin{fulllineitems}
\phantomsection\label{\detokenize{source/api/setup_rad_ch:nucleardatapy.setup_rad_ch.SetupRadCh.nucRch}}
\pysigstartsignatures
\pysigline
{\sphinxbfcode{\sphinxupquote{nucRch}}}
\pysigstopsignatures
\sphinxAtStartPar
Attribue R\_ch (charge radius) in fm.

\end{fulllineitems}

\index{nucRch\_err (nucleardatapy.setup\_rad\_ch.SetupRadCh attribute)@\spxentry{nucRch\_err}\spxextra{nucleardatapy.setup\_rad\_ch.SetupRadCh attribute}}

\begin{fulllineitems}
\phantomsection\label{\detokenize{source/api/setup_rad_ch:nucleardatapy.setup_rad_ch.SetupRadCh.nucRch_err}}
\pysigstartsignatures
\pysigline
{\sphinxbfcode{\sphinxupquote{nucRch\_err}}}
\pysigstopsignatures
\sphinxAtStartPar
Attribue uncertainty in R\_ch (charge radius) in fm.

\end{fulllineitems}

\index{nucSymb (nucleardatapy.setup\_rad\_ch.SetupRadCh attribute)@\spxentry{nucSymb}\spxextra{nucleardatapy.setup\_rad\_ch.SetupRadCh attribute}}

\begin{fulllineitems}
\phantomsection\label{\detokenize{source/api/setup_rad_ch:nucleardatapy.setup_rad_ch.SetupRadCh.nucSymb}}
\pysigstartsignatures
\pysigline
{\sphinxbfcode{\sphinxupquote{nucSymb}}}
\pysigstopsignatures
\sphinxAtStartPar
Attribute symb (symbol) of the element, e.g., Fe.

\end{fulllineitems}

\index{nucZ (nucleardatapy.setup\_rad\_ch.SetupRadCh attribute)@\spxentry{nucZ}\spxextra{nucleardatapy.setup\_rad\_ch.SetupRadCh attribute}}

\begin{fulllineitems}
\phantomsection\label{\detokenize{source/api/setup_rad_ch:nucleardatapy.setup_rad_ch.SetupRadCh.nucZ}}
\pysigstartsignatures
\pysigline
{\sphinxbfcode{\sphinxupquote{nucZ}}}
\pysigstopsignatures
\sphinxAtStartPar
Attribute Z (charge of the nucleus).

\end{fulllineitems}

\index{print\_outputs() (nucleardatapy.setup\_rad\_ch.SetupRadCh method)@\spxentry{print\_outputs()}\spxextra{nucleardatapy.setup\_rad\_ch.SetupRadCh method}}

\begin{fulllineitems}
\phantomsection\label{\detokenize{source/api/setup_rad_ch:nucleardatapy.setup_rad_ch.SetupRadCh.print_outputs}}
\pysigstartsignatures
\pysiglinewithargsret
{\sphinxbfcode{\sphinxupquote{print\_outputs}}}
{}
{}
\pysigstopsignatures
\sphinxAtStartPar
Method which print outputs on terminal’s screen.

\end{fulllineitems}

\index{ref (nucleardatapy.setup\_rad\_ch.SetupRadCh attribute)@\spxentry{ref}\spxextra{nucleardatapy.setup\_rad\_ch.SetupRadCh attribute}}

\begin{fulllineitems}
\phantomsection\label{\detokenize{source/api/setup_rad_ch:nucleardatapy.setup_rad_ch.SetupRadCh.ref}}
\pysigstartsignatures
\pysigline
{\sphinxbfcode{\sphinxupquote{ref}}}
\pysigstopsignatures
\sphinxAtStartPar
Attribute providing the full reference to the paper to be citted.

\end{fulllineitems}


\end{fulllineitems}

\index{tables\_rad\_ch() (in module nucleardatapy.setup\_rad\_ch)@\spxentry{tables\_rad\_ch()}\spxextra{in module nucleardatapy.setup\_rad\_ch}}

\begin{fulllineitems}
\phantomsection\label{\detokenize{source/api/setup_rad_ch:nucleardatapy.setup_rad_ch.tables_rad_ch}}
\pysigstartsignatures
\pysiglinewithargsret
{\sphinxcode{\sphinxupquote{nucleardatapy.setup\_rad\_ch.}}\sphinxbfcode{\sphinxupquote{tables\_rad\_ch}}}
{}
{}
\pysigstopsignatures
\sphinxAtStartPar
Return a list of the tables available in this toolkit for the charge radiuus and
print them all on the prompt.  These tables are the following
ones: ‘2013\sphinxhyphen{}Angeli’.
\begin{quote}\begin{description}
\sphinxlineitem{Returns}
\sphinxAtStartPar
The list of tables.

\sphinxlineitem{Return type}
\sphinxAtStartPar
list{[}str{]}.

\end{description}\end{quote}

\end{fulllineitems}


\begin{figure}[htbp]
\centering
\capstart

\noindent\sphinxincludegraphics[scale=0.7]{{plot_SetupRadCh}.png}
\caption{Charge radii for Zn, Sn, and Pb isotopes and for the models available in the nuda toolkit.}\label{\detokenize{source/api/setup_rad_ch:id1}}\end{figure}

\sphinxstepscope


\section{SetupISGMR}
\label{\detokenize{source/api/setup_ISGMR:setupisgmr}}\label{\detokenize{source/api/setup_ISGMR::doc}}\index{module@\spxentry{module}!nucleardatapy.setup\_ISGMR@\spxentry{nucleardatapy.setup\_ISGMR}}\index{nucleardatapy.setup\_ISGMR@\spxentry{nucleardatapy.setup\_ISGMR}!module@\spxentry{module}}\index{SetupISGMR (class in nucleardatapy.setup\_ISGMR)@\spxentry{SetupISGMR}\spxextra{class in nucleardatapy.setup\_ISGMR}}\phantomsection\label{\detokenize{source/api/setup_ISGMR:module-nucleardatapy.setup_ISGMR}}

\begin{fulllineitems}
\phantomsection\label{\detokenize{source/api/setup_ISGMR:nucleardatapy.setup_ISGMR.SetupISGMR}}
\pysigstartsignatures
\pysiglinewithargsret
{\sphinxbfcode{\sphinxupquote{class\DUrole{w}{ }}}\sphinxcode{\sphinxupquote{nucleardatapy.setup\_ISGMR.}}\sphinxbfcode{\sphinxupquote{SetupISGMR}}}
{\sphinxparam{\DUrole{n}{table}\DUrole{o}{=}\DUrole{default_value}{\textquotesingle{}2018\sphinxhyphen{}ISGMR\sphinxhyphen{}GARG\textquotesingle{}}}}
{}
\pysigstopsignatures
\sphinxAtStartPar
Instantiate the object with microscopic results choosen    by the toolkit practitioner.
This choice is defined in the variable \sphinxtitleref{table}.

\sphinxAtStartPar
The \sphinxtitleref{table} can chosen among the following ones:    ‘2010\sphinxhyphen{}ISGMR\sphinxhyphen{}LI’, ‘2018\sphinxhyphen{}ISGMR\sphinxhyphen{}GARG’.
\begin{quote}\begin{description}
\sphinxlineitem{Parameters}
\sphinxAtStartPar
\sphinxstyleliteralstrong{\sphinxupquote{table}} (\sphinxstyleliteralemphasis{\sphinxupquote{str}}\sphinxstyleliteralemphasis{\sphinxupquote{, }}\sphinxstyleliteralemphasis{\sphinxupquote{optional.}}) \textendash{} Fix the name of \sphinxtitleref{table}. Default value: ‘2018\sphinxhyphen{}ISGMR\sphinxhyphen{}GARG’, ‘2018\sphinxhyphen{}ISGMR\sphinxhyphen{}GARG\sphinxhyphen{}LATEX’.

\end{description}\end{quote}

\sphinxAtStartPar
\sphinxstylestrong{Attributes:}
\index{E\_unit (nucleardatapy.setup\_ISGMR.SetupISGMR attribute)@\spxentry{E\_unit}\spxextra{nucleardatapy.setup\_ISGMR.SetupISGMR attribute}}

\begin{fulllineitems}
\phantomsection\label{\detokenize{source/api/setup_ISGMR:nucleardatapy.setup_ISGMR.SetupISGMR.E_unit}}
\pysigstartsignatures
\pysigline
{\sphinxbfcode{\sphinxupquote{E\_unit}}}
\pysigstopsignatures
\sphinxAtStartPar
Attribute energy unit.

\end{fulllineitems}

\index{label (nucleardatapy.setup\_ISGMR.SetupISGMR attribute)@\spxentry{label}\spxextra{nucleardatapy.setup\_ISGMR.SetupISGMR attribute}}

\begin{fulllineitems}
\phantomsection\label{\detokenize{source/api/setup_ISGMR:nucleardatapy.setup_ISGMR.SetupISGMR.label}}
\pysigstartsignatures
\pysigline
{\sphinxbfcode{\sphinxupquote{label}}}
\pysigstopsignatures
\sphinxAtStartPar
Attribute providing the label the data is references for figures.

\end{fulllineitems}

\index{note (nucleardatapy.setup\_ISGMR.SetupISGMR attribute)@\spxentry{note}\spxextra{nucleardatapy.setup\_ISGMR.SetupISGMR attribute}}

\begin{fulllineitems}
\phantomsection\label{\detokenize{source/api/setup_ISGMR:nucleardatapy.setup_ISGMR.SetupISGMR.note}}
\pysigstartsignatures
\pysigline
{\sphinxbfcode{\sphinxupquote{note}}}
\pysigstopsignatures
\sphinxAtStartPar
Attribute providing additional notes about the data.

\end{fulllineitems}

\index{nucA (nucleardatapy.setup\_ISGMR.SetupISGMR attribute)@\spxentry{nucA}\spxextra{nucleardatapy.setup\_ISGMR.SetupISGMR attribute}}

\begin{fulllineitems}
\phantomsection\label{\detokenize{source/api/setup_ISGMR:nucleardatapy.setup_ISGMR.SetupISGMR.nucA}}
\pysigstartsignatures
\pysigline
{\sphinxbfcode{\sphinxupquote{nucA}}}
\pysigstopsignatures
\sphinxAtStartPar
Attribute A (mass of the nucleus).

\end{fulllineitems}

\index{nucM12Mm1\_cent (nucleardatapy.setup\_ISGMR.SetupISGMR attribute)@\spxentry{nucM12Mm1\_cent}\spxextra{nucleardatapy.setup\_ISGMR.SetupISGMR attribute}}

\begin{fulllineitems}
\phantomsection\label{\detokenize{source/api/setup_ISGMR:nucleardatapy.setup_ISGMR.SetupISGMR.nucM12Mm1_cent}}
\pysigstartsignatures
\pysigline
{\sphinxbfcode{\sphinxupquote{nucM12Mm1\_cent}}}
\pysigstopsignatures
\sphinxAtStartPar
Attribute energy centroid.

\end{fulllineitems}

\index{nucM12Mm1\_errm (nucleardatapy.setup\_ISGMR.SetupISGMR attribute)@\spxentry{nucM12Mm1\_errm}\spxextra{nucleardatapy.setup\_ISGMR.SetupISGMR attribute}}

\begin{fulllineitems}
\phantomsection\label{\detokenize{source/api/setup_ISGMR:nucleardatapy.setup_ISGMR.SetupISGMR.nucM12Mm1_errm}}
\pysigstartsignatures
\pysigline
{\sphinxbfcode{\sphinxupquote{nucM12Mm1\_errm}}}
\pysigstopsignatures
\sphinxAtStartPar
Attribute (\sphinxhyphen{}) uncertainty in the energy centroid.

\end{fulllineitems}

\index{nucM12Mm1\_errp (nucleardatapy.setup\_ISGMR.SetupISGMR attribute)@\spxentry{nucM12Mm1\_errp}\spxextra{nucleardatapy.setup\_ISGMR.SetupISGMR attribute}}

\begin{fulllineitems}
\phantomsection\label{\detokenize{source/api/setup_ISGMR:nucleardatapy.setup_ISGMR.SetupISGMR.nucM12Mm1_errp}}
\pysigstartsignatures
\pysigline
{\sphinxbfcode{\sphinxupquote{nucM12Mm1\_errp}}}
\pysigstopsignatures
\sphinxAtStartPar
Attribute (+) uncertainty in the energy centroid.

\end{fulllineitems}

\index{nucSymbol (nucleardatapy.setup\_ISGMR.SetupISGMR attribute)@\spxentry{nucSymbol}\spxextra{nucleardatapy.setup\_ISGMR.SetupISGMR attribute}}

\begin{fulllineitems}
\phantomsection\label{\detokenize{source/api/setup_ISGMR:nucleardatapy.setup_ISGMR.SetupISGMR.nucSymbol}}
\pysigstartsignatures
\pysigline
{\sphinxbfcode{\sphinxupquote{nucSymbol}}}
\pysigstopsignatures
\sphinxAtStartPar
Attribute the symbol of the element.

\end{fulllineitems}

\index{nucZ (nucleardatapy.setup\_ISGMR.SetupISGMR attribute)@\spxentry{nucZ}\spxextra{nucleardatapy.setup\_ISGMR.SetupISGMR attribute}}

\begin{fulllineitems}
\phantomsection\label{\detokenize{source/api/setup_ISGMR:nucleardatapy.setup_ISGMR.SetupISGMR.nucZ}}
\pysigstartsignatures
\pysigline
{\sphinxbfcode{\sphinxupquote{nucZ}}}
\pysigstopsignatures
\sphinxAtStartPar
Attribute Z (charge of the nucleus).

\end{fulllineitems}

\index{print\_outputs() (nucleardatapy.setup\_ISGMR.SetupISGMR method)@\spxentry{print\_outputs()}\spxextra{nucleardatapy.setup\_ISGMR.SetupISGMR method}}

\begin{fulllineitems}
\phantomsection\label{\detokenize{source/api/setup_ISGMR:nucleardatapy.setup_ISGMR.SetupISGMR.print_outputs}}
\pysigstartsignatures
\pysiglinewithargsret
{\sphinxbfcode{\sphinxupquote{print\_outputs}}}
{}
{}
\pysigstopsignatures
\sphinxAtStartPar
Method which print outputs on terminal’s screen.

\end{fulllineitems}

\index{ref (nucleardatapy.setup\_ISGMR.SetupISGMR attribute)@\spxentry{ref}\spxextra{nucleardatapy.setup\_ISGMR.SetupISGMR attribute}}

\begin{fulllineitems}
\phantomsection\label{\detokenize{source/api/setup_ISGMR:nucleardatapy.setup_ISGMR.SetupISGMR.ref}}
\pysigstartsignatures
\pysigline
{\sphinxbfcode{\sphinxupquote{ref}}}
\pysigstopsignatures
\sphinxAtStartPar
Attribute providing the full reference to the paper to be citted.

\end{fulllineitems}

\index{table (nucleardatapy.setup\_ISGMR.SetupISGMR attribute)@\spxentry{table}\spxextra{nucleardatapy.setup\_ISGMR.SetupISGMR attribute}}

\begin{fulllineitems}
\phantomsection\label{\detokenize{source/api/setup_ISGMR:nucleardatapy.setup_ISGMR.SetupISGMR.table}}
\pysigstartsignatures
\pysigline
{\sphinxbfcode{\sphinxupquote{table}}}
\pysigstopsignatures
\sphinxAtStartPar
Attribute table.

\end{fulllineitems}


\end{fulllineitems}

\index{tables\_isgmr() (in module nucleardatapy.setup\_ISGMR)@\spxentry{tables\_isgmr()}\spxextra{in module nucleardatapy.setup\_ISGMR}}

\begin{fulllineitems}
\phantomsection\label{\detokenize{source/api/setup_ISGMR:nucleardatapy.setup_ISGMR.tables_isgmr}}
\pysigstartsignatures
\pysiglinewithargsret
{\sphinxcode{\sphinxupquote{nucleardatapy.setup\_ISGMR.}}\sphinxbfcode{\sphinxupquote{tables\_isgmr}}}
{}
{}
\pysigstopsignatures
\sphinxAtStartPar
Return a list of tables available in this toolkit for the ISGMR energy and
print them all on the prompt. These tables are the following
ones: ‘2010\sphinxhyphen{}ISGMR\sphinxhyphen{}LI’, ‘2018\sphinxhyphen{}ISGMR\sphinxhyphen{}GARG’, ‘2018\sphinxhyphen{}ISGMR\sphinxhyphen{}GARG\sphinxhyphen{}LATEX’.
\begin{quote}\begin{description}
\sphinxlineitem{Returns}
\sphinxAtStartPar
The list of tables.

\sphinxlineitem{Return type}
\sphinxAtStartPar
list{[}str{]}.

\end{description}\end{quote}

\end{fulllineitems}


\begin{figure}[htbp]
\centering
\capstart

\noindent\sphinxincludegraphics[scale=0.7]{{plot_SetupISGMR}.png}
\caption{ISGMR energies available in the nucleardatapy toolkit.}\label{\detokenize{source/api/setup_ISGMR:id1}}\end{figure}

\sphinxstepscope


\section{SetupEsymLsym}
\label{\detokenize{source/api/setup_EsymLsym:setupesymlsym}}\label{\detokenize{source/api/setup_EsymLsym::doc}}\index{module@\spxentry{module}!nucleardatapy.setup\_EsymLsym@\spxentry{nucleardatapy.setup\_EsymLsym}}\index{nucleardatapy.setup\_EsymLsym@\spxentry{nucleardatapy.setup\_EsymLsym}!module@\spxentry{module}}\index{SetupEsymLsym (class in nucleardatapy.setup\_EsymLsym)@\spxentry{SetupEsymLsym}\spxextra{class in nucleardatapy.setup\_EsymLsym}}\phantomsection\label{\detokenize{source/api/setup_EsymLsym:module-nucleardatapy.setup_EsymLsym}}

\begin{fulllineitems}
\phantomsection\label{\detokenize{source/api/setup_EsymLsym:nucleardatapy.setup_EsymLsym.SetupEsymLsym}}
\pysigstartsignatures
\pysiglinewithargsret
{\sphinxbfcode{\sphinxupquote{class\DUrole{w}{ }}}\sphinxcode{\sphinxupquote{nucleardatapy.setup\_EsymLsym.}}\sphinxbfcode{\sphinxupquote{SetupEsymLsym}}}
{\sphinxparam{\DUrole{n}{constraint}\DUrole{o}{=}\DUrole{default_value}{\textquotesingle{}2014\sphinxhyphen{}IAS\textquotesingle{}}}}
{}
\pysigstopsignatures
\sphinxAtStartPar
Instantiate the values of Esym and Lsym from the constraint.

\sphinxAtStartPar
The name of the constraint to be chosen in the     following list: ‘2009\sphinxhyphen{}HIC’, ‘2010\sphinxhyphen{}RNP’, ‘2012\sphinxhyphen{}FRDM’, ‘2013\sphinxhyphen{}NS’,     ‘2014\sphinxhyphen{}IAS’, ‘2014\sphinxhyphen{}IAS+RNP’, ‘2015\sphinxhyphen{}POL\sphinxhyphen{}208PB’, ‘2015\sphinxhyphen{}POL\sphinxhyphen{}120SN’,     ‘2015\sphinxhyphen{}POL\sphinxhyphen{}68NI’, ‘2017\sphinxhyphen{}UG’, ‘2021\sphinxhyphen{}PREXII\sphinxhyphen{}Reed’,     ‘2021\sphinxhyphen{}PREXII\sphinxhyphen{}Reinhard’, ‘2021\sphinxhyphen{}PREXII+CREX\sphinxhyphen{}Zhang’.
\begin{quote}\begin{description}
\sphinxlineitem{Parameters}
\sphinxAtStartPar
\sphinxstyleliteralstrong{\sphinxupquote{constraint}} (\sphinxstyleliteralemphasis{\sphinxupquote{str}}\sphinxstyleliteralemphasis{\sphinxupquote{, }}\sphinxstyleliteralemphasis{\sphinxupquote{optional.}}) \textendash{} Fix the name of \sphinxtitleref{constraint}. Default value: ‘2014\sphinxhyphen{}IAS’.

\end{description}\end{quote}

\sphinxAtStartPar
\sphinxstylestrong{Attributes:}
\index{Esym (nucleardatapy.setup\_EsymLsym.SetupEsymLsym attribute)@\spxentry{Esym}\spxextra{nucleardatapy.setup\_EsymLsym.SetupEsymLsym attribute}}

\begin{fulllineitems}
\phantomsection\label{\detokenize{source/api/setup_EsymLsym:nucleardatapy.setup_EsymLsym.SetupEsymLsym.Esym}}
\pysigstartsignatures
\pysigline
{\sphinxbfcode{\sphinxupquote{Esym}}}
\pysigstopsignatures
\sphinxAtStartPar
Attribute Esym.

\end{fulllineitems}

\index{Esym\_err (nucleardatapy.setup\_EsymLsym.SetupEsymLsym attribute)@\spxentry{Esym\_err}\spxextra{nucleardatapy.setup\_EsymLsym.SetupEsymLsym attribute}}

\begin{fulllineitems}
\phantomsection\label{\detokenize{source/api/setup_EsymLsym:nucleardatapy.setup_EsymLsym.SetupEsymLsym.Esym_err}}
\pysigstartsignatures
\pysigline
{\sphinxbfcode{\sphinxupquote{Esym\_err}}}
\pysigstopsignatures
\sphinxAtStartPar
Attribute with uncertainty in Esym.

\end{fulllineitems}

\index{Esym\_max (nucleardatapy.setup\_EsymLsym.SetupEsymLsym attribute)@\spxentry{Esym\_max}\spxextra{nucleardatapy.setup\_EsymLsym.SetupEsymLsym attribute}}

\begin{fulllineitems}
\phantomsection\label{\detokenize{source/api/setup_EsymLsym:nucleardatapy.setup_EsymLsym.SetupEsymLsym.Esym_max}}
\pysigstartsignatures
\pysigline
{\sphinxbfcode{\sphinxupquote{Esym\_max}}}
\pysigstopsignatures
\sphinxAtStartPar
Attribute max of Esym.

\end{fulllineitems}

\index{Esym\_min (nucleardatapy.setup\_EsymLsym.SetupEsymLsym attribute)@\spxentry{Esym\_min}\spxextra{nucleardatapy.setup\_EsymLsym.SetupEsymLsym attribute}}

\begin{fulllineitems}
\phantomsection\label{\detokenize{source/api/setup_EsymLsym:nucleardatapy.setup_EsymLsym.SetupEsymLsym.Esym_min}}
\pysigstartsignatures
\pysigline
{\sphinxbfcode{\sphinxupquote{Esym\_min}}}
\pysigstopsignatures
\sphinxAtStartPar
Attribute min of Esym.

\end{fulllineitems}

\index{Lsym (nucleardatapy.setup\_EsymLsym.SetupEsymLsym attribute)@\spxentry{Lsym}\spxextra{nucleardatapy.setup\_EsymLsym.SetupEsymLsym attribute}}

\begin{fulllineitems}
\phantomsection\label{\detokenize{source/api/setup_EsymLsym:nucleardatapy.setup_EsymLsym.SetupEsymLsym.Lsym}}
\pysigstartsignatures
\pysigline
{\sphinxbfcode{\sphinxupquote{Lsym}}}
\pysigstopsignatures
\sphinxAtStartPar
Attribute Lsym.

\end{fulllineitems}

\index{Lsym\_err (nucleardatapy.setup\_EsymLsym.SetupEsymLsym attribute)@\spxentry{Lsym\_err}\spxextra{nucleardatapy.setup\_EsymLsym.SetupEsymLsym attribute}}

\begin{fulllineitems}
\phantomsection\label{\detokenize{source/api/setup_EsymLsym:nucleardatapy.setup_EsymLsym.SetupEsymLsym.Lsym_err}}
\pysigstartsignatures
\pysigline
{\sphinxbfcode{\sphinxupquote{Lsym\_err}}}
\pysigstopsignatures
\sphinxAtStartPar
Attribute with uncertainty in Lsym.

\end{fulllineitems}

\index{Lsym\_max (nucleardatapy.setup\_EsymLsym.SetupEsymLsym attribute)@\spxentry{Lsym\_max}\spxextra{nucleardatapy.setup\_EsymLsym.SetupEsymLsym attribute}}

\begin{fulllineitems}
\phantomsection\label{\detokenize{source/api/setup_EsymLsym:nucleardatapy.setup_EsymLsym.SetupEsymLsym.Lsym_max}}
\pysigstartsignatures
\pysigline
{\sphinxbfcode{\sphinxupquote{Lsym\_max}}}
\pysigstopsignatures
\sphinxAtStartPar
Attribute max of Lsym.

\end{fulllineitems}

\index{Lsym\_min (nucleardatapy.setup\_EsymLsym.SetupEsymLsym attribute)@\spxentry{Lsym\_min}\spxextra{nucleardatapy.setup\_EsymLsym.SetupEsymLsym attribute}}

\begin{fulllineitems}
\phantomsection\label{\detokenize{source/api/setup_EsymLsym:nucleardatapy.setup_EsymLsym.SetupEsymLsym.Lsym_min}}
\pysigstartsignatures
\pysigline
{\sphinxbfcode{\sphinxupquote{Lsym\_min}}}
\pysigstopsignatures
\sphinxAtStartPar
Attribute min of Lsym.

\end{fulllineitems}

\index{alpha (nucleardatapy.setup\_EsymLsym.SetupEsymLsym attribute)@\spxentry{alpha}\spxextra{nucleardatapy.setup\_EsymLsym.SetupEsymLsym attribute}}

\begin{fulllineitems}
\phantomsection\label{\detokenize{source/api/setup_EsymLsym:nucleardatapy.setup_EsymLsym.SetupEsymLsym.alpha}}
\pysigstartsignatures
\pysigline
{\sphinxbfcode{\sphinxupquote{alpha}}}
\pysigstopsignatures
\sphinxAtStartPar
Attribute the plot alpha

\end{fulllineitems}

\index{constraint (nucleardatapy.setup\_EsymLsym.SetupEsymLsym attribute)@\spxentry{constraint}\spxextra{nucleardatapy.setup\_EsymLsym.SetupEsymLsym attribute}}

\begin{fulllineitems}
\phantomsection\label{\detokenize{source/api/setup_EsymLsym:nucleardatapy.setup_EsymLsym.SetupEsymLsym.constraint}}
\pysigstartsignatures
\pysigline
{\sphinxbfcode{\sphinxupquote{constraint}}}
\pysigstopsignatures
\sphinxAtStartPar
Attribute constraint.

\end{fulllineitems}

\index{label (nucleardatapy.setup\_EsymLsym.SetupEsymLsym attribute)@\spxentry{label}\spxextra{nucleardatapy.setup\_EsymLsym.SetupEsymLsym attribute}}

\begin{fulllineitems}
\phantomsection\label{\detokenize{source/api/setup_EsymLsym:nucleardatapy.setup_EsymLsym.SetupEsymLsym.label}}
\pysigstartsignatures
\pysigline
{\sphinxbfcode{\sphinxupquote{label}}}
\pysigstopsignatures
\sphinxAtStartPar
Attribute providing the label the data is references for figures.

\end{fulllineitems}

\index{note (nucleardatapy.setup\_EsymLsym.SetupEsymLsym attribute)@\spxentry{note}\spxextra{nucleardatapy.setup\_EsymLsym.SetupEsymLsym attribute}}

\begin{fulllineitems}
\phantomsection\label{\detokenize{source/api/setup_EsymLsym:nucleardatapy.setup_EsymLsym.SetupEsymLsym.note}}
\pysigstartsignatures
\pysigline
{\sphinxbfcode{\sphinxupquote{note}}}
\pysigstopsignatures
\sphinxAtStartPar
Attribute providing additional notes about the constraint.

\end{fulllineitems}

\index{print\_outputs() (nucleardatapy.setup\_EsymLsym.SetupEsymLsym method)@\spxentry{print\_outputs()}\spxextra{nucleardatapy.setup\_EsymLsym.SetupEsymLsym method}}

\begin{fulllineitems}
\phantomsection\label{\detokenize{source/api/setup_EsymLsym:nucleardatapy.setup_EsymLsym.SetupEsymLsym.print_outputs}}
\pysigstartsignatures
\pysiglinewithargsret
{\sphinxbfcode{\sphinxupquote{print\_outputs}}}
{}
{}
\pysigstopsignatures
\sphinxAtStartPar
Method which print outputs on terminal’s screen.

\end{fulllineitems}

\index{ref (nucleardatapy.setup\_EsymLsym.SetupEsymLsym attribute)@\spxentry{ref}\spxextra{nucleardatapy.setup\_EsymLsym.SetupEsymLsym attribute}}

\begin{fulllineitems}
\phantomsection\label{\detokenize{source/api/setup_EsymLsym:nucleardatapy.setup_EsymLsym.SetupEsymLsym.ref}}
\pysigstartsignatures
\pysigline
{\sphinxbfcode{\sphinxupquote{ref}}}
\pysigstopsignatures
\sphinxAtStartPar
Attribute providing the full reference to the paper to be citted.

\end{fulllineitems}


\end{fulllineitems}

\index{constraints\_EsymLsym() (in module nucleardatapy.setup\_EsymLsym)@\spxentry{constraints\_EsymLsym()}\spxextra{in module nucleardatapy.setup\_EsymLsym}}

\begin{fulllineitems}
\phantomsection\label{\detokenize{source/api/setup_EsymLsym:nucleardatapy.setup_EsymLsym.constraints_EsymLsym}}
\pysigstartsignatures
\pysiglinewithargsret
{\sphinxcode{\sphinxupquote{nucleardatapy.setup\_EsymLsym.}}\sphinxbfcode{\sphinxupquote{constraints\_EsymLsym}}}
{}
{}
\pysigstopsignatures
\sphinxAtStartPar
Return a list of constraints available in this toolkit in the     following list: ‘2009\sphinxhyphen{}HIC’, ‘2010\sphinxhyphen{}RNP’, ‘2012\sphinxhyphen{}FRDM’, ‘2013\sphinxhyphen{}NS’,     ‘2014\sphinxhyphen{}IAS’, ‘2014\sphinxhyphen{}IAS+RNP’, ‘2015\sphinxhyphen{}POL\sphinxhyphen{}208PB’, ‘2015\sphinxhyphen{}POL\sphinxhyphen{}120SN’,     ‘2015\sphinxhyphen{}POL\sphinxhyphen{}68NI’, ‘2017\sphinxhyphen{}UG’, ‘2021\sphinxhyphen{}PREXII\sphinxhyphen{}Reed’,     ‘2021\sphinxhyphen{}PREXII\sphinxhyphen{}Reinhard’, ‘2023\sphinxhyphen{}PREXII+CREX\sphinxhyphen{}Zhang’; and     print them all on the prompt.
\begin{quote}\begin{description}
\sphinxlineitem{Returns}
\sphinxAtStartPar
The list of constraints.

\sphinxlineitem{Return type}
\sphinxAtStartPar
list{[}str{]}.

\end{description}\end{quote}

\end{fulllineitems}


\sphinxAtStartPar
Here are a set of figures which are produced with the Python sample: /sample/nucleardatapy\_plots/plot\_setupEsymLsym.py

\begin{figure}[htbp]
\centering
\capstart

\noindent\sphinxincludegraphics[scale=0.7]{{plot_SetupEsymLsym}.png}
\caption{This figure shows the Esym,2 versus Lsym,2 correlation for the different constraints availble in the nucleardatapy toolkit.}\label{\detokenize{source/api/setup_EsymLsym:id1}}\end{figure}


\chapter{Indices and tables}
\label{\detokenize{index:indices-and-tables}}\begin{itemize}
\item {} 
\sphinxAtStartPar
\DUrole{xref}{\DUrole{std}{\DUrole{std-ref}{genindex}}}

\item {} 
\sphinxAtStartPar
\DUrole{xref}{\DUrole{std}{\DUrole{std-ref}{modindex}}}

\item {} 
\sphinxAtStartPar
\DUrole{xref}{\DUrole{std}{\DUrole{std-ref}{search}}}

\end{itemize}


\renewcommand{\indexname}{Python Module Index}
\begin{sphinxtheindex}
\let\bigletter\sphinxstyleindexlettergroup
\bigletter{n}
\item\relax\sphinxstyleindexentry{nucleardatapy}\sphinxstyleindexpageref{source/generated/nucleardatapy:\detokenize{module-nucleardatapy}}
\item\relax\sphinxstyleindexentry{nucleardatapy.setup\_crust}\sphinxstyleindexpageref{source/api/setup_crust:\detokenize{module-nucleardatapy.setup_crust}}
\item\relax\sphinxstyleindexentry{nucleardatapy.setup\_EsymLsym}\sphinxstyleindexpageref{source/api/setup_EsymLsym:\detokenize{module-nucleardatapy.setup_EsymLsym}}
\item\relax\sphinxstyleindexentry{nucleardatapy.setup\_hic\_matter}\sphinxstyleindexpageref{source/api/setup_hic_matter:\detokenize{module-nucleardatapy.setup_hic_matter}}
\item\relax\sphinxstyleindexentry{nucleardatapy.setup\_ISGMR}\sphinxstyleindexpageref{source/api/setup_ISGMR:\detokenize{module-nucleardatapy.setup_ISGMR}}
\item\relax\sphinxstyleindexentry{nucleardatapy.setup\_masses\_exp}\sphinxstyleindexpageref{source/api/setup_masses_exp:\detokenize{module-nucleardatapy.setup_masses_exp}}
\item\relax\sphinxstyleindexentry{nucleardatapy.setup\_masses\_theory}\sphinxstyleindexpageref{source/api/setup_masses_theory:\detokenize{module-nucleardatapy.setup_masses_theory}}
\item\relax\sphinxstyleindexentry{nucleardatapy.setup\_micro\_LP}\sphinxstyleindexpageref{source/api/setup_micro_LP:\detokenize{module-nucleardatapy.setup_micro_LP}}
\item\relax\sphinxstyleindexentry{nucleardatapy.setup\_micro\_matter}\sphinxstyleindexpageref{source/api/setup_micro_matter:\detokenize{module-nucleardatapy.setup_micro_matter}}
\item\relax\sphinxstyleindexentry{nucleardatapy.setup\_micro\_matter\_band}\sphinxstyleindexpageref{source/api/setup_micro_matter_band:\detokenize{module-nucleardatapy.setup_micro_matter_band}}
\item\relax\sphinxstyleindexentry{nucleardatapy.setup\_pheno\_matter}\sphinxstyleindexpageref{source/api/setup_pheno_matter:\detokenize{module-nucleardatapy.setup_pheno_matter}}
\item\relax\sphinxstyleindexentry{nucleardatapy.setup\_rad\_ch}\sphinxstyleindexpageref{source/api/setup_rad_ch:\detokenize{module-nucleardatapy.setup_rad_ch}}
\end{sphinxtheindex}

\renewcommand{\indexname}{Index}
\printindex
\end{document}